\documentclass[UTF8,12pt]{ctexart}
\usepackage{amsmath,amssymb,geometry,bm,graphicx,fontspec,amssymb,amsthm}
\usepackage[mathscr]{euscript}

\usepackage{tabularray}

\usepackage[colorlinks,
linkcolor=black,
anchorcolor=blue,
citecolor=green
]{hyperref} % 目录中的超链接

\newtheorem{Def}{定义}[section]
\newtheorem{Theo}{定理}[section]
\newtheorem{Lemm}{引理}[section]
\newtheorem{Prop}{命题}[section]
\newtheorem{Assu}{假设}[section]
\newtheorem{Axiom}{Axiom}

\numberwithin{equation}{section} % 按章节进行排序与编号
\numberwithin{figure}{section}
\numberwithin{table}{section}

\usepackage{draftwatermark} % 所有页加水印
\SetWatermarkText{EconNerd} % 设置水印内容
\SetWatermarkLightness{0.99} % 设置水印透明度 0-1
\SetWatermarkScale{1} % 设置水印大小    

\title{税法} % 文档相关信息
\author{EconNerd}
\date{\today}
\geometry{scale=0.8}

\begin{document}
	\maketitle
	\tableofcontents
	\newpage
	
	\section{税法总论}
	\subsection{税法概念}
	\subsubsection{税收与税法的概念}
	税收是一个经济学概念,是取得财政收入的一种重要工具。主要特征在于\textbf{强制性、无偿性、固定性}。调整的对象是征税形成的\textbf{分配关系}
	
	税法是一个法学概念,是税收制度的核心内容。税法属于义务性规定,以规定纳税人的义务为主,具有综合性。调整的对象是税收法律关系主体的\textbf{权利与义务关系}
	
	\subsubsection{税收法律关系}
	税收法律关系主要包含三个内容
	\begin{enumerate}
		\item 权利主体。双主体,一方是国家以及代表国家行使征税职责的税务机关、海关;另一方是履行纳税义务的法人、自然人和其他组织
		
		\item 权利客体。税收法律关系主体的权利、义务所共同指向的对象,即征税对象
		
		\item 内容。是权利主体所享有的权利和所应承担的义务,是税收法律关系中最实质的东西,也是税法的灵魂
	\end{enumerate}
	
	税收法律关系的产生、变更由\textbf{税收法律事实}来决定——税收法律事实分为税收法律事件和税收法律行为
	
	税收法律关系的保护对权利主体双方是平等的,同时对其享有权利的保护就是对其承担义务的制约
	
	\subsubsection{税法与其他法律关系}
	\begin{enumerate}
		\item 宪法。税法依据宪法的原则制定
		
		《宪法》第五十六条规定:“中华人民共和国公民有依照法律纳税的义务”
		
		《宪法》第三十三条规定:“中华人民共和国公民在法律面前一律平等”
		
		\item 民法。 民法调整方法的主要特点是平等、等价、有偿;税法调整方法要采用命令、服从的方法
		
		民法与税法不发生冲突时,税法不再另行规定;出现不一致时,一般按税法规定纳税
		
		\item 刑法。调整范围不同,违反了税法,并不一定就是刑事犯罪,但违反税法情节严重触及刑律者,将受刑事处罚
		
		\item 行政法。税法与行政法有着十分密切的联系,税法具有行政法的一般特性

		税法又与一般行政法有所不同——税法具有经济分配的性质,并且经济利益由纳税人向国家无偿单向转移,这是一般行政法所不具备的
	\end{enumerate}
	
	\subsection{税法原则}
	税法原则包含两类,分别是税法基本原则和税法适用原则
	\subsubsection{税法基本原则}
	\paragraph{税收法定原则}也称为税收法定主义,是指税法主体的权利义务必须由法律加以规定,税法的各类构成要素都必须且只能由法律予以明确
	
	税收法定主义贯穿税收立法和执法的全部领域,其内容包括税收要件法定原则和税务合法性原则
	\begin{enumerate}
		\item 税收要件法定原则
		\begin{enumerate}
			\item 税种实施法定——国家对其开征的任何税种都必须由法律对其进行专门确定才能实施
			
			\item 要素变动法定——国家对任何税种征税要素的变动都应当按相关法律的规定进行
			
			\item 征税要素明确——征税的各个要素不仅应当法定,还应当尽量明确,避免歧义
		\end{enumerate}
		
		\item 税务合法性原则
		\begin{enumerate}
			\item 立良法——在立法的过程中要对法定征收程序加以明确规定,提高工作效率,节约社会成本,尊重并保护税收债务人的程序性权利,促使其提高纳税的意识
			
			\item 求善治——要求征税机关及其工作人员在征税过程中,必须按照税收程序法和税收实体法律的规定来行使自己的职权,履行自己的职责,充分尊重纳税人的各项权利
		\end{enumerate}
	\end{enumerate}
	
	
	\paragraph{税收公平原则}
	一般认为税收公平原则包括税收横向公平和纵向公平,即税收负担必须根据纳税人的负担能力分配,负担能力相同的税负相同(横向公平);负担能力不同的税负不同(纵向公平)。税收公平原则源于法律上的平等性原则
	
	\paragraph{税收效率原则}
	税收效率原则包括两个方面:经济效率和行政效率。前者要求税法的制定要有利于资源的有效配置和经济体制的有效运行,后者要求提高税收行政效率,节约税收征管成本
	
	\paragraph{实质课税原则}
	是指应根据客观事实确定是否符合课税要件,并根据纳税人的真实负担能力决定纳税人的税负,而不能仅考虑相关外观和形式
	
	
	\subsubsection{税法适用原则}
	税法适用原则是指税务行政机关和司法机关运用税收法律规范解决具体问题所必须遵循的准则。与税法基本原则相比,税法适用原则含有更多的法律技术性准则,更为具体化。
	
	\paragraph{法律优位原则}其基本含义为法律的效力高于行政立法的效力,还可进一步推论为税收行政法规的效力优于税收行政规章的效力;效力低的税法与效力高的税法发生冲突时,效力低的税法即是无效的。
	
	\paragraph{法律不溯及既往原则}即新法实施后,对新法实施之前人们的行为不得适用新法,而只能沿用旧法。
	
	\paragraph{新法优于旧法原则}也称后法优于先法原则,即新法、旧法对同一事项有不同规定时,新法的效力优于旧法。
	
	\paragraph{特别法优于普通法原则}对同一事项两部法律分别订有一般和特别规定时,特别规定的效力高于一般规定的效力。本原则打破了税法效力等级的限制,在授权范围内,居于特别法地位的级别比较低的税法,其效力可高于作为普通法的级别比较高的税法。
	
	\paragraph{实体从旧、程序从新原则}即实体性税法不具备溯及力,而程序性税法在特定条件下具备一定溯及力。
	
	\paragraph{程序优于实体原则}即在诉讼发生时,税收程序法优于税收实体法,以保证国家课税权的实现。
	
	\subsection{税法(种)要素}
	不同考试教材对于税法(种)构成要素的表述是有差异。CPA教材收录了11个要素,分别是:总则、纳税义务人、征税对象(课税对象)、税目、税率、纳税环节、纳税期限、纳税地点、减税免税(税收优惠)、罚则、附则。重点为以下七个
	
	\paragraph{纳税义务人}
	又称“纳税主体”,是税法规定的直接负有纳税义务的单位和个人。解决的是国家对谁征税的问题。纳税人有两种基本形式:自然人和法人。
	
	自然人可划分为居民个人和非居民个人;法人可划分为居民企业和非居民企业,还可按企业的不同所有制性质来进行分类。也可以根据民法典分为营利法人、非营利法人和特别法人。
	
	与纳税人紧密联系的两个概念是代扣代缴义务人和代收代缴义务人。两者共同简称扣缴义务人
	\begin{enumerate}
		\item 代扣代缴义务人:指虽不承担纳税义务,但依照有关规定,在向纳税人支付收入、结算货款、支付费用时有义务代扣代缴其应纳税款的单位和个人
		
		支付个人稿酬的单位代扣代缴个人所得税;支付境外企业股息红利的单位代扣代缴企业所得税
		
		\item 代收代缴义务人:指虽不承担纳税义务,但依照有关规定,在向纳税人收取商品或劳务收入时,有义务代收代缴其应纳税款的单位和个人
		
		受托加工应税消费品的单位,代收代缴消费税;办理交强险的保险机构代收代缴车船税
	\end{enumerate}
	
	\paragraph{征税对象}
	征税对象又叫课税对象、征税客体,指税法规定\textbf{对什么征税},是征纳税双方权利义务共同指向的客体或标的。征税对象是区别一种税与另一种税的重要标志。
	
	与征税对象相关的两个重要概念:税目与税基。
	
	在CPA教材中,税目也是税法要素之一。税目是在税法中对征税对象分类规定的具体征税项目,反映具体的征税范围,是对课税对象质的界定。税目体现征税的广度。
	
	税基又叫计税依据,是据以计算征税对象应纳税款的直接数量依据,它解决对征税对象课税的计算问题,是对课税对象量的规定。(计税依据只是与征税对象密切相关的概念,不属于税法要素)
	
	\paragraph{税率}
	税率主要分为三类
	\begin{enumerate}
		\item 比例税率。比如增值税、城市维护建设税、企业所得税等
		
		\item 定额税率。比如城镇土地使用税、车船税等
		
		\item 累进税率。又有四类
		\begin{enumerate}
			\item 全额累进税率。目前我国没有采用
			
			\item 超额累进税率。个人所得税中的综合所得、经营所得
			
			\item 全率累进税率。目前我国没有采用
			
			\item 超率累进税率。比如土地增值税
		\end{enumerate}
	\end{enumerate}
	
	\paragraph{纳税环节}
	指税法规定的征税对象在从生产到消费的流转过程中应当缴纳税款的环节。要掌握生产、批发、零售、进出口等各个环节上的税种分布。
	
	\paragraph{纳税期限}
	指税法规定的关于税款缴纳时间方面的限定。有三个相关概念:纳税义务发生时间、纳税期限、缴库期限。应掌握各税种纳税期限、结算缴款期限与滞纳金计算的关系。
	
	\paragraph{纳税地点}
	指税法规定纳税人(包括代征、代扣、代缴义务人)申报纳税的地点。纳税地点关系到税收管辖权和是否便利纳税等问题,在税法中明确规定纳税地点有助于防止漏征或重复征税。各种纳税地点的规定都易出客观题。
	
	\paragraph{减免税收(税收优惠)}
	指对某些纳税人和征税对象采取减少征税或免予征税的特殊规定。各章节的减税、免税规定往往存在大量考点。
	
	\subsection{税收立法与我国税法体系}
	
	\subsubsection{税收立法原则}
	税收立法不仅包括对税收法律、法规、规章的制定、公布、修改和补充,也包括对税收法律、法规、规章的废止(不包括试行)
	
	\subsubsection{税收立法权及其划分}
	\textbf{立法权可以授予某级政府},行政上的执行权给予另一级——我国的税收立法权的划分就属于这种模式
	
	中央税、中央与地方共享税以及全国统一实行的地方税的\textbf{立法权集中在中央}。
	
	\subsubsection{税收立法机关}
	税收法律分类不同,立法机关及形式也存在不同
	\begin{enumerate}
		\item 税收法律。全国人大及常委会制定的税收法律
		
		除《宪法》外,在税法体系中,税收法律具有最高的法律效力。例如:《企业所得税法》《个人所得税法》《车船税法》《环境保护税法》《烟叶税法》《船舶吨税法》《耕地占用税法》《车辆购置税法》《资源税法》《契税法》《城市维护建设税法》《印花税法》《关税法》《增值税法》《税收征收管理法》
		
		\textbf{目前尚未立法的税收为消费税、房产税、城镇土地使用税、土地增值税。}
		
		\item 授权立法。全国人大及常委会授权国务院指定的暂行规定或条例。
		
		属于准法律。具有国家法律的性质和地位,为待条件成熟上升为法律做好准备。例如消费税暂行条例
		
		\item 税收法规
		\begin{enumerate}
			\item 国务院指定的税收行政法规。
			
			\item 地方人大及常委会指定的税收地方性法规。目前仅限于海南省、民族自治区。
		\end{enumerate}
		
		\item 税收规章
		\begin{enumerate}
			\item 国务院税务主管部门制定的税收部门规章
			
			国务院税务主管部门指财政部、国家税务总局和海关总署。该级次规章不得与宪法、税收法律、行政法规相抵触。例如:《税务代理试行办法》
			
			\item 地方政府制定的税收地方规章
			
			省、自治区、直辖市的人民政府可以根据法律、
			行政法规和本省、自治区、直辖市的地方性法规,
			制定规章,报国务院和本级人民代表大会常务委
			员会备案。设区的市、自治州的人民政府可以根
			据法律、行政法规和本省、自治区的地方性法规,
			依照法律规定的权限制定规章,报国务院和省、
			自治区的人民代表大会常务委员会、人民政府以
			及本级人民代表大会常务委员会备案。例如:房
			产税等地方性税种的实施细则等
		\end{enumerate}
	\end{enumerate}
	
	\subsubsection{税收立法程序}
	税收立法程序主要包括三个阶段:(1)提议阶段;(2)审议阶段;(3)通过和公布阶段。
	
	\subsubsection{我国现行税法体系}
	税法体系就是通常所说的税收制度。一个国家的税收制度,可按构成方法和形式分为简单型税制及复合型税制。我国税收制度属于复合型税制。
	
	税法体系中各税法按基本内容和效力、职能作用、权限范围的不同,可分为不同类型。
	\begin{enumerate}
		\item 按照税法的基本内容和效力的不同,可以分为税收基本法和税收普通法。
		
		\item 按照税法的职能作用可以分为税收实体法和税收程序法。
		
		\item 按照主权国家形式税收管辖权的不同可以分为国内税法和国际税法。 
	\end{enumerate}
	
	税种的分类不具有法定性。我国现行税收实体法体系可以按照以下进行分类
	\begin{enumerate}
		\item 按照征税对象不同分为
		\begin{enumerate}
			\item 商品(货物)和劳务税类。增值税、消费税、关税
			
			\item 资源税和环境保护税。资源税、环境保护税、城镇土地使用税
			
			\item 所得税类。企业所得税、个人所得税、\textbf{土地增值税}。
			
			\item 特定的税类。城市维护建设税、车辆购置税、耕地占用税、烟叶税、船舶吨税 
			
			\item 财产和行为税类。房产税、车船税、印花税、契税
		\end{enumerate}
		
		\item 按照税负是否容易转嫁分为
		\begin{enumerate}
			\item 直接税。如企业所得税、个人所得税、契税
			
			\item 间接税。处于生产流通环节的税种一般属于间接税,如增值税、消费税、关税
		\end{enumerate}
		
		\item 按照计税价格中是否包含税款可以分为
		\begin{enumerate}
			\item 价内税。如消费税、资源税
			
			\item 价外税。如增值税、关税、车辆购置税
		\end{enumerate}
	\end{enumerate}
	
	除税收实体法外,我国还有一系列税务管理流程和时限的法律制度,是按照税收管理机关的不同而分别规定的:
	\begin{enumerate}
		\item 由税务机关负责征收的税种的征收管理,按照全国人大常委会发布实施的《税收征收管理法》及各实体税法中的征管规定执行。
		
		\item 由海关负责征收的税种的征收管理,按照《中华人民共和国海关法》及《中华人民共和国关税法》中的征管规定执行。
	\end{enumerate}
	
	
	\subsection{税收执法}
	税法的实施即税法的执行。它包括税收执法和守法两个方面:一方面,执法机关和人员要依法执法;另一方面,征纳双方都要守法。由于税法具有多层次的特点,在税收执法过程中,对其适用性或法律效力的判断,一般按以下原则掌握:
	\begin{enumerate}
		\item 层次高的法律优于层次低的法律;
		
		\item 同一层次的法律中特别法优于普通法;
		
		\item 国际法优于国内法;
		
		\item 实体法从旧、程序法从新。
	\end{enumerate}
	税收执法权具体包括税款征收管理权、税务稽查权、税务检查权、税务行政复议裁决权及其他税务管理权。
	
	\subsubsection{税务机构设置与职能}
	2018年,根据我国经济和社会发展及推进国家治理体系和治理能力现代化的需要,对国税地税征管体制进行了改革。现行税务机构设置是中央政府设立国家税务总局(正部级),原有的省及省以下国税地税机构两个系统通过合并整合,统一设置为省、市、县三级税务局,实行以国家税务总局为主与省(自治区、直辖市)人民政府双重领导管理体制。
	
	\subsubsection{税收征收管理范围划分}
	目前,我国的税收分别由税务、海关两个系统负责征收管理。
	\begin{enumerate}
		\item 税务系统(国家税务总局系统):增值税;消费税;车辆购置税;城市维护建设税;企业所得税;个人所得税;资源税;城镇土地使用税;耕地占用税;土地增值税;房产税;车船税;印花税;契税;烟叶税;环境保护税
		
		\item 海关:\textbf{关税;船舶吨税};代征进口环节的增值税、消费税
	\end{enumerate}
	
	\subsubsection{税收收入划分}
	收入划分如下
	\begin{enumerate}
		\item 中央固定收入:消费税(含税务机关征收及进口环节海关代征的全部消费税);车辆购置税;关税;船舶吨税;海关代征的进口环节增值税;铁路建设基金营改增
		
		\item 地方固定收入:房产税;城镇土地使用税;耕地占用税;土地增值税;车船税;契税;烟叶税;环境保护税
		
		\item 中央、地方共享收入:增值税;城市维护建设税;企业所得税;个人所得税;印花税;资源税
	\end{enumerate}
	
	其中对于中央、地方共享收入的方式和比例如下
	\begin{enumerate}
		\item 海关代征的部分;铁路建设基金营
		税务机关征收部分扣除营改增固定
		增值税
		改增的部分;其余部分的50%
		给中央项目之后的50%
		
		\item 中国国家铁路集团、海洋石油企业、
		除已列举归属中央之外的企业所得
		中石油和中石化企业、国有邮政企
		企业所得税
		税的40%
		业、各银行总行的企业所得税;除
		上述之外企业所得税的60%
		
		\item 个人所得税。中央收入60\%,地方收入40\%
		
		\item 中国国家铁路集团、各银行总行、
		城市维护建设税
		其他城建税
		各保险总公司集中缴纳的部分
		
		\item 印花税。中央收入为证券交易印花税,地方收入为其他印花税
		
		\item 资源税。中央收入为海洋石油企业资源税,地方收入为其他资源税
	\end{enumerate}
	
	\subsection{税收权利与义务}
	
	\subsection{国际税收关系}
	
	\section{增值税法}
	
	\section{消费税法}
	
	\section{企业所得税}
	
	\section{个人所得税}
	
	\section{城市维护建设税法和烟叶税法}
	
	\section{关税法和船舶吨税法}
	
	\section{资源税法和资源保护税法}
	
	\section{城镇土地使用税法和耕地占用税法}
	
	\section{房产税法、契税法和土地增值税法}
	
	\section{车辆购置税法、车窗税法和印花税法}
	
	\section{国际税收管理实务}
	
	\section{税收征收管理法}
	
	\section{税务行政法则}
	

	
\end{document}