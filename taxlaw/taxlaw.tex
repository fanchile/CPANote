\documentclass[UTF8,12pt]{ctexart}
\usepackage{amsmath,amssymb,geometry,bm,graphicx,fontspec,amssymb,amsthm}
\usepackage[mathscr]{euscript}

\usepackage{tabularray}

\usepackage[colorlinks,
linkcolor=black,
anchorcolor=blue,
citecolor=green
]{hyperref} % 目录中的超链接

\newtheorem{Def}{定义}[section]
\newtheorem{Theo}{定理}[section]
\newtheorem{Lemm}{引理}[section]
\newtheorem{Prop}{命题}[section]
\newtheorem{Assu}{假设}[section]
\newtheorem{Axiom}{Axiom}

\numberwithin{equation}{section} % 按章节进行排序与编号
\numberwithin{figure}{section}
\numberwithin{table}{section}

\usepackage{draftwatermark} % 所有页加水印
\SetWatermarkText{EconNerd} % 设置水印内容
\SetWatermarkLightness{0.99} % 设置水印透明度 0-1
\SetWatermarkScale{1} % 设置水印大小    

\title{税法} % 文档相关信息
\author{EconNerd}
\date{\today}
\geometry{scale=0.8}

\begin{document}
	\maketitle
	\tableofcontents
	\newpage
	
	\section{税法总论}
	\subsection{税法概念}
	\subsubsection{税收与税法的概念}
	税收是一个经济学概念,是取得财政收入的一种重要工具。主要特征在于\textbf{强制性、无偿性、固定性}。调整的对象是征税形成的\textbf{分配关系}
	
	税法是一个法学概念,是税收制度的核心内容。税法属于义务性规定,以规定纳税人的义务为主,具有综合性。调整的对象是税收法律关系主体的\textbf{权利与义务关系}
	
	\subsubsection{税收法律关系}
	税收法律关系主要包含三个内容
	\begin{enumerate}
		\item 权利主体。双主体,一方是国家以及代表国家行使征税职责的税务机关、海关;另一方是履行纳税义务的法人、自然人和其他组织
		
		\item 权利客体。税收法律关系主体的权利、义务所共同指向的对象,即征税对象
		
		\item 内容。是权利主体所享有的权利和所应承担的义务,是税收法律关系中最实质的东西,也是税法的灵魂
	\end{enumerate}
	
	税收法律关系的产生、变更由\textbf{税收法律事实}来决定——税收法律事实分为税收法律事件和税收法律行为
	
	税收法律关系的保护对权利主体双方是平等的,同时对其享有权利的保护就是对其承担义务的制约
	
	\subsubsection{税法与其他法律关系}
	\begin{enumerate}
		\item 宪法。税法依据宪法的原则制定
		
		《宪法》第五十六条规定:“中华人民共和国公民有依照法律纳税的义务”
		
		《宪法》第三十三条规定:“中华人民共和国公民在法律面前一律平等”
		
		\item 民法。 民法调整方法的主要特点是平等、等价、有偿;税法调整方法要采用命令、服从的方法
		
		民法与税法不发生冲突时,税法不再另行规定;出现不一致时,一般按税法规定纳税
		
		\item 刑法。调整范围不同,违反了税法,并不一定就是刑事犯罪,但违反税法情节严重触及刑律者,将受刑事处罚
		
		\item 行政法。税法与行政法有着十分密切的联系,税法具有行政法的一般特性

		税法又与一般行政法有所不同——税法具有经济分配的性质,并且经济利益由纳税人向国家无偿单向转移,这是一般行政法所不具备的
	\end{enumerate}
	
	\subsection{税法原则}
	税法原则包含两类,分别是税法基本原则和税法适用原则
	\subsubsection{税法基本原则}
	\paragraph{税收法定原则}也称为税收法定主义,是指税法主体的权利义务必须由法律加以规定,税法的各类构成要素都必须且只能由法律予以明确
	
	税收法定主义贯穿税收立法和执法的全部领域,其内容包括税收要件法定原则和税务合法性原则
	\begin{enumerate}
		\item 税收要件法定原则
		\begin{enumerate}
			\item 税种实施法定——国家对其开征的任何税种都必须由法律对其进行专门确定才能实施
			
			\item 要素变动法定——国家对任何税种征税要素的变动都应当按相关法律的规定进行
			
			\item 征税要素明确——征税的各个要素不仅应当法定,还应当尽量明确,避免歧义
		\end{enumerate}
		
		\item 税务合法性原则
		\begin{enumerate}
			\item 立良法——在立法的过程中要对法定征收程序加以明确规定,提高工作效率,节约社会成本,尊重并保护税收债务人的程序性权利,促使其提高纳税的意识
			
			\item 求善治——要求征税机关及其工作人员在征税过程中,必须按照税收程序法和税收实体法律的规定来行使自己的职权,履行自己的职责,充分尊重纳税人的各项权利
		\end{enumerate}
	\end{enumerate}
	
	
	\paragraph{税收公平原则}
	一般认为税收公平原则包括税收横向公平和纵向公平,即税收负担必须根据纳税人的负担能力分配,负担能力相同的税负相同(横向公平);负担能力不同的税负不同(纵向公平)。税收公平原则源于法律上的平等性原则
	
	\paragraph{税收效率原则}
	税收效率原则包括两个方面:经济效率和行政效率。前者要求税法的制定要有利于资源的有效配置和经济体制的有效运行,后者要求提高税收行政效率,节约税收征管成本
	
	\paragraph{实质课税原则}
	是指应根据客观事实确定是否符合课税要件,并根据纳税人的真实负担能力决定纳税人的税负,而不能仅考虑相关外观和形式
	
	
	\subsubsection{税法适用原则}
	税法适用原则是指税务行政机关和司法机关运用税收法律规范解决具体问题所必须遵循的准则。与税法基本原则相比,税法适用原则含有更多的法律技术性准则,更为具体化。
	
	\paragraph{法律优位原则}其基本含义为法律的效力高于行政立法的效力,还可进一步推论为税收行政法规的效力优于税收行政规章的效力;效力低的税法与效力高的税法发生冲突时,效力低的税法即是无效的。
	
	\paragraph{法律不溯及既往原则}即新法实施后,对新法实施之前人们的行为不得适用新法,而只能沿用旧法。
	
	\paragraph{新法优于旧法原则}也称后法优于先法原则,即新法、旧法对同一事项有不同规定时,新法的效力优于旧法。
	
	\paragraph{特别法优于普通法原则}对同一事项两部法律分别订有一般和特别规定时,特别规定的效力高于一般规定的效力。本原则打破了税法效力等级的限制,在授权范围内,居于特别法地位的级别比较低的税法,其效力可高于作为普通法的级别比较高的税法。
	
	\paragraph{实体从旧、程序从新原则}即实体性税法不具备溯及力,而程序性税法在特定条件下具备一定溯及力。
	
	\paragraph{程序优于实体原则}即在诉讼发生时,税收程序法优于税收实体法,以保证国家课税权的实现。
	
	\subsection{税法(种)要素}
	不同考试教材对于税法(种)构成要素的表述是有差异。CPA教材收录了11个要素,分别是:总则、纳税义务人、征税对象(课税对象)、税目、税率、纳税环节、纳税期限、纳税地点、减税免税(税收优惠)、罚则、附则。重点为以下七个
	
	\paragraph{纳税义务人}
	又称“纳税主体”,是税法规定的直接负有纳税义务的单位和个人。解决的是国家对谁征税的问题。纳税人有两种基本形式:自然人和法人。
	
	自然人可划分为居民个人和非居民个人;法人可划分为居民企业和非居民企业,还可按企业的不同所有制性质来进行分类。也可以根据民法典分为营利法人、非营利法人和特别法人。
	
	与纳税人紧密联系的两个概念是代扣代缴义务人和代收代缴义务人。两者共同简称扣缴义务人
	\begin{enumerate}
		\item 代扣代缴义务人:指虽不承担纳税义务,但依照有关规定,在向纳税人支付收入、结算货款、支付费用时有义务代扣代缴其应纳税款的单位和个人
		
		支付个人稿酬的单位代扣代缴个人所得税;支付境外企业股息红利的单位代扣代缴企业所得税
		
		\item 代收代缴义务人:指虽不承担纳税义务,但依照有关规定,在向纳税人收取商品或劳务收入时,有义务代收代缴其应纳税款的单位和个人
		
		受托加工应税消费品的单位,代收代缴消费税;办理交强险的保险机构代收代缴车船税
	\end{enumerate}
	
	\paragraph{征税对象}
	征税对象又叫课税对象、征税客体,指税法规定\textbf{对什么征税},是征纳税双方权利义务共同指向的客体或标的。征税对象是区别一种税与另一种税的重要标志。
	
	与征税对象相关的两个重要概念:税目与税基。
	
	在CPA教材中,税目也是税法要素之一。税目是在税法中对征税对象分类规定的具体征税项目,反映具体的征税范围,是对课税对象质的界定。税目体现征税的广度。
	
	税基又叫计税依据,是据以计算征税对象应纳税款的直接数量依据,它解决对征税对象课税的计算问题,是对课税对象量的规定。(计税依据只是与征税对象密切相关的概念,不属于税法要素)
	
	\paragraph{税率}
	税率主要分为三类
	\begin{enumerate}
		\item 比例税率。比如增值税、城市维护建设税、企业所得税等
		
		\item 定额税率。比如城镇土地使用税、车船税等
		
		\item 累进税率。又有四类
		\begin{enumerate}
			\item 全额累进税率。目前我国没有采用
			
			\item 超额累进税率。个人所得税中的综合所得、经营所得
			
			\item 全率累进税率。目前我国没有采用
			
			\item 超率累进税率。比如土地增值税
		\end{enumerate}
	\end{enumerate}
	
	\paragraph{纳税环节}
	指税法规定的征税对象在从生产到消费的流转过程中应当缴纳税款的环节。要掌握生产、批发、零售、进出口等各个环节上的税种分布。
	
	\paragraph{纳税期限}
	指税法规定的关于税款缴纳时间方面的限定。有三个相关概念:纳税义务发生时间、纳税期限、缴库期限。应掌握各税种纳税期限、结算缴款期限与滞纳金计算的关系。
	
	\paragraph{纳税地点}
	指税法规定纳税人(包括代征、代扣、代缴义务人)申报纳税的地点。纳税地点关系到税收管辖权和是否便利纳税等问题,在税法中明确规定纳税地点有助于防止漏征或重复征税。各种纳税地点的规定都易出客观题。
	
	\paragraph{减免税收(税收优惠)}
	指对某些纳税人和征税对象采取减少征税或免予征税的特殊规定。各章节的减税、免税规定往往存在大量考点。
	
	\subsection{税收立法与我国税法体系}
	
	\subsubsection{税收立法原则}
	税收立法不仅包括对税收法律、法规、规章的制定、公布、修改和补充,也包括对税收法律、法规、规章的废止(不包括试行)
	
	\subsubsection{税收立法权及其划分}
	\textbf{立法权可以授予某级政府},行政上的执行权给予另一级——我国的税收立法权的划分就属于这种模式
	
	中央税、中央与地方共享税以及全国统一实行的地方税的\textbf{立法权集中在中央}。
	
	\subsubsection{税收立法机关}
	税收法律分类不同,立法机关及形式也存在不同
	\begin{enumerate}
		\item 税收法律。全国人大及常委会制定的税收法律
		
		除《宪法》外,在税法体系中,税收法律具有最高的法律效力。例如:《企业所得税法》《个人所得税法》《车船税法》《环境保护税法》《烟叶税法》《船舶吨税法》《耕地占用税法》《车辆购置税法》《资源税法》《契税法》《城市维护建设税法》《印花税法》《关税法》《增值税法》《税收征收管理法》
		
		\textbf{目前尚未立法的税收为消费税、房产税、城镇土地使用税、土地增值税。}
		
		\item 授权立法。全国人大及常委会授权国务院指定的暂行规定或条例。
		
		属于准法律。具有国家法律的性质和地位,为待条件成熟上升为法律做好准备。例如消费税暂行条例
		
		\item 税收法规
		\begin{enumerate}
			\item 国务院指定的税收行政法规。
			
			\item 地方人大及常委会指定的税收地方性法规。目前仅限于海南省、民族自治区。
		\end{enumerate}
		
		\item 税收规章
		\begin{enumerate}
			\item 国务院税务主管部门制定的税收部门规章
			
			国务院税务主管部门指财政部、国家税务总局和海关总署。该级次规章不得与宪法、税收法律、行政法规相抵触。例如:《税务代理试行办法》
			
			\item 地方政府制定的税收地方规章
			
			省、自治区、直辖市的人民政府可以根据法律、
			行政法规和本省、自治区、直辖市的地方性法规,
			制定规章,报国务院和本级人民代表大会常务委
			员会备案。设区的市、自治州的人民政府可以根
			据法律、行政法规和本省、自治区的地方性法规,
			依照法律规定的权限制定规章,报国务院和省、
			自治区的人民代表大会常务委员会、人民政府以
			及本级人民代表大会常务委员会备案。例如:房
			产税等地方性税种的实施细则等
		\end{enumerate}
	\end{enumerate}
	
	\subsubsection{税收立法程序}
	税收立法程序主要包括三个阶段:(1)提议阶段;(2)审议阶段;(3)通过和公布阶段。
	
	\subsubsection{我国现行税法体系}
	税法体系就是通常所说的税收制度。一个国家的税收制度,可按构成方法和形式分为简单型税制及复合型税制。我国税收制度属于复合型税制。
	
	税法体系中各税法按基本内容和效力、职能作用、权限范围的不同,可分为不同类型。
	\begin{enumerate}
		\item 按照税法的基本内容和效力的不同,可以分为税收基本法和税收普通法。
		
		\item 按照税法的职能作用可以分为税收实体法和税收程序法。
		
		\item 按照主权国家形式税收管辖权的不同可以分为国内税法和国际税法。 
	\end{enumerate}
	
	税种的分类不具有法定性。我国现行税收实体法体系可以按照以下进行分类
	\begin{enumerate}
		\item 按照征税对象不同分为
		\begin{enumerate}
			\item 商品(货物)和劳务税类。增值税、消费税、关税
			
			\item 资源税和环境保护税。资源税、环境保护税、城镇土地使用税
			
			\item 所得税类。企业所得税、个人所得税、\textbf{土地增值税}。
			
			\item 特定的税类。城市维护建设税、车辆购置税、耕地占用税、烟叶税、船舶吨税 
			
			\item 财产和行为税类。房产税、车船税、印花税、契税
		\end{enumerate}
		
		\item 按照税负是否容易转嫁分为
		\begin{enumerate}
			\item 直接税。如企业所得税、个人所得税、契税
			
			\item 间接税。处于生产流通环节的税种一般属于间接税,如增值税、消费税、关税
		\end{enumerate}
		
		\item 按照计税价格中是否包含税款可以分为
		\begin{enumerate}
			\item 价内税。如消费税、资源税
			
			\item 价外税。如增值税、关税、车辆购置税
		\end{enumerate}
	\end{enumerate}
	
	除税收实体法外,我国还有一系列税务管理流程和时限的法律制度,是按照税收管理机关的不同而分别规定的:
	\begin{enumerate}
		\item 由税务机关负责征收的税种的征收管理,按照全国人大常委会发布实施的《税收征收管理法》及各实体税法中的征管规定执行。
		
		\item 由海关负责征收的税种的征收管理,按照《中华人民共和国海关法》及《中华人民共和国关税法》中的征管规定执行。
	\end{enumerate}
	
	
	\subsection{税收执法}
	税法的实施即税法的执行。它包括税收执法和守法两个方面:一方面,执法机关和人员要依法执法;另一方面,征纳双方都要守法。由于税法具有多层次的特点,在税收执法过程中,对其适用性或法律效力的判断,一般按以下原则掌握:
	\begin{enumerate}
		\item 层次高的法律优于层次低的法律;
		
		\item 同一层次的法律中特别法优于普通法;
		
		\item 国际法优于国内法;
		
		\item 实体法从旧、程序法从新。
	\end{enumerate}
	税收执法权具体包括税款征收管理权、税务稽查权、税务检查权、税务行政复议裁决权及其他税务管理权。
	
	\subsubsection{税务机构设置与职能}
	2018年,根据我国经济和社会发展及推进国家治理体系和治理能力现代化的需要,对国税地税征管体制进行了改革。现行税务机构设置是中央政府设立国家税务总局(正部级),原有的省及省以下国税地税机构两个系统通过合并整合,统一设置为省、市、县三级税务局,实行以国家税务总局为主与省(自治区、直辖市)人民政府双重领导管理体制。
	
	\subsubsection{税收征收管理范围划分}
	目前,我国的税收分别由税务、海关两个系统负责征收管理。
	\begin{enumerate}
		\item 税务系统(国家税务总局系统):增值税;消费税;车辆购置税;城市维护建设税;企业所得税;个人所得税;资源税;城镇土地使用税;耕地占用税;土地增值税;房产税;车船税;印花税;契税;烟叶税;环境保护税
		
		\item 海关:\textbf{关税;船舶吨税};代征进口环节的增值税、消费税
	\end{enumerate}
	
	\subsubsection{税收收入划分}
	收入划分如下
	\begin{enumerate}
		\item 中央固定收入:消费税(含税务机关征收及进口环节海关代征的全部消费税);车辆购置税;关税;船舶吨税;海关代征的进口环节增值税;铁路建设基金营改增
		
		\item 地方固定收入:房产税;城镇土地使用税;耕地占用税;土地增值税;车船税;契税;烟叶税;环境保护税
		
		\item 中央、地方共享收入:增值税;城市维护建设税;企业所得税;个人所得税;印花税;资源税
	\end{enumerate}
	
	其中对于中央、地方共享收入的方式和比例如下
	\begin{enumerate}
		\item 增值税。中央收入为海关代征的部分;铁路建设基金营改增的部分;其余部分的50\%
		
		地方收入为税务机关征收部分扣除营改增固定给中央项目之后的50\%
		
		\item 企业所得税。中央收入为中国国家铁路集团、海洋石油企业、中石油和中石化企业、国有邮政企业、各银行总行的企业所得税;除上述之外企业所得税的60\%
		
		地方收入为除已列举归属中央之外的企业所得税的40\%

		\item 个人所得税。中央收入60\%,地方收入40\%
		
		\item 城市维护建设税。中央收入为中国国家铁路集团、各银行总行、各保险总公司集中缴纳的部分。地方收入为其他城建税
		
		\item 印花税。中央收入为证券交易印花税,地方收入为其他印花税
		
		\item 资源税。中央收入为海洋石油企业资源税,地方收入为其他资源税
	\end{enumerate}
	
	\subsection{税收权利与义务}
	\subsubsection{税务行政主体的权利与义务}
	税务机关应当广泛宣传税收法律、行政法规,普及纳税知识,无偿地为纳税人提供纳税咨询服务。
	
	税务机关负责征收、管理、稽查,行政复议人员的职责应当明确,并相互分离、相互制约。
	
	税务人员在核定应纳税额、调整税收定额、进行税务检查、实施税务行政处罚、办理税务行政复议时,与纳税人、扣缴义务人或者其法定代表人、直接责任人有下列关系之一的,应当回避:①夫妻关系;②直系血亲关系;③三代以内旁系血亲关系;④近姻亲关系;⑤可能影响公正执法的其他利益关系。
	
	\subsubsection{纳税人、扣缴义务人的权利与义务}
	纳税人、扣缴义务人有权要求税务机关为纳税人、扣缴义务人的情况保密。税务机关应当为纳税人、扣缴义务人的情况保密。
	
	保密的内容是指纳税人、扣缴义务人的商业秘密及个人隐私。纳税人、扣缴义务人的\textbf{税收违法行为不属于保密范围}。
	
	纳税人、扣缴义务人对税务机关所作出的决定,享有陈述权、申辩权,税收法律救济权(依法享有申请行政复议、提起行政诉讼、请求国家赔偿)等权利。
	
	\subsubsection{涉税专业服务机构的范围和服务内容}
	涉税专业服务是指涉税专业服务机构接受委托,利用专业知识和技能,就涉税事项向委托人提供的税务代理等服务。涉税专业服务机构是指税务师事务所和从事涉税专业服务的会计师事务所、律师事务所、代理记账机构、税务代理公司、财税类咨询公司等机构。
	\begin{enumerate}
		\item 纳税申报代理。对纳税人、扣缴义务人提供的资料进行归集和专业判断,代理纳税人、扣缴义务人进行纳税申报准备和签署纳税申报表、扣缴税款报告表以及相关文件
		
		\item 一般税务咨询。对纳税人、扣缴义务人的日常办税事项提供税务咨询服务
		
		\item 专业税务顾问。对纳税人、扣缴义务人的涉税事项提供长期的专业税务顾问服务
		
		\item 税收策划。对纳税人、扣缴义务人的经营和投资活动提供符合税收法律法规及相关规定的纳税计划、纳税方案
		
		\item 涉税鉴证。按照法律、法规以及依据法律、法规制定的相关规定要求,对涉税事项真实性和合法性出具鉴定和证明
		
		\item 纳税情况审查。接受行政机关、司法机关委托,依法对企业纳税情况进行审查,作出专业结论
		
		\item 其他税务事项代理。接受纳税人、扣缴义务人的委托,代理建账记账、减免退税申请等税务事项
		
		\item 发票服务、其他涉税服务
	\end{enumerate}
	
	\subsubsection{税务机关对涉税专业服务机构实施监管内容}
	
	税务机关应当对税务师事务所实施行政登记管理。税务机关对涉税专业服务机构及其从事涉税服务人员进行实名制管理。
	
	税务机关应当建立业务信息采集制度,利用现有的信息化平台分类采集业务信息,加强内部信息共享,提高分析利用水平。
	
	税务机关对涉税专业服务机构从事涉税专业服务的执业情况进行检查,根据举报、投诉情况进行调查。
	
	税务机关应当建立信用评价管理制度,对涉税专业服务机构从事涉税专业服务情况进行信用评价,对其从事涉税服务人员进行信用记录。
	
	税务机关应当加强对税务师行业协会的监督指导,与其他相关行业协会建立工作联系制度。
	
	税务机关应当在门户网站、电子税务局和办税服务场所公告纳入监管的涉税专业服务机构名单及其信用情况,同时公告未经行政登记的税务师事务所名单。
	
	税务机关应当为涉税专业服务机构提供便捷的服务,依托信息化平台为信用等级高的涉税专业服务机构开展批量纳税申报、信息报送等业务提供便利化服务。
	
	\subsection{国际税收关系}
	
	\subsubsection{国际重复征税与国际税收协定}
	国际税收是指两个或两个以上的主权国家或地区,各自基于其课税主权,在对跨国纳税人进行分别课税而形成的征纳关系中,所发生的国家或地区之间的税收分配关系。
	
	国际税收分配关系中的一系列矛盾的产生都与税收管辖权有关。税收管辖权属于国家主权在税收领域中的体现,是一个主权国家在征税方面的主权范围。税收管辖权划分原则有属地原则和属人原则两种。
	\begin{enumerate}
		\item 属地原则。地域管辖权
		
		\item 属人原则。居民管辖权、公民管辖权
	\end{enumerate}
	
	国际重复征税有狭义和广义之分。狭义的国际重复征税强调被重复征税的纳税主体与征税
	对象都具有同一性,是指两个或两个以上国家对同一跨国纳税人的同一征税对象所进行的重复
	征税。广义的国际重复征税则在狭义基础上还包括纳税主体与征税对象具有非同一性时所发生
	的国际重复征税,涉及不同纳税人的同一征税对象以及因对同一笔所得或收入的确定标准和计
	算方法的不同所引起的国际重复征税。
	
	国际重复征税一般包括法律性国际重复征税、经济性国际重复征税、税制性国际重复征税。国际税收中所指的国际重复征税一般属于法律性国际重复征税。
	\begin{table}
		\centering
		\begin{tabular}{|l|l|l|l|l|} 
			\hline
			国际重复征税类型  & 产生原因 & 征税主体 & 纳税人 & 税源  \\
			\hline
			法律性国际重复征税 & 不同征税原则 & 不同 & 相同  & 相同  \\
			\hline
			经济性国际重复征税 & 股份公司经济组织形式 & 不同 & 不同  & 相同     \\
			\hline
			税制性国际重复征税 & 复合税制度 & 不同 & 相同  & 相同 \\
			\hline    
		\end{tabular}
	\end{table}
	
	国际税收协定是指两个或两个以上的主权国家为了协调相互间在处理跨国纳税人征税事务和其他有关方面的税收关系,本着对等原则,经由政府谈判所签订的一种书面协议或条约,也称为国际税收条约。
	
	\textbf{国际税收协定属于国际税法,是以国内税法为基础的}。在国际税收协定与其他国内税法的地位关系上,有两种模式:模式一是国际税收协定优于国内税法;模式二是国际税收协定与国内税法效力等同,在出现冲突时按照“新法优于旧法”和“特别法优于普通法”等处理法律冲突的一般性原则来协调。
	
	在国际税收协定中,国际认可的所得主要有经营所得、劳务所得、投资所得和财产所得四大类,\textbf{其中经营所得(营业利润)是税收协定处理重复征税问题的重点项目}。
	
	国际双重征税的免除是签订国际税收协定的重要内容,也是国际税收协定的首要任务。缔约国各方对避免或免除国际双重征税所采取的方法和条件,以及是否在一定范围和程度上给予饶让抵免,都必须在协定中明确,而不论缔约国各方在其国内税法中有无上述规定。一般常用的避免或免除国际重复征税的方法有免税法、低税法、抵免法等,缔约国具体使用哪种方法要在税收协定中明确,并保持双方协调一致。
	
	\subsubsection{国际避税、反避税与国际税收合作}
	各国政府以及国际社会不仅要采取措施避免所得的国际重复征税,而且也要采取措施防范跨国纳税人的国际避税。
	
	国际避税是指纳税人利用两个或两个以上国家的税法和国家间的税收协定的漏洞、特例和缺陷,规避或减轻其全球总纳税义务的行为。在国外,“避税”与“税务筹划”或“合法节税”基本上是一个概念,它们都是指纳税人利用税法的漏洞或不明之处,规避或减少纳税义务的一种不违法的行为。
	
	中国在国际税收征管合作方面的进展:已签署了《多边税收征管互助公约》《金融账户涉税信息自动交换多边主管当局间协议》《实施税收协定相关措施以防止税基侵蚀和利润转移(BEPS)的多边公约》(简称《BEPS公约》)等3项国际多边合作协定。
	
	\section{增值税法}
	
	\subsection{征税范围与纳税义务人}
	\subsubsection{征税范围的一般规定}
	2016.5.1全面营改增之后,增值税范围包括五项,其中后三项为营改增的范围
	\begin{enumerate}
		\item 销售或者进口的货物
		
		\item 销售劳务
		
		\item 销售服务
		
		\item 销售无形资产
		
		\item 销售不动产
	\end{enumerate}
	
	\paragraph{一般征税范围的基本规定}
	\textbf{销售或者进口的货物},其中货物是指包括电力、热力和气体在内的有形动产。
	
	\textbf{销售劳务}中的劳务是指纳税人有偿提供的\textbf{加工、修理修配}劳务。但是单位或者个体工商户聘用的员工为本单位或者雇主提供劳务不包括在内
	\begin{enumerate}
		\item 其中加工是指受托加工货物,即委托方提供原料及主要材料,受托方按照委托方的妖气,制造货物并收取加工费的业务。
		
		\item 修理修配是指受托对损伤和丧失功能的货物进行修复,使期回复原状和功能的业务。
	\end{enumerate}
	
	销售服务包括交通运输服务、邮政服务、电信服务、建筑服务、金融服务、现代服务、生活服务。前五项服务体现了特定行业或特定业务,第六、七项包含的服务比较琐碎,以往考试多次命制客观题。
	\begin{enumerate}
		\item 交通运输服务:括陆路运输服务、水路运输服务、航空运输服务和管道运输服务。
		\begin{enumerate}
			\item 出租车公司向使用本公司自有出租车的出租车司机收取的管理费用,按照陆路运输服务缴纳增值税。
			
			\item 水路运输的程租、期租业务,属于水路运输服务。航空运输的湿租业务,属于航空运输服务。
			
			\item 无运输工具承运业务,按照“交通运输服务”缴纳增值税。
			
			\item 纳税人已售票但客户逾期未消费取得的运输逾期票证收入,按照“交通运输服务”缴纳增值税。
			
			\item 在运输工具舱位承包业务中,发包方和承包方均按照“交通运输服务”缴纳增值税。
			
			\item 在运输工具舱位互换业务中,互换运输工具舱位的双方均按照“交通运输服务”缴纳增值税。
		\end{enumerate}
		
		\item 邮政服务:邮政服务的提供服务主体是\textbf{中国邮政集团公司机器所属邮政企业},包括邮政普遍服务、邮政特殊服务和其他邮政服务
		
		\item 电信服务:包括基础电信服务和增值电信服务(包括卫星电视信号落地转接服务)
		
		\item 建筑服务:包括工程服务、安装服务、修缮服务、装饰服务和其他建筑服务(针对各类建筑物、构筑物即期附属设施)
		\begin{enumerate}
			\item 物业服务企业为业主提供的装修服务,按照“建筑服务”缴纳增值税。
			
			\item 钻井(打井)、拆除建筑物或者构筑物、平整土地、园林绿化、疏浚(不包括航道疏浚)、建筑物平移、搭脚手架、爆破、矿山穿孔、表面附着物(包括岩层、土层、沙层等)剥离和清理等工程作业属于其他建筑服务。
			
			\item 纳税人将建筑施工设备出租给他人使用并配备操作人员的,按照“建筑服务”缴纳增值税。
		\end{enumerate}
		
		\item 金融服务:括贷款服务、直接收费金融服务、保险服务和金融商品转让。金融商品转让是指转让外汇、有价证券、非货物期货和其他金融商品(基金、信托、理财等资产管理产品和各种金融衍生品)所有权的业务活动。
		\begin{enumerate}
			\item 各种占用、拆借资金取得的收入,包括金融商品持有期间(含到期)利息(保本收益、报酬、资金占用费、补偿金等)收入、信用卡透支利息收入、买人返售金融商品利息收入、融资融券收取的利息收入,以及融资性售后回租、押汇、罚息、票据贴现、转贷等业务取得的利息及利息性质的收入,按照贷款服务缴纳增值税。
			
			金融商品持有期间(含到期)取得的非保本的上述收益,不属于利息或利息性质的收入,不征收增值税。
			
			\item 以货币资金\textbf{投资收取的固定利润或者保底利润},按照贷款服务缴纳增值税。
			
			\item 资管产品投资人购入各类资管产品持有期间(含到期)取得的保本收益,应按贷款服务缴纳增值税;资管产品投资人购入各类资管产品,在未到期之前转让其所有权的,应按金融商品转让缴纳增值税。
			
			\item 纳税人转让因同时实施股权分置改革和重大资产重组而首次公开发行股票并上市形成的限售股,以及上市首日至解禁日期间由上述股份孳生的送、转股,按照“金融商品转让”缴纳增值税。
		\end{enumerate}
		
		\item 现代服务:包括研发和技术服务、信息技术服务、文化创意服务、物流辅助服务、租赁服务、鉴证咨询服务、广播影视服务、商务辅助服务和其他现代服务。
		\begin{enumerate}
			\item 文化创意服务中的广告服务包括广告代理和广告的发布、播映、宣传、展示等。
			
			\item 宾馆、旅馆、旅社、度假村和其他经营性住宿场所提供会议场地及配套服务的活动,按照文化创意服务中的“会议展览服务”缴纳增值税。
			
			\item 将建筑物、构筑物等不动产或者飞机、车辆等有形动产的广告位出租给其他单位或者个人用于发布广告,按照经营租赁服务缴纳增值税。
			
			\item 车辆停放服务、道路通行服务(包括过路费、过桥费、过闸费等)等按照不动产经营租赁服务缴纳增值税。
			
			\item 水路运输的\textbf{光租}业务、航空运输的\textbf{干租}业务,属于经营租赁服务。
			
			\item 翻译服务和市场调查服务,应按照咨询服务缴纳增值税。
			
			\item 企业管理服务、经纪代理服务、人力资源服务、安全保护服务,都属于商务辅助服务。武装守护押运服务按照“安全保护服务”缴纳增值税。
			
			\item 拍卖行受托拍卖取得的手续费或佣金收入,按照“经纪代理服务”缴纳增值税。
			
			\item 货运客运场站服务、打捞救助服务、装卸搬运服务、仓储服务和收派服务,属于物流辅助服务。
			
			\item 纳税人为客户办理退票而向客户收取的 退票费、手续费等收入,按照“其他现代服务”缴纳增值税。
			
			\item 纳税人对安装运行后的机器设备提供的维护保养服务,按照“其他现代服务”缴纳增值税。
		\end{enumerate}
		
		\item 生活服务:是指为满足城乡居民日常生活需求提供的各类服务活动。包括文化体育服务、教育医疗服务、旅游娱乐服务、餐饮住宿服务、居民日常服务和其他生活服务。
		\begin{enumerate}
			\item 提供餐饮服务的纳税人销售的 外卖食品,按照“餐饮服务”缴纳增值税。
			
			\item 纳税人现场制作食品并直接销售给消费者,按照“餐饮服务”缴纳增值税。
			
			\item 纳税人在游览场所经营索道、摆渡车、电瓶车、游船等取得的收入,按照“文化体育服务”缴纳增值税。
			
			\item 纳税人提供植物养护服务,按照“其他生活服务”缴纳增值税。
		\end{enumerate}
	\end{enumerate}
	
	其中关于和租相关的服务需要进行区分
	
	
	\textbf{销售无形资产}是指转让无形资产所有权或者使用权的业务活动。无形资产,包括技术、商标、著作权、商誉、自然资源使用权和其他权益性无形资产。(包括土地使用权)
	
	其他权益性无形资产,包括基础设施资产经营权、公共事业特许权、配额、经营权(包括特许经营权、连锁经营权、其他经营权)、经销权、分销权、代理权、会员权、席位权、网络游戏虚拟道具、域名、名称权、肖像权、冠名权、转会费等。
	
	纳税人通过省级土地行政主管部门设立的交易平台转让补充耕地指标,按照“销售无形资产”缴纳增值税,税率为6\%
	
	
	\textbf{销售不动产}是指转让不动产\textbf{所有权}的业务活动。不动产,包括建筑物、构筑物等。
	
	转让建筑物有限产权或者永久使用权的,转让在建的建筑物或者构筑物所有权的,以及在转让建筑物或者构筑物时一并转让其所占土地的使用权的,按照销售不动产缴纳增值税。
	
	\paragraph{地域界定——在境内销售服务或无形资产、不动产}
	不属于在境内销售服务或者无形资产的行为
	\begin{enumerate}
		\item 境外单位或者个人向境内单位或者个人销售完全在境外发生的服务
		
		\item 境外单位或者个人向境内单位或者个人销售完全在境外使用的无形资产
		
		\item 境外单位或者个人向境内单位或者个人出租完全在境外使用的有形动产
		
		\item 财政部和国家税务总局规定的其他情形
	\end{enumerate}
	
	
	\paragraph{基本行为界定——有偿、经营性}
	销售服务、无形资产或不动产是指有偿提供应税服务、有偿转让无形资产或不动产,但不包括非经营活动中提供的相应项目。有偿,是指取得货币、货物或者其他经济利益。
	
	非经营活动,是指:
	\begin{enumerate}
		\item 行政单位收取的符合条件的政府性基金或者行政事业性收费。
		
		\item 单位或者个体工商户聘用的员工为本单位或者雇主提供取得工资的服务。
		
		\item 单位或者个体工商户为聘用的员工提供服务。
		
		\item 财政部和国家税务总局规定的其他情形。
	\end{enumerate}
	
	\subsubsection{征税范围的特殊规定}
	其中可以分为特殊项目与特殊行为。
	
	有些特殊项目属于征税范围
	\begin{enumerate}
		\item 纳税人取得的财政补贴收入,与其销售货物、劳务、服务、无形资产、不动产的收入或者数量直接挂钩的
		
		\item 经营罚没物品(未上缴财政的)收入
		
		\item 单用途卡售卡方因发行或者销售单用途卡并办理相关资金收付结算业务取得的手续费、结算费、服务费、管理费等收入
	\end{enumerate}
	
	有些特殊项目不属于征税范围
	\begin{enumerate}
		\item 纳税人取得的与销售收入或数量不直接挂钩的其他情形的财政补贴收入
		
		\item 视罚没物品收入归属确定征税与否,凡作为罚没变价或拍卖收入如数上缴财政的
		
		\item 单用途卡发卡企业或者售卡企业销售单用途卡,或者接受单用途卡持卡人充值取得的预收资金
		
		\item 根据国家指令无偿提供的铁路运输服务、航空运输服务,属于用于公益事业服务
		
		\item 融资性售后回租业务中,承租方出售资产的行为
		
		\item 存款利息
		
		\item 被保险人获得的保险赔付
		
		\item 房地产主管部门或者其指定机构、公积金管理中心、开发企业以及物业管理单位代收的住宅专项维修资金
		
		\item 纳税人在资产重组过程中,通过合并、分立、出售、置换等方式,将全部或部分实物资产以及与其相关联的债权、负债和劳动力一并转让给其他单位和个人,不属于增值税的征税范围,其中涉及的货物、不动产、土地使用权转让行为(调整)
		
		\item 金融商品持有期间(含到期)取得的非保本收益
		
		\item 转让非上市公司股权
	\end{enumerate}
	
	此外,除了财政部、国家税务总局规定的其他情形外,还有一些情形视同销售
	\begin{enumerate}
		\item 将货物交付其他单位或者个人代销
		
		\item 销售代销货物
		
		\item 设有两个以上机构并实行统一核算的纳税人,将货物从一个机构移送至其他机构用于销售,但相关机构设在同一县(市)的除外
		
		这里的“用于销售”是指受货机构发生以下情形之一的经营行为:a.向购货方开具发票;b.向购货方收取货款。受货机构的货物移送行为有上述两项情形之一的,应当向所在地税务机关缴纳增值税。未发生上述两项情形的,则应由总机构统一缴纳增值税
		
		\item 将自产、委托加工的货物用于集体福利或者个人消费(含交际应酬)
		
		\item 将自产、委托加工或者购进的货物作为投资,提供给其他单位或者个体工商户
		
		\item 将自产、委托加工或者购进的货物分配给股东或者投资者
		
		\item 将自产、委托加工或者购进的货物无偿赠送其他单位或者个人
		
		\item 单位或者个体工商户向其他单位或者个人无偿提供服务,但用于公益、事业或者以社会公众为对象的除外
		
		\item 单位或者个人向其他单位或者个人无偿转让无形资产或者不动产,但用于公益事业或者以社会公众为对象的除外
	\end{enumerate}
	
	
	
	\subsubsection{纳税义务人和扣缴义务人的规定}
	\paragraph{纳税义务人}在中华人民共和国境内销售或者进口货物,销售劳务、服务、无形资产和不动产的单位和个人,为增值税的纳税人。
	
	所称单位,是指企业、行政单位、事业单位、军事单位、社会团体及其他单位。所称个人,是指个体工商户和其他个人。
	
	发包人名义对外经营,且承担法律责任的,则以发包人为纳税人。否则以承包人为纳税人。
	
	\paragraph{扣缴义务人}
	境外的单位或个人在境内销售应税劳务而在境内未设有经营机构的,其应纳税款以代理人为扣缴义务人;没有代理人的,以购买者为扣缴义务人。
	
	中华人民共和国境外的单位或者个人在境内发生应税行为,在境内未设有经营机构的,以购买方为扣缴义务人。财政部和国家税务总局另有规定的除外。
	
	\subsubsection{一般纳税人和小规模纳税人划分的具体标准}
	一般纳税人和小规模纳税人划分的基本标准是纳税人\textbf{年应税销售额的大小和会计核算水平};年应税销售额不能达到规定标准但符合资格条件的也可以办理增值税一般纳税人资格登记。
	\begin{enumerate}
		\item 基本资格(销售规模的金额)登记标准。增值税纳税人年应税销售额超过财政部、国家税务局规定的小规模纳税人标准的,除依规定选择按照小规模纳税人纳以及其他个人以外,应当向其机构所在地主管税务机关办理一般纳税人登记。
		
		2018年5月1日起,年应税销售额(不含增值税销售额)500万元(含)以下的为小规模纳税人。
		
		年应税销售额是指纳税人再连续不超过12个月或四个季度的经营期内累计应征增值税销售额,包括纳税申报销售额、稽查查补销售额、纳税评估调整销售额(时间区间上计入查补税款申报当月)。偶然发生的销售无形资产、转让不动产的销售额,不计入年应税销售额
		
		\item 资格条件。能够按照国家统一的会计制度规定设置账簿,根据合法、有效凭证核算,能够准确提供税务资料。除国家税务总局另有规定外,\textbf{一经登记为一般纳税人后,不得转为小规模纳税人}。
		
		\item 不得办理一般纳税人的情形。个人或者政策规定必须小规模纳税人的。
	\end{enumerate}
	
	\subsubsection{一般纳税人的登记}
	增值税一般纳税人资格实行登记制,符合一般纳税人条件的纳税人应当向主管税务机关办理一般纳税人资格登记。
	
	纳税人应在年应税销售额超过规定标准的月份(或季度)的所属申报期\textbf{结束后15日内}按照规定办理资格登记手续;未按规定时限办理的,主管税务机关应当在规定时限结束后\textbf{5日内}制作《税务事项通知书》,告知纳税人应当在\textbf{5日内}向主管税务机关办理相关手续;逾期仍不办理的,次月起按销售额依照增值税税率计算应纳税额,不得抵扣进项税额,直至纳税人办理相关手续为止
	
	登记生效之日,是指纳税人办理登记的\textbf{当月1日或者次月1日},由纳税人在办理登记手续时自行选择。纳税人自一般纳税人生效之日起,按照增值税一般计税方法计算应纳税额。
	
	\subsection{税率和征收率}
	\subsubsection{增值税税率}
	增值税基本税率为13\%,主要用于
	\begin{enumerate}
		\item 纳税人销售或者进口货物
		
		\item 纳税人提供加工、修理修配劳务(以下简称应税劳务)
		
		\item 有形动产租赁服务,包括融资租赁和经营租赁
	\end{enumerate}
	
	低税率9\%主要用于两类
	\begin{enumerate}
		\item 货物类
		\begin{enumerate}
			\item 粮食等农产品(含动物骨粒、干姜、姜黄)、食用植物油、鲜奶。【辨析】不含麦芽、复合胶、人发、淀粉、环氧大豆油、氢化植物油、肉桂油、桉油、香茅油、调制乳
			
			\item 自来水、暖气、冷气、热水、煤气、石油液化气、天然气、沼气、居民用煤炭制品、二甲醚
			
			\item 图书、报纸、杂志;音像制品;电子出版物
			
			\item 饲料、化肥、农药、农机(各类农用相关设备)、农膜
			
			\item 食用盐
		\end{enumerate}
		
		\item 服务、无形资产、不动产类
		\begin{enumerate}
			\item 交通运输服务
			
			\item 邮政服务
			
			\item 基础电信服务
			
			\item 建筑服务
			
			\item \textbf{不动产租赁服务}
			
			\item 销售不动产
			
			\item 转让土地使用权(不含转让补充耕地指标)
		\end{enumerate}
	\end{enumerate}
	
	6\%的低税率主要用于如下的项目
	\begin{enumerate}
		\item 现代服务(\textbf{租赁服务除外})
		
		\item 增值电信服务
		
		\item 金融服务
		
		\item 生活服务
		
		\item 销售无形资产(\textbf{含转让补充耕地指标,不含转让土地使用权})
	\end{enumerate}
	
	还有部分项目为零税率(零税率不同于免税,零税率可以抵扣进项税额)
	\begin{enumerate}
		\item 出口货物,税率为0
		
		\item 境内单位和个人跨境销售国务院规定范围内的服务、无形资产,税率为0。
		
		主要包括国际运输服务;航天运输服务;向境外单位提供的完全在境外消费的研发服务、合同能源管理服务、设计服务、广播影视节目制作和发行服务、软件服务、电路设计及测试服务、信息系统服务、业务流程管理服务、连服务外包业务、转让技术
		
		\item 照国家有关规定应取得相关资质的\textbf{国际运输服务}项目,纳税人取得相关资质的适用零税率政策,未取得的,适用增值税免税政策。境内单位和个人以无运输工具承运方式提供的国际运输服务,由境内实际承运人适用增值税零税率;无运输工具承运业务的经营者适用增值税免税政策
	\end{enumerate}

	\subsubsection{增值税征收率}
	增值税征收率适用于简易计税方法。(通常是效果莫纳税人适用,一般纳税人在特殊情况下也会适用简易计税法。对增值税一般纳税人的一些特殊经营项目,符合规定的采用征收率计税,属于
	增值税“链条”不完整或不公平情形下的特殊举措。)
	
	基本规定:一般情况下,征收率为3\%殊征收率为5\%
	
	2027年12月31日前,小规模纳税人适用3\%收率的应税销售收入,减按1\%收率征收增值税;适用3\%征率的预缴增值税项目,减按1\%征率预缴增值税。
	
	除了常规的征收率外,某些特殊销售项目有“减按”规定,下面进行详细的介绍
	\begin{enumerate}
		\item 按照3\%收率减按2\%收增值税。
		
		\item 按照5\%收率减按1.5\%收增值税。
		
		\item 减按0.5\%收增值税。
	\end{enumerate}
	
	一般纳税人发生特定的应税销售行为,可选择按照或暂时适用简易计税办法依照3\%的征收率计算缴纳增值税
	\begin{enumerate}
		\item (1)县级及县级以下小型水力发电单位生产的自产电力。小型水力发电单位,是指各类投
		资主体建设的装机容量为5万千瓦以下(含5万千瓦)的小型水力发电单位。
		(2)自产建筑用和生产建筑材料所用的砂、土、石料。
		(3)以自己采掘的砂、土、石料或其他矿物连续生产的砖、瓦、石灰(不含粘土实心
		砖、瓦)。
		(4)自己用微生物、微生物代谢产物、动物毒素、人或动物的血液或组织制成的生物制品。
		(5)自产的自来水。
		(6)自来水公司销售自来水。(必须选择)
		(7)自产的商品混凝土(仅限于以水泥为原料生产的水泥混凝土)。
		(8)单采血浆站销售非临床用人体血液。
		(9)寄售商店代销寄售物品(包括居民个人寄售的物品在内)。(必须选择)
		(10)典当业销售死当物品。(必须选择)
		(11)药品经营企业销售生物制品。
		(12)公共交通运输服务,包括轮客渡、公交客运、轨道交通(含地铁、城市轻轨)、出租车、
		长途客运、班车。其中,班车是指按固定路线、固定时间运营并在固定站点停靠的运送旅客的
		陆路运输服务。
		(13)经认定的动漫企业为开发动漫产品提供的动漫脚本编撰、形象设计,背景设计、动
		画设计、分镜、动画制作、摄制、描线、上色、画面合成、配音、配乐、音效合成、剪辑、字
		幕制作、压缩转码(面向网络动漫、手机动漫格式适配)服务,以及在境内转让动漫版权(包
		括动漫品牌、形象或者内容的授权及再授权)。
		(14)电影放映服务、仓储服务、装卸搬运服务、收派服务和文化体育服务。
		(15)资管产品管理人运营资管产品过程中发生的增值税应税行为,暂适用简易计税方法,
		按照3??征收率缴纳增值税。(必须选择)
		(16)提供物业管理服务的纳税人,向服务接受方收取的自来水水费,以扣除其对外支付的
		自来水水费后的余额为销售额,按照简易计税方法依3??征收率计算缴纳增值税。(必须选择)
		(17)提供非学历教育服务、教育辅助服务。
		(18)从事再生资源回收的增值税一般纳税人销售其收购的再生资源。
		(19)销售自产机器设备的同时提供安装服务,应分别核算机器设备和安装服务的销售额,
		安装服务可以按照甲供工程选择适用简易计税方法计税。
		销售外购机器设备的同时提供安装服务,如果已经按照兼营的有关规定,分别核算机器设
		备和安装服务的销售额,安装服务可以按照甲供工程选择适用简易计税方法计税。
		(20)生产销售和批发、零售下列药品,可选择按照简易办法依照3??收率计算缴纳增值
		税:①抗癌药品;②罕见病药品。
		
		(21)非企业性单位中的一般纳税人提供的研发和技术服务、信息技术服务、鉴证咨询服务,
		以及销售技术、著作权等无形资产,可以选择简易计税方法按照3??收率计算缴纳增值税。
		(22)一般纳税人可以选择适用简易计税方法计税的建筑服务:以清包工方式提供的建筑服务、为甲供工程提供的建筑服务、为建筑工程老项目提供的建筑服务(开工日期在20160430前)
	\end{enumerate}
	
	适用3\%征收率计税的一般纳税人和小规模纳税人,其某些特殊销售项目按照3\%征收率减按2\%收增值税
	\begin{enumerate}
		\item 小规模纳税人销售自己使用过的固定资产(有形动产,下同),适用简易办法依照3\%征收率减按2\%征收增值税政策的,可以放弃减税,按照简易办法依照3\%征收率缴纳增值税,并可自行开具或由主管税务机关代开增值税专用发票。
		
		\item 一般纳税人销售自己使用过的不得抵扣且未抵扣进项税的固定资产,适用简易办法依照3\%收率减按2\%收增值税政策的,可以放弃减税,按照简易办法依照3\%收率缴纳增值税,并可以开具增值税专用发票。
		
		\item 纳税人(含一般纳税人和小规模纳税人)销售除二手车以外的旧货,按照简易办法依照3??收率减按2??收增值税,不能放弃减税。
	\end{enumerate}
	
	2027年12月31日前,对从事二手车经销的纳税人销售其收购的二手车,按照简易办法减按0.5\%收增值税。
	
	全面“营改增”过程中的特殊项目,适用5\%征收率计算增值税:
	第一类,与“房地”相关的适用5??收率的项目。
	\begin{enumerate}
		\item (1)小规模纳税人销售自建或者取得的不动产。
		(2)房地产开发企业中的小规模纳税人,销售自行开发的房地产项目。
		(3)小规模纳税人出租(经营租赁)其取得的不动产(不含个人出租住房以及住房租赁企
		业向个人出租住房)。
		(4)其他个人销售其取得的不动产。
		(5)其他个人出租(经营租赁)其取得的不动产(不含住房)。
		(6)个人出租住房,应按照5??征收率减按1.5??算应纳税额。
		(7)一般纳税人选择简易计税方法计税的不动产销售。
		(8)房地产开发企业中的一般纳税人购入未完工的房地产老项目(2016年4月30日之前
		的建筑工程项目)继续开发后,以自己名义立项销售的不动产,属于房地产老项目,可以选择
		适用简易计税方法按照5??征收率计算缴纳增值税。
		(9)一般纳税人选择简易计税方法计税的不动产经营租赁(不含住房租赁企业向个人出租
		住房)。
		(10)住房租赁企业中的增值税一般纳税人向个人出租住房取得的全部出租收入,可以选
		择适用简易计税方法,按照5??征收率减按1.5??算缴纳增值税;住房租赁企业中的增值
		税小规模纳税人向个人出租住房,按照5??征收率减按1.5??算缴纳增值税。
		住房租赁企业向个人出租住房适用上述简易计税方法并进行预缴的,减按1.5??征率预
		缴增值税。(教材未收录)
		(11)一般纳税人2016年4月30日前签订的不动产融资租赁合同,或以2016年4月30
		日前取得的不动产提供的融资租赁服务,选择适用简易计税方法的。
		(12)一般纳税人收取试点前开工的一级公路、二级公路、桥、闸通行费,选择适用简易
		计税方法的。
		(13)纳税人转让2016年4月30日前取得的土地使用权,选择适用简易计税方法的。
	\end{enumerate}
	第二类,其他适用5\%征收率的项目
	\begin{enumerate}
		\item (1)纳税人提供劳务派遣服务,选择差额纳税的。
		(2)一般纳税人提供人力资源外包服务,选择适用简易计税方法的。
		(3)纳税人提供安全保护服务,选择差额纳税的。
	\end{enumerate}
	
	\section{消费税法}
	
	\section{企业所得税}
	
	\section{个人所得税}
	
	\section{城市维护建设税法和烟叶税法}
	
	\section{关税法和船舶吨税法}
	
	\section{资源税法和资源保护税法}
	
	\section{城镇土地使用税法和耕地占用税法}
	
	\section{房产税法、契税法和土地增值税法}
	
	\section{车辆购置税法、车窗税法和印花税法}
	
	\section{国际税收管理实务}
	
	\section{税收征收管理法}
	
	\section{税务行政法则}
	

	
\end{document}