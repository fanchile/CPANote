\documentclass[UTF8,12pt]{ctexart}
\usepackage{amsmath,amssymb,geometry,bm,graphicx,fontspec,amssymb,amsthm}
\usepackage[mathscr]{euscript}

\usepackage[colorlinks,
linkcolor=black,
anchorcolor=blue,
citecolor=green
]{hyperref} % 目录中的超链接

\newtheorem{Def}{定义}[section]
\newtheorem{Theo}{定理}[section]
\newtheorem{Lemm}{引理}[section]
\newtheorem{Prop}{命题}[section]
\newtheorem{Assu}{假设}[section]
\newtheorem{Axiom}{Axiom}

\numberwithin{equation}{section} % 按章节进行排序与编号
\numberwithin{figure}{section}
\numberwithin{table}{section}

\usepackage{draftwatermark} % 所有页加水印
\SetWatermarkText{EconNerd} % 设置水印内容
\SetWatermarkLightness{0.99} % 设置水印透明度 0-1
\SetWatermarkScale{1} % 设置水印大小    

\title{经济法} % 文档相关信息
\author{EconNerd}
\date{\today}
\geometry{scale=0.8}

\begin{document}
	\maketitle
	\tableofcontents
	\newpage
	
	\section{法律基本原理}
	
	\subsection{法律基本概念}
	\subsubsection{我国的法律渊源}
	我国的法律渊源主要表现为制定法,不包括判例法。在效力等级上:(1)宪法>法律>行政法规>地方性法规>本级和下级地方政府规章;(2)宪法>法律>行政法规>部门规章
	
	\paragraph{宪法}由全国人民代表大会制定
	
	\paragraph{法律}
	\begin{enumerate}
		\item 基本法律:由全国人民代表大会制定
		
		\item 一般法律:由全国人民代表大会常务委员会制定
	\end{enumerate}
	全国人民代表大会制定和修改的,调整国家和社会生活中带有普遍性的社会关系的规范性法律文件,属于基本法律。如《刑法》、《民法总则》。
	
	全国人民代表大会常务委员会制定和修改的,调整国家和社会生活中某一方面社会关系的规范性法律文件,属于一般法律。如《公司法》、《证券法》。
	
	在全国人民代表大会闭会期间,全国人民代表大会常务委员会可对基本法律进行部分补充和修改,但是不得同该法律的基本原则相抵触。
	
	全国人民代表大会常务委员会负责解释法律,其作出的法律解释与法律具有同等效力。例如,2016年11月7日,全国人民代表大会常务委员会对《香港特别行政区基本法》第104条进行了解释,明确了相关公职人员“就职时必须依法宣誓”的具体含义。
	
	\paragraph{法规}
	\begin{enumerate}
		\item 行政法规:由国务院制定
		
		\item 地方性法规:由有地方立法权的地方人民代表大会及其常务委员会制定
	\end{enumerate}
	省、自治区、直辖市的人民代表大会及其常务委员会,有权制定地方性法规;自治州和设区的市的人民代表大会及其常务委员会有权对\textbf{城乡建设与管理、环境保护、历史文化保护}等方面的事项制定地方性法规。
	
	\paragraph{规章}
	\begin{enumerate}
		\item 部门规章:由国务院各部、委员会、中国人民银行、审计署和具有行政管理职能的直属机构制定
		
		\item 地方政府规章:由有地方立法权的地方人民政府制定
	\end{enumerate}
	自治州和设区的市的人民政府有权就城乡建设与管理、环境保护、历史文化保护等方面的事项制定地方政府规章。
	
	没有法律或者国务院的行政法规、决定、命令的依据,部门规章不得设定减损公民、法人和其他组织权利或者增加其义务的规范,不得增加本部门的权力或者减少本部门的法定职责。
	
	没有法律、行政法规、地方性法规的依据,地方政府规章不得设定减损公民、法人和其他组织权利或者增加其义务的规范。
	
	\paragraph{司法解释}由最高人民法院、最高人民检察院制定
	最高人民法院和最高人民检察院的解释如果有原则性的分歧,报请全国人民代表大会常务委员会解释或者决定。法律解释权归全国人民代表大会常务委员会,司法解释权归最高人民法院和最高人民检察院。

	
	\subsubsection{法律规范}
	\paragraph{法律规范的概念}法律规范是由国家制定或者认可的,具体规定主体权利、义务及法律后果的行为准则。
	
	\paragraph{法律规范与法律条文}
	\begin{enumerate}
		\item 法律规范不同于法律条文,法律规范是法律条文的内容,法律条文是法律规范的表现形式。
		
		\item 法律规范是法律条文的主要内容,但法律条文的内容还可能包含其他法要素(如法律原则)。
		
		\item 法律规范与法律条文不是一一对应的,一项法律规范的内容可以表现在不同法律条文甚至不同的规范性法律文件中。同样,一个法律条文中也可以反映若干法律规范的内容。
	\end{enumerate}
	
	\paragraph{法律规范的种类}
	\begin{enumerate}
		\item 授权性规范和义务性规范。这是根据法律规范为主体提供行为模式的不同方式进行的区分。其中,义务性规范可分为命令性规范和禁止性规范。
		\begin{enumerate}
			\item 授权性规范是规定人们可以作出一定行为或者可以要求别人作出一定行为的法律规范。授权性规范的立法语言表达式为“可以……”、“有权……”、“享有……权利”等。
			
			\item 命令性规范是指规定人们的积极义务,即规定主体应当或者必须作出一定积极行为的规范。命令性规范的立法语言表达式为“应当……”、“必须……”、“有……义务”等。
			
			\item 禁止性规范是指规定人们的消极义务(不作为义务),即禁止人们作出一定行为的规范。禁止性规范的立法语言表达式为“不得……”、“禁止……”等。
		\end{enumerate}
		
		\item 强行性规范和任意性规范。
		\begin{enumerate}
			\item 强行性规范是指所规定的义务具有确定的性质,不允许任意变动和伸缩。
			
			\item 任意性规范是指在法定范围内允许行为人自行确定其权利义务的具体内容。
		\end{enumerate}
		
		
		\item 确定性规范和非确定性规范。
		\begin{enumerate}
			\item 确定性规范是指内容已经完备明确,无须再援引或者参照其他规范来确定其内容的法律规范。
			
			\item 非确定性规范是指没有明确具体的行为模式或者法律后果,需要引用其他法律规范来说明或者补充的规范,具体包括委任性规范和准用性规范。
		\end{enumerate}
		
	\end{enumerate}
	
	
	\subsection{法律关系}
	\subsubsection{法律关系的主体}
	\paragraph{法律关系的概念}
	法律关系是根据法律规范产生的,以主体之间的权利与义务为内容的特殊的社会关系(如合同关系)。法律关系包括三个要素:主体、内容和客体。
	
	并非所有的社会关系均属于法律关系。法律关系(如夫妻关系)是以相应法律规范的存在为前提的社会关系,没有相应的法律规范就不可能产生相应的法律关系。例如,同学关系、恋人关系,因不存在相应的法律规范,也就不存在相应的法律关系。
	
	\paragraph{法律关系主体的种类}
	\begin{enumerate}
		\item 自然人
		
		\item 法人和非法人组织
		
		\item 国家
	\end{enumerate}

	自然人既包括本国公民,也包括居住在一国境内或者在境内活动的外国公民和无国籍人。国家可以直接以自己的名义参与国内法律关系(如发行国债)。
	
	\paragraph{法人的分类}
	\begin{enumerate}
		\item 法人分为营利法人、非营利法人和特别法人。
		
		\item 营利法人包括有限责任公司、股份有限公司和其他企业法人等。
		
		\item 非营利法人包括事业单位、社会团体、基金会、社会服务机构等。
		
		\item 特别法人包括特定的机关法人、农村集体经济组织法人、城镇农村的合作经济组织法人、基层群众性自治组织法人。
	\end{enumerate}
	
	\paragraph{非法人组织}
	\begin{enumerate}
		\item 非法人组织是不具有法人资格,但是能够依法以自己的名义从事民事活动的组织。
	
		\item 非法人组织包括个人独资企业、合伙企业、不具有法人资格的专业服务机构等。
	\end{enumerate}
	
	\subsubsection{权利能力和行为能力}
	\paragraph{权利能力和行为能力}
	\begin{enumerate}
		\item 权利能力是指权利主体享有权利和承担义务的能力,它反映了权利主体取得权利和承担义务的资格。行为能力是指权利主体能够通过自己的行为取得权利和承担义务的能力。
		
		\item 法律关系主体要自己参与法律活动,必须具备相应的行为能力。
		
		\item 行为能力必须以权利能力为前提,无权利能力就谈不上行为能力。
		
		\item 作为民事法律关系主体的法人,其权利能力从法人成立时产生,其行为能力伴随着权利能力的产生而同时产生;法人终止时,其权利能力和行为能力同时消灭。
		
		\item 自然人从出生时起到死亡时止,具有民事权利能力,依法享有民事权利,承担民事义务。自然人的民事权利能力一律平等。
	\end{enumerate}
	
	
	\paragraph{自然人的民事行为能力}
	\begin{enumerate}
		\item 完全民事行为能力人。
		\begin{enumerate}
			\item 18周岁以上(≥18周岁)的自然人为成年人,成年人为完全民事行为能力人,可以独立实施民事法律行为。
			
			\item 16周岁以上(≥16周岁)的未成年人,以自己的劳动收入为主要生活来源的,视为完全民事行为能力人,可以独立实施民事法律行为。
		\end{enumerate}
		
		\item 限制民事行为能力人。8周岁以上(≥8周岁)的未成年人和不能完全辨认自己行为的成年人为限制民事行为能力人。
		
		\item 无民事行为能力人。不满8周岁(<8周岁)的未成年人,不能辨认自己行为的成年人,以及8周岁以上的未成年人不能辨认自己行为的,为无民事行为能力人,由其法定代理人代理实施民事法律行为。
	\end{enumerate}
	无民事行为能力人、限制民事行为能力人的监护人是其法定代理人。
	
	\subsubsection{法律关系的客体}
	法律关系的客体,是指法律关系主体间权利义务所指向的对象。法律关系的客体通常包括:
	\begin{enumerate}
		\item 物。物(如土地、机器设备、货币、有价证券)是物权法律关系的客体。
		
		\item 行为。行为包括作为和不作为(如竞业禁止合同的客体是不从事相同或者相似的经营活动)。给付行为是债权法律关系(如合同之债)的客体。
		
		\item 人格利益。人格利益(如公民的肖像、名誉、人身)是人身权法律关系的客体,也是诸多行政、刑事法律关系的客体。
		
		\item 智力成果。智力成果(如文学艺术作品、科学著作、专利、商标)是知识产权法律关系的客体。
	\end{enumerate}
	
	
	\subsubsection{法律事实}
	法律事实是指法律规范所规定的,能够引起法律关系产生、变更或者消灭的客观现象。根据是否以权利主体的意志为转移,法律事实分为行为和事件两类。
	\begin{enumerate}
		\item 事件(与当事人的意志无关)
		(1)人的出生与死亡
		(2)自然灾害与意外事件
		(3)时间的经过
		
		\item 行为
		(1)法律行为(以行为人的\textbf{意思表示}为要素,如订立合同)
		(2)事实行为(与意思表示无关,如创作行为、侵权行为)
	\end{enumerate}
	
	
	
	\subsection{全面依法治国基本方略}
	
	\section{基本民事法律制度}
	
	\subsection{民事法律行为制度}
	\paragraph{民事法律行为的分类}
	民事法律行为是民事主体通过\textbf{意思表示}设立、变更、终止民事法律关系的行为。民事法律行为是法律关系变动的原因之一,是民法最重要的\textbf{法律事实}
	
	民事法律行为是\textbf{有目的}的行为(包括后果、不包括动机),此处的“目的”仅指当事人实施民事法律行为所追求的法律后果,不包括行为人实施行为的动机。这一特征使得民事法律行为区别于其他法律事实,如侵权行为。侵权行为虽然也产生一定的法律后果,但该法律后果并非由当事人自己主张,而是由法律规定的
	
	民事法律行为有很多中分类方式,最基础的可以分为以下几类
	\begin{enumerate}
		\item 单方民事法律行为,是指根据一方当事人的意思表示而成立的民事法律行为(如订立遗嘱、撤销权的行使、解除权的行使、效力待定行为的追认等)。
		
		\item 双方民事法律行为,是指两个当事人之间意思表示一致而成立的民事法律行为。
		
		\item 多方民事法律行为,是指三个以上的当事人意思表示一致而成立的民事法律行为。
	\end{enumerate}
	
	合同是常见的双方民事法律行为,决议则是典型的多方民事法律行为。决议是指多个主体依据表决规则作出的决定,决议当事人的意思表示可以多数决的方式作出,而且对没有表示同意的成员也具有拘束力;决议中的意思表示不仅对发出表示的成员有拘束力,而且主要对表示者共同代表的法人有拘束力。例如公司股东会依法作出的决议,对全体股东(包括投反对票的股东、弃权的股东、未出席会议的股东)均有约束力
	
	负担行为与处分行为
	\begin{enumerate}
		\item \textbf{负担行为}是使一方(义务人)相对于他方(权利人)承担一定给付义务的法律行为,这种给付义务既可以是作为的(如交付货物),也可以是不作为的(如保密义务)。因此负担行为产生的是债法上的法律效果,其中负有给付义务的主体是债务人。负担行为中的权利人可以享有要求履行的请求权,义务人的履行行为是请求权实现的重要前提。
		
		\item \textbf{处分行为}是直接导致权利发生变动的法律行为(如甲抛弃自己的动产),并不需要义务人积极履行给付义务。物权行为是典型的处分行为。
	\end{enumerate}
	
	要式民事法律行为与不要式民事法律行为
	\begin{enumerate}
		\item 要式民事法律行为,是指法律规定必须采取一定的形式或者履行一定的程序才能成立,否则民事法律行为不能成立。例如,票据行为属于法定要式民事法律行为。根据《民法典》的规定,融资租赁合同、建设工程合同应当采用书面形式。
		
		\item 不要式民事法律行为,是指法律不要求采取特定形式,当事人自由选择形式即可成立。例如,买卖合同为非要式合同。
	\end{enumerate}
	
	主民事法律行为与从民事法律行为
	(1)主民事法律行为(如借款合同)不成立,从民事法律行为(如担保合同)则不能成立;主民事法律行为无效,从民事法律行为当然不能生效。
	(2)主民事法律行为履行完毕,并不必然导致从民事法律行为效力的丧失。
	
	\paragraph{意思表示}
	民事法律行为以意思表示为核心,对于单方面民事法律行为可以分为有相对人(撤销权的行使、法定代理人的追认、授予代理权)和无相对人(遗嘱、抛弃动产)的意思表示
	
	有无相对人也影响了民事法律行为的生效时间。无相对人的意思表示,表示完成时生效。法律另有规定的,依照其规定。
	
	有相对人的意思表示分为对话的意思表示和非对话的意思表示。
	\begin{enumerate}
		\item 以对话方式作出的意思表示,相对人知道其内容时生效。
		
		\item 以非对话方式作出的意思表示,到达相对人时生效。订立合同过程中的要约和承诺、授予代理权、合同解除等意思表示,均采取到达主义。
		
		\item 以非对话方式作出的采用数据电文形式的意思表示,相对人指定特定系统接收数据电文的,该数据电文进入该特定系统时生效;未指定特定系统的,相对人知道或者应当知道该数据电文进入其系统时生效。当事人对采用数据电文形式的意思表示的生效时间另有约定的,按照其约定。
		
		\item 以公告方式作出的意思表示,公告发布时生效。
	\end{enumerate}
	
	行为人可以明示或者默示作出意思表示。沉默只有在有法律规定、当事人约定或者符合当事人之间的交易习惯时,才可以视为意思表示。(遗产中的沉默视为接受继承)
	
	行为人可以撤回意思表示。撤回意思表示的通知应当在意思表示到达相对人前或者与意思表示同时到达相对人。
	
	对意思表示的解释:有相对人应当按照所用词句,无相对人不能完全拘泥于所用词句
	
	\paragraph{民事法律行为的生效}
	民事法律行为主要可以分为四种,以下会分别进行介绍
	\begin{enumerate}
		\item 有效的
		
		\item 无效的
		
		\item 可撤销的
		
		\item 效力待定的
	\end{enumerate}
	
	民事法律行为有效需要分别满足实质要件和形式要件,实质要件包括
	\begin{enumerate}
		\item 行为人具有相应的民事行为能力;
		
		\item 行为人的意思表示真实;
		
		\item 不违反法律、行政法规的强制性规定,不违背公序良俗。
	\end{enumerate}
	
	形式要件包括
	\begin{enumerate}
		\item 民事法律行为可以采用\textbf{书面形式、口头形式或者其他形式}(如推定形式、沉默形式);法律、行政法规规定或者当事人约定采用特定形式的,应当采用特定形式。
		
		\item \textbf{推定形式},是指当事人并不直接用书面形式或者口头形式进行意思表示,而是通过实施某种\textbf{积极的行为},使得他人可以推定其意思表示。例如,王某在超市购物,王某向售货员交付货币的行为可以推定为王某具有购买商品的意思。
		
		\item \textbf{沉默形式},是指行为人没有以积极的作为进行意思表示,而是以\textbf{消极的不作为代替意思表示}。根据《民法典》的规定,沉默只有在有法律规定、当事人约定或者符合当事人之间的交易习惯时,才可以视为意思表示。
	\end{enumerate}
	
	\paragraph{无效的民事法律行为}
	无效的民事法律行为有着如下的三个特征
	\begin{enumerate}
		\item 自始无效:无效的民事法律行为从行为开始时就没有法律约束力。
		
		\item 当然无效:不论当事人是否主张,是否知道,也不论是否经过人民法院或者仲裁机构的确认,该民事法律行为当然无效。
		
		\item 绝对无效:无效的民事法律行为绝对不发生法律效力,不能通过当事人的行为进行补正。
	\end{enumerate}
	
	无效民事法律行为当事人通过一定行为消除无效原因,使之有效,这不是对无效民事法律行为的补正,而是消灭旧的民事法律行为,成立新的民事法律行为。
	
	无效的民事法律行为有以下几种
	\begin{enumerate}
		\item 无民事行为能力人独立实施的民事法律行为无效。
		
		\item 违背公序良俗的民事法律行为无效。
		
		\item 行为人与相对人恶意串通,损害他人合法权益的民事法律行为无效。
		
		\item 行为人与相对人以虚假的意思表示实施的民事法律行为无效。
		
		\item 违反法律、行政法规的强制性规定的民事法律行为无效,但是该强制性规定不导致该民事法律行为无效的除外。
	\end{enumerate}
	
	合同违反法律、行政法规的强制性规定,有下列情形之一,由行为人承担行政责任或者刑事责任能够实现强制性规定的立法目的的,人民法院可以认定该合同不因违反强制性规定无效:
	\begin{enumerate}
		\item 强制性规定虽然旨在维护社会公共秩序,但是合同的实际履行对社会公共秩序造成的影响显著轻微,认定合同无效将导致案件处理结果有失公平公正;
		
		\item 强制性规定旨在维护政府的税收、土地出让金等国家利益或者其他民事主体的合法利益而非合同当事人的民事权益,认定合同有效不会影响该规范目的的实现;
		
		\item 强制性规定旨在要求当事人一方加强风险控制、内部管理等,对方无能力或者无义务审查合同是否违反强制性规定,认定合同无效将使其承担不利后果;
		
		\item 当事人一方虽然在订立合同时违反强制性规定,但是在合同订立后其已经具备补正违反强制性规定的条件却违背诚信原则不予补正;
		
		\item 法律、司法解释规定的其他情形。
	\end{enumerate}
	
	法律、行政法规的强制性规定旨在规制合同订立后的履行行为,当事人以合同违反强制性规定为由请求认定合同无效的,人民法院不予支持。但是,合同履行必然导致违反强制性规定或者法律、司法解释另有规定的除外。(2024年新增)
	
	\paragraph{可撤销的民事法律行为}
	首先应当区别一下可撤销和无效的民事法律行为之间的区别
	\begin{enumerate}
		\item \textbf{法律效力不同}。可撤销的民事法律行为在撤销前已经生效,在被撤销之前,其法律效果可以对抗除撤销权人以外的任何人。而无效的民事法律行为在法律上当然无效,从一开始即不发生法律效力。
		
		\item \textbf{主张权利的主体不同}。可撤销的民事法律行为的撤销,应由撤销权人申请,人民法院不主动干预。而无效的民事法律行为的确认,不以当事人的意志为转移,人民法院或者仲裁机构可以在诉讼或者仲裁程序中主动宣告其无效。
		
		\item \textbf{行为效果不同}。可撤销的民事法律行为的撤销权人对权利的行使拥有选择权,如果撤销权人未在法定的期限内行使撤销权的,可撤销的民事法律行为将终局有效,不得再被撤销。可撤销的民事法律行为一经撤销,则视同无效的民事法律行为,其效力溯及至行为开始,即自行为开始时无效。而无效的民事法律行为则自始无效、绝对无效。
		
		\item \textbf{行使时间不同}。可撤销的民事法律行为,其撤销权的行使有时间限制。而无效的民事法律行为不存在此种限制。
	\end{enumerate}
	
	在以下几种情况下,民事法律行为属于可撤销的民事法律行为
	\begin{enumerate}
		\item 重大误解。基于重大误解实施的民事法律行为,行为人(误解方)有权请求人民法院或者仲裁机构予以撤销。交易习惯除外(知道90日内,发生5年内)
		
		\item 显失公平。一方利用对方处于危困状态、缺乏判断能力等情形,致使民事法律行为\textbf{成立时}显失公平的,受损害方有权请求人民法院或者仲裁机构予以撤销。(知道1年内,发生5年内)
		
		在民事法律行为\textbf{成立之后}发生的情势变化,导致双方利益显失公平的,不属于可撤销的民事法律行为,而应当按照诚实信用原则处理。
		
		\item 一方或者第三人以胁迫手段,使对方在违背真实意思的情况下实施的民事法律行为,受胁迫方有权请求人民法院或者仲裁机构予以撤销。(胁迫终止1年内,发生5年内)
		
		\item 一方以欺诈手段,使对方在违背真实意思的情况下实施的民事法律行为,受欺诈方有权请求人民法院或者仲裁机构予以撤销。(欺诈终止1年内,发生5年内)
		
		第三人实施欺诈行为,使一方在违背真实意思的情况下实施的民事法律行为,对方知道或者应当知道该欺诈行为的,受欺诈方有权请求人民法院或者仲裁机构予以撤销。(善意第三人不可撤销)
	\end{enumerate}
	
	撤销权在性质上属于\textbf{形成权},依撤销权人单方的意思表示即可产生相应的法律效力,无须相对人同意。形成权是指依照权利人单方意思表示就可以使已经成立的民事法律关系发生变化的权利。如追认权、解除权、撤销权等
	
	撤销权的存续期间为除斥期间
	
	无效的或者被撤销的民事法律行为\textbf{自始没有}法律约束力。民事法律行为部分无效,不影响其他部分效力的,其他部分仍然有效。
	
	民事法律行为无效、被撤销或者确定不发生效力后,行为人因该行为取得的财产,应当予以返还;不能返还或者没有必要返还的,应当折价补偿。有过错的一方应当赔偿对方由此所受到的损失;各方都有过错的,应当各自承担相应的责任。法律另有规定的,依照其规定。(占有资金按LPR或存款利率来计算利息)
	
	\paragraph{效力待定的民事法律行为}
	效力待定的民事法律行为,是指民事法律行为成立时尚未生效,须经权利人追认才能生效。追认的意思表示自到达相对人时生效。一旦追认,则民事法律行为自成立时起生效;如果权利人拒绝追认,则民事法律行为自成立时起无效。
	
	可能导致效力待定的情况只有两个:限制民事行为能力人独立实施的民事法律行为和无权代理
	
	限制民事行为能力人实施的\textbf{纯获利益}(如接受奖励、赠与)的民事法律行为或者与其年龄、智力、精神健康状况相适应的民事法律行为\textbf{直接有效}。除此以外效力待定。
	
	相对人可以催告法定代理人自收到通知之日起30日内予以追认;法定代理人未作表示的,视为拒绝追认。民事法律行为被追认前,善意相对人有撤销的权利。撤销应当以通知的方式作出。
	
	无权代理还要细分两种情况。狭义的无权代理效力待定、表见代理直接有效。
	
	狭义的无权代理:行为人没有代理权、超越代理权或者代理权终止后,仍然实施代理行为,未经被代理人追认的,\textbf{对被代理人不发生效力}。撤销权和催告权与限制性民事行为能力人类似。
	
	无权代理人以被代理人的名义订立合同,被代理人已经开始履行合同义务或者接受相对人履行的,视为对合同的追认。
	
	行为人实施的行为未被追认的:(1)善意相对人有权请求行为人履行债务或者就其受到的损害请求行为人赔偿,但是赔偿的范围不得超过被代理人追认时相对人所能获得的利益;(2)相对人知道或者应当知道行为人无权代理的,相对人和行为人按照各自的过错承担责任。
	
	表见代理:行为人没有代理权、超越代理权或者代理权终止后,仍然实施代理行为,相对人有理由相信行为人有代理权的,\textbf{代理行为有效}。要成立表见代理,应当具备如下构成要件:
	\begin{enumerate}
		\item 代理人无代理权
		
		\item 相对人主观上为善意且无过失
		
		\item 客观上有使相对人相信无权代理人具有代理权的情形,即存在代理权的外观
		
		\item 相对人基于这种客观情形而与无权代理人成立民事法律行为
	\end{enumerate}
	
	相对人有理由相信无权代理人具有代理权的情形包括但不限于:
	\begin{enumerate}
		\item 被代理人对相对人表示已将代理权授予无权代理人,而实际并未授权
		
		\item 无权代理人持有被代理人的介绍信或者盖有印章的空白合同书,使得相对人相信其有代理权
		
		\item 代理关系终止后,被代理人未采取必要的措施而使相对人仍然相信行为人有代理权,并与之进行民事法律行为
	\end{enumerate}
	
	\paragraph{附条件和附期限的民事法律行为}
	对于附条件的民事法律行为
	\begin{enumerate}
		\item 附\textbf{生效条件}(延缓条件)的民事法律行为,自条件成就时生效。
		
		\item 附\textbf{解除条件}的民事法律行为,自条件成就时失效。
		
		\item 附条件的民事法律行为,当事人为自己的利益不正当地阻止条件成就的,视为条件已成就;不正当地促成条件成就的,视为条件不成就。
	\end{enumerate}
	
	延缓条件亦称“停止条件”,在延缓条件成就之前,民事法律行为已经成立,但是效力却处于停止状态。条件成就之后,民事法律行为发生法律效力。
	
	解除条件亦称“消灭条件”,附解除条件的民事法律行为,在所附条件成就之前,已经发生法律效力,行为人已经开始行使权利和承担义务。当条件成就时,权利和义务则失去法律效力。
	
	\textbf{所附条件应当是双方当事人约定的},如果是法律规定的特定民事法律行为的成立条件,不属于此处所谓的“条件”。
	
	所附条件,可以是自然现象、事件,也可以是人的行为。但应当具备下列特征:(1)必须是将来发生的事实;(2)必须是将来不确定的事实;(3)条件应当是双方当事人约定的;(4)条件必须合法;(5)条件是可能发生的事实。
	
	下列民事法律行为不得附条件:(1)条件与行为性质相违背的,如根据《民法典》的规定,法定抵销不得附条件或者附期限;(2)条件违背社会公共利益或者社会公德的,如结婚、离婚等身份性民事法律行为,原则上不得附条件。
	
	如果条件不可能发生,对于生效条件,视为法律行为不发生效力。对于解除条件,视为未附条件。
	
	附期限的民事法律行为
	\begin{enumerate}
		\item 附生效期限(延缓期限,也称初期)的民事法律行为,自期限届至时生效。
		
		\item 附终止期限(解除期限,也称终期)的民事法律行为,自期限届满时失效。
	\end{enumerate}
	
	所附的期限可以是未来一个确定的日期(如2028年11月11日),也可以是一个不确定的日期(如雷某死亡之日),但无论是不是一个确定的日期,期限的到来是一个必然发生的事件。因此,附期限的民事法律行为的效力的产生或者消灭是确定的、可预知的。
	
	
	\subsection{代理制度}
	
	\paragraph{代理的概念}
	代理是指代理人在代理权限内,以被代理人的名义与第三人实施民事法律行为,由此产生的法律后果直接由被代理人承担的一种法律制度。应当由本人实施的法律行为不得代理。
	
	行纪是指行纪人接受他人委托以自己的名义从事商业活动的行为。拍卖公司(行纪人)与委托人之间的合同是一种典型的行纪合同。与代理的主要区别在于
	\begin{enumerate}
		\item 行纪人以自己的名义实施民事法律行为,而代理人以被代理人的名义实施民事法律行为。
		
		\item 行纪的法律后果由行纪人自行承担,然后会通过其他法律关系(如委托合同)转给委托人;而代理的法律效果直接由被代理人承受。
	
		\item 行纪必须为有偿民事法律行为,而代理既可以有偿,也可以无偿
	\end{enumerate}
	
	代理与传达的区别
	\begin{enumerate}
		\item 传达的任务是忠实传递委托人的意思表示,传达人自己不进行意思表示,传达人不以具有民事行为能力为条件。
		
		\item 代理人在代理权限内可以独立向第三人进行意思表示,因此代理人必须具有相应的民事行为能力。
		
		\item 身份行为(如结婚行为、收养行为)\textbf{不能代理,但可以借助传达人传递意思表示}。
		
	\end{enumerate}

	
	\paragraph{委托代理}
	委托代理是指基于被代理人授权的意思表示而发生的代理。
	2、委托授权为不要式行为,既可以采用书面形式,也可以采用口头或者其他方式授权。
	3、委托代理中的授权行为是一种单方民事法律行为,仅凭被代理人一方的意思表示,即可发生授权的效果。被代理人的授权行为,既可以向代理人进行,也可以向相对人进行,二者效力相同。
	
	执行法人或者非法人组织工作任务的人员,就其职权范围内的事项,以法人或者非法人组织的名义实施民事法律行为,对法人或者非法人组织发生效力。法人或者非法人组织对执行其工作任务的人员职权范围的限制,\textbf{不得对抗善意相对人}。
	
	代理权会存在如下的滥用情况
	\begin{enumerate}
		\item 自己代理:代理人不得以被代理人的名义与自己实施民事法律行为,但是被代理人同意或者追认的除外。
		
		\item 双方代理:代理人不得以被代理人的名义与自己同时代理的其他人实施民事法律行为,但是被代理的双方同意或者追认的除外。
		
		\item 恶意串通:代理人和相对人恶意串通,损害被代理人合法权益的,代理人和相对人应当承担连带责任
	\end{enumerate}
	
	\subsection{诉讼时效制度}
	
	\paragraph{诉讼时效的基本理论}
	诉讼时效的概念
	\begin{enumerate}
		\item 诉讼时效期间届满的,义务人可以提出不履行义务的抗辩。诉讼时效期间届满后,义务人同意履行的,不得以诉讼时效期间届满为由抗辩;义务人已经自愿履行的,不得请求返还。
		
		\item 诉讼时效期间届满时债务人获得抗辩权,但债权人的实体权利并不消灭。
		
		\item 权利人超过诉讼时效期间后起诉的,人民法院应当受理(起诉权并不丧失)。义务人提出诉讼时效抗辩的,人民法院查明无中止、中断、延长事由的,判决驳回权利人的诉讼请求(权利人丧失胜诉权),但权利人的实体权利并不消灭。
		
		\item 义务人未提出诉讼时效抗辩的,人民法院不应对诉讼时效问题进行释明及主动适用诉讼时效的规定进行裁判。
		
		\item 当事人在一审期间未提出诉讼时效抗辩,在二审期间提出的,人民法院不予支持;但其基于新的证据能够证明对方当事人的请求权已过诉讼时效期间的情形除外。
		
	\end{enumerate}
	
	
	\textbf{诉讼时效具有强制性}
	(1)当事人对诉讼时效利益的预先放弃无效。
	(2)诉讼时效的期间、计算方法以及中止、中断的事由由法律规定,当事人约定无效。
	
	下列\textbf{请求权不适用诉讼时效}的规定
	(1)请求停止侵害、排除妨碍、消除危险;
	(2)不动产物权和登记的动产物权的权利人请求返还财产;
	(3)请求支付抚养费、赡养费或者扶养费;
	(4)依法不适用诉讼时效的其他请求权。
	
	下列\textbf{债权请求权不适用诉讼时效}的规定
	(1)支付存款本金及利息请求权;
	(2)兑付国债、金融债券以及向不特定对象发行的企业债券本息请求权;
	(3)基于投资关系产生的缴付出资请求权;
	(4)其他依法不适用诉讼时效规定的债权请求权。
	
	诉讼时效和除斥区间一般有着如下的不同
	\begin{enumerate}
		\item 适用对象不同
		①诉讼时效一般适用于债权请求权;
		②除斥期间一般适用于形成权(如追认权、解除权、撤销权等),也可能适用于请求权(如受遗赠权)。
		
		\item 可以援用的主体不同
		①人民法院不能主动援用诉讼时效,诉讼时效须由当事人主张后,人民法院才能审查;
		②除斥期间无论当事人是否主张,人民法院均可主动审查。
		
		\item 法律效力不同
		①诉讼时效届满只是让债务人取得抗辩权,债权人的实体权利并不消灭;
		②除斥期间届满,实体权利消灭。
	\end{enumerate}
	
	\paragraph{诉讼时效的种类和起算}
	诉讼时效有以下两种
	\begin{enumerate}
		\item 普通诉讼时效。向人民法院请求保护民事权利的诉讼时效期间为3年。法律另有规定的,依照其规定(可以中止、中断,不可以延长)。因国际货物买卖合同和技术进出口合同争议提起诉讼或者申请仲裁的时效期间为4年。
		
		\item 最长诉讼时效。自权利受到损害之日起超过20年的,人民法院不予保护;有特殊情况的,人民法院可以根据权利人的申请决定延长。(不可以中止、中断,可以延长)
	\end{enumerate}
	
	诉讼时效期间的起算
	(1)附条件或者附期限的债的请求权,从条件成就或者期限届满之日起算。
	(2)约定有履行期限的债的请求权,从清偿期限届满之日起算;当事人约定同一债务分期履行的,诉讼时效期间从最后一期履行期限届满之日起计算。
	(3)未约定履行期限或者履行期限不明确的债的请求权,依照《民法典》的规定可以确定履行期限的,诉讼时效期间从履行期限届满之日起计算;不能确定履行期限的,诉讼时效期间从债权人要求债务人履行义务的宽限期届满之日起计算,但债务人在债权人第一次向其主张权利之时明确表示不履行义务的,诉讼时效期间从债务人明确表示不履行义务之日起计算。
	(4)请求他人不作为的债权请求权,应当自权利人知道义务人违反不作为义务时起算。
	(5)国家赔偿的诉讼时效的起算,自赔偿请求人知道或者应当知道国家机关及其工作人员行使职权时的行为侵犯其人身权、财产权之日起计算,但被羁押等限制人身自由期间不计算在内。
	(6)未成年人遭受性侵害的损害赔偿请求权的诉讼时效期间,自受害人年满18周岁之日起算。
	(7)无民事行为能力人或者限制民事行为能力人对其法定代理人的请求权,诉讼时效期间自该法定代理终止之日起算。
	(8)无民事行为能力人或者限制民事行为能力人的权利受到损害的,诉讼时效期间自其法定代理人知道或者应当知道权利受到损害以及义务人之日起计算。法律另有规定的,依照其规定。
	
	\paragraph{诉讼时效的中止}
	诉讼时效中止的事由
	在诉讼时效期间的最后6个月内,因下列障碍,不能行使请求权的,\textbf{诉讼时效中止}:
	\begin{enumerate}
		\item 不可抗力;
		
		\item 无民事行为能力人或者限制民事行为能力人没有法定代理人,或者法定代理人死亡、丧失民事行为能力、丧失代理权;
		
		\item 继承开始后未确定继承人或者遗产管理人;
		
		\item 权利人被义务人或者其他人控制;
		
		\item 其他导致权利人不能行使请求权的障碍。
	\end{enumerate}
	
	自中止时效的原因消除之日起满6个月,诉讼时效期间届满。
	
	\paragraph{诉讼时效的中断事由}
	有下列情形之一的,诉讼时效中断,从中断、有关程序终结时起,\textbf{诉讼时效期间重新计算}:
	\begin{enumerate}
		\item 权利人向义务人、义务人的代理人、财产代管人或者遗产管理人等提出履行请求;以下情形认定为权利人向义务人提出履行请求
		\begin{enumerate}
			\item 当事人一方直接向对方当事人送交主张权利文书,对方当事人在文书上签名、盖章、按指印或者虽未签名、盖章、按指印但能够以其他方式证明该文书到达对方当事人的。
			
			\item 当事人一方以发送信件或者数据电文方式主张权利,信件或者数据电文到达或者应当到达对方当事人的。
			
			\item 当事人一方为金融机构,依照法律规定或者当事人约定从对方当事人账户中扣收欠款本息的。
			
			\item 当事人一方下落不明,对方当事人在国家级或者下落不明的当事人一方住所地的省级有影响的媒体上刊登具有主张权利内容的公告的,但法律和司法解释另有特别规定的,适用其规定。
			
			\item 权利人对同一债权中的部分债权主张权利,诉讼时效中断的效力及于剩余债权,但权利人明确表示放弃剩余债权的情形除外。
		\end{enumerate}
		
		
		\item 义务人同意履行义务;义务人作出分期履行、部分履行、提供担保、请求延期履行、制定清偿债务计划等承诺或者行为的,应当认定为“义务人同意履行义务”
		
		\item 权利人提起诉讼或者申请仲裁;
		\begin{enumerate}
			\item 当事人一方向人民法院提交起诉状或者口头起诉的,诉讼时效从提交起诉状或者口头起诉之日起中断。
			
			\item 权利人向人民调解委员会以及其他依法有权解决相关民事纠纷的国家机关、事业单位、社会团体等社会组织提出保护相应民事权利的请求,诉讼时效从提出请求之日起中断。
			
			\item 权利人向公安机关、人民检察院、人民法院报案或者控告,请求保护其民事权利的,诉讼时效从其报案或者控告之日起中断。
			
			\item 上述机关决定不立案、撤销案件、不起诉的,诉讼时效期间从权利人知道或者应当知道不立案、撤销案件、不起诉之日起重新计算。
		\end{enumerate}
	
		
		\item 与提起诉讼或者申请仲裁具有同等效力的其他情形。
		\begin{enumerate}
			\item 申请支付令;
			
			\item 申请破产、申报破产债权;
			
			\item 为主张权利而申请宣告义务人失踪或者死亡;
			
			\item 申请诉前财产保全、诉前临时禁令等诉前措施;
			
			\item 申请强制执行;
			
			\item 申请追加当事人或者被通知参加诉讼;
		\end{enumerate}
	\end{enumerate}
	
	诉讼时效中断的其他情形
	\begin{enumerate}
		\item 对于连带债权人、连带债务人中的一人发生诉讼时效中断效力的事由,应当认定对其他连带债权人、连带债务人也发生诉讼时效中断的效力。
		
		\item 债权人提起代位权诉讼的,应当认定对债权人的债权和债务人的债权均发生诉讼时效中断的效力。
		
		\item 债权转让的,应当认定诉讼时效从债权转让通知到达债务人之日起中断。
		
		\item 债务承担情形下,构成原债务人对债务承认的,应当认定诉讼时效从债务承担意思表示到达债权人之日起中断。
	\end{enumerate}
	
	
\end{document}