\documentclass[UTF8,12pt]{ctexart}
\usepackage{amsmath,amssymb,geometry,bm,graphicx,fontspec,amssymb,amsthm}
\usepackage[mathscr]{euscript}

\usepackage{tabularray}

\usepackage[colorlinks,
linkcolor=black,
anchorcolor=blue,
citecolor=green
]{hyperref} % 目录中的超链接

\newtheorem{Def}{定义}[section]
\newtheorem{Theo}{定理}[section]
\newtheorem{Lemm}{引理}[section]
\newtheorem{Prop}{命题}[section]
\newtheorem{Assu}{假设}[section]
\newtheorem{Axiom}{Axiom}

\numberwithin{equation}{section} % 按章节进行排序与编号
\numberwithin{figure}{section}
\numberwithin{table}{section}

\usepackage{draftwatermark} % 所有页加水印
\SetWatermarkText{EconNerd} % 设置水印内容
\SetWatermarkLightness{0.99} % 设置水印透明度 0-1
\SetWatermarkScale{1} % 设置水印大小    

\title{经济法} % 文档相关信息
\author{EconNerd}
\date{\today}
\geometry{scale=0.8}

\begin{document}
	\maketitle
	\tableofcontents
	\newpage
	
	\section{法律基本原理}
	
	\subsection{法律基本概念}
	\subsubsection{我国的法律渊源}
	我国的法律渊源主要表现为制定法,不包括判例法。在效力等级上:(1)宪法>法律>行政法规>地方性法规>本级和下级地方政府规章;(2)宪法>法律>行政法规>部门规章
	
	\paragraph{宪法}由全国人民代表大会制定
	
	\paragraph{法律}
	\begin{enumerate}
		\item 基本法律:由全国人民代表大会制定
		
		\item 一般法律:由全国人民代表大会常务委员会制定
	\end{enumerate}
	全国人民代表大会制定和修改的,调整国家和社会生活中带有普遍性的社会关系的规范性法律文件,属于基本法律。如《刑法》、《民法总则》。
	
	全国人民代表大会常务委员会制定和修改的,调整国家和社会生活中某一方面社会关系的规范性法律文件,属于一般法律。如《公司法》、《证券法》。
	
	在全国人民代表大会闭会期间,全国人民代表大会常务委员会可对基本法律进行部分补充和修改,但是不得同该法律的基本原则相抵触。
	
	全国人民代表大会常务委员会负责解释法律,其作出的法律解释与法律具有同等效力。例如,2016年11月7日,全国人民代表大会常务委员会对《香港特别行政区基本法》第104条进行了解释,明确了相关公职人员“就职时必须依法宣誓”的具体含义。
	
	\paragraph{法规}
	\begin{enumerate}
		\item 行政法规:由国务院制定
		
		\item 地方性法规:由有地方立法权的地方人民代表大会及其常务委员会制定
	\end{enumerate}
	省、自治区、直辖市的人民代表大会及其常务委员会,有权制定地方性法规;自治州和设区的市的人民代表大会及其常务委员会有权对\textbf{城乡建设与管理、环境保护、历史文化保护}等方面的事项制定地方性法规。
	
	\paragraph{规章}
	\begin{enumerate}
		\item 部门规章:由国务院各部、委员会、中国人民银行、审计署和具有行政管理职能的直属机构制定
		
		\item 地方政府规章:由有地方立法权的地方人民政府制定
	\end{enumerate}
	自治州和设区的市的人民政府有权就城乡建设与管理、环境保护、历史文化保护等方面的事项制定地方政府规章。
	
	没有法律或者国务院的行政法规、决定、命令的依据,部门规章不得设定减损公民、法人和其他组织权利或者增加其义务的规范,不得增加本部门的权力或者减少本部门的法定职责。
	
	没有法律、行政法规、地方性法规的依据,地方政府规章不得设定减损公民、法人和其他组织权利或者增加其义务的规范。
	
	\paragraph{司法解释}由最高人民法院、最高人民检察院制定
	最高人民法院和最高人民检察院的解释如果有原则性的分歧,报请全国人民代表大会常务委员会解释或者决定。法律解释权归全国人民代表大会常务委员会,司法解释权归最高人民法院和最高人民检察院。

	
	\subsubsection{法律规范}
	\paragraph{法律规范的概念}法律规范是由国家制定或者认可的,具体规定主体权利、义务及法律后果的行为准则。
	
	\paragraph{法律规范与法律条文}
	\begin{enumerate}
		\item 法律规范不同于法律条文,法律规范是法律条文的内容,法律条文是法律规范的表现形式。
		
		\item 法律规范是法律条文的主要内容,但法律条文的内容还可能包含其他法要素(如法律原则)。
		
		\item 法律规范与法律条文不是一一对应的,一项法律规范的内容可以表现在不同法律条文甚至不同的规范性法律文件中。同样,一个法律条文中也可以反映若干法律规范的内容。
	\end{enumerate}
	
	\paragraph{法律规范的种类}
	\begin{enumerate}
		\item 授权性规范和义务性规范。这是根据法律规范为主体提供行为模式的不同方式进行的区分。其中,义务性规范可分为命令性规范和禁止性规范。
		\begin{enumerate}
			\item 授权性规范是规定人们可以作出一定行为或者可以要求别人作出一定行为的法律规范。授权性规范的立法语言表达式为“可以……”、“有权……”、“享有……权利”等。
			
			\item 命令性规范是指规定人们的积极义务,即规定主体应当或者必须作出一定积极行为的规范。命令性规范的立法语言表达式为“应当……”、“必须……”、“有……义务”等。
			
			\item 禁止性规范是指规定人们的消极义务(不作为义务),即禁止人们作出一定行为的规范。禁止性规范的立法语言表达式为“不得……”、“禁止……”等。
		\end{enumerate}
		
		\item 强行性规范和任意性规范。
		\begin{enumerate}
			\item 强行性规范是指所规定的义务具有确定的性质,不允许任意变动和伸缩。
			
			\item 任意性规范是指在法定范围内允许行为人自行确定其权利义务的具体内容。
		\end{enumerate}
		
		
		\item 确定性规范和非确定性规范。
		\begin{enumerate}
			\item 确定性规范是指内容已经完备明确,无须再援引或者参照其他规范来确定其内容的法律规范。
			
			\item 非确定性规范是指没有明确具体的行为模式或者法律后果,需要引用其他法律规范来说明或者补充的规范,具体包括委任性规范和准用性规范。
		\end{enumerate}
		
	\end{enumerate}
	
	
	\subsection{法律关系}
	\subsubsection{法律关系的主体}
	\paragraph{法律关系的概念}
	法律关系是根据法律规范产生的,以主体之间的权利与义务为内容的特殊的社会关系(如合同关系)。法律关系包括三个要素:主体、内容和客体。
	
	并非所有的社会关系均属于法律关系。法律关系(如夫妻关系)是以相应法律规范的存在为前提的社会关系,没有相应的法律规范就不可能产生相应的法律关系。例如,同学关系、恋人关系,因不存在相应的法律规范,也就不存在相应的法律关系。
	
	\paragraph{法律关系主体的种类}
	\begin{enumerate}
		\item 自然人
		
		\item 法人和非法人组织
		
		\item 国家
	\end{enumerate}

	自然人既包括本国公民,也包括居住在一国境内或者在境内活动的外国公民和无国籍人。国家可以直接以自己的名义参与国内法律关系(如发行国债)。
	
	\paragraph{法人的分类}
	\begin{enumerate}
		\item 法人分为营利法人、非营利法人和特别法人。
		
		\item 营利法人包括有限责任公司、股份有限公司和其他企业法人等。
		
		\item 非营利法人包括事业单位、社会团体、基金会、社会服务机构等。
		
		\item 特别法人包括特定的机关法人、农村集体经济组织法人、城镇农村的合作经济组织法人、基层群众性自治组织法人。
	\end{enumerate}
	
	\paragraph{非法人组织}
	\begin{enumerate}
		\item 非法人组织是不具有法人资格,但是能够依法以自己的名义从事民事活动的组织。
	
		\item 非法人组织包括个人独资企业、合伙企业、不具有法人资格的专业服务机构等。
	\end{enumerate}
	
	\subsubsection{权利能力和行为能力}
	\paragraph{权利能力和行为能力}
	\begin{enumerate}
		\item 权利能力是指权利主体享有权利和承担义务的能力,它反映了权利主体取得权利和承担义务的资格。行为能力是指权利主体能够通过自己的行为取得权利和承担义务的能力。
		
		\item 法律关系主体要自己参与法律活动,必须具备相应的行为能力。
		
		\item 行为能力必须以权利能力为前提,无权利能力就谈不上行为能力。
		
		\item 作为民事法律关系主体的法人,其权利能力从法人成立时产生,其行为能力伴随着权利能力的产生而同时产生;法人终止时,其权利能力和行为能力同时消灭。
		
		\item 自然人从出生时起到死亡时止,具有民事权利能力,依法享有民事权利,承担民事义务。自然人的民事权利能力一律平等。
	\end{enumerate}
	
	
	\paragraph{自然人的民事行为能力}
	\begin{enumerate}
		\item 完全民事行为能力人。
		\begin{enumerate}
			\item 18周岁以上(≥18周岁)的自然人为成年人,成年人为完全民事行为能力人,可以独立实施民事法律行为。
			
			\item 16周岁以上(≥16周岁)的未成年人,以自己的劳动收入为主要生活来源的,视为完全民事行为能力人,可以独立实施民事法律行为。
		\end{enumerate}
		
		\item 限制民事行为能力人。8周岁以上(≥8周岁)的未成年人和不能完全辨认自己行为的成年人为限制民事行为能力人。
		
		\item 无民事行为能力人。不满8周岁(<8周岁)的未成年人,不能辨认自己行为的成年人,以及8周岁以上的未成年人不能辨认自己行为的,为无民事行为能力人,由其法定代理人代理实施民事法律行为。
	\end{enumerate}
	无民事行为能力人、限制民事行为能力人的监护人是其法定代理人。
	
	\subsubsection{法律关系的客体}
	法律关系的客体,是指法律关系主体间权利义务所指向的对象。法律关系的客体通常包括:
	\begin{enumerate}
		\item 物。物(如土地、机器设备、货币、有价证券)是物权法律关系的客体。
		
		\item 行为。行为包括作为和不作为(如竞业禁止合同的客体是不从事相同或者相似的经营活动)。给付行为是债权法律关系(如合同之债)的客体。
		
		\item 人格利益。人格利益(如公民的肖像、名誉、人身)是人身权法律关系的客体,也是诸多行政、刑事法律关系的客体。
		
		\item 智力成果。智力成果(如文学艺术作品、科学著作、专利、商标)是知识产权法律关系的客体。
	\end{enumerate}
	
	
	\subsubsection{法律事实}
	法律事实是指法律规范所规定的,能够引起法律关系产生、变更或者消灭的客观现象。根据是否以权利主体的意志为转移,法律事实分为行为和事件两类。
	\begin{enumerate}
		\item 事件(与当事人的意志无关)
		(1)人的出生与死亡
		(2)自然灾害与意外事件
		(3)时间的经过
		
		\item 行为
		(1)法律行为(以行为人的\textbf{意思表示}为要素,如订立合同)
		(2)事实行为(与意思表示无关,如创作行为、侵权行为)
	\end{enumerate}
	
	
	
	\subsection{全面依法治国基本方略}
	
	\newpage
	
	\section{基本民事法律制度}
	
	\subsection{民事法律行为制度}
	\paragraph{民事法律行为的分类}
	民事法律行为是民事主体通过\textbf{意思表示}设立、变更、终止民事法律关系的行为。民事法律行为是法律关系变动的原因之一,是民法最重要的\textbf{法律事实}
	
	民事法律行为是\textbf{有目的}的行为(包括后果、不包括动机),此处的“目的”仅指当事人实施民事法律行为所追求的法律后果,不包括行为人实施行为的动机。这一特征使得民事法律行为区别于其他法律事实,如侵权行为。侵权行为虽然也产生一定的法律后果,但该法律后果并非由当事人自己主张,而是由法律规定的
	
	民事法律行为有很多中分类方式,最基础的可以分为以下几类
	\begin{enumerate}
		\item 单方民事法律行为,是指根据一方当事人的意思表示而成立的民事法律行为(如订立遗嘱、撤销权的行使、解除权的行使、效力待定行为的追认等)。
		
		\item 双方民事法律行为,是指两个当事人之间意思表示一致而成立的民事法律行为。
		
		\item 多方民事法律行为,是指三个以上的当事人意思表示一致而成立的民事法律行为。
	\end{enumerate}
	
	合同是常见的双方民事法律行为,决议则是典型的多方民事法律行为。决议是指多个主体依据表决规则作出的决定,决议当事人的意思表示可以多数决的方式作出,而且对没有表示同意的成员也具有拘束力;决议中的意思表示不仅对发出表示的成员有拘束力,而且主要对表示者共同代表的法人有拘束力。例如公司股东会依法作出的决议,对全体股东(包括投反对票的股东、弃权的股东、未出席会议的股东)均有约束力
	
	负担行为与处分行为
	\begin{enumerate}
		\item \textbf{负担行为}是使一方(义务人)相对于他方(权利人)承担一定给付义务的法律行为,这种给付义务既可以是作为的(如交付货物),也可以是不作为的(如保密义务)。因此负担行为产生的是债法上的法律效果,其中负有给付义务的主体是债务人。负担行为中的权利人可以享有要求履行的请求权,义务人的履行行为是请求权实现的重要前提。
		
		\item \textbf{处分行为}是直接导致权利发生变动的法律行为(如甲抛弃自己的动产),并不需要义务人积极履行给付义务。物权行为是典型的处分行为。
	\end{enumerate}
	
	要式民事法律行为与不要式民事法律行为
	\begin{enumerate}
		\item 要式民事法律行为,是指法律规定必须采取一定的形式或者履行一定的程序才能成立,否则民事法律行为不能成立。例如,票据行为属于法定要式民事法律行为。根据《民法典》的规定,融资租赁合同、建设工程合同应当采用书面形式。
		
		\item 不要式民事法律行为,是指法律不要求采取特定形式,当事人自由选择形式即可成立。例如,买卖合同为非要式合同。
	\end{enumerate}
	
	主民事法律行为与从民事法律行为
	(1)主民事法律行为(如借款合同)不成立,从民事法律行为(如担保合同)则不能成立;主民事法律行为无效,从民事法律行为当然不能生效。
	(2)主民事法律行为履行完毕,并不必然导致从民事法律行为效力的丧失。
	
	\paragraph{意思表示}
	民事法律行为以意思表示为核心,对于单方面民事法律行为可以分为有相对人(撤销权的行使、法定代理人的追认、授予代理权)和无相对人(遗嘱、抛弃动产)的意思表示
	
	有无相对人也影响了民事法律行为的生效时间。无相对人的意思表示,表示完成时生效。法律另有规定的,依照其规定。
	
	有相对人的意思表示分为对话的意思表示和非对话的意思表示。
	\begin{enumerate}
		\item 以对话方式作出的意思表示,相对人知道其内容时生效。
		
		\item 以非对话方式作出的意思表示,到达相对人时生效。订立合同过程中的要约和承诺、授予代理权、合同解除等意思表示,均采取到达主义。
		
		\item 以非对话方式作出的采用数据电文形式的意思表示,相对人指定特定系统接收数据电文的,该数据电文进入该特定系统时生效;未指定特定系统的,相对人知道或者应当知道该数据电文进入其系统时生效。当事人对采用数据电文形式的意思表示的生效时间另有约定的,按照其约定。
		
		\item 以公告方式作出的意思表示,公告发布时生效。
	\end{enumerate}
	
	行为人可以明示或者默示作出意思表示。沉默只有在有法律规定、当事人约定或者符合当事人之间的交易习惯时,才可以视为意思表示。(遗产中的沉默视为接受继承)
	
	行为人可以撤回意思表示。撤回意思表示的通知应当在意思表示到达相对人前或者与意思表示同时到达相对人。
	
	对意思表示的解释:有相对人应当按照所用词句,无相对人不能完全拘泥于所用词句
	
	\paragraph{民事法律行为的生效}
	民事法律行为主要可以分为四种,以下会分别进行介绍
	\begin{enumerate}
		\item 有效的
		
		\item 无效的
		
		\item 可撤销的
		
		\item 效力待定的
	\end{enumerate}
	
	民事法律行为有效需要分别满足实质要件和形式要件,实质要件包括
	\begin{enumerate}
		\item 行为人具有相应的民事行为能力;
		
		\item 行为人的意思表示真实;
		
		\item 不违反法律、行政法规的强制性规定,不违背公序良俗。
	\end{enumerate}
	
	形式要件包括
	\begin{enumerate}
		\item 民事法律行为可以采用\textbf{书面形式、口头形式或者其他形式}(如推定形式、沉默形式);法律、行政法规规定或者当事人约定采用特定形式的,应当采用特定形式。
		
		\item \textbf{推定形式},是指当事人并不直接用书面形式或者口头形式进行意思表示,而是通过实施某种\textbf{积极的行为},使得他人可以推定其意思表示。例如,王某在超市购物,王某向售货员交付货币的行为可以推定为王某具有购买商品的意思。
		
		\item \textbf{沉默形式},是指行为人没有以积极的作为进行意思表示,而是以\textbf{消极的不作为代替意思表示}。根据《民法典》的规定,沉默只有在有法律规定、当事人约定或者符合当事人之间的交易习惯时,才可以视为意思表示。
	\end{enumerate}
	
	\paragraph{无效的民事法律行为}
	无效的民事法律行为有着如下的三个特征
	\begin{enumerate}
		\item 自始无效:无效的民事法律行为从行为开始时就没有法律约束力。
		
		\item 当然无效:不论当事人是否主张,是否知道,也不论是否经过人民法院或者仲裁机构的确认,该民事法律行为当然无效。
		
		\item 绝对无效:无效的民事法律行为绝对不发生法律效力,不能通过当事人的行为进行补正。
	\end{enumerate}
	
	无效民事法律行为当事人通过一定行为消除无效原因,使之有效,这不是对无效民事法律行为的补正,而是消灭旧的民事法律行为,成立新的民事法律行为。
	
	无效的民事法律行为有以下几种
	\begin{enumerate}
		\item 无民事行为能力人独立实施的民事法律行为无效。
		
		\item 违背公序良俗的民事法律行为无效。
		
		\item 行为人与相对人恶意串通,损害他人合法权益的民事法律行为无效。
		
		\item 行为人与相对人以虚假的意思表示实施的民事法律行为无效。
		
		\item 违反法律、行政法规的强制性规定的民事法律行为无效,但是该强制性规定不导致该民事法律行为无效的除外。
	\end{enumerate}
	
	合同违反法律、行政法规的强制性规定,有下列情形之一,由行为人承担行政责任或者刑事责任能够实现强制性规定的立法目的的,人民法院可以认定该合同不因违反强制性规定无效:
	\begin{enumerate}
		\item 强制性规定虽然旨在维护社会公共秩序,但是合同的实际履行对社会公共秩序造成的影响显著轻微,认定合同无效将导致案件处理结果有失公平公正;
		
		\item 强制性规定旨在维护政府的税收、土地出让金等国家利益或者其他民事主体的合法利益而非合同当事人的民事权益,认定合同有效不会影响该规范目的的实现;
		
		\item 强制性规定旨在要求当事人一方加强风险控制、内部管理等,对方无能力或者无义务审查合同是否违反强制性规定,认定合同无效将使其承担不利后果;
		
		\item 当事人一方虽然在订立合同时违反强制性规定,但是在合同订立后其已经具备补正违反强制性规定的条件却违背诚信原则不予补正;
		
		\item 法律、司法解释规定的其他情形。
	\end{enumerate}
	
	法律、行政法规的强制性规定旨在规制合同订立后的履行行为,当事人以合同违反强制性规定为由请求认定合同无效的,人民法院不予支持。但是,合同履行必然导致违反强制性规定或者法律、司法解释另有规定的除外。(2024年新增)
	
	\paragraph{可撤销的民事法律行为}
	首先应当区别一下可撤销和无效的民事法律行为之间的区别
	\begin{enumerate}
		\item \textbf{法律效力不同}。可撤销的民事法律行为在撤销前已经生效,在被撤销之前,其法律效果可以对抗除撤销权人以外的任何人。而无效的民事法律行为在法律上当然无效,从一开始即不发生法律效力。
		
		\item \textbf{主张权利的主体不同}。可撤销的民事法律行为的撤销,应由撤销权人申请,人民法院不主动干预。而无效的民事法律行为的确认,不以当事人的意志为转移,人民法院或者仲裁机构可以在诉讼或者仲裁程序中主动宣告其无效。
		
		\item \textbf{行为效果不同}。可撤销的民事法律行为的撤销权人对权利的行使拥有选择权,如果撤销权人未在法定的期限内行使撤销权的,可撤销的民事法律行为将终局有效,不得再被撤销。可撤销的民事法律行为一经撤销,则视同无效的民事法律行为,其效力溯及至行为开始,即自行为开始时无效。而无效的民事法律行为则自始无效、绝对无效。
		
		\item \textbf{行使时间不同}。可撤销的民事法律行为,其撤销权的行使有时间限制。而无效的民事法律行为不存在此种限制。
	\end{enumerate}
	
	在以下几种情况下,民事法律行为属于可撤销的民事法律行为
	\begin{enumerate}
		\item 重大误解。基于重大误解实施的民事法律行为,行为人(误解方)有权请求人民法院或者仲裁机构予以撤销。交易习惯除外(知道90日内,发生5年内)
		
		\item 显失公平。一方利用对方处于危困状态、缺乏判断能力等情形,致使民事法律行为\textbf{成立时}显失公平的,受损害方有权请求人民法院或者仲裁机构予以撤销。(知道1年内,发生5年内)
		
		在民事法律行为\textbf{成立之后}发生的情势变化,导致双方利益显失公平的,不属于可撤销的民事法律行为,而应当按照诚实信用原则处理。
		
		\item 一方或者第三人以胁迫手段,使对方在违背真实意思的情况下实施的民事法律行为,受胁迫方有权请求人民法院或者仲裁机构予以撤销。(胁迫终止1年内,发生5年内)
		
		\item 一方以欺诈手段,使对方在违背真实意思的情况下实施的民事法律行为,受欺诈方有权请求人民法院或者仲裁机构予以撤销。(欺诈终止1年内,发生5年内)
		
		第三人实施欺诈行为,使一方在违背真实意思的情况下实施的民事法律行为,对方知道或者应当知道该欺诈行为的,受欺诈方有权请求人民法院或者仲裁机构予以撤销。(善意第三人不可撤销)
	\end{enumerate}
	
	撤销权在性质上属于\textbf{形成权},依撤销权人单方的意思表示即可产生相应的法律效力,无须相对人同意。形成权是指依照权利人单方意思表示就可以使已经成立的民事法律关系发生变化的权利。如追认权、解除权、撤销权等
	
	撤销权的存续期间为除斥期间
	
	无效的或者被撤销的民事法律行为\textbf{自始没有}法律约束力。民事法律行为部分无效,不影响其他部分效力的,其他部分仍然有效。
	
	民事法律行为无效、被撤销或者确定不发生效力后,行为人因该行为取得的财产,应当予以返还;不能返还或者没有必要返还的,应当折价补偿。有过错的一方应当赔偿对方由此所受到的损失;各方都有过错的,应当各自承担相应的责任。法律另有规定的,依照其规定。(占有资金按LPR或存款利率来计算利息)
	
	\paragraph{效力待定的民事法律行为}
	效力待定的民事法律行为,是指民事法律行为成立时尚未生效,须经权利人追认才能生效。追认的意思表示自到达相对人时生效。一旦追认,则民事法律行为自成立时起生效;如果权利人拒绝追认,则民事法律行为自成立时起无效。
	
	可能导致效力待定的情况只有两个:限制民事行为能力人独立实施的民事法律行为和无权代理
	
	限制民事行为能力人实施的\textbf{纯获利益}(如接受奖励、赠与)的民事法律行为或者与其年龄、智力、精神健康状况相适应的民事法律行为\textbf{直接有效}。除此以外效力待定。
	
	相对人可以催告法定代理人自收到通知之日起30日内予以追认;法定代理人未作表示的,视为拒绝追认。民事法律行为被追认前,善意相对人有撤销的权利。撤销应当以通知的方式作出。
	
	无权代理还要细分两种情况。狭义的无权代理效力待定、表见代理直接有效。
	
	狭义的无权代理:行为人没有代理权、超越代理权或者代理权终止后,仍然实施代理行为,未经被代理人追认的,\textbf{对被代理人不发生效力}。撤销权和催告权与限制性民事行为能力人类似。
	
	无权代理人以被代理人的名义订立合同,被代理人已经开始履行合同义务或者接受相对人履行的,视为对合同的追认。
	
	行为人实施的行为未被追认的:(1)善意相对人有权请求行为人履行债务或者就其受到的损害请求行为人赔偿,但是赔偿的范围不得超过被代理人追认时相对人所能获得的利益;(2)相对人知道或者应当知道行为人无权代理的,相对人和行为人按照各自的过错承担责任。
	
	表见代理:行为人没有代理权、超越代理权或者代理权终止后,仍然实施代理行为,相对人有理由相信行为人有代理权的,\textbf{代理行为有效}。要成立表见代理,应当具备如下构成要件:
	\begin{enumerate}
		\item 代理人无代理权
		
		\item 相对人主观上为善意且无过失
		
		\item 客观上有使相对人相信无权代理人具有代理权的情形,即存在代理权的外观
		
		\item 相对人基于这种客观情形而与无权代理人成立民事法律行为
	\end{enumerate}
	
	相对人有理由相信无权代理人具有代理权的情形包括但不限于:
	\begin{enumerate}
		\item 被代理人对相对人表示已将代理权授予无权代理人,而实际并未授权
		
		\item 无权代理人持有被代理人的介绍信或者盖有印章的空白合同书,使得相对人相信其有代理权
		
		\item 代理关系终止后,被代理人未采取必要的措施而使相对人仍然相信行为人有代理权,并与之进行民事法律行为
	\end{enumerate}
	
	\paragraph{附条件和附期限的民事法律行为}
	对于附条件的民事法律行为
	\begin{enumerate}
		\item 附\textbf{生效条件}(延缓条件)的民事法律行为,自条件成就时生效。
		
		\item 附\textbf{解除条件}的民事法律行为,自条件成就时失效。
		
		\item 附条件的民事法律行为,当事人为自己的利益不正当地阻止条件成就的,视为条件已成就;不正当地促成条件成就的,视为条件不成就。
	\end{enumerate}
	
	延缓条件亦称“停止条件”,在延缓条件成就之前,民事法律行为已经成立,但是效力却处于停止状态。条件成就之后,民事法律行为发生法律效力。
	
	解除条件亦称“消灭条件”,附解除条件的民事法律行为,在所附条件成就之前,已经发生法律效力,行为人已经开始行使权利和承担义务。当条件成就时,权利和义务则失去法律效力。
	
	\textbf{所附条件应当是双方当事人约定的},如果是法律规定的特定民事法律行为的成立条件,不属于此处所谓的“条件”。
	
	所附条件,可以是自然现象、事件,也可以是人的行为。但应当具备下列特征:(1)必须是将来发生的事实;(2)必须是将来不确定的事实;(3)条件应当是双方当事人约定的;(4)条件必须合法;(5)条件是可能发生的事实。
	
	下列民事法律行为不得附条件:(1)条件与行为性质相违背的,如根据《民法典》的规定,法定抵销不得附条件或者附期限;(2)条件违背社会公共利益或者社会公德的,如结婚、离婚等身份性民事法律行为,原则上不得附条件。
	
	如果条件不可能发生,对于生效条件,视为法律行为不发生效力。对于解除条件,视为未附条件。
	
	附期限的民事法律行为
	\begin{enumerate}
		\item 附生效期限(延缓期限,也称初期)的民事法律行为,自期限届至时生效。
		
		\item 附终止期限(解除期限,也称终期)的民事法律行为,自期限届满时失效。
	\end{enumerate}
	
	所附的期限可以是未来一个确定的日期(如2028年11月11日),也可以是一个不确定的日期(如雷某死亡之日),但无论是不是一个确定的日期,期限的到来是一个必然发生的事件。因此,附期限的民事法律行为的效力的产生或者消灭是确定的、可预知的。
	
	
	\subsection{代理制度}
	
	\paragraph{代理的概念}
	代理是指代理人在代理权限内,以被代理人的名义与第三人实施民事法律行为,由此产生的法律后果直接由被代理人承担的一种法律制度。应当由本人实施的法律行为不得代理。
	
	行纪是指行纪人接受他人委托以自己的名义从事商业活动的行为。拍卖公司(行纪人)与委托人之间的合同是一种典型的行纪合同。与代理的主要区别在于
	\begin{enumerate}
		\item 行纪人以自己的名义实施民事法律行为,而代理人以被代理人的名义实施民事法律行为。
		
		\item 行纪的法律后果由行纪人自行承担,然后会通过其他法律关系(如委托合同)转给委托人;而代理的法律效果直接由被代理人承受。
	
		\item 行纪必须为有偿民事法律行为,而代理既可以有偿,也可以无偿
	\end{enumerate}
	
	代理与传达的区别
	\begin{enumerate}
		\item 传达的任务是忠实传递委托人的意思表示,传达人自己不进行意思表示,传达人不以具有民事行为能力为条件。
		
		\item 代理人在代理权限内可以独立向第三人进行意思表示,因此代理人必须具有相应的民事行为能力。
		
		\item 身份行为(如结婚行为、收养行为)\textbf{不能代理,但可以借助传达人传递意思表示}。
		
	\end{enumerate}

	
	\paragraph{委托代理}
	委托代理是指基于被代理人授权的意思表示而发生的代理。
	2、委托授权为不要式行为,既可以采用书面形式,也可以采用口头或者其他方式授权。
	3、委托代理中的授权行为是一种单方民事法律行为,仅凭被代理人一方的意思表示,即可发生授权的效果。被代理人的授权行为,既可以向代理人进行,也可以向相对人进行,二者效力相同。
	
	执行法人或者非法人组织工作任务的人员,就其职权范围内的事项,以法人或者非法人组织的名义实施民事法律行为,对法人或者非法人组织发生效力。法人或者非法人组织对执行其工作任务的人员职权范围的限制,\textbf{不得对抗善意相对人}。
	
	代理权会存在如下的滥用情况
	\begin{enumerate}
		\item 自己代理:代理人不得以被代理人的名义与自己实施民事法律行为,但是被代理人同意或者追认的除外。
		
		\item 双方代理:代理人不得以被代理人的名义与自己同时代理的其他人实施民事法律行为,但是被代理的双方同意或者追认的除外。
		
		\item 恶意串通:代理人和相对人恶意串通,损害被代理人合法权益的,代理人和相对人应当承担连带责任
	\end{enumerate}
	
	\subsection{诉讼时效制度}
	
	\paragraph{诉讼时效的基本理论}
	诉讼时效的概念
	\begin{enumerate}
		\item 诉讼时效期间届满的,义务人可以提出不履行义务的抗辩。诉讼时效期间届满后,义务人同意履行的,不得以诉讼时效期间届满为由抗辩;义务人已经自愿履行的,不得请求返还。
		
		\item 诉讼时效期间届满时债务人获得抗辩权,但债权人的实体权利并不消灭。
		
		\item 权利人超过诉讼时效期间后起诉的,人民法院应当受理(起诉权并不丧失)。义务人提出诉讼时效抗辩的,人民法院查明无中止、中断、延长事由的,判决驳回权利人的诉讼请求(权利人丧失胜诉权),但权利人的实体权利并不消灭。
		
		\item 义务人未提出诉讼时效抗辩的,人民法院不应对诉讼时效问题进行释明及主动适用诉讼时效的规定进行裁判。
		
		\item 当事人在一审期间未提出诉讼时效抗辩,在二审期间提出的,人民法院不予支持;但其基于新的证据能够证明对方当事人的请求权已过诉讼时效期间的情形除外。
		
	\end{enumerate}
	
	
	\textbf{诉讼时效具有强制性}
	(1)当事人对诉讼时效利益的预先放弃无效。
	(2)诉讼时效的期间、计算方法以及中止、中断的事由由法律规定,当事人约定无效。
	
	下列\textbf{请求权不适用诉讼时效}的规定
	(1)请求停止侵害、排除妨碍、消除危险;
	(2)不动产物权和登记的动产物权的权利人请求返还财产;
	(3)请求支付抚养费、赡养费或者扶养费;
	(4)依法不适用诉讼时效的其他请求权。
	
	下列\textbf{债权请求权不适用诉讼时效}的规定
	(1)支付存款本金及利息请求权;
	(2)兑付国债、金融债券以及向不特定对象发行的企业债券本息请求权;
	(3)基于投资关系产生的缴付出资请求权;
	(4)其他依法不适用诉讼时效规定的债权请求权。
	
	诉讼时效和除斥区间一般有着如下的不同
	\begin{enumerate}
		\item 适用对象不同
		①诉讼时效一般适用于债权请求权;
		②除斥期间一般适用于形成权(如追认权、解除权、撤销权等),也可能适用于请求权(如受遗赠权)。
		
		\item 可以援用的主体不同
		①人民法院不能主动援用诉讼时效,诉讼时效须由当事人主张后,人民法院才能审查;
		②除斥期间无论当事人是否主张,人民法院均可主动审查。
		
		\item 法律效力不同
		①诉讼时效届满只是让债务人取得抗辩权,债权人的实体权利并不消灭;
		②除斥期间届满,实体权利消灭。
	\end{enumerate}
	
	\paragraph{诉讼时效的种类和起算}
	诉讼时效有以下两种
	\begin{enumerate}
		\item 普通诉讼时效。向人民法院请求保护民事权利的诉讼时效期间为3年。法律另有规定的,依照其规定(可以中止、中断,不可以延长)。因国际货物买卖合同和技术进出口合同争议提起诉讼或者申请仲裁的时效期间为4年。
		
		\item 最长诉讼时效。自权利受到损害之日起超过20年的,人民法院不予保护;有特殊情况的,人民法院可以根据权利人的申请决定延长。(不可以中止、中断,可以延长)
	\end{enumerate}
	
	诉讼时效期间的起算
	(1)附条件或者附期限的债的请求权,从条件成就或者期限届满之日起算。
	(2)约定有履行期限的债的请求权,从清偿期限届满之日起算;当事人约定同一债务分期履行的,诉讼时效期间从最后一期履行期限届满之日起计算。
	(3)未约定履行期限或者履行期限不明确的债的请求权,依照《民法典》的规定可以确定履行期限的,诉讼时效期间从履行期限届满之日起计算;不能确定履行期限的,诉讼时效期间从债权人要求债务人履行义务的宽限期届满之日起计算,但债务人在债权人第一次向其主张权利之时明确表示不履行义务的,诉讼时效期间从债务人明确表示不履行义务之日起计算。
	(4)请求他人不作为的债权请求权,应当自权利人知道义务人违反不作为义务时起算。
	(5)国家赔偿的诉讼时效的起算,自赔偿请求人知道或者应当知道国家机关及其工作人员行使职权时的行为侵犯其人身权、财产权之日起计算,但被羁押等限制人身自由期间不计算在内。
	(6)未成年人遭受性侵害的损害赔偿请求权的诉讼时效期间,自受害人年满18周岁之日起算。
	(7)无民事行为能力人或者限制民事行为能力人对其法定代理人的请求权,诉讼时效期间自该法定代理终止之日起算。
	(8)无民事行为能力人或者限制民事行为能力人的权利受到损害的,诉讼时效期间自其法定代理人知道或者应当知道权利受到损害以及义务人之日起计算。法律另有规定的,依照其规定。
	
	\paragraph{诉讼时效的中止}
	诉讼时效中止的事由
	在诉讼时效期间的最后6个月内,因下列障碍,不能行使请求权的,\textbf{诉讼时效中止}:
	\begin{enumerate}
		\item 不可抗力;
		
		\item 无民事行为能力人或者限制民事行为能力人没有法定代理人,或者法定代理人死亡、丧失民事行为能力、丧失代理权;
		
		\item 继承开始后未确定继承人或者遗产管理人;
		
		\item 权利人被义务人或者其他人控制;
		
		\item 其他导致权利人不能行使请求权的障碍。
	\end{enumerate}
	
	自中止时效的原因消除之日起满6个月,诉讼时效期间届满。
	
	\paragraph{诉讼时效的中断事由}
	有下列情形之一的,诉讼时效中断,从中断、有关程序终结时起,\textbf{诉讼时效期间重新计算}:
	\begin{enumerate}
		\item 权利人向义务人、义务人的代理人、财产代管人或者遗产管理人等提出履行请求;以下情形认定为权利人向义务人提出履行请求
		\begin{enumerate}
			\item 当事人一方直接向对方当事人送交主张权利文书,对方当事人在文书上签名、盖章、按指印或者虽未签名、盖章、按指印但能够以其他方式证明该文书到达对方当事人的。
			
			\item 当事人一方以发送信件或者数据电文方式主张权利,信件或者数据电文到达或者应当到达对方当事人的。
			
			\item 当事人一方为金融机构,依照法律规定或者当事人约定从对方当事人账户中扣收欠款本息的。
			
			\item 当事人一方下落不明,对方当事人在国家级或者下落不明的当事人一方住所地的省级有影响的媒体上刊登具有主张权利内容的公告的,但法律和司法解释另有特别规定的,适用其规定。
			
			\item 权利人对同一债权中的部分债权主张权利,诉讼时效中断的效力及于剩余债权,但权利人明确表示放弃剩余债权的情形除外。
		\end{enumerate}
		
		
		\item 义务人同意履行义务;义务人作出分期履行、部分履行、提供担保、请求延期履行、制定清偿债务计划等承诺或者行为的,应当认定为“义务人同意履行义务”
		
		\item 权利人提起诉讼或者申请仲裁;
		\begin{enumerate}
			\item 当事人一方向人民法院提交起诉状或者口头起诉的,诉讼时效从提交起诉状或者口头起诉之日起中断。
			
			\item 权利人向人民调解委员会以及其他依法有权解决相关民事纠纷的国家机关、事业单位、社会团体等社会组织提出保护相应民事权利的请求,诉讼时效从提出请求之日起中断。
			
			\item 权利人向公安机关、人民检察院、人民法院报案或者控告,请求保护其民事权利的,诉讼时效从其报案或者控告之日起中断。
			
			\item 上述机关决定不立案、撤销案件、不起诉的,诉讼时效期间从权利人知道或者应当知道不立案、撤销案件、不起诉之日起重新计算。
		\end{enumerate}
	
		
		\item 与提起诉讼或者申请仲裁具有同等效力的其他情形。
		\begin{enumerate}
			\item 申请支付令;
			
			\item 申请破产、申报破产债权;
			
			\item 为主张权利而申请宣告义务人失踪或者死亡;
			
			\item 申请诉前财产保全、诉前临时禁令等诉前措施;
			
			\item 申请强制执行;
			
			\item 申请追加当事人或者被通知参加诉讼;
		\end{enumerate}
	\end{enumerate}
	
	诉讼时效中断的其他情形
	\begin{enumerate}
		\item 对于连带债权人、连带债务人中的一人发生诉讼时效中断效力的事由,应当认定对其他连带债权人、连带债务人也发生诉讼时效中断的效力。
		
		\item 债权人提起代位权诉讼的,应当认定对债权人的债权和债务人的债权均发生诉讼时效中断的效力。
		
		\item 债权转让的,应当认定诉讼时效从债权转让通知到达债务人之日起中断。
		
		\item 债务承担情形下,构成原债务人对债务承认的,应当认定诉讼时效从债务承担意思表示到达债权人之日起中断。
	\end{enumerate}
	
	\newpage
	
	\section{物权法律制度}
	
	首先基本介绍一下什么是物权,其次就从动产和不动产的角度分别考虑两者的物权是如何变动的。
	
	\subsection{物权法律制度概述}
	\paragraph{物的种类} 民法典规定物分为动产和不动产,如果有其他的法律规定,则权利也可以作为物权客体。物权有着如下的特征
	\begin{enumerate}
		\item 有体性
		
		\item 可支配性
		
		\item 在人的身体之外
	\end{enumerate}
	
	物有着如下的分类
	\begin{enumerate}
		\item \textbf{动产与不动产}。不动产包括土地、海域以及房屋、林木等地上定着物。
		 
		\item \textbf{可分物与不可分物}。可分物是指不因分割而变更其性质或者减损其价值的物。牛肉属于可分物,一头牛则属于 不可分物。
		 
		\item (仅限于动产)\textbf{可替代物与不可替代物}。交易客体为可替代物(如冰棍)时,可以同类物替代履行;不可替代 物 ( 如 齐 白 石 的 某 一幅 字 画 ) 一旦 发 生 毁 损 、 灭 失 , 就 只 能 转 化 为 金 钱 赔 偿 。
		
		\item (仅限于动产) \textbf{消耗(费)物与非消耗(费)物}。消耗物只能 一次性使用或者让与,非消耗物则相反。以让与为目的的 消耗物(如金钱)转移占有即转移所有权。
		
		\item \textbf{流通物、限制流通物与禁止流通物。}流通物可以自由进入市场流通(如冰棍),限制流通物是指被法律限制市场流通之物(如 文物、黄金、药品),禁止流通物是指法律规定禁止流通之物。根据《民法典》的规定,法律 规定专属于国家所有的不动产和动产,任何组织或者个人不能取得所有权。
		
		\item \textbf{主物与从物}。认定主物、从物关系,必须同时具备两个条件:(1 )二者在物理上互相独立;(2 )二者在 经济用途上存在主从关系。A物脱离B物,不损害A物的独立用途,则A物为主物;B物脱 离A物,丧失其本来的用途,则B物为从物。
	\end{enumerate}
	
	物的划分上还有一种划分方式,即原物与孳息
	\begin{enumerate}
		\item 孳息是独立于原物的物,原物 、孳息属 于两个物。因此,尚在母牛身体里的小牛属于母牛的组成部分,不属于孳息;尚未与苹果树相分离的苹果,也不属于孳息。

		\item 孳息分为天然孳息和法定孳息(如储蓄存款的利息、出租房屋获得的租金)。
		
		\item 天然孳息,由所有权人取得;既有所有权人又有用益物权人的,由用益物权人取得。 当事人另有约定的,按照其约定。
		
		\item 法定孳息,当事人有约定的,按照约定取得;没有约定或者约定不明确的,按照交易习惯取得。
	\end{enumerate}
	
	\paragraph{物权的概念}
	物权是权利人依法对特定的物享有直接支配和排他的权利。与债权相比,物权具有以下特征:
	\begin{enumerate}
		\item 支配性:物权人有权仅以自己的意志实现权利,无须第三人的积极行为协助,属于支配权。而债权属于请求权,其实现有赖于债务人的履行行为。
		
		\item 排他性:物权具有排他性, 一物之上只能成立一项所有权。而债权具有兼容性,同一标的物之上可 以成立数个买卖合同,几个买卖合同均可有效,并不相互排斥。
		
		\item 绝对性:物权是可以对抗所有人的财产权,排除任何他人的干涉,他人有义务予以尊重,属于绝对权、对世权。而债权仅对特定的债务人存在,属于相对权、对人权。
	\end{enumerate}
	
	物权可以分为以下几类
	\begin{enumerate}
		\item 所有权
		
		\item 用益物权。《民法典》规定的用益物权包括土地承包经营权、建设用地使用权、宅基地使用权、居住权和地役权
		
		\item 担保物权。担保物权包括抵押权、质权和留置权
	\end{enumerate}
	
	物权的分类上可以分类为自物权和他物权以及独立物权和从物权。独立物权是指能够独立存在的物权,包括所有权、建设用地使用权、土地承包经营权、 宅基地使用权和居住权。从物权是指从属于其他权利、不能独立存在的物权,包括担保物权和地役权。具体的分类如下表所示
	
	
	\begin{table}
		\centering
		\begin{tblr}{
				width = \linewidth,
				colspec = {Q[154]Q[256]Q[119]Q[119]Q[154]Q[119]},
				cell{1}{1} = {c=2}{0.41\linewidth},
				cell{2}{1} = {c=2}{0.41\linewidth},
				cell{3}{1} = {r=5}{},
				cell{8}{1} = {c=2}{0.41\linewidth},
				vlines,
				hline{1-3,8-9} = {-}{},
				hline{4-7} = {2-6}{},
			}
			物权的类型 &         & 自物权 & 他物权 & 独立物权 & 从物权 \\
			所有权   &         & 是   &     & 是    &     \\
			用益物权  & 建设用地使用权 &     & 是   & 是    &     \\
			& 土地承包经营权 &     & 是   & 是    &     \\
			& 宅基地使用权  &     & 是   & 是    &     \\
			& 居住权     &     & 是   & 是    &     \\
			& 地役权     &     & 是   &      & 是   \\
			担保物权  &         &     & 是   &      & 是   
		\end{tblr}
	\end{table}
	
	\paragraph{物权法律制度的基本原则}
	主要有以下三个原则
	\begin{enumerate}
		\item 物权法定原则。根据《民法典》第116条的规定,物权的种类和内容,由法律规定。
		
		\item 物权客体特定原则(一物一权原则)
		\begin{enumerate}
			\item 物权只存在于确定的一物之上,物尚未确定谈不上物权;而债权的客体是当事人的给 付行为,即使物尚未确定、尚不存在,也不影响债权合同(如贾某将正在研发的汽车作价10 0 万卖给孙某)的有效性。
			
			\item 一物之上只能有 一个所有权,但所有权人可以为多人(如甲、乙按份共有或者共同共 有 一辆 汽 车 )。
			
			\item 一物之上只能有 一个所有权,但一物之上可以成立数个互不冲突的物权 。如所有权和他物权的共容、用益物权与担保物权的共容。
		\end{enumerate}
		
		\item 物权公示原则
		\begin{enumerate}
			\item 不动产物权的设立、变更、转让和消灭,应当依照法律规定登记。
			
			\item 动产物权的设立和转让,应当依照法律规定交付。
		\end{enumerate}
	\end{enumerate}
	
	\paragraph{物权行为和债权行为}
	我们对比一下物权行为和债权行为,首先先区分一下两者的概念
	
	债权行为的效力在当事人之间确立债权债务关系,债务人为此负有法律上的义务。例如, 甲、乙双方就某套商品房订立买卖合同,买卖合同生效后,出卖人甲负有向买受人乙转移房屋所有权的义务,乙负有向甲支付相应价款的义务。
	
	买卖合同只是债权行为,并不会直接导致房 屋所有权的转移。\textbf{房屋所有权的转移依赖于出卖人向买受人为了履行买卖合同而转移所有权的行为},该行为在消灭合同之债的意义上称为合同的履行行为,在转移物权的意义上称为物权 行为。
	
	接下来从法律效果、处分权和兼容性三方面进行分析
	
	\textbf{法律效果}上,\textbf{债权行为不会直接引起积极财产 (物权)的减少},\textbf{却会导致消极财产(义务)的增加};\textbf{物权行为则直接导致积极财产的减少}。例如,房屋买卖合同生效后,出卖人负有向买受人转移所 有权的义务,但在履行义务之前,标的物的所有权仍属于出卖人。在出卖人实际履行义务(向 买受人转移所有权)之后,出卖人才失去标的物的所有权。
	
	处分权上
	\begin{enumerate}
		\item 物权行为直接导致物权发生变动,因此出让人应当对标的物享有处分权,否则将构成 无权处分。无权处分行为处于效力待定状态,在得到真权利人的追认或者出让人取得处分权之 后,该行为有效;否则,该行为归于无效。
		
		\item 债权行为只是负担行为而不直接转移物权,因此对出卖人无处分权的要求。出卖他人 之物的买卖合同亦可有效,当出卖人无法履行合同时,买受人可以基于有效的买卖合同主张违 约救济。根据《民法典》的规定,因出卖人未取得处分权致使标的物所有权不能转移的,买受 人可以解除合同并请求出卖人承担违约责任。
	\end{enumerate}
	
	
	兼容性上
	\begin{enumerate}
		\item 物权只能被转让一次,出让人在实施转让物权的物权行为后,即失去所转让标的物的 物权,因此对 于同一标的物不能实施两次有效的处分行为。
		
		\item 债权行为因其仅负担义务,而不涉及物权变动,因此可以反复作出,在同一标的物上 成立的数个买卖合同均可有效,但出卖人只能履行其中一项买卖合同,其他未能获得标的物所 有权的买受人有权基于有效的买卖合同请求出卖人承担违约责任(具体规定见第四单元)。
	\end{enumerate}
	
	\subsection{不动产的物权变动}
	物权变动可能基于法律行为,也可能不基于。以下分别进行讨论
	
	\subsubsection{基于法律行为的物权变动}
	\paragraph{登记生效} 
	不动产物权(包括抵押权)的设立、变更、转让和消灭,\textbf{经依法登记,发生效力}(包括建设用地使用权、居住权);未经登记,不发生效力,但是法律另有规定的除外。

	当事人之间订立有关设立、变更、转让和消灭不动产物权的合同,除法律另有规定或 者当事人另有约定外,自合同成立时生效;未办理物权登记的,不影响合同效力。(2016年案 例分析题 )
	
	\paragraph{登记对抗} 
	土地承包经营权自土地承包经营权合同生效时设立。土地承包经营权互换、转让的, 当事人可以向登记机构申请登记
	
	地役权自地役权合同生效时设立。当事人要求登记的,可以向登记机构申请地役权登记
	
	以上两种情况下,\textbf{未经登记,不得对抗善意第三人}。
	
	\subsubsection{非基于法律行为的物权变动}
	物权变动也可能基于事实行为、法律规定以及公法行为。基于非法律行为的物权变动则不以登记为前提。
	\begin{enumerate}
		\item 基于事实行为。因合法建造、拆除房屋等事实行为设立或者消灭物权的,自事实行为成就时发生效力。
		
		非基于法律行为的不动产物权变动\textbf{不以登记为前提},但获得不动产物权之人\textbf{再处分该不动产时},依照法律规定需要办理登记的,未经登记,不发生物权效力
		
		\item 基于法律规定。因继承取得物权的,自继承开始时发生效力。
		
		\item 基于公法行为。因人民法院、仲裁机构的法律文书或者人民政府的征收决定等,导致物权设立、变更、转 让或者消灭的,自法律文书或者征收决定等生效时发生效力。
		
		【解释】人民法院、仲裁机构在分割共有不动产或者动产等案件中作出并依法生效的 改变原有物权关系的判决书、裁决书、调解书,以及人民法院在执行程序中作出的拍卖成 交裁定书、变卖成交裁定书、以物抵债裁定书,应当认定为“ 导致物权设立、变更、转让 或者消灭的人民法院、仲裁机构的法律文书”。这些 “法律文书” 具有直接改变原有物权 关系、不必由当事人履行的形成效力 。如果判决内容是一方当事人向另一方履行,那么, 让物权发生变动的,是当事人的履行行为而非判决本身。
		
	\end{enumerate}

	\subsubsection{不动产登记制度}
	登记这块主要解释各种登记是什么意思,以及在什么情况下可以进行登记。指的注意的是异议登记有15天的限制。预告登记有90天的限制
	
	\paragraph{首次登记} 是指不动产权利第一次登记。未办理不动产首次登记的,不得办理不动 产其他类型登记,但法律、行政法规另有规定的除外。
	
	除了首次登记外,不动产还会进行\textbf{变更登记、转移登记、注销登记、更正登记、异议登记和预告登记}
	
	\paragraph{变更登记}
	变更登记,是指不动产登记事项发生\textbf{不涉及权利转移的变更}所需进行的登记 。有下列情形之一的,不动产权利人可以向不动产登记机构申请变更登记:
	\begin{enumerate}
		\item 权利人的姓名、名称、身份证明类型或者身份证明号码发生变更的;
		
		\item 不动产的坐落、界址、用途、面积等状况变更的;
		
		\item 不动产权利期限、来源等状况发生变化的;
		
		\item 同一权利人分割或者合并不动产的;
		
		\item 抵押担保的范围、主债权数额、债务履行期限、抵押权顺位发生变化的;
		
		\item 最高额抵押担保的债权范围、最高债权额、债权确定期间等发生变化的;
		
		\item 地役权的利用目的、方法等发生变化的;
		
		\item 共有性质发生变更的;
		
		\item 法律、行政法规规定的其他不涉及不动产权利转移的变更情形。
	\end{enumerate}
	
	
	\paragraph{转移登记} 
	转移登记,是指\textbf{不动产权利在不同主体之间发生转移}所需进行的登记。因下列情形导致不动产权利转移的,当事人可以向不动产登记机构申请转移登记: 
	\begin{enumerate}
		\item 买卖、互换、赠与不动产的;
		
		\item 以不动产作价出资 (入股 )的;
		
		\item 法人或者其他组织因合并、分立等原因致使不动产权利发生转移的;
		
		\item 不动产分割、合并导致权利发生转移的;
		
		\item 继承、受遗赠导致权利发生转移的;
		
		\item 共有人增加或者减少以及共有不动产份额变化的;
		
		\item 因人民法院、仲裁委员会的生效法律文书导致不动产权利发生转移的;
		
		\item 因主债权转移引起不动产抵押权转移的;
		
		\item 因需役地不动产权利转移引起地役权转移的; (10)法律、行政法规规定的其他不动产权利转移情形。
	\end{enumerate}
	
	
	\paragraph{注销登记} 
	\textbf{不动产权利消灭时,需要办理注销登记}。有下列情形之一的,当事人可以申请办理注销登记: 
	\begin{enumerate}
		\item 不动产灭失的;
		
		\item 权利人放弃不动产权利的;
		
		\item 不动产被依法没收、征收或者收回的;
		
		\item 人民法院、仲裁委员会的生效法律文书导致不动产权利消灭的;
		
		\item 法律、行政法规规定的其他情形。
	\end{enumerate}
	
	
	\paragraph{更正登记} 权利人、利害关系人认为不动产登记簿记载的事项错误的,可以申请更正登记。不动产登 记簿记载的权利人书面同意更正或者有证据证明登记确有错误的,登记机构应当予以更正。
	
	\paragraph{异议登记}
	不动产登记簿记载的权利人不同意更正的,利害关系人可以申请异议登记。登记机构予以异议登记,申请人自\textbf{异议登记之日起15日内不提起诉讼的},异议登记失效。异议登记不当, 造成权利人损害的,权利人可以向申请人请求损害赔偿。
	
	\paragraph{预告登记}
	当事人签订买卖房屋的协议或者签订其他不动产物权的协议,\textbf{为保障将来实现物权, 按照约定可以向登记机构申请预告登记}。如以下情形
	\begin{enumerate}
		\item 商品房 等不动产预售的;
		
		\item 不动产买卖、抵押的;
		
		\item 以预购商品房设定抵押权的;
		
		\item 法律、 行政法规规定的其他情形。
	\end{enumerate}
	
	预告登记后,未经预告登记的权利人同意,处分该不动产的(包括转让不动产所有权等物权,或者设立建设用地 使用权、居住权、地役权、抵押权等其他物权),不发生物权效力。
	
	预告登记后,\textbf{债权消灭或者自能够进行不动产登记之日起90日内未申请登记的},预告登记失效。
	
	其中债权消灭是指买卖不动产物权的协议被认定无效、被撤销,或者预告登记的权利人放弃债 权的
	
	\subsection{动产动物权变动}
	不动产的物权变动通常以登记作为节点,动产的物权变动通常以交付为节点
	
	\subsubsection{动产的所有权}
	对于一般动产交付生效。动产物权的设立和转让,自交付时发生效力,但是法律另有规定的除外。
	
	对于特殊动产(船舶、航空器和机动车等)则是交付生效+登记对抗。
	
	\subsubsection{特殊的交付方式}
	除了普通交付,还有简易交付(提前给)、指示交付(其他人给)、占有交付(以后给)等特殊的交付方式
	\begin{enumerate}
		\item 简易交付。动产物权设立和转让前,权利人已经占有该动产的,物权自民事法律行为生效时发生效力。
		
		\item 指示交付。动产物权设立和转让前,第三人占有该动产的,负有交付义务的人可以通过转让请求第 三 人返还原物的权利代替交付。
		
		\item 占有交付。动产物权转让时,当事人又约定由出让人继续占有该动产的,物权自该约定生效时发生效力。
	\end{enumerate}

	\subsubsection{动产所有权的特殊取得方式}
	特殊取得方式有先占和添附两种
	\begin{enumerate}
		\item 先占是指以所有权人的意思\textbf{占有无主动产}。先占人基于先占行为取得无主动产的所有权。
		
		\item 添附是附合、混合与加工的总称。
		\begin{enumerate}
			\item 附合是指不同所有权人的物密切结合,构成不可分割的一物。 
			
			动产附合于不动产,由不动产所有权人取得该附合物的所有权。例如,甲错拿乙的钢筋 建造自己的房屋,由甲取得该房屋的所有权。
			
			动产附合于动产, 一般情况下,各动产所有权人按其动产附合时的价值,共有附合物。 但附合的动产,有可视为主物者,则该主物的所有权人取得附合物的所有权。例如,甲错拿乙 的油漆粉刷自己的办公桌,办公桌是主物,因此甲单独取得新办公桌的所有权。 
			
			\item 混合,是指不同所有权人的动产相互混杂合并,不能识别或者识别所需费用过大。例 如,甲错拿乙的牛奶 ,将其倒人自己的咖啡中,难以识别分离。
			
			\item 加工,是指在他人的动产之上进行改造或者劳作,并生成新物的法律事实。例如,甲 将 乙 的 木 板 加 工成 办 公 桌 。
		\end{enumerate}
	\end{enumerate}
	
	\subsection{有权处分和无权处分}
	处分行为是民事法律行为的一种,而处分权也是物权中重要的关注点。因此本节主要考虑分别在什么情况下有权处分以及无权处分
	
	\subsubsection{有权处分}
	这里考虑一下针对普通动产以及特殊动产的一物二卖,什么样的买受人有权处分。总结来讲判断标准分别为: 普通动产:交付 > 付款 > 合同成立时间 ;特殊动产 : 交付 > 登记 > 合同成立时间
	
	\paragraph{普通动产的一物二卖} ( 第 四 章 ) 
	出卖人就同一普通动产订立多重买卖合同,在买卖合同均有效的情况下,买受人均要求实 际履行合同的,应当按照以下情形分别处理: 
	\begin{enumerate}
		\item 先行受领交付的买受人请求确认所有权已经转移的,人民法院应予支持;
		
		\item 均未受领交付,先行支付价款的买受人请求出卖人履行交付标的物等合同义务的,人 民法院应予支持;
		
		\item 均未受领交付,也未支付价款,依法成立在先合同的买受人请求出卖人履行交付标的 物等合同义务的,人民法院应予支持。
	\end{enumerate}
	
	\paragraph{特殊动产的 一物二卖}(第四章)
	出卖人就同 一船舶、航空器、机动车等特殊动产订立多重买卖合同,在买卖合同均有效的 情况下,买受人均要求实际履行合同的,应当按照以下情形分别处理: 
	\begin{enumerate}
		\item 先行受领交付的买受人请求出卖人履行办理所有权转移登记手续等合同义务的,人民 法院应予支持;
		
		\item 均未受领交付,先行办理所有权转移登记手续的买受人请求出卖人履行交付标的物等 合同义务的,人民法院应予支持;
		
		\item 均未受领交付,也未办理所有权转移登记手续,依法成立在先合同的买受人请求出卖 人履行交付标的物和办理所有权转移登记手续等合同义务的,人民法院应予支持; (4)出卖人将标的物交付给买受人之一,又 其他买受人办理所有权转移登记,已受领交 付的买受人请求将标的物所有权登记在自己名下的,人民法院应予支持。
	\end{enumerate}
	
	\subsubsection{无权处分与善意取得制度}
	出卖人因未取得处分权致使标的物所有权不能转移,此时合同有效,买受人可以解除合同并请求出卖人承担违约责任(区分无权处分和无权代理)
	
	无处分权人将不动产或者动产转让给受让人的,所有权人有权追回;除法律另有规定外, 符合下列情形的,受让人取得该不动产或者动产的所有权(善意取得制度):
	\begin{enumerate}
		\item 受让人受让该不动产或者动产时是善意;
		
		\item 以合理的价格转让;
	
		\item 转让的不动产或者动产依照法律规定应当登记的已经登记,不需要登记的已经交付给 受让人。
	\end{enumerate} 
	受让人依据 上述规定取得不动产或者动产的所有权的,原所有权人有权向无处分权人请求损害赔偿。
	
	\subsubsection{拾得遗失物}
	拾得漂流物、发现埋藏物或者隐藏物的,参照适用拾得遗失物的有关规定。 法律另有规定的,依照其规定。这属于准用性规范。
	\begin{enumerate}
		\item 拾得遗失物,应当返还权利人。拾得人应当及时通知权利人领取,或者送交公安等有 关部门。有关部门收到遗失物,知道权利人的,应当及时通知其领取;不知道的,应当及时发 布招领公告。遗失物自发布招领公告之日起1 年内无人认领的,归国家所有。
		
		\item 权利人领取遗失物时,应当向抬得人或者有关部门支付保管遗失物等支出的必要费用。
		
		\item 权利人悬赏寻找遗失物的,领取遗失物时应当按照承诺履行义务。
		
		\item 所有权人或者其他权利人有权追回遗失物。该遗失物通过转让被他人占有的,权利人有权向无处分权人请求损害赔偿,或者自知道或者应当知道受让人之日起\textbf{2年内向受让人请求返还原物};
		
		但是,受让人通过\textbf{拍卖或者向具有经营资格的经营者购得该遗失物的},权利人请求返还原物时应当支付受让人所付的费用。权利人向受让人支付所付费用后,有权向无处分权人 追偿。
		
	\end{enumerate}
	
	\subsection{共有}
	
	\subsubsection{基本规定}
	\paragraph{共有的确定}共有可以分为按份共有和共同共有,无约定一般为按份共有(除共有人具有家庭关系等外)。按份共有的份额无约定一般按照出资额确定,不能确定出资额的视为等额享有
	
	\paragraph{共有物的管理}在管理上,共有人按照约定管理共有的不动产或者动产;没有约定或者约定不明确的,\textbf{各共有人都有管理的权利和义务}。共有人对共有物的管理费用以及其他负担,有约定的,按照其约定;没有约定或者约 定不明确的,\textbf{按份共有人按照其份额负担,共同共有人共同负担}。(2019 年案例分析题)
	
	\paragraph{共有物的分割}共有人约定不得分割共有的不动产或者动产,以维持共有关系的,应当按照约定,但是共 有人有重大理由需要分割的,可以请求分割;没有约定或者约定不明确的,按份共有人可以随 时请求分割,共同共有人在共有的基础丧失或者有重大理由需要分割时可以请求分割。因分割 造成其他共有人损害的,应当给予赔偿。
	
	\paragraph{共有物的对内以及对外责任}因共有的不动产或者动产产生的债权债务,在对外关系上,共有人享有连带债权、承 担连带债务,但是法律另有规定或者第三人知道共有人不具有连带债权债务关系的除外。(2019 年案例分析题 )
	
	在共有人内部关系上,除共有人另有约定外,按份共有人按照份额享有债权、承担债务,共同共有人共同享有债权、承担债务 。偿还债务超过自己应当承担份额的按份共有人,有权向其他共有人追偿。
	
	总结:\textbf{按份共有人}对外承担连带责任、内部承担按份责任。
	
	\subsubsection{按份共有人转让自己的个人份额}
	 按份共有人可以转让其享有的共有的不动产或者动产份额,其他共有人在同等条件下 享有优先购买的权利。民法典所称的“同等条件”,应当综合共有份额的转让价格、价款履行 方式及期限等因素确定。
	 
	按份共有人转让其享有的共有的不动产或者动产份额的,应当将转让条件及时通知其 他共有人。其他共有人应当在合理期限内行使优先购买权。(2019 年案例分析题)
	
	优先购买权的行使期间,按份共有人之间有约定的,按照约定处理;没有约定或者约定不明的,按照下列情形确定:
	\begin{enumerate}
		\item 转让人向其他按份共有人发出的包含同等条件内容的通知中载明行使期间的,以该期 间为准;
		
		\item 通知中未载明行使期间,或者载明的期间短于通知送达之日起15 日的,为15 日;
		
		\item 转让人未通知的,为其他按份共有人知道或者应当知道最终确定的同等条件之日起 15 日;
		
		\item 转让人未通知,且无法确定其他按份共有人知道或者应当知道最终确定的同等条件的, 为共有份额权属转移之日起6 个月。
	\end{enumerate}
	
	两个以上其他共有人主张行使优先购买权的,协商确定各自的购买比例;协商不成的, 按照转让时各自的共有份额比例行使优先购买权。
	
	按份共有人向共有人之外的人转让其份额,其他按份共有人根据法律、司法解释规定, 请求按照同等条件优先购买该共有份额的,人民法院应予支持。其他按份共有人的请求具有下 列情形之 一的,人民法院不予支持: 
	\begin{enumerate}
		\item 未在司法解释规定的期间内主张优先购买,或者虽主张优先购买,但提出减少转让价 款 、增加转让人负担等实质性变更要求;
		
		\item 以其优先购买权受到侵害为由,仅请求撤销共有份额转让合同或者认定该合同无效。 
	\end{enumerate}
	
	按\textbf{份共有人之间转让共有份额}以及\textbf{共有份额的权利主体因继承、遗赠等原因发生变化时},其他按份共有人主张依据《民法典》的规定优先购买的, \textbf{人民法院不予支持},但按份共有人之间另有约定的除外。
	
	
	\subsubsection{共有物的处分}
	处分共有的不动产或者动产以及对共有的不动产或者动产作重大修缮、变更性质或者用途的。不同类型的共有有着不同的条件
	\begin{enumerate}
		\item 按份共有,\textbf{应当经占份额2/3以上的按份共有人同意},但是共有人之间另有约定的除外 。(2019年案 例分析题 )
		
		\item 共同共有 ,\textbf{应当经全体共同共有人同意},但是共有人之间另有约定的除外。(2019 年案例分析题)
	\end{enumerate}
	
	\subsection{建设用地使用权}
	之前主要从物的种类角度考虑了物权,现在我们着重考虑几个特殊的物权(建设用地使用权、担保物权(包括抵押权、质权、留置权))
	
	\subsubsection{建设用地使用权的设立}
	设立建设用地使用权的,应当向登记机构申请建设用地使用权登记。\textbf{建设用地使用权自登记时设立}。设立建设用地使用权,\textbf{可以采取出让或者划拨}等方式。
	
	严格限制以划拨方式设立建设用地使用权(用于商业开发的建设用地,不得以划拨方式取得建设用地使用权)。下列建设用地的土地使用权,确属必需的, 可以由县级以 上人民政府依法批准划拨:
	\begin{enumerate}
		\item 国家机关用地和军事用地;
		
		\item 城市基础设施用地和公益事业用地; 
		
		\item 国家重点扶持的能源、交通、水利等项目用地;
		
		\item 法律、行政法规规定的其他用地。
	\end{enumerate}
	
	建设用地使用权出让,可以采取\textbf{拍卖、招标或者双方协议}的方式。其中,工业 、商业、旅游、娱乐和商品住宅等经营性用地以及同 一土地有两个以上意向用地者的,应当采取招标、 拍卖等公开竞价的方式出让;没有条件,不能采取拍卖、招标方式的,可以采取双方协议的 方式。
	
	\subsubsection{建设用地使用权的期限}
	以无偿划拨方式取得的建设用地使用权,除法律、行政法规另有规定外,没有使用期 限的限制。
	
	以有偿出让方式取得的建设用地使用权,出让最高年限按下列用途确定:
	\begin{enumerate}
		\item 居住用地70年 ;
		
		\item 工业用地50年 ;
		
		\item 教育、科技、文化、卫生、体育用地5 0年;
		
		\item 商业 、 旅 游 、 娱 乐 用 地 4 0 年 ;
		
		\item 综合或者其他用地 5 0 年 。
	\end{enumerate}

	 
	住宅建设用地使用权期限届满的,自动续期。续期费用的缴纳或者减免,依照法律、 行政法规的规定办理。
	
	非住宅建设用地使用权期限届满后的续期,依照法律规定办理。该土地上的房屋以及 其他不动产的归属,有约定的,按照约定;没有约定或者约定不明确的 ,依照法律、行政法规 的规定办理。
	
	\subsubsection{建设用地使用权的转让}
	建设用地使用权转让、互换、出资、赠与或者抵押的,当事人应当采用\textbf{书面形式订立相应的合同}。使用期限由当事人约定 ,但是\textbf{不得超过建设用地使用权的剩余期限}。
	
	建设用地使用权转让、互换、出资或者赠与的,应当向\textbf{登记机构申请变更登记}。
	
	以划拨方式取得土地使用权的,转让房地产时,\textbf{应当按照国务院规定,报有批准权的人民政府审批}。有批准权的人民政府准予转让的,应当由受让方办理土地使用权出让手续,并 依照国家有关规定\textbf{缴纳土地使用权出让金}。
	
	以出让方式取得土地使用权的,转让房地产时,应当符合下列条件:
	\begin{enumerate}
		\item 按照出让合同约定已经支付全部土地使用权出让金,并取得 土地使用权证 书;
		
		\item 按照出让合同约定进行投资开发,属于房屋建设工程的,完成开发投资总额的25\% 以 上,属于成片开发 土地的,形成 工业用地或者其他建设用地条件; 
		
		\item 转让房地产时房屋已经建成的,还应当持有房屋所有权证书。
	\end{enumerate}
	
	
	\subsubsection{集体土地的建设使用}
	对于集体土地有着不同的法律
	\paragraph{农田}
	建设占用土地,涉及农用地转为建设用地的,应当办理农用地转用审批手续。其中:
	\begin{enumerate}
		\item \textbf{永久基本农田转为建设用地的,由国务院批准}。
		
		\item 在土地利用总体规划确定的城市和村庄、集镇建设用地\textbf{规模范围内},为实施该规划而将永久基本农田以外的农用地转为建设用地的,按 土地利用年度计划分批次按照国务院规定由\textbf{原批准土地利用总体规划的机关或者其授权的机关批准}。在已批准的农用地转用范围内,\textbf{具体建设项目用地可以由市、县人民政府批准}。
		
		\item 在土地利用总体规划确定的城市和村庄、集镇建设用地\textbf{规模范围外},将永久基本农田 以外的农用地转为建设用地的,由\textbf{国务院或者国务院授权的省、自治区、直辖市人民政府批准}。
	\end{enumerate}
	
	\paragraph{集体经营性建设用地}城市规划区内的集体所有的 土地,经依法征收转为国有土地后,该幅国有土地的使用权方 可有偿出让,但法律另有规定的除外。
	
	\subsection{担保的一般规定}
	
	\subsubsection{担保的类型}
	保证、抵押、质押和定金,都是依据当事人的合同而设立,称为\textbf{约定担保}。留置则是 直接依据法律的规定而设立,无须当事人之间特别约定,称为\textbf{法定担保}。担保物权分为
	\begin{enumerate}
		\item 意定担保物权(抵押权、质权)
		
		\item 法定担保物权(留置权)。
	\end{enumerate}
	
	担保也可以分为人保、物保、金钱担保
	\begin{enumerate}
		\item 保证是以保证人的财产和信用为担保的基础,属于人的担保 。
		
		\item 抵押、质押和留置,是 以 一定的财产为担保的基础,属于物的担保。
		
		\item 定金是以一定的金钱为担保的基础,称为金钱担保。
	\end{enumerate}
	此外,所有权保留、融资租赁也可具有担保的功能。
	
	
	在担保上还可以有反担保,当第三人为债务人向债权人提供担保的,可以要求债务人提供反担保。反担保方式可以是债务人提供的抵押或者质押,也可以是其他人提供的 保证、抵押、 质押。留置和定金不能作为反担保方式。
	
	\subsubsection{担保合同的效力}
	登记为\textbf{营利法人}的学校、幼儿园、医疗机构、养老机构等提供担保,当事人以其不具有担保资格为由\textbf{主张担保合同无效的,人民法院不予支持}。
	 
	以\textbf{公益为目的的非营利性}学校、幼儿园、医疗机构、养老机构等提供担保的,\textbf{人民法院应当认定担保合同无效},但是有下列情形之 一的除外: 
	\begin{enumerate}
		\item 在购人或者以融资租赁方式承租教育设施、医疗卫生设施、养老服务设施和其他公益 设施时,出卖人、出租人 担保价款或者租金实现而在该公益设施上保留所有权;
		
		\item 以教育设施、医疗卫生设施、养老服务设施和其他公益设施以外的不动产、动产或者 财产权利设立担保物权。
	\end{enumerate}
	
	\subsubsection{担保合同无效的法律责任}
	
	\begin{enumerate}
		\item 主合同有效而第三人提供的担保合同无效
		\begin{enumerate}
			\item 债权人与担保人均有过错的,担保人承担的赔偿责任不应超过债务人不能清偿部分的1/2; 
			
			\item 担保人有过错而债权人无过错的,担保人对债务人不能清偿的部分承担赔偿责任; 
			
			\item 债权人有过错而担保人无过错的,担保人不承担赔偿责任。
		\end{enumerate}
		
		\item 主合同无效导致第三人提供的担保合同无效
		\begin{enumerate}
			\item 担保人无过错的,不承担赔偿责任;
			
			\item 担 保 人 有 过 错 的 , 其 承 担 的 赔 偿 责 任 不 应 超 过 债 务 人 不 能 清 偿 部 分 的 1 / 3 。
		\end{enumerate}
	\end{enumerate}
	
	【相关链接】主合同解除后,担保人对债务人应当承担的民事责任仍应当承担担保责任, 但是担保合同另有约定的除外。
	
	【解释】主合同解除后,担保合同继续有效,担保人仍应按照担保合同承担担保责任, 除非担保合同另有约定。担保合同被确认无效后,债务人、担保人、债权人有过错的,应 当根据其过错各自承担相应的民事责任,即承担《民法典》规定的缔约过失责任。
	
	\subsubsection{借新还旧的担保责任}
	
	\begin{enumerate}
		\item 主合同当事人协议以新贷偿还旧贷,债权人请求旧贷的担保人承担担保责任的,人民 法 院 不 予 支持 。(2 0 2 3 年 案 例 分 析 题 )
		
		\item 主合同当事人协议以新贷偿还旧贷,债权人请求新货的担保人承担担保责任的,按照 下列情形处理:
		\begin{enumerate}
			\item 新货与旧贷的担保人相同的,人民法院应 予支持;
			
			\item 新贷与旧贷的担保人不同,或者旧贷无担保新贷有担保的,人民法院不予支持,但 是债权人有证据证明新贷的担保人提供担保时对以新贷偿还旧贷的事实知道或者应当知道的 除外。
		\end{enumerate}
		
		\item 主合同当事人协议以新贷偿还旧贷,旧贷的物的担保人在登记尚未注销的情形下同意继续为新贷提供担保,在订立新的贷款合同前又以该担保财产为其他债权人设立担保物权,其 他债权人主张其担保物权顺位优先于新贷债权人的,人民法院不予支持。
	\end{enumerate}

	
	
	\subsubsection{不可分性}
	\begin{enumerate}
		\item 主债权未受全部清偿,担保物权人主张就担保财产的全部行使担保物权的,人民法院 应子支持,但是留置权人行使留置权,如果留置财产 可分物的,留置财产的价值应当相当于 债务的金额。
		
		\item 担保财产被分割或者部分转让,担保物权人主张就分割或者转让后的担保财产行使担 保物权的,人民法院应予支持,但是法律或者司法解释另有规定的除外。
		
		\item 主债权被分割或者部分转让,各债权人主张就其享有的债权份额行使担保物权的,人 民法院应予支持,但是法律另有规定或者当事人另有约定的除外。
		
		\item 主债务被分割或者部分转移,债务人自己提供物的担保,债权人请求以该担保财产担 保全部债务履行的,人民法院应予支持;第三人提供物的担保,主张对未经其书面同意转移的 债务不再承担担保责任的,人民法院应予支持。
	\end{enumerate}
	
	
	\subsection{抵押权}
	
	\subsubsection{抵押财产}
	所谓抵押,是指为担保债务的履行,债务人或者第三人\textbf{不转移财产的占有}, 将该财产抵押给债权人,债务人不履行到期债务或者发生当事人约定的实现抵押权的情形, 债权人有权\textbf{就该财产优先受偿}。其中,债务人或者第三人为抵押人,债权人为抵押权人, 提供担保的财产为抵押财产。
	
	债务人或者第三人有权处分的下列财产可以抵押: 
	\begin{enumerate}
		\item 建筑物和其他土地附着物;
		
		\item 建设用地使用权;
		
		\item 海 域 使 用 权 ;
		
		\item 生产设备、原材料、半成品、产品;
		
		\item 正在建造的建筑物、船舶、航空器;
		
		\item 交通运输 工具;
		
		\item 法律、行政法规未禁止抵押的其他财产。
	\end{enumerate}
	
	下列财产不得抵押:
	\begin{enumerate}
		\item 土地所有权;
		
		\item 宅基地、自留地、自留山等集体所有土地的使用权,但是法律规定可以抵押的除外; 
		
		\item 学校、幼儿园、医疗机构等为公益目的成立的非营利法人的教育设施、医疗卫生设施 和其他公益设施;
		
		\item 所有权、使用权不明或者有争议的财产;
		 
		\item 依 法 被 查 封 、 扣 押 、 监 管 的 财 产 ;
		
		\item 法律、行政法规规定不得抵押的其他财产
	\end{enumerate}
	
	此外抵押中还有房地一体原则
	\begin{enumerate}
		\item 城市房地产(房随地走、地随房走):以建筑物抵押的,该建筑物占用范围内的建设用地使用权一并抵押。以建设用地使用权抵 押的,该土地上的建筑物一并抵押。
		
		\item 农村集体土地(地随房走):乡镇、村企业的建设用地使用权不得单独抵押。以乡镇、村企业的厂房等建筑物抵押的, 其占用范围内的建设用地使用权一并抵押。
	\end{enumerate}
	
	
	\subsubsection{抵押合同}
	设立抵押权,当事人应当采用书面形式订立抵押合同
	
	流押条款:抵押权人在债务履行期限届满前,与抵押人约定债务人不履行到期债务时抵押财产归债权 人所有的,只能依法就抵押财产优先受偿(所有权不直接归债权人所有,质押和抵押相同)
	
	担保物权的担保范围包括主债权及其利息、违约金、损害赔偿金、保管担保财产和实 现担保物权的费用。当事人另有约定的,按照其约定。保证的范围不包括保管担保财产的费用
	
	\subsubsection{不动产的抵押}
	以建筑物和其他土地附着物、建设用地使用权、海域使用权和正在建造的建筑物抵押的, 应当办理抵押登记 ,\textbf{抵押权自登记时设立} 
	
	\paragraph{不能办理抵押登记时的责任承担}
	\begin{enumerate}
		\item  不动产抵押合同生效后未办理抵押登记手续,债权人请求抵押人办理抵押登记手续的 , 人民法院应予支持。
		
		\item 抵押财产因不可归责于抵押人自身的原因灭失或者被征收等导致不能办理抵押登记, 债权人请求抵押人在约定的担保范围内承担责任的,人民法院不予支持;但是抵押人已经获得 保险金、赔偿金或者补偿金等,债权人请求抵押人在其所获金额范围内承担赔偿责任的,人民法院依法予以支持 。
		
		\item 因抵押人转让抵押财产或者其他可归责于抵押人自身的原因导致不能办理抵押登记, 债权人请求抵押人在约定的担保范围内承担责任的,人民法院依法予以支持,但是不得超过抵 押权能够设立时抵押人应当承担的责任范围。
		
		\item 当事人申请办理抵押登记手续时,因登记机构的过错致使其不能办理抵押登记,当事 人请求登记机构承担赔偿责任的,人民法院依法予以支持。
	\end{enumerate}
	
	
	\paragraph{以划拨方式取得的建设用地使用权}
	当事人以划拨方式取得的建设用地使用权抵押,抵押人以未办理批准手续为由主张抵 押合同无效或者不生效的,人民法院不予支持。已经依法办理抵押登记,抵押权人主张行使抵 押权的,人民法院应予支持。抵押权依法实现时所得的价款,应当优先用于补缴建设用地使用 权 出 让 金 。 (2 0 1 3 年 案 例 分 析 题 )
	
	抵押人以划拨建设用地上的建筑物抵押,当事人以该建设用地使用权不能抵押或者未 办理批准手续为由主张抵押合同无效或者不生效的,人民法院不子支持。抵押权依法实现时, 拍卖、变卖建筑物所得的价款,应当优先用于补缴建设用地使用权出让金。
	
	拍卖、变卖建筑物所得的价款,应当优先用于补缴建设用地使用权出让金。
	
	\paragraph{土地上新增的建筑物} 
	\begin{enumerate}
		\item 建设用地使用权抵押后,该土地上新增的建筑物不属于抵押财产。该建设用地使用权 实现抵押权时,应当将该土地上新增的建筑物与建设用地使用权一并处分。但是,新增建筑物 所得的价款,抵押权人无权优先受偿。(2013 年案例分析题)
		
		\item 当事人仅以建设用地使用权抵押,债权人主张抵押权的效力及于土地上已有的建筑物 以及正在建造的建筑物已完成部分的,人民法院应予支持。债权人主张抵押权的效力及于正在 建造的建筑物的续建部分以及新增建筑物的,人民法院不予支持。 
		
		\item 当事人以正在建造的建筑物抵押,抵押权的效力范围限于已办理抵押登记的部分。当 事人按照担保合同的约定,主张抵押权的效力及于续建部分、新增建筑物以及规划中尚未建造 的建筑物的,人民法院不予支持。
		
		\item 抵押人将建设用地使用权、土地上的建筑物或者正在建造的建筑物分别抵押给不同债 权人的,人民法院应当根据抵押登记的时间先后确定清偿顺序。		
	\end{enumerate}

	\paragraph{预告登记}如果存在尚未办理建筑物所有权首次登记、预告登记的财产与办理建筑物所有权首次登记时的财产不一致、 抵押预告登记已经失效等情形,则预告登记失效。
	
	如果不存在预告登记失效等情形,认定抵押权自预告登记之日起设立。
	
	
	\subsubsection{动产的抵押}
	以动产抵押的,抵押权自\textbf{抵押合同生效时设立};未经登记,不得对抗善意第三人
	
	动产抵押合同订立后\textbf{未办理抵押登记},动产抵押权的效力按照下列情形分别处理:
	\begin{enumerate}
		\item 抵押人转让抵押财产,受让人占有抵押财产后,抵押权人向受让人请求行使抵押权的, 人民法院不予支持,但是抵押权人能够举证证明受让人知道或者应当知道已经订立抵押合同的 除外;
		
		\item 抵押人将抵押财产出租给他人并移转占有,抵押权人行使抵押权的,租赁关系不受影 响,但是抵押权人能够举证证明承租人知道或者应当知道已经订立抵押合同的除外; 
		
		\item 抵押人的其他债权人向人民法院申请保全或者执行抵押财产,人民法院已经作出财产 保全裁定或者采取执行措施,抵押权人主张对抵押财产优先受偿的,人民法院不予支持;
		
		\item 抵押人破产,抵押权人主张对抵押财产优先受偿的,人民法院不予支持。
	\end{enumerate}
	
	
	以动产抵押的,不得对抗\textbf{正常经营活动中已经支付合理价款并取得抵押财产的买受人}。(和善意取得要件进行对比)
	
	 买受人在出卖人正常经营活动中通过支付合理对价取得已被设立担保物权的动产,担保物权人请求就该动产优先受偿的,人民法院不予支持,但是有下列情形之一的 除外:
	 \begin{enumerate}
	 	\item  购买商品的数量明显超过一般买受人;
	 	
	 	\item 购买出卖人的生产设备;
	 	
	 	\item 订立买卖合同的目的在于担保出卖人或者第三人履行债务;
	 	
	 	\item 买受人与出卖人存在直接或者 间接的控制关系;
	 	
	 	\item 买受人应当查询抵押登记而未查询的其他情形。
	 \end{enumerate}
	
	\subsubsection{动产的浮动抵押}
	企业、个体工商户、农业生产经营者可以将\textbf{现有的以及将有的生产设备、原材料、半成品、 产品抵押},债务人不履行到期债务或者发生当事人约定的实现抵押权的情形,债权人有权就抵 押财产确定时的动产优先受偿。抵押权自抵押合同生效时设立;未经登记,不得对抗善意第 三人。
	
	动产浮动抵押\textbf{无论是否办理抵押登记,均不得对抗正常经营活动中已支付合理价款并取得抵押财产的买受人}。
	
	抵押财产自下列情形之一发生时确定:
	\begin{enumerate}
		\item 债务履行期限届满,债权未实现;
		
		\item 抵押人被宣告破产或者解散;
		
		\item 当事人约定的实现抵押权的情形;
		
		\item 严重影响债权实现的其他情形 。
	\end{enumerate}
	
	
	\subsubsection{抵押物的转让}

	抵押期间,抵押人可以转让抵押财产。当事人另有约定的,按照其约定。抵押财产转 让的,抵押权不受影响。(2021年案例分析题)
	
	抵押人转让抵押财产的,\textbf{应当及时通知抵押权人}。抵押权人能够证明抵押财产转让\textbf{可能损害抵押权的},\textbf{可以请求抵押人将转让所得的价款向抵押权人提前清偿债务或者提存}。转让 的价款超过债权数额的部分归抵押人所有,不足部分由债务人清偿。
	
	关于“当事人约定禁止或者限制转让抵押财产” 的司法解释
	\begin{enumerate}
		\item 未将约定登记
		\begin{enumerate}
			\item 抵押权人请求确认转让合同无效的,人民法院不予支持; 
			
			\item 抵押财产已经交付或者登记,抵押权人请求确认转让不发生物权效力的,人民法院不予 支持,但是抵押权人有证据证明受让人知道的除外; 
			
			\item 抵押权人请求抵押人承担违约责任的,人民法院依法予以支持。
		\end{enumerate}
		
		\item 已经将约定登记
		\begin{enumerate}
			\item 抵押权人请求确认转让合同无效的,人民法院不予支持; 
			
			\item 抵押财产已经交付或者登记,抵押权人主张转让不发生物权效力的,人民法院应予支持, 但是因受让人代替债务人清偿债务导致抵押权消灭的除外
		\end{enumerate}
	\end{enumerate} 
	
	动产抵押合同订立后\textbf{未办理抵押登记,抵押人转让抵押财产},受让人占有抵押财产后, 抵押权人向受让人请求行使抵押权的,人民法院不予支持,但是抵押权人能够举证证明受让人 知道或者应当知道已经订立抵押合同的除外。(动产抵押中未办理抵押登记就转让抵押财产,不对对抗善意第三人)
	
	
	\subsubsection{抵押物的出租}
	
	\paragraph{先出租后抵押} 抵押权设立前,抵押财产已经出租并转移占有的,原租赁关系不受该抵押权的影响。
	
	\paragraph{先抵押后出租} 动产抵押合同订立后未办理抵押登记,抵押人将抵押财产出租给他人并移转占有,抵押权 人行使抵押权的,租赁关系不受影响,但是抵押权人能够举证证明承租人知道或者应当知道已 经订立抵押合同的除外。
	
	\subsubsection{抵押权的效力}
	
	\paragraph{物上代位性}
	\begin{enumerate}
		\item 担保期间,担保财产毁损、灭失或者被征收等,担保物权人可以就获得的保险金、赔偿金或者补偿金等优先受偿。
		
		\item 被担保债权的履行限未届满的,也可以提存该保险金、赔偿金或者补偿金等。
	\end{enumerate} 
	
	\paragraph{孳息} 债务人不履行到期债务或者发生当事人约定的实现抵押权的情形,致使抵押财产被人民法院依法扣押的,自扣押之日起,\textbf{抵押权人有权收取该抵押财产的天然孳息或者法定孳息},但是 抵押权人未通知应当清偿法定孳息义务人的除外。收取的孳息应当先充抵收取孳息的费用。关于孳息的处理总结如下
	\begin{enumerate}
		\item 抵押物被扣押之前的孳息归抵押人,与抵押权人没有关系;
		
		\item 抵押物被 扣押之后,抵押权人有权收取该抵押物的天然孳息(不用通知),但并非取得该孽息的所 有 权 , 而 是 将 孳 息 一 并 计 入 抵 押 财 产 ; 
		
		\item 如果收取法定孳息(如房租), 应通知义务人(如承租人)。
	\end{enumerate}

	
	\paragraph{从物}(2022年案例分析题)
	
	这里主要考虑从物的抵押权效力如何判断。如果产生于抵押权设立前,效力及于从物。反之同理。具体来说
	\begin{enumerate}
		\item 从物产生于抵押权依法设立前,抵押权人主张抵押权的效力及于从物的,人民法院应予支持,但是当事人另有约定的除外。
		
		\item 从物产生于抵押权依法设立后,抵押权人主张抵押权的效力及于从物的,人民法院不予支持,但是在抵押权实现时可以 一并处分。
	\end{enumerate}
	
	\paragraph{添附}
	这里主要分析抵押财产被添附后抵押权的效力问题
	\begin{enumerate}
		\item 抵押权依法设立后,抵押财产被添附,添附物归第 三人所有,抵押权人主张抵押权效 力及于补偿金的,人民法院应予支持。 
		
		\item 抵押权依法设立后,抵押财产被添附,抵押人对添附物享有所有权,抵押权人主张抵 押权的效力及于添附物的,人民法院应予支持,但是添附导致抵押财产价值增加的,抵押权的 效力不及于增加的价值部分。 
		
		\item 抵押权依法设立后,抵押人与第三人因添附成为添附物的共有人,抵押权人主张抵押权的效力及于抵押人对共有物享有的份额的,人民法院应予支持 。		
	\end{enumerate}

	
	\paragraph{同一财产向两个以上债权人设定抵押时的清偿顺序}(2018 年案例分析题) 
	
	同一财产向两个以上债权人抵押的,拍卖、变卖抵押财产所得的价款依照下列规定清偿: 
	\begin{enumerate}
		\item 抵押权已经登记的,按照登记的时间先后确定清偿顺序;
		
		\item 抵押权已经登记的先于未登记的受偿;
		
		\item 抵押权未登记的,按照债权比例清偿。 【解释】抵押权均未登记的,按照债权比例(而非抵押权设立的时间先后)清偿。
	\end{enumerate}
	
	\paragraph{抵押权顺位的变更} 抵押权人与抵押人可以协议变更抵押权顺位以及被担保的债权数额等内容。但是,抵押权 的变更未经其他抵押权人书面同意的,不得对其他抵押权人产生不利影响。
	
	债务人以自己的财产设定抵押,抵押权人放弃该抵押权、抵押权顺位或者变更抵押权的, 其他担保人在抵押权人丧失优先受偿权益的范围内免除担保责任,但是其他担保人承诺仍然提 供担保的除外。
	
	\paragraph{抵押权的消灭}
	\begin{enumerate}
		\item 债权消灭
		
		\item 抵押权实现
		
		\item 抵押物灭失。抵押物灭失,将导致抵押权消灭。但是,如果抵押物灭失之后存在保险金、赔偿金等价值 转换形态,则抵押权并不消灭。
		
		\item 混同。如果抵押权人获得抵押物的所有权,集抵押权人与抵押人于一身,将导致抵押权消灭。
	\end{enumerate}
	
	
	\subsection{质权}
	
	\subsubsection{动产质押}
	动产、不动产均可抵押。质押包括动产质押和权利质押。所谓动产质押,是 指为担保债务的履行 ,\textbf{债务人或者第三人将其动产出质给债权人占有的},债务人不履行到 期债务或者发生当事人约定的实现质权的情形,债权人有权就该动产优先受偿。其中,债 务人或者第三人为出质人,债权人为质权人,交付的动产为质押财产。
	
	 动产质权自出质人交付质押财产时设立。(和动产抵押对比)
	 
	质权人对于质物处分也存在限制 
	\begin{enumerate}
		\item 质权人在质权存续期间,未经出质人同意,擅自使用、处分质押财产,造成出质人损 害的,应当承担赔偿责任。
		
		\item 质权人在质权存续期间,未经出质人同意转质,造成质押财产毁损、灭失的,应当承 担赔偿责任。
	\end{enumerate}
	
	质权人有着妥善保管质物的义务
	\begin{enumerate}
		\item 因保管不善致使质押财产毁损、灭失的,应当承担赔偿责任。
		
		\item 质权人的行为可能使质押财产毁损、灭失的,出质人可以请求质权人将质押财产提存, 或者请求提前清偿债务并返还质押财产。
	\end{enumerate} 
	
	出质人有及时行使质权请求权 
	\begin{enumerate}
		\item 出质人可以请求质权人在债务履行期限届满后及时行使质权;质权人不行使的,出质 人可以请求人民法院拍卖、变卖质押财产。
		
		\item 出质人请求质权人及时行使质权,因质权人怠于行使权利造成出质人损害的,由质权 人承担赔偿责任。
	\end{enumerate}
	
	\subsubsection{权利质押}
	债务人或者第三人有权处分的下列权利可以出质:
	\begin{enumerate}
		\item 汇票、本票、支票;
		
		\item 债券、存款单;
		
		\item 仓单、提单;
		
		\item 可以转让的基金份额、股权;
		
		\item 可以转让的注册商标专用权、专利权、著作权等知识产权中的财产权;
		
		\item 现有的以及将有的应收账款;
		
		\item 法律、行政法规规定可以出质的其他财产权利。 
	\end{enumerate}【解释】根据物权法定原则,不动产、建设用地使用权、海域使用权可以设定抵押, 但不能设定质押。可以转让的股权、应收账款等权利可以设定质押,但不能设定抵押。
	
	质权设立的时点上以汇票、本票、支票、债券、存款单、仓单、提单出质的,\textbf{质权自权利凭证交付质权人时设立};\textbf{没有权利凭证的,质权自办理出质登记时设立}。
	\begin{enumerate}
		\item 以汇票出质,当事人以背书记载“质押” 字样并在汇票上签章,汇票已经交付质权人 的,人民法院应当认定质权自汇票交付质权人时设立。 
		
		\item 存货人或者仓单持有人在仓单上以背书记载“质押” 字样,并经保管人签章,仓单已 经交付质权人的,人民法院应当认定质权自仓单交付质权人时设立。没有权利凭证的仓单,依 法可以办理出质登记的,仓单质权自办理出质登记时设立。
		
		\item 出质人既以仓单出质,又以仓储物设立担保,按照公示的先后确定清偿顺序;难以确 定先后的,按照债权比例清偿。 
		
		\item 保管人为同一货物签发多份仓单,出质人在多份仓单上设立多个质权,按照公示的先 后确定清偿顺序;难以确定先后的,按照债权比例受偿。
	\end{enumerate} 
	
	以\textbf{可以转让的基金份额、股权出质}的,质权自办理出质登记时设立。基金份额、股权出质后,不得转让,但是出质人与质权人协商同意的除外。(2022年案例分析题)
	
	以\textbf{可以转让的注册商标专用权、专利权、著作权等知识产权中的财产权出质}的,质权 自办理出质登记时设立。知识产权中的财产权出质后,出质人不得转让或者许可他人使用,但 是出质人与质权人协商同意的除外。
	
	以现有的以及将有的应收账款出质的,质权自办理出质登记时设立。应收账款出质后, 不得转让,但是出质人与质权人协商同意的除外。
	\begin{enumerate}
		\item  以现有的应收账款出质,应收账款债务人向质权人确认应收账款的真实性后,又以应 收账款不存在或者已经消灭为由主张不承担责任的,人民法院不予支持。
		
		\item 以现有的应收账款出质,应收账款债务人未确认应收账款的真实性,质权人以应收账 款债务人为被告,请求就应收账款优先受偿,能够举证证明办理出质登记时应收账款真实存在 的,人民法院应予支持;质权人不能举证证明办理出质登记时应收账款真实存在,仅以已经办 理出质登记为由,请求就应收账款优先受偿的,人民法院不予支持。 
		
		\item 以现有的应收账款出质,应收账款债务人已经向应收账款债权人履行了债务,质权人 请求应收账款债务人履行债务的,人民法院不予支持,但是应收账款债务人接到质权人要求向 其履行的通知后,仍然向应收账款债权人履行的除外。
		
		\item 以基础设施和公用事业项目收益权、提供服务或者劳务产生的债权以及其他将有的应 收账款出质,当事人为应收账款设立特定账户,发生法定或者约定的质权实现事由时,质权人 请求就该特定账户内的款项优先受偿的,人民法院应予支持;特定账户内的款项不足以清偿债 务或者未设立特定账户,质权人请求折价或者拍卖、变卖项目收益权等将有的应收账款,并以 所得的价款优先受偿的,人民法院依法予以支持。
	\end{enumerate}


	\begin{table}[h!]
		\centering
		\begin{tblr}{
				width = \linewidth,
				colspec = {Q[129]Q[535]Q[273]},
				cell{1}{1} = {r=4}{},
				cell{5}{1} = {r=7}{},
				cell{5}{3} = {r=4}{},
				cell{9}{3} = {r=3}{},
				cell{12}{1} = {r=4}{},
				cell{12}{3} = {r=3}{},
				vlines,
				hline{1,5,12,16} = {-}{},
				hline{2-4,9,15} = {2-3}{},
				hline{6-8,10-11,13-14} = {2}{},
			}
			动产   & 一般动产的所有权              & 交付生效      \\
			& 船舶、航空器、机动车的所有权        & 交付生效、登记对抗 \\
			& 动产的抵押权                & 登记生效      \\
			& 动产的质权                 & 交付生效      \\
			不动产  & 房屋的转让、抵押              & 登记生效      \\
			& 建设用地使用权的设立、转让、抵押      &           \\
			& 海域使用权的抵押              &           \\
			& 居住权的设立                &           \\
			& 土地承包经营权的设立            & 登记对抗      \\
			& 地役权的设立                &           \\
			& 以家庭承包方式取得的土地使用权的抵押    &           \\
			权利质押 & 可以转让的基金份额、股权          & 登记生效      \\
			& 可以转让的知识产权中的财产权        &           \\
			& 现 有的 以 及将 有 的 应 收 账 款 &           \\
			& 票据、债券、存款单、仓单、提单       & 交付(登记)生效  
		\end{tblr}
	\end{table}
	
	\subsection{留置权}
	
	\subsubsection{留置权}
	债务人不履行到期馈务,债权人可以留置已经合法占有的债务人或者第 三人的动产, 并有权就该动产优先受偿。其中,债权人为留置权人,占有的动产为留置财产 。(2022年案例 分析题 )
	
	【解释1】留置权的行使对象\textbf{仅限于动产}。留置权属于法定担保物权,只要具备法定要件,债权人就可以依法行使留置权 。但是,\textbf{当事人可以通过合同约定事先排除留置权的适用}。例如,甲、乙公司订立保管合同,双方事先约定,即使债务人甲公司不能按期支付保 管费,债权人乙公司也不得行使留置权,该约定有效。
	
	【解释2 】留置财产为可分物的,留置财产的价值应当相当于债务的金额。
	
	
	债权人留置的动产,\textbf{应当与债权属于同一法律关系},但是企业之间留置的除外。 【解释】所谓“同一法律关系 ”,是指占有人交付或者返还占有物的义务与留置所担保 的债权属于同一法律关系。因此在同一法律关系以及非同一法律关系的情况下,债权人与第三人有着不同的权利
	\begin{enumerate}
		\item 债务人不履行到期债务,债权人\textbf{因同一法律关系}留置合法占有的第 三人的动产,并主 张就该留置财产优先受偿的,人民法院应予支持。第三人以该留置财产并非债务人的财产为由 请求返还的,人民法院不予支持。
		
		\item 企业之间留置的动产与债权\textbf{并非同一法律关系},债权人留置第 三人的财产,第三人请 求债权人返还留置财产的,人民法院应予支持。
	\end{enumerate}
	
	企业之间留置的动产与债权并非同 一法律关系,债务人以该债权不属于企业持续经营 中发生的债权为由请求债权人返还留置财产的,人民法院应予支持。
	
	留置权人与债务人应当约定留置财产后的债务履行期限;没有约定或者约定不明确的, 留置权人应当给债务人60 日以上履行债务的期限,但是鲜活易腐等不易保管的动产除外。债 务人逾期未履行的,留置权人可以与债务人协议以留置财产折价,也可以就拍卖、变卖留置财 产所得的价款优先受偿。留置财产折价或者变卖的,应当参照市场价格。
	
	债务人可以请求留置权人在债务履行期限届满后行使留置权;留置权人不行使的,债 务人可以请求人民法院拍卖、变卖留置财产。
	【相关链接】动产质押的出质人可以请求质权人在债务履行期限届满后及时行使质权; 质权人不行使的,出质人可以请求人民法院拍卖、变卖质押财产。
	
	留置权的消灭原因
	\begin{enumerate}
		\item 债权消灭;
		
		\item 留置权人对留置财产丧失占有; 
		
		\item 债务人另行提供担保并被留置权人接受。
	\end{enumerate}
	
	
	\subsubsection{担保物权与诉讼时效}
	
	\paragraph{抵押权} 抵押权人应当在主债权诉讼时效期间行使抵押权;未行使的,人民法院不予保护。抵押人 以主债权诉讼时效期间届满为由,主张不承担担保责任的,人民法院应予支持。
	
	\paragraph{留置权 }主债权诉讼时效期间届满后,财产被留置的债务人或者对留置财产享有所有权的第三人请 求债权人返还留置财产的,人民法院不予支持;债务人或者第 三人请求拍卖、变卖留置财产并 以所得价款清偿债务的,人民法院应予支持。
	
	\paragraph{质权}
	主债权诉讼时效期间届满的法律后果,以 登记作为公示方式的权利质权,参照适用抵押权 的规定;动产质权、以交付权利凭证作为公示方式的权利质权,参照适用留置权的规定。
	
	【相关链接1 】以可以转 让的基金份额、股权出质的,质权自办理出质登记时设立。 
	
	【相关链接2 】以汇票、本票、支票、债券、存款单、仓单、提单出质的,质权自权利 凭证交付质权人时设立;没有权利凭证的,质权自办理出质登记时设立。
	
	\subsection{担保的并存}
	
	\subsubsection{人保+物保}
	被担保的债权既有物的担保又有人的担保的,债务人不履行到期债务或者发生当事人约定 的实现担保物权的情形,债权人应当按照约定实现债权;没有约定或者约定不明确:
	
	\textbf{债务人自己提供物的担保的},债权人应当先就该物的担保实现债权。同一债权既有债 务人自己提供的物的担保,又有第 三人提供的担保,承担了担保责任或者赔偿责任的第 三人, 主张行使债权人对债务人享有的担保物权的,人民法院应予支持。
	
	【解释】之所以先就债务人提供的物保实现债权,是因为这样既可以避免法律关系的 复杂化,又有助于节省司法成本。如果先由保证人承担责任,那保证人必然再向债务人追偿, 其仍然可能要就债务人的物保变价求偿,会造成较多资源浪费。在债务人的物保与第三人 的物保并存时,也应同样处理,除非债务人提供的物保不足以清偿全部债务。(2024 年新增)
	
	\textbf{第三人提供物的担保的},债权人可以就物的担保实现债权,也可以请求保证人承担保 证责任。提供担保的第三人承担担保责任后,有权向债务人追偿。
	
	【相关链接】主债务被分割或者部分转移,债务人自己提供物的担保,债权人请求以 该担保财产担保全部债务履行的,人民法院应予支持;第三人提供物的担保,主张对未经 其书面同意转移的债务不再承担担保责任的,人民法院应予支持。

	
	\subsubsection{动产抵押权、质权与留置权的竞存}
	同一动产向两个以上债权人设定抵押时的清偿顺序
	\begin{enumerate}
		\item 抵押权已经登记的,按照登记的时间先后确定清偿顺序;
		
		\item 抵押权已经登记的先于未登记的受偿;
		
		\item 抵押权未登记的,按照债权比例清偿。
	\end{enumerate}
	
	同一财产既设立抵押权又设立质权的,拍卖、变卖该财产所得的价款按照登记、交付的时 间先后确定清偿顺序。
	
	同一动产上已经设立抵押权或者质权,该动产又被留置的,留置权人优先受偿。
	
	\subsubsection{动产抵押权人的超级优先权}
	《民法典》的规定,动产抵押担保的主债权是抵押物的价款,标的物交付后10 日内办理抵押登记的,该抵押 权人优先于抵押物买受人的其他担保物权人受偿,但是留置权人除外。
	\begin{enumerate}
		\item 担保人在设立动产浮动抵押并办理抵押登记后又购人或者以融资租赁方式承租新的 动产,下列权利人为担保价款债权或者租金的实现而订立担保合同,并在该动产交付后10 日内 办理登记,主张其权利优先于在先设立的浮动抵押权的,人民法院应予支持:
		\begin{enumerate}
			\item 在该动产上设立抵押权或者保留所有权的出卖人; 
			
			\item 为价款支付提供融资而在该动产上设立抵押权的债权人; 
			
			\item 以融资租赁方式出租该动产的出租人。
		\end{enumerate}
		
		\item 买受人取得动产但未付清价款或者承租人以融资租赁方式占有租赁物但是未付清全部 租金, 又以标的物 为他 人设 立担保物权, 上述权利 人为担保价款债权或者租 金的实现而订立担保合同,并在该动产交付后10 日内办理登记,主张其权利优先于买受人为他人设立的担保物 权的,人民法院应予支持。(2022 年案例分析题)
		
		\item 同一动产上存在多个价款优先权的,人民法院应当按照登记的时间先后确定清偿顺序。
	\end{enumerate}
	
	
	抵押人购买动产时,一般需要向出卖人或者金融机构融资,并就该债权提供 担保,通常是在该动产上设定抵押。与此同时,抵押人可能之前已经设定过浮动抵押,该 动产可能会被纳入抵押财产中,进而成为其他债权担保的一部分。在这种情况下,对于该 动产的拍卖价款,谁优先受偿?
	
	\subsubsection{让与担保}
	\paragraph{让与担保中所有权的效力}
	根据《民法典担保制度解释》第68 条第 一款的规定,债务人或者第三人与债权人约定将 财产形式上转移至债权人名下,债务人不履行到期债务,债权人有权对财产折价或者以拍卖、 变卖该财产所得价款偿还债务的,人民法院应当认定该约定有效。当事人已经完成财产权利变 动的公示,债务人不履行到期债务,债权人请求参照 《民法典》关于担保物权的有关规定就该 财产优先受偿的,人民法院应予支持。
	
	【解释】所谓“将财产形式上转移”,意味着当事人所转移的所有权并非真正意义上的 所有权,而是仅具有担保功能的所有权。形式上的受让人并不享有对财产的全面支配权, 而只享有就该财产进行变价、优先受偿的权利。
	
	\paragraph{流质的禁止}
	(1 )根据 《民法典担保制度解释》第 68 条第二款的规定,债务人或者第三人与债权人约 定将财产形式上转移至债权人名下,债务人不履行到期债务,财产归债权人所有的,人民法院 应当认定该约定无效,但是不影响当事人有关提供担保的意思表示的效力。当事人已经完成财 产权利变动的公示,债务人不履行到期债务,债权人请求对该财产享有所有权的,人民法院不 予支持;债权人请求参照《民法典》关于担保物权的规定对财产折价或者以拍卖 、变卖该财产 所得的价款优先受偿的,人民法院应予支持;债务人履行债务后请求返还财产,或者请求对财 产折价或者以拍卖、变卖所得的价款清偿债务的,人民法院应予支持。
	
	【解释】在让与担保等非典型担保中,仍应适用流质(流押)禁止规定 。換言之,在 债务人不履行到期债务时,债权人不能直接获得真正的所有权,而只能对财产进行合理折 价,或者以拍卖、变卖该财产所得的价款优先受偿。
	
	(2)根据《民法典合同编通则解释》第28条的规定,债务人或者第 三人与债权人在债务 履行期限届满前达成以物抵债协议的,人民法院应当在审理债权债务关系的基础上认定该协议 的效力。当事人约定债务人到期没有清偿债务,债权人可以对抵债财产拍卖、变卖、折价以实 现债权的,人民法院应当认定该约定有效。当事人约定债务人到期没有清偿债务,抵债财产归 债权人所有的,人民法院应当认定该约定无效,但是不影响其他部分的效力;债权人请求对抵 债财产拍卖、变卖、折价以实现债权的,人民法院应子支持。当事人订立上述以物抵债协议后, 债务人或者第 三人未将财产权利转移至债权人名下,债权人主张优先受偿的,人民法院不予支 持;债务人或者第三人已将财产权利转移至债权人名下的,依据《民法典担保制度解释》第 68 条的规定处理。
	
	\newpage
	\section{合同法律制度}
	
	\subsection{通则}
	要约和承诺说明了订立合同前所需要进行的步骤。
	
	经过了要约和承诺后,合同就会成立,通常成立的时间和地点都取决于承诺生效的时间或地点。在具体的不同订立合同的方式下有着不同的生效时间和地点
	
	订立合同过程中,需要注意合同的格式条款以及免责条款,其中有些条款可能无效。此外在过程中,当事人有可能承担缔约过失责任
	
	合同生效有着特殊和普通的情况
	
	生效后就需要履行合同,履行时需要考虑约定不明确的情况。以及提前履行、部分履行、履行发生困难和情势变更的情况
	
	
	在通则中讲到了很多债权的内容,那是因为合同与债权有着紧密的关系
	
	\subsubsection{要约}
	当事人订立合同,可以采用\textbf{书面形式、口头形式}或者其他形式。
	
	\paragraph{要约及要约邀请}(2022年案例分析题) 要约是希望与他人订立合同的意思表示,要约可以向特定人发出,也可以向非特定人发出。 该意思表示应当符合下列条件:
	(1)内容具体确定;
	(2)表明经受要约人承诺,要约人即受该意思表示约束。
	
	要约邀请是希望他人向自己发出要约的表示。拍卖公告、招标公告、招股说明书、债券募集办法、基金招募说明书、商业广告和宣传、寄送的价目表等为要约邀请。值得注意的是,如果商业广告和宣传的内容符合要约条件的,如悬赏广告,构成\textbf{要约}。
	
	要约可能会有生效、撤回、撤销以及失效的情形
	
	\paragraph{要约的生效}
	根据不同方式作出的意思表示,生效时间不同
	\begin{enumerate}
		\item 以\textbf{对话方式}作出的意思表示,相对人\textbf{知道其内容时生效}。
		
		\item 以\textbf{非对话方式}作出的意思表示,\textbf{到达相对人时生效} 。以非对话方式作出的采用数据电 文形式的意思表示,相对人指定特定系统接收数据电文的,该数据电文进人该特定系统时生效; 未指定特定系统的,相对人知道或者应当知道该数据电文进入其系统时生效。当事人对采用数据电文形式的意思表示的生效时间另有约定的,按照其约定。
	\end{enumerate}
	
	
	\paragraph{要约的撤回}
	要约在\textbf{发出后、生效前},要约人可以\textbf{撤回要约}。但是在\textbf{时间上},撤回意思表示的通知早于应当在意思表示到达相对人前或者与意思表示同时到达相对人。
	
	\paragraph{要约的撤销}
	在要约\textbf{生效后、受要约人承诺前},要约人可以撤销要约,但是有下列情形之一的除外: 
	\begin{enumerate}
		\item \textbf{要约人}以确定承诺期限或者其他形式明示\textbf{要约不可撤销};
		
		\item \textbf{受要约人}有理由认为要约是不可撤销的,并已经\textbf{为履行合同做了合理准备工作}。
	\end{enumerate}
	
	【解释】撤销要约的意思表示以对话方式作出的,该意思表示的内容应当在受要约人 作出承诺之前为受要约人所知道 ;撤销要约的意思表示以非对话方式作出的,应当在受要 约人作出承诺之前到达受要约人。
	
	\paragraph{要约的失效}
	有下列情形之 一的,要约失效:
	\begin{enumerate}
		\item 要约被拒绝;
		
		\item 要约被依法撤销;
		
		\item 承诺期限届满,受要约人未作出承诺;
		
		\item 受要约人对要约的内容作出实质性变更。 
	\end{enumerate}
	
	【解释】有关合同标的、数量、质量、价款或者报酬、履行期限、履行地点和方式、 违约责任和解决争议方法等的变更,是对要约内容的实质性变更。
	
	\subsubsection{承诺}
	
	承诺是\textbf{受要约人同意要约的意思表示}。承诺应当在要约\textbf{确定的期限内到达要约人}。
	
	承诺存在期限,如无约定,以对话方式作出要约的应当即时作出承诺;以非对话方式作出要约的,承诺应当在合理期限内到达。
	 
	承诺期限的起算 
	\begin{enumerate}
		\item 要约以信件或者电报作出的,承诺期限自信件载明的日期或者电报交发之日开始计算。 信件未载明日期的,自投寄该信件的邮戳日期开始计算。
		
		\item 要约以电话、传真、电子邮件等快速通讯方式作出的,承诺期限自要约到达受要约人时开始计算 。
	\end{enumerate}
	
	承诺可能会有生效、撤回、迟延与迟到
	
	\paragraph{承诺的生效} 
	\textbf{承诺自通知到达要约人时生效}。承诺不需要通知的,根据交易习惯或者要约的要求作出承诺的行为时生效。此外\textbf{承诺生效时合同成立},但是法律另有规定或者当事人另有约定的除外。
	
	\paragraph{承诺的撤回} 承诺可以撤回,撤回承诺的通知应当在承诺通知到达相对人前或者与承诺通知同时到达相对人。(承诺不可以撤销,因为承诺生效时合同已经成立)
	
	
	\paragraph{承诺的迟延与迟到} 
	承诺的迟延是指\textbf{受要约人超过承诺期限发出承诺},或者在承诺期限内发出承诺,按照通常情形不能及时到 达要约人的,为新要约;但是,要约人及时通知受要约人该承诺有效的除外。
	
	承诺的迟到是指\textbf{受要约人在承诺期限内发出承诺},按照通常情形能够及时到达要约人,\textbf{但是因其他原因致使承诺到达要约人时超过承诺期限的},除要约人及时通知受要约人因承诺超过期限不接受该承诺外,\textbf{该承诺有效}。
	
	\paragraph{承诺的内容}
	内容上可以区分\textbf{是否作出实质性变更},如果作出实质性变更的,则为新要约。
	
	承诺对要约的内容作出\textbf{非实质性变更的},除要约人及时表示反对或者要约表明承诺不得对要约的内容作出任何变更外,该承诺有效,合同的内容以承诺的内容为准。
	
	\subsubsection{合同成立的时间和地点}
	\paragraph{合同成立的时间}
	一般来说,\textbf{承诺生效时合同成立},但是法律另有规定或者当事人另有约定的除外。(202 0年案例 分析题 )
	
	具体来说,但是人可能采用合同书形式、信件等形式来订立合同
	\begin{enumerate}
		\item 当事人采用合同书形式订立合同的,自当事人均签名、盖章或者按指印时合同成立。 在签名、盖章或者按指印之前,当事人 一方已经履行主要义务,对方接受时,该合同成立。(“对方是否接受” 是判断合同是否处于实际履行状态的关键。)(2022 年案例分析题 )
		
		\item 当事人采用信件、数据电文等形式订立合同要求签订确认书的,签订确认书时合同成立。
		
		\item 法律、行政法规规定或者当事人约定合同应当采用书面形式订立,当事人未采用书面 形式但是 一方已经履行主要义务,对方接受时,该合同成立。
	\end{enumerate}
	
	
	\paragraph{合同成立的地点}
	一般来说,\textbf{承诺生效的地点为合同成立的地点}。针对具体的订立合同方式有着不同的成立地点
	\begin{enumerate}
		\item 采用数据电文形式订立合同的 ,收件人的主营业地为合同成立的地点;没有主营业地 的,其住所地为合同成立的地点。当事人另有约定的,按照其约定。
		
		\item 当事人采用合同书形式订立合同的,最后签名、盖章或者按指印的地点为合同成立的 地点,但是当事人另有约定的除外。
	\end{enumerate}

	
	\subsubsection{格式条款与免责条款}
	\paragraph{格式条款}
	格式条款是当事人\textbf{为了重复使用而预先拟定,并在订立合同时未与对方协商的条款}。
	
	【解释】合 同 条 款 符 合 《民 法 典 》 第 4 9 6 条 规 定 的 情 形 ( 当事 人 为 了 重 复 使 用 而 预 先 拟定,并在订立合同时未与对方协商),当事人仅以合同系依据合同示范文本制作或者双 方已经明确约定合同条款不属于格式条款 由主张该条款不是格式条款的,人民法院不予 支 持 。( 2 0 2 4 年 新 增 )
	
	【解释】从事经营活动的当事人一方仅以未实际重复使用为由主张其预先拟定且未与 对方协商的合同条款不是格式条款的,人民法院不予支持。但是,有证据证明该条款不是 为 了 重 复 使 用 而 预 先 拟 定 的 除 外 。 (2 0 2 4 年 新 增 )
	
	
	采用格式条款订立合同的,提供格式条款的 一方应当遵循公平原则确定当事人之间的 权利和义务,并采取合理的方式提示对方注意免除或者减轻其责任等与对方有重大利害关系的 条款,按照对方的要求,对该条款予以说明。提供格式条款的 一方对已尽合理提示及说明义务 承担举证责任。
	
	对格式条款的理解发生争议的,应当按照通常理解予以解释。对格式条款有两种以上解释的,应当作出\textbf{不利于提供格式条款一方的解释}。格式条款和非格式条款不 一致的,应当采 用非格式条款。
	
	以下情况,格式条款无效
	\begin{enumerate}
		\item 提供格式条款的一方不合理地免除或者减轻其责任、加重对方责任、限制对方主要权利。 
		
		\item 提供格式条款的一方排除对方主要权利。 
		
		\item 格式条款具有《民法典》规定的无效情形(“无效的免责条款” 和第二章 “无效的民 事 法 律 行 为 ” )。
	\end{enumerate}
	
	
	\paragraph{免责条款}
	以下情况,免责条款无效
	\begin{enumerate}
		\item 造成对方人身损害的;
		
		\item 因故意或者重大过失造成对方财产损失的。
	\end{enumerate}
	
	\subsubsection{缔约过失责任}
	缔约过失责任是指当事人在\textbf{订立合同过程}中有下列情形之一,造成对方损失的,应当承担赔偿责任: 
	\begin{enumerate}
		\item 假借订立合同,恶意进行磋商;
		
		\item 故意隐瞒与订立合同有关的重要事实或者提供虚假情况; 
		
		\item 有其他违背诚信原则的行为。
	\end{enumerate}
	
	当事人在订立合同过程中知悉的商业秘密或者其他应当保密的信息,无论合同是否成 立,不得泄露或者不正当地使用;泄露、不正当地使用该商业秘密或者信息,造成对方损失的,应当承担赔偿责任 。 (2 0 2 2 年 案 例 分 析 题 )
	
	\textbf{缔约过失责任与违约责任}的区别如下 
	\begin{enumerate}
		\item 缔约过失责任适用于\textbf{合同未成立、合同未生效或者合同无效}等情形;违约责任适用于\textbf{生效合同}。 
		
		\item 缔约过失责任赔偿的是\textbf{信赖利益的损失};而违约责任赔偿的是\textbf{可期待利益的损失 }。可期待利益的损失要大于或者等于信赖利益的损失。
	\end{enumerate}
	
	
	\subsubsection{合同的生效}
	\paragraph{通常情况}
	合同生效有着不同的方式,最常见的就是诺成合同和实践合同以及附条件、附期限的合同
	\begin{enumerate}
		\item 诺成合同:保证合同、抵押合同、质押合同、赠与合同、金融机构贷款的借款合同等均属于诺成合同, \textbf{当事人意思表示一致时,合同成立}。依法成立的合同,原则上自成立时生效。
		
		\item 实践合同
		\begin{enumerate}
			\item 定金合同自实际交付定 金时成立。 
			
			\item 自然人之间的借款合同,自贷款人提供借款时成立。 
			
			\item 保管合同自保管物交付时成立,但是当事人另有约定的除外。
		\end{enumerate}
		
		\item 附条件、附期限的合同。附生效条件的合同,自条件成就时生效。附生效期限的合同,自期限届至时生效。
	\end{enumerate}
	
	\paragraph{特殊情况}
	部分情况下应当登记备案,但登记备案并非合同的生效要件 
	\begin{enumerate}
		\item 当事人之间订立有关设立、变更、转让和消灭不动产物权的合同,除法律另有规定或 者当事人另有约定外,自合同成立时生效;未办理物权登记的,不影响合同的效力。
		
		\item 当事人未依照法律、行政法规规定办理租赁合同登记备案手续的,不影响合同的效力。 
		
		\item 对自由进出又的技术,实行合同登记制度。但是,该技术进出又合同自依法成立时生 效,不以登记作为合同生效的条件。(第十二章)
		
	\end{enumerate}
	
	对限制进出又的技术,实行许可证管理,该技术进出由合同自许可证颁发之日起生效。 (第 十二 章 )
	
	法律、行政法规规定应当办理批准等手续生效的,在依照其规定办理批准等手续后生效。 
	\begin{enumerate}
		\item 合同依法成立后,负有报批义务的当事人不履行报批义务或者履行报批义务不符合合 同的约定或者法律、行政法规的规定,对方请求其继续履行报批义务的,人民法院应 予支持; 对方主张解除合同并请求其承担违反报批义务的赔偿责任的,人民法院应予支持。(2024年新增) 
		
		\item 人民法院判决当事人 一方履行报批义务后,其仍不履行,对方主张解除合同并参照违 反合同的违约责任请求其承担赔偿责任的,人民法院应予支持 。(2024年新增)
		
		\item 合同获得批准前,当事人一方起诉请求对方履行合同约定的主要义务,经释明后拒绝 变更诉讼请求的,人民法院应当判决驳回其诉讼请求,但是不影响其另行提起诉讼。(2024 年 新增 )
		
	\end{enumerate}
	
	\subsubsection{合同的履行规则}
	\paragraph{约定不明的处理} 
	合同生效后,当事人就质量、价款或者报酬、履行地点等内容没有约定或者约定不明确的, 可以协议补充;不能达成补充协议的,按照合同相关条款或者交易习惯确定。仍不能确定的, 适 用 下列 规 定 : 
	\begin{enumerate}
		\item 质量要求不明确的,按照强制性国家标准履行;没有强制性国家标准的,按照推荐性 国家标准履行;没有推荐性国家标准的,按照行业标准履行;没有国家标准、行业标准的,按 照通常标准或者符合合同目的的特定标准履行。
		
		\item 价款或者报酬不明确的,按照订立合同时履行地的市场价格履行;依法应当执行政府 定价或者政府指导价的,依照规定履行。
		
		\item 履行地点不明确,给付货币的,在接受货币一方所在地履行;交付不动产的,在不动 产所在地履行;其他标的,在履行义务一方所在地履行。
		
		\item 履行期限不明确的,债务人可以随时履行,债权人也可以随时请求履行,但是应当给 对方必要的准备时间。
		
		\item 履行方式不明确的,按照有利于实现合同目的的方式履行。 
		
		\item 履行费用的负担不明确的,由履行义务 一方负担;因债权人原因增加的履行费用,由 债权人负担。
	\end{enumerate}

	
	\paragraph{提前履行} 债权人可以拒绝债务人提前履行債务,但是提前履行不损害债权人利益的除外。债务人提 前履行债务给债权人增加的费用,由债务人负担。
	
	【相关链接】在借款合同中,借款人提前返还借款的,除当事人另有约定外,应当按 照实际借款的期间计算利息。
	
	\paragraph{部分履行}
	债权人可以拒绝债务人部分履行债务,但是部分履行不损害债权人利益的除外。债务人部 分履行债务给债权人增加的费用,由债务人负担。
	
	\paragraph{履行发生困难}债权人分立、合并或者变更住所没有通知债务人,致使履行债务发生困难的,债务人 可以中止履行或者将标的物提存。
	
	\paragraph{情势变更} 合同成立后,合同的基础条件发生了当事人在订立合同时无法预见的、不属于商业风险的重大变化,继续履行合同对于当事人 一方明显不公平的,受不利影响的当事人可以与对方重新 协商;在合理期限内协商不成的,\textbf{当事人可以请求人民法院或者仲裁机构变更或者解除合同}。
	
	【解释】(1 )合同的基础条件发生重大变化可能是因不可抗力造成的,也可能是因其他不可归责于双方当事人的事由造成的。例如,因政策调整或者市场供求关系异常变动等 原因导致价格发生当事人在订立合同时无法预见的、异常剧烈的涨跌。但是,合同涉及市 场属性活跃、长期以来价格波动较大的大宗商品以及股票、期货等风险投资型金融产品的除外。
	
	(2 )构成情势变更时,当事人负有重新协商的义务。
	
	(3 )当事人请求变更合同的, 人民法院不得解除合同;当事人一方请求变更合同,对方请求解除合同的,或者当事人一 方请求解除合同,对方请求变更合同的,人民法院应当结合案件的实际情况,根据公平原 则判决变更或者解除合同 。(2024年新增)
	
	
	
	\subsubsection{涉及第三人的合同}
	涉及第三人主要有向第三人履行、由第三人履行以及第三人单方自愿代为履行三种情况。这里可以结合最后合同的相对性进行考虑
	
	\paragraph{向第三人履行的合同}
	当事人约定\textbf{由债务人向第三人履行債务},债务人未向第 三人履行债务或者履行債务不 符合约定的,应当向债权人(而非第 三人)承担违约责任(合同的相对性)。
	
	法律规定或者当事人约定\textbf{第三人可以直接请求债务人向其履行债务},第 三人未在合理 期限内明确拒绝,债务人未向第三人履行债务或者履行债务不符合约定的,第三人可以请求债 务人承担违约责任。债务人对债权人的抗辩,可以向第 三人主张。(2022年案例分析题)
	
	【解释】该条款规定的是“ 利他合同”,\textbf{在利他合同中},虽然第三人并非合同的当事人, 但是\textbf{合同的效力可以拓展到第三人,第三人可以取得履行请求权}。利他合同的构成要件:
	\begin{enumerate}
		\item 必须约定由债务人向第三人履行债务;
		
		\item 根据法律规定或者当事人的约定,第三人可以直接请求债务人向其履行债务 。如果债务人未向第三人履行債务或者履行债务不符合约定的,第三人可以请求债务人承担违约责任。
	\end{enumerate} 
	
	\paragraph{由第三人履行的合同} 当事人约定由第三人向债权人履行債务,第三人不履行债务或者履行债务不符合约定的, 债务人(而非第三人)应当向债权人承担违约责任。(合同的相对性)
	
	\paragraph{第三人单方自愿代为履行} (2 0 2 3 年 案 例 分 析 题 ) 
	\textbf{债务人不履行债务,第三人对履行该债务具有合法利益的,第 三人有权向债权人代为 履行};但是,根据债务性质、按照当事人约定或者依照法律规定只能由债务人履行的除外。 
	
	债权人接受第 三人履行后,\textbf{其对债务人的债权转让给第三人,担保权利亦一同转让}, 但是债务人和第三人另有约定的除外。

	区分“由第三人履行的合同”和“第三人单方自愿代为履行”主要在于后者的前提是,债权人和债务人在合同中未 约定第三人具有履行义务。
	
	
	下列民事主体,人民法院可以认定为“对履行债务具有合法利益的第三人” ; 
	\begin{enumerate}
		\item 保证人或者提供物的担保的第三人;
		
		\item 担保财产的受让人、用益物权人、合法占有人; 
		
		\item 担保财产上的后顺位担保权人;
		
		\item 对债务人的财产享有合法权益且该权益将因财产 被强制执行而丧失的第三人;
		
		\item 债务人为法人或者非法人组织的,其出资人或者设立人; 
		
		\item 债务人为自然人的,其近亲属;
		
		\item 其他对履行债务具有合法利益的第三人(如承租 人 拖 欠 租 金 场 合 的 次 承 租 人 )。(2 0 2 4 年 新 增 )
	\end{enumerate}
	
	不具有合法利益的第三人包括
	\begin{enumerate}
		\item 债务人为法人或者非法人组织的,其 普通债权人;
		
		\item 债务人为自然人的,其同事、同学或者希望提供热心帮助的陌生人。 
	\end{enumerate}(2 02 4 年新增 )
	
	\subsubsection{按份之债与连带之债}
	\paragraph{按份之债}
	债权人为二人以上,标的可分,按照份额各自享有债权的,为\textbf{按份债权}。
	
	债务人为二人以上,标的可分,按照份额各自负担债务的,为\textbf{按份债务}。 
	
	按份债权人或者按份债务人的份额难以确定的 ,视为\textbf{份额相同}。
	
	\paragraph{连带之债}
	债权人为 二人以上,部分或者全部债权人均可以请求债务人履行债务的,为\textbf{连带债权}。 
	
	债务人为 二人以上,债权人可以请求部分或者全部债务人履行全部债务的,为\textbf{连带债务}。 
	
	连带债权或者连带债务,由法律规定或者当事人约定。 
	
	【相关链接】代理人和相对人恶意串通,损害被代理人合法权益的,代理人和相对人 应当承担连带责任。
	
	\paragraph{连带債务}
	连带债务人之间的\textbf{份额难以确定的,视为份额相同}。实际承担债务超过自己份额的连带债务人,有权就超出部分在其他连带债务人未履行的份额范围内向其追偿,并相应地享有债权人 的权利,但是不得损害债权人的利益。(承担超出可以向内追偿并享有权利)其他连带债务人对债权人的抗辩,可以向该债务人主张。 被追偿的连带债务人不能履行其应分担份额的,其他连带债务人应当在相应范围内按比例分担。(实在不能分担,其他人分担)
	
	【解释】追偿权和法定代位权是两个权利,两个权利可以 一并行使,但是追偿权人受偿的数额不得超出其份额。
	
	部分连带\textbf{债务人履行、抵销债务或者提存标的物的},其他债务人对债权人的债务\textbf{在相应范围内消灭};该债务人可以依据前述规定\textbf{向其他债务人追偿}。
	
	部分连带\textbf{债务人的债务被债权人免除的},在该连带债务人应当承担的份额范围内,\textbf{其他债务人对债权人的债务消灭}。
	
	部分连带债务人的债务与债权人的债权同归于 一人(\textbf{混同})的,在扣除该债务人应当承担的份额后,\textbf{债权人对其他债务人的债权继续存在}。
	
	债权人对部分连带债务人的给付\textbf{受领迟延的,对其他连带债务人发生效力}。
	
	\subsubsection{双务合同履行中的抗辩权}
	双务合同代表双方当事人在合同中具有互相给付义务,在这种情况下有着三种不同的抗辩权(前提是当事人基于同一双务合同)
	\begin{enumerate}
		\item 同时履行抗辩权: (2 0 1 6 年 案 例 分 析 题 ) 当事人互负债务,没有先后履行顺序的,应当同时履行。一方在对方履行之前有权拒绝其 履行请求。 一方在对方履行债务不符合约定时,有权拒绝其相应的履行请求。
		
		\item 先履行抗辩权:当事人互负债务,有先后履行顺序,应当先履行债务一方未履行的,后履行一方有权拒绝 其履行请求。先履行一方履行债务不符合约定的,后履行一方有权拒绝其相应的履行请求。
		
		\item 不安抗辩权(2017年案例分析题)在以下两种情况可以行使
		\begin{enumerate}
			\item \textbf{中止履行}。应当先履行债务的当事人,有确切证据证明对方有下列情形之 一的,可以中止履行: 1经营状况严重恶化;2转移财产、抽逃资金,以逃避债务;3丧失商业信誉;4有丧失或者可能丧失履行债务能力的其他情形。
			
			\item \textbf{解除合同}。当事人中止履行的,应当及时通知对方。对方提供适当担保的,应当恢复履行 。中止履行后,对方在合理期限内未恢复履行能力且未提供适当担保的,视为以自己的行为表明不履行主 要债务,中止履行的 一方可以解除合同并可以请求对方承担违约责任。
		\end{enumerate}
	\end{enumerate}
	
	
	
	\subsubsection{债权人代位权}
	如果债务人出现一些情况,债权人可能可以行使代位权和撤销权。
	
	\paragraph{代位权简介}代位权是指因\textbf{债务人怠于行使其债权或者与该债权有关的从权利},影响债权人的到期債权实现的,债权人可以向人民法院请求以自己的名义代位行使债务人对相对人的权利,但是该权利专属于债务人自身的除外(债权人的债权不受是否专属于债权人自身的限制)。
	
	债务人不履行其对债权人的到期债务,又不以诉讼或者仲裁方式向相对人主 张其享有的债权或者与该债权有关的从权利,致使债权人的到期债权未能实现的,人民法 院可以认定为“债务人急于行使其债权或者与该债权有关的从权利,影响债权人的到期债 权 实现 ” 。(2 0 2 4 年新 增 ) 
	
	下列权利,人民法院可以认定为“专属于债务人自身的权利” :
	\begin{enumerate}
		\item 抚养货、 赡养费或者扶养费请求权;
		
		\item 人身损害赔偿请求权;
		
		\item 劳动报酬请求权,但是超过债 务人及其所扶养家属的生活必需费用的部分除外;
		
		\item 请求支付基本养老保险金、失业保 险金、最低生活保障金等保障当事人基本生活的权利;
		
		\item 其他专属于债务人自身的权利。
	\end{enumerate} (2 0 2 4 年 调 整 )
	
	
	\paragraph{代位权行使的条件}代位权行使有着以下的条件
	\begin{enumerate}
		\item 债权人对债务人的债权合法;
		
		\item 债务人怠于行使其到期债权或者与该债权有关的从权利,影响债权人的到期债权实现; 
		
		\item 债务人的债权已到期;
		
		\item 债务人的债权不是专属 于债务人自身的债权。
	\end{enumerate}
	
	
	\paragraph{代位权诉讼中的主体和管辖}
	(1)债权人必须以自己的名义通过诉讼形式行使代位权。
	(2 )债权人以债务人的相对人(次债务人)为被告向人民法院提起代位权诉讼,未将债务 人列为第 三人的,人民法院应当追加债务人为第 三人。
	(3 )债权人依法对债务人的相对人提起代位权诉讼的,由被告住所地人民法院管辖,但是 依法应当适用专属管辖规定的除外 。(2024年调整)
	(4 )债权人行使代位权的必要费用,由债务人负担。
	
	
	【解释】在代位权诉讼中,如果债权人胜诉,由次债务人承担诉讼费用,且从实现的 债权中优先支付。代位权诉讼的其他必要费用则由债务人承担。
	4. 代位权的行使范围以债权人的到期债权为限。
	
	5. 债权人的债权到期前,债务人的债权或者与该债权有关的从权利存在诉讼时效期间即 将届满或者未及时申报破产债权等情形,影响债权人的债权实现的,债权人可以代位向债务人 的相对人请求其向债务人(而非债权人)履行、向破产管理人申报或者作出其他必要的行为。 
	
	6. 相对人对债务人的抗辩,可以向债权人主张。债权人提起代位权诉讼后,债务人无正 当理由减免相对人的债务或者延长相对人的履行期限,债务人及其相对人均不得以此对抗债权 人。(2024年新增)
	7. 人民法院认定代位权成立的,由债务人的相对人向债权人履行义务,债权人接受履行后, 债权人与债务人、債务人与相对人之间相应的权利义务终止。
	【相关链接】债权人提起代位权诉讼的,应当认定对债权人的债权和债务人的债权均 发生诉讼时效中断的效力。
	
	\subsubsection{债权人撤销权}
	在以下情况下,债权人可以撤销债务人的行为,且撤销权的行使范围以债权人的债权为限(可以是到期或未到期债权)
	\begin{enumerate}
		\item 债务人以放弃其债权(到期、未到期均可)、放弃债权担保、无偿转让财产等方式无 偿处分财产权益,或者恶意延长其到期债权的履行期限,影响债权人的债权实现的,债权人可 以请求人民法院撤销债务人的行为。
		
		\item 债务人以明显不合理的低价转让财产、以明显不合理的高价受让他人财产或者为他人 的债务提供担保,影响债权人的债权实现,债务人的相对人知道或者应当知道该情形的,债权 人可以请求人民法院撤销债务人的行为。
	\end{enumerate}
	
	【解释1】“明显不合理” 价格的判断标准,人民法院应当按照交易当地一般经营者的 判断,并参考交易时交易地的市场交易价或者物价部门指导价予以认定。低价不得低于70\%,高价不得高于130 \%。债务人与相对人存在亲属关系、关联关系的,不受上述规定的70\%、30\% 的 限 制 。 (2 0 2 4 年 调 整 )
	
	【解释2 】债务人以明显不合理的价格,实施互易财产、以物抵債、出租或者承租财产、 知识产权许可使用等行为,影响债权人的债权实现,债务人的相对人知道或者应当知道该 情形,债权人请求撤销债务人的行为的,人民法院应当予以支持。(2024年新增) 
	
	【解释3】(1)对于债务人的“无偿处分” 行为,不论相对人是否善意,均可撤销; (2)对于债务人的“有偿处分” 行 ,只有相对人恶意的,才可以撤销。
	
	\paragraph{撤销权的行使期限}撤销权自债权人知道或者应当知道撤销事由之日起1年内行使。自债务人的行为发生 之日起5年内没有行使撤销权的,该撤销权消灭。此处的“5年” 期间为除斥期间,不适用诉 讼时效中止、中断或者延长的规定。
	【解释】撤销权的消灭以1 年、5 年两个时间先到期的为准。
	
	\paragraph{撤销权诉讼中的主体和管辖}
	(1 )债权人必须以自己的名义通过诉讼形式行使撤销权。
	(2 )债权人依法提起撤销权诉讼的,应当以债务人和债务人的相对人为共同被告,由债务 人或者相对人的住所地人民法院管辖,但是依法应当适用专属管辖规定的除外。
	(3 )债权人行使撤销权的必要费用,包括合理的律师代理费、差旅费等费用,由债务人负担。 (4)债务人影响债权人的债权实现的行为被撤销的,自始没有法律约束力。撤销权行使的 目的是恢复债务人的责任财产,债权人就撤销权行使的结果并无优先受偿的权利。
	
	
	\subsubsection{债权转让}
	债权人可以将债权的全部或者部分转让给第三人,但是有下列情形之 一的除外: 
	\begin{enumerate}
		\item 根据债权性质不得转让;(出版合同中出版公司的债权、委托合同中委托人的债权,根据債权性质不得 转让)
		
		\item 按照当事人约定不得转让,这里还有个关于对抗第三人的问题。
		\begin{enumerate}
			\item 当事人约定非金钱债权不得转让的,不得对抗善意第三人。
			
			\item 当事人约定金钱债权不得转让的,不得对抗第三人。 
			
			【解释】金钱是普遍接受的交换媒介,禁止金钱债权转让的约定仅在当事人之间具有相对效力。金钱债权不得转让的约定,对于任何第 三人均不发生效力,无论第三人是否善意。
		\end{enumerate}
		
		\item 依照法律规定不得转让。
	\end{enumerate}
	
	债权转让不以债务人的同意为生效条件,但是要对债务人发生效力,则必须 通知债务人。(2016年案例分析题、2021年案例分析题) 
	
	债务人接到债权转让通知后,债权让与行为对债务人生效,债务人应当对受让人履行 义务。
	
	债务人在接到债权转让通知前已经向让与人履行,受让人请求债务人履行的, 人民法院不予支持;债务人接到债权转让通知后仍然向让与人履行,受让人请求债务人履 行 的 , 人 民 法 院 应 予 支 持 。(2 0 2 4 年 新 增 )
	
	因债权转让增加的履行费用,由让与人负担。
	
	债权人转让债权的,受让人取得与债权有关的从权利,但是该从权利专属于债权人自身的除外。受让人取得从权利不因该从权利未办理转移登记手续或者未转移占有而受到影响。
	
	债务人接到债权转让通知后,债务人对让与人的抗辩(如诉讼时效抗辩),可以向受让人主张。
	
	有下列情形之 一的,债务人可以向受让人主张抵销: 
	\begin{enumerate}
		\item 债务人接到债权转让通知时,债务人对让与人享有债权,且债务人的债权先于转让的 債权到期或者同时到期;
		
		\item 债务人的债权与转让的债权是基于同一合同产生。
	\end{enumerate}

	
	\paragraph{债权的多重让与}(2024年新增)
	(1)让与人将同 一债权转让给两个以上受让人,债务人以已经向最先通知的受让人履行 为由主张其不再履行债务的,人民法院应子支持。债务人明知接受履行的受让人不是最先通知 的受让人,最先通知的受让人请求债务人继续履行债务或者依据债权转让协议请求让与人承担 违约责任的,人民法院应予支持;最先通知的受让人请求接受履行的受让人返还其接受的财产 的,人民法院不予支持,但是接受履行的受让人明知该债权在其受让前已经转让给其他受让人 的除外。

	(2)所谓 “最先通知的受让人”,是指最先到达债务人的转让通知中载明的受让人。当事 人之间对通知到达时间有争议的,人民法院应当结合通知的方式等因素综合判断,而不能仅根 据债务人认可的通知时间或者通知记载的时间予以认定。当事人采用邮寄、通讯电子系统等方 式发出通知的,人民法院应当以邮戳时间或者通讯电子系统记载的时间等作为认定通知到达时 间的依据。
	
	\subsubsection{债务承担}
	债务承担中除了转移还有加入
	\begin{enumerate}
		\item 债务转移 
		\begin{enumerate}
			\item 债务人将债务的全部或者部分转移给第三人的,应当经债权人同意。
			
			\item 债务人或者第三人可以催告债权人在合理期限内 予以同意,债权人未作表示的,视为 不同意。
			
			\item 债务人转移债务的,新债务人可以主张原债务人对债权人的抗辩;原债务人对债权人 享有债权的,新债务人不得向债权人主张抵销。
			
			\item 债务人转移债务的,新债务人应当承担与主债务有关的从债务,但是该从债务专属于 原债务人自身的除外。
		\end{enumerate}
		
		\item 债务加人:第三人与债务人约定加入债务并通知债权人,或者第三人向债权人表示愿意加人债务,债 权人未在合理期限内明确拒绝的,债权人可以请求第 三人在其愿意承担的债务范围内和债务人 承担连带债务。
		
		所谓 “债务加入” ,是指原债务人并不脱离债权债务关系,第三人加入后,与 原债务人共同向债权人承担债务(原债务人和第三人均为债务人)。第三人的加入,不仅 对债权人没有风险,反而增加了债权实现的安全性。因此,在债务加入的情形下,无须征 得债权人的同意。第三人与债务人约定加入债务时,只需通知债权人,只要债权人未在合 理期限肉明确拒绝,債权人就可以请求第三人(在其愿意承担的债务范围内)和债务人承 担连带债务。
	\end{enumerate}
	
	
	\subsubsection{合同解除}
	\paragraph{合同解除的类型}
	合同解除的情形主要有协商解除、约定解除、法定解除与随时解除
	\begin{enumerate}
		\item 协商解除。当事人协商一致,可以解除合同。
		
		\item 约定解除。当事人可以约定 一方解除合同的事由。解除合同的事由发生时,解除权人可以解除合同。 
		
		\item 法定解除 (2 0 1 7 年 案 例 分 析 题 )
		有下列情形之一的,当事人可以解除合同:
		\begin{enumerate}
			\item 因不可抗力致使不能实现合同目的;
			
			\item 在履行期限届满前,当事人 一方明确表示或者以自己的行为表明不履行主要债务; 
			
			\item 当事人一方迟延履行主要債务,经催告后在合理期限内仍未履行;
			
			\item 当事人一方迟延履行债务或者有其他违约行为致使不能实现合同目的;
			
			\item 法律规定的其他情形。
		\end{enumerate}
		【相关链接】当事人行使不安抗辩权,中止履行后,对方在合理期限内未恢复履行能 力且未提供适当担保的,视为以自己的行为表明不履行主要债务,中止履行的一方可以解 除合同并可以请求对方承担违约责任。
		
		\item 随时解除。以下情况下可以随时解除
		\begin{enumerate}
			\item 以持续履行的债务为内容的不定期合同,当事人可以随时解除合同,但是应当在合理 期限之前通知对方。
			
			\item 定作人在承揽人完成工作前可以随时解除合同,造成承揽人损失的,应当赔偿损失。 
			
			\item 在承运人将货物交付收货人之前,托运人可以要求承运人中止运输、返还货物、变更 到达地或者将货物交给其他收货人,但是应当赔偿承运人因此受到的损失。 
			
			\item 委托人或者受托人可以随时解除委托合同 。因解除合同造成对方损失的,除不可归责 于该当事人的事由外,无偿委托合同的解除方应当赔偿因解除时间不当造成的直接损失,有偿 委托合同的解除方应当赔偿对方的直接损失和合同履行后可以获得的利益。
		\end{enumerate} 
	\end{enumerate}
	
	\paragraph{解除权的行使}
	法律\textbf{规定或者当事人约定解除权行使期限},\textbf{期限届满}当事人不行使的,该权利 消灭。
	
	法律\textbf{没有规定或者当事人没有约定解除权行使期}限,自解除权人\textbf{知道或者应当知道解除事由之日起 1 年内不行使},或者经对方催告后在合理期限内不行使的,该权利消灭。
	
	当 事 人 一方 依 法 主 张 解 除 合 同 的 , 应 当 通 知 对 方 。合 同 自 通 知 到 达 对 方 时 解 除 ; 通 知 载明债务人在一定期限内不履行债务则合同自动解除,债务人在该期限内未履行债务的,合同 自通知载明的期限届满时解除。
	
	对方对解除合同有异议的,任何 一方当事人均可以请求人民法院或者仲裁机构确认解 除行为的效力。
	当事人 一方未通知对方,直接以提起诉讼或者申请仲裁的方式依法主张解除合同,人 民法院或者仲裁机构确认该主张的,合同自起诉状副本或者仲裁申请书副本送达对方时解除。 
	
	\paragraph{合同解除的效力}
	(1 )合同解除后,尚未履行的,终止履行;已经履行的,根据履行情况和合同性质,当事 人可以请求恢复原状或者 取其他补救措施,并有权请求赔偿损失。
	(2 )合同因违约解除的,解除权人可以请求违约方承担违约责任,但是当事人另有约定的 除外。
	(3 )主合同解除后,担保人对债务人应当承担的民事责任仍应当承担担保责任,但是担保 合同另有约定的除外。
	
	\subsubsection{抵销}
	合同的抵销有两种情况,约定抵销和法定抵销
	\begin{enumerate}
		\item 约定抵销 当事人互负债务,标的物种类、品质不相同的,经协商一致,可以抵销。
		【解释】合同标的物 一般分为四类:(1)有形財产;(2)无形财产;(3)劳务;(4)工 作成果(如建设工程合同中由承包人完成的建设项目)。
		
		\item 法定抵销 
		\begin{enumerate}
			\item 当事人互负债务,该债务的标的物种类、品质相同的,任何一方可以将自己的债务 与对方的到期债务抵销;但是,根据债务性质、按照当事人约定或者依照法律规定不得抵销的 除外。
			
			\item 提供劳务的债务 、不作为的债务,根据债务性质不得抵销。
			
			\item 因侵害自然人人身权益,或者故意、重大过失侵害他人财产权益产生的损害赔偿債务, 侵 权 人 主 张 抵 销 的 , 人 民 法 院 不 予 支 持 。 (2 0 2 4 年 调 整 )
			
			\item 法定抵销必须以对方的债务已届清偿期为前提。在 一方当事人主张抵销的情形下,并 不要求双方当事人的债务均届清偿期。若一项债务已届清偿期,而另一项债务未届清偿期,则 未到期的债务人可以主张抵销。因为期限利益原则上属于债务人。
			
			\item 法定抵销中的抵销权在性质上属于形成权,当事人主张抵销的,应当通知对方。通知 为非要式(可以是又头形式),通知自到达对方时生效 。抵销不得附条件或者附期限。
			
			\item 当事人一方依法主张抵销,人民法院经审理认为抵销权成立的,应当认定通知到达对 方时双方互负的主债务、利息、违约金或者损害赔偿金等债务在同等数额内消灭。换言之,法 定 抵 销 不 具 有 溯 及 既 往 的 效 力 。(2 0 2 4 年 调 整 )
		\end{enumerate}
	\end{enumerate}
	1. 
	
	
	
	\subsubsection{提存}
	有下列情形之一,难以履行债务的,债务人可以将标的物提存:
	\begin{enumerate}
		\item 债权人无正当理由拒绝受领;
		
		\item 债 权 人 下落 不 明 ;
		
		\item 债权人死亡未确定继承人、遗产管理人,或者丧失民事行为能力未确定监护人; 
		
		\item 法律规定的其他情形。
	\end{enumerate}
	【解释】标的物不适于提存或者提存费用过高的,债务人依法可以拍卖或者变卖标的物, 提存所得的价款。
	【相关链接】债权人分立、合并或者变更住所没有通知债务人,致使履行债务发生困 难的,债务人可以中止履行或者将标的物提存。
	
	提存成立的,视为债务人在其提存范围内已经交付标的物。
	
	标的物提存后,债务人应当及时通知债权人或者债权人的继承人、遗产管理人、监护人、 财产代管人。
	
	提存的法律效力
	(1)标的物提存后,毁损、灭失的风险由债权人承担。
	(2 )提存期间,标的物的孳息归债权人所有。
	(3)提存费用由债权人负担。
	
	债权人可以随时领取提存物。但是,债权人对债务人负有到期债务的,在债权人未履 行债务或者提供担保之前,提存部门根据债务人的要求应当拒绝其领取提存物。
	
	债权人领取提存物的权利,自提存之日起5年内不行使而消灭,提存物扣除提存费用后归国家所有。但是,债权人未履行对债务人的到期债务,或者债权人向提存部门书面表示放弃领取提存物权利的,债务人负担提存费用后有权取回提存物。 【相关链接】遗失物自发布招领公告之日起 1 年内无人认领的,归国家所有。
	
	\subsubsection{债权债务的终止}
	有下列情形之一的,债权债务终止:
	\begin{enumerate}
		\item 债 务已经履行 ;
		
		\item 债 务 相 互 抵 销 ;
		
		\item 债务人依法将标的物提存;
		
		\item 债权人免除债务;
		
		\item 债权债务同归 于一人(但是损害第 三人利益的除外);
		
		\item 法律规定或者当事人约定终止的其他情形。
	\end{enumerate}
	
	债权人免除债务人部分或者全部债务的,债权债务部分或者全部终止,但是债务人在 合理期限内拒绝的除外。
	
	债务人在履行主债务外还应当支付利息和实现债权的有关费用,其给付不足以清偿全 部 债 务 的 , 除 当 事 人 另 有 约 定 外 , 应 当 按 照 下 列 顺 序 履 行 :( 1 ) 实 现 债 权 的 有 关 费 用 ;( 2 ) 利 息 ; (3 )主债务。
	
	\subsubsection{违约责任}
	
	\paragraph{违约的种类}
	\begin{enumerate}
		\item 预期违约 当事人一方明确表示或者以自己的行为表明不履行合同义务的,对方可以在履行期限届满 前请求其承担违约责任。
		【相关链接】在履行期限届满前,当事人 一方明确表示或者以自己的行为表明不履行 主要债务,对方可以解除合同。
		
		\item 届期违约 (1)当事人一方不履行合同义务或者履行合同义务不符合约定的,应当承担继续履行、采 取补救措施或者赔偿损失等违约责任。
		(2 )因当事人 一方的违约行为,损害对方人身权益、财产权益的,受损害方有权选择请求 其承担违约责任或者侵权责任。
		
	\end{enumerate}
	
	
	\paragraph{债务的种类}
	
	\begin{enumerate}
		\item 金钱债务。当事人 一方未支付价款、报酬、租金、利息,或者不履行其他金钱债务的,对方可以请求其支付。 
		
		\item 非金钱债务。当事人一方不履行非金钱债务或者履行非金钱债务不符合约定的,对方可以请求履行,但 是 有 下 列 情 形 之 一的 除 外 :
		(1 )法律上或者事实 上不能履行;
		(2 )债务的标的不适于强制履行或者履行费用过高; (3)债权人在合理期限内未请求履行。
		【解释】非金钱债务存在上述除外情形之一,致使不能实现合同目的的,人民法院或者 仲裁机构可以根据当事人的请求终止合同权利义务关系,但是不影响违约责任的承担。
	\end{enumerate}
	
	\paragraph{采取补救措施}(2017年案例分析题) 履行不符合约定的,应当按照当事人的约定承担违约责任。对违约责任没有约定或者约定 不明确,可以协议补充;不能达成补充协议的,按照合同相关条款或者交易习惯确定;仍不能 确定的,受损害方根据标的的性质以及损失的大小,可以合理选择请求对方承担修理、重作、 更换、退货、减少价款或者报酬等违约责任。
	
	\paragraph{赔偿损失} 
	(1)当事人一方不履行合同义务或者履行合同义务不符合约定的,在履行义务或者采取补 救措施后,对方还有其他损失的,应当赔偿损失。(2016 年案例分析题、2022 年案例分析题) 
	
	(2 )当事人一方不履行合同义务或者履行合同义务不符合约定,造成对方损失的,损失赔 偿额应当相当于因违约所造成的损失,包括合同履行后可以获得的利益;但是,不得超过违约 一方订立合同时预见到或者应当预见到的因违约可能造成的损失。
	【解释1】人民法院确定“合同履行后可以获得的利益” 时,可以在扣除非违约方为订 立、履行合同支出的费用等合理成本后,按照非违约方能够获得的生产利润、经营利润或 者转 售 利 润 等 计 算 。(2 0 2 4 年新 增 )
	【解释2 】非违约方依法行使合同解除权并实施了替代交易,主张按照替代交易价格与 合同价格的差额确定合同履行后可以获得的利益的,人民法院依法子以支持;替代交易价 格明显偏离替代交易发生时当地的市场价格,违约方主张按照市场价格与合同价格的差额 确定合同履行后可以获得的利益的,人民法院应予支持。(2024 年新增)
	【解释3 】非违约方依法行使合同解除权但是未实施替代交易,主张按照违约行为发 生后合理期间内合同履行地的市场价格与合同价格的差额确定合同履行后可以获得的利益 的,人民法院应予支持。(2024 年新增)
	【解释4 】在以持续履行的债务为内容的定期合同中,一方不履行支付价款、租金等金 钱债务,对方请求解除合同,人民法院经审理认为合同应当依法解除的,可以根据当事人 的主张,参考合同主体、交易类型、市场价格变化、剩余履行期限等因素确定非违约方寻 找替代交易的合理期限,并按照该期限对应的价款、租金等扣除非违约方应当支付的相应 履约成本确定合同履行后可以获得的利益。非违约方主张按照合同解除后剩余履行期限相 应的价款、租金等扣除履约成本确定合同履行后可以获得的利益的,人民法院不予支持。 但是,剩余履行期限少于寻找替代交易的合理期限的除外。(2024 年新增)
	
	(3)当事人一方违约后,对方应当采取适当措施防止损失的扩大;没有采取适当措施致使 损失扩大的,不得就扩大的损失请求赔偿。当事人因防止损失扩大而支出的合理费用,由违约 方负担。
	
	(4 )当事人 一方违约造成对方损失,对方对损失的发生有过错的,可以减少相应的损失赔 偿额。
	【案例】张某与甲公司订立汽车买卖合同,后张某驾驶该汽车超载行驶,由于汽车钢 圈(质量不合格)破损导致翻车,给张某造成损失。人民法院经审理认为,甲公司(产品 质量不合格)应承担主要责任,张某(超载)也应承担一定责任。
	
	\paragraph{支付违约金}
	(1 )约定的违约金低 于造成的损失的,人民法院或者仲裁机构可以根据当事人的请求予以 增加;约定的违约金过分高于造成的损失的,人民法院或者仲裁机构可以根据当事人的请求予 以 适 当 减 少 。 (2 0 2 2 年 案 例 分 析 题 、 2 0 2 3 年 案 例 分 析 题 )
	
	 (2)当事人就迟延履行约定违约金的,违约方支付违约金后,还应当履行债务。(2022年 案例分析题 )
	
	\paragraph{定金}
	当 事 人 既 约 定 违 约 金 , 又 约 定 定 金 的 , 一方 违 约 时 , 对 方 可 以 选 择 适 用 违 约 金 或 者 定 金条款。(2020 年案例分析题)
	
	定金不足以弥补一方违约造成的损失的,对方可以请求赔偿超过定金数额的损失。 (2 0 1 7 年 案 例 分 析 题 )
	
	
	\paragraph{不可抗力} 
	(1)当事人一方因不可抗力不能履行合同的,根据不可抗力的影响,部分或者全部免除责 任 , 但 是 法 律 另 有 规 定 的 除 外 。 (2 0 2 2 年 案 例 分 析 题 )
	(2 )当事人迟延履行后发生不可抗力的,不免除其违约责任。
	【解释】《民法典》规定的违约损害赔偿法定的免责事由仅限于不可抗力。不可抗力是 指不能预见、不能避免且不能克服的客观情况。 常见的不可抗力有:
	\begin{enumerate}
		\item 自然灾害 (如地 震、台风、洪水、海啸等):
		
		\item 政府行为(如运输合同订立后,由于政府颁布禁运的法律, 致使合同不能履行)
		
		\item 社会异常现象(如罢工骚乱等)。
		
	\end{enumerate}
	
	\subsubsection{合同的相对性}
	由于合同关系具有相对性,因此,违约责任也具有相对性,即违约责任只能在特定的 具有合同关系的当事人之间发生。(2023 年案例分析题)
	
	当事人一方因第 三人的原因造成违约的,应当依法向对方承担违约责任。当事人一方 和第 三人之间的纠 ,依照法律规定或者按照约定处理。(2022年案例分析题)
	
	基于合同的相对性可以分析两个具体的例子
	\begin{enumerate}
		\item 转租。承租人经出租人同意,可以将租赁物转租给第三人。承租人转租的,承租人与出租人之间 的租赁合同继续有效;第 三人造成租赁物损失的,承租人应当赔偿损失。
		
		\item 承揽合同。承揽人将其承揽的主要工作交由第 三人完成的,应当就该第 三人完成的工作成果向定作人 负责;未经定作人同意的,定作人也可以解除合同。
	\end{enumerate}
	
	合同相对性有着如下例外
	\begin{enumerate}
		\item 在代位权中,突破了合同的相对性,使得债权人可以向合同关系以外的第 三人提起诉 讼,主张权利。 
		
		\item 租赁物在承租人按照租赁合同占有期限内发生所有权变动的,不影响租赁合同的 效力。即租赁合同在有效期限内对承租人和新的所有权人继续有效,突破了合同关系的相 对性。 
		
		\item 总承包人或者勘察、设计、施工承包人经发包人同意,可以将自己承包的部分工作交 由第 三人完成。第 三人就其完成的 工作成果与总承包人或者勘察、设计、施 工承包人向发包人 承担连带责任。
		
		\item 法律规定或者 当事人约定第 三人可以直接请求债务人向其履行债务,第 三人未在合理 期限内明确拒绝,债务人未向第三人履行债务或者履行债务不符合约定的,第 三人可以请求债 务人承担违约责任,突破 了合同关系的相对性。
	\end{enumerate} 
	
	\subsection{合同的担保}
	这块主要分两部分介绍,一部分是介绍保证相关的内容,另一部分介绍定金
	
	\subsubsection{保证合同}
	\paragraph{保证合同}
	保证合同为单务合同、无偿合同、诺成合同。 保证合同可以是单独订立的书面合同,也可以是主债权债务合同中的保证条款。 
	
	第三人单方以书面形式向债权人作出保证,债权人接收且未提出异议的,保证合同成 立。(2014年案例分析题、2022 年案例分析题)
	
	(根据承诺来认定是保证还是债务加入,不能确定就按保证来处理)第三人向债权人提供差额补足、流动性支持等类似承诺文件作为增信措施,具有提供担保的意思表示,债权人请求第三人承担保证责任的,人民法院应当依照保证的有关规定处理。 第三人向债权人提供的承诺文件,具有加入债务或者与债务人共同承担债务等意思表示的,人民法院应当认定为债务加入。第 三人提供的承诺文件难以确定是保证还是债务加入的,人民法 院应当将其认定为保证。
	
	\paragraph{保证人}
	机关法人不得为保证人,但是经国务院批准为使用外国政府或者国际经济组织贷款进 行转贷的除外 。
	
	以公益为目的的非营利法人、非法人组织不得为保证人。
	
	\subsubsection{保证方式}
	当事人在保证合同中对保证方式没有约定或者约定不明确的,\textbf{按照一般保证承担保证责任} 。 (2 0 2 1 年 案 例 分 析 题 、 2 0 2 2 年 案 例 分 析 题 、 2 0 2 3 年 案 例 分 析 题 )
	
	\paragraph{一般保证} 当事人在保证合同中约定,债务人不能履行债务时,由保证人承担保证责任的,为 一般保证。 一般保证人享有先诉抗辩权,即在主合同纠纷未经审判或者仲裁,并就债务人财产依法强 制执行仍不能履行债务前,有权拒绝向债权人承担保证责任。(2021 年案例分析题、2023 年 案例分析题 )
	
	有下列情形之一的,一般保证人不得行使先诉抗辩权 :
	\begin{enumerate}
		\item 债务人下落不明 , 且无财产可供执行;
		
		\item 人民法院已经受理债务人破产案件;
		
		\item 债权人有证据证明债务人的财产不足以履行全部债务或者丧失履行债务能力;
		
		\item 保证人书面表示放弃先诉抗辩权。
	\end{enumerate}
	
	\paragraph{连带责任保证} 当事人在保证合同中约定保证人和债务人对债务承担连带责任的,为连带责任保证。连带 责任保证的债务人不履行到期债务或者发生当事人约定的情形时,债权人可以请求债务人履行 债务,也可以请求保证人在其保证范围内承担保证责任。
	
	\subsubsection{保证期间}
	债权人与保证人可以约定保证期间,但是约定的保证期间早于主债务履行期限或者与 主债务履行期限同时届满的,视为没有约定;没有约定或者约定不明确的,保证期间为主债务 履行期限届满之日起6 个月。 
	【解释1】保证合同约定“保证人承担保证责任直至主债务本息还清时为止” 等类似内 容的,视为约定不明,保证期间为主债务履行期限届满之日起6 个月。
	【解释2 】债权人与债务人对主债务履行期限没有约定或者约定不明确的,保证期间自 债权人请求债务人履行债务的宽限期届满之日起计算。
	
	债权人在保证期间内未依法行使权利的 ,保证责任消灭。保证责任消灭后,债权人书 面通知保证人要求承担保证责任,保证人在通知书上签字、盖章或者按指印,债权人请求保证 人继续承担保证责任的,人民法院不予支持,但是债权人有证据证明成立了新的保证合同的 除外。
	
	\paragraph{连带责任保证} (1)连带责任保证的债权人未在保证期间请求保证人承担保证责任的,保证人不再承担保 证责任。 (2)连带责任保证的债权人在保证期间届满前请求保证人承担保证责任的,从债权人请求 保证人承担保证责任之日起,开始计算保证债务的诉讼时效。
	
	\paragraph{一般保证} (1)一般保证的债权人未在保证期间对债务人提起诉讼或者申请仲裁的,保证人不再承担 保证责任。
	(2) 一般保证的债权人在保证期间届满前对债务人提起诉讼或者申请仲裁的,从保证人拒 绝承担保证责任的权利消灭之日起,开始计算保证债务的诉讼时效。
	
	
	\subsubsection{保证责任}
	\paragraph{保证的范围} 保证的范围包括主债权及其利息、违约金、损害赔偿金和实现债权的费用。当事人另有约 定的,按照其约定。
	
	保证人承担保证责任后,除当事人另有约定外,有权在其承担保证责任的范围内向债 务人追偿,享有债权人对债务人的权利,但是不得损害债权人的利益。
	
	保证人知道或者应当知道主债权诉讼时效期间届满仍然提供保证或者承担保证责任, 又以诉讼时效期间届满为由拒绝承担保证责任或者请求返还财产的,人民法院不予支持;保证 人承担保证责任后向债务人追偿的,人民法院不予支持,但是债务人放弃诉讼时效抗辩的除外。
	
	\paragraph{债权转让} (1)债权人转让全部或者部分债权,未通知保证人的,该转让对保证人不发生效力。
	(2 )保证人与债权人约定禁止债权转让,债权人未经保证人书面同意转让债权的,保证人 对受让人不再承担保证责任。(2021 年案例分析题)
	
	\paragraph{债务转移} 债权人未经保证人书面同意,允许债务人转移全部或者部分债务,保证人对未经其同意转 移的债务不再承担保证责任,但是债权人和保证人另有约定的除外。
	
	\paragraph{债务加入}
	第 三人加入债务的,保证人的保证责任不受影响。
	
	\paragraph{主合同的变更}(2017年案例分析题)
	(1 )债权人和债务人未经保证人书面同意,协商变更主债权债务合同内容,减轻债务的, 保证人仍对变更后的债务承担保证责任;加重债务的,保证人对加重的部分不承担保证责任。 (2 )债权人和债务人变更主债权债务合同的履行期限,未经保证人书面同意的,保证期间 不受影 响。
	【解释】未经保证人书面同意的,避重就轻,不加重保证人的责任:100 万元变更为 8 0万元的,按照80 万元承担保证责任;100 万元变更为180 万元的,按照100 万元承担 保证责任。
	
	\paragraph{共同保证} 同一债务有两个以上保证人的,保证人应当按照保证合同约定的保证份额,承担保证责任; 没有约定保证份额的,债权人可以请求任何 一个保证人在其保证范围内承担保证责任。
	\subsubsection{定金}
	\paragraph{共同保证}
	(1)定金合同自实际交付定金时成立。
	(2 )定金的数额由当事人约定;但是,不得超过主合同标的额的20\%,超过部分不产生 定 金 的 效 力 。 (2 0 1 7 年 案 例 分 析 题 、 2 0 2 2 年 案 例 分 析 题 )
	
	
	\paragraph{定金罚则}
	(1 )给付定金的一方不履行债务或者履行债务不符合约定,致使不能实现合同目的的,无 权请求返还定金;收受定金的一方不履行债务或者履行债务不符合约定,致使不能实现合同目 的的,应当双倍返还定金。
	【解释】在迟延履行或者有其他违约行为时,并不能当然适用定金罚则 。只有因当事 人一方迟延履行或者其他违约行为,致使合同目的不能实现,才可以适用定金罚则。法律 另有规定或者当事人另有约定的除外。
	(2 )因不可抗力致使主合同不能履行的,不适用定金罚则。
	(3 )因合同关系以外的第 三人的过错,致使主合同不能履行时,适用定金罚则。受定金处 罚的一方当事人,可以依法向第 三人追偿。
	【相关链接】当事人 一方因第三人的原因造成违约的,应当依法向对方承担违约责任。 当事人一方和第三人之间的纠纷,依照法律规定或者按照约定处理。
	(4 )当事人一方不完全履行合同的,应当按照未履行部分所占合同约定内容的比例,适用 定金罚则。
	(5 )当事人既约定违约金,又约定定金的, 一方违约时,对方可以选择适用违约金或者定 金条款。
	(6 )定金不足以弥补一方违约造成的损失的,对方可以请求赔偿超过定金数额的损失。
	(7)双方当事人均具有致使不能实现合同目的的违约行为,其中 一方请求适用定金罚则的, 人民法院不予支持。当事人一方仅有轻微违约,对方具有致使不能实现合同目的的违约行为, 轻微违约方主张适用定金罚则,对方以轻微违约方也构成违约为由抗辩的,人民法院对该抗辩
	不予支持。(2024年新增)
	(8 )当事人一方已经部分履行合同,对方接受并主张按照未履行部分所占比例适用定金罚
	则的,人民法院应 予支持。对方主张按照合同整体适用定金罚则的,人民法院不予支持,但是 部分未履行致使不能实现合同目的的除外。(2024年新增)
	
	\paragraph{未交付定金的情况}当事人约定以交付定金作为合同成立或者生效条件,应当交付定金的一方未交付定金, 但是合同主要义务已经履行完毕并为对方所接受的,人民法院应当认定合同在对方接受履行时 已经成立或者生效。(2024 年新增)
	
	\subsection{典型合同}
	合同中需要重点关注买卖合同、商品房买卖合同、租赁合同、融资租赁合同、借款合同、民间借贷合同、建设工程合同
	
	\subsubsection{买卖合同}
	\paragraph{标的物的风险负担}
	(1 )标的物毁损、灭失的风险,在标的物交付之前由出卖人承担,交付之后由买受人
	承担,但是法律另有规定或者当事人另有约定的除外。(2017 年案例分析题、2022 年案例 分析题 )
	(2 )因标的物不符合质量要求,致使不能实现合同目的的,买受人可以拒绝接受标的物或 者解除合同。买受人拒绝接受标的物或者解除合同的,标的物毁损、灭失的风险由出卖 人承担。
	(3 )因买受人的原因致使标的物未按照约定的期限交付的,买受人应当自违反约定时起承 担标的物毁损、灭失的风险。
	(4 )出卖人按照约定或者依照法律规定将标的物置于交付地点,买受人违反约定没有收取 的,标的物毁损、灭失的风险自违反约定时起由买受 人承担。
	(5 )出卖人出卖交由承运人运输的在途标的物,除当事人另有约定外,毁损、灭失的风险 自合同成立时起由买受人承担。但是,如果出卖人出卖交由承运人运输的在途标的物,在合同 成立时知道或者应当知道标的物已经毁损、灭失却没有告知买受人,买受人主张出卖人负担标
	的物毁损、灭失风险的,人民法院应予支持。
	 (6)当事人没有约定交付地点或者约定不明确的 ,可以协议补充;不能达成补充协议的,
	按照合同相关条款或者交易习惯确定;仍不能确定,标的物需要运输的,出卖人将标的物交付 给 第 一 承 运 人 后 , 标 的 物 毁 损 、 灭 失 的 风 险 由 买 受 人 承 担 。 (2 0 2 0 年 案 例 分 析 题 )
	
	(7 )出卖人按照约定未交付有关标的物的单证和资料的,不影响标的物毁损、灭失风险的转移。 
	(8 )标的物毁损、灭失的风险由买受人承担的,不影响因出卖人履行义务不符合约定,买 受人请求其承担违约责任的权利。
	【相关链接】出卖人依法将标的物提存后,毁损、灭失的风险由买受人承担。
	
	\paragraph{标的物的检验}
	(1 )约定检验期限的 当事人约定检验期限的,买受人应当在检验期限内将标的物的数量或者质量不符合约定的 情形通知出卖人。买受人怠于通知的,视为标的物的数量或者质量符合约定。(2015年案例分 析 题 、2 0 1 7 年 案 例 分 析 题 )
	(2 )未约定检验期限的 当事人没有约定检验期限的,买受人应当在发现或者应当发现标的物的数量或者质量不符 合约定的合理期限内通知出卖人。买受人在合理期限内未通知或者自收到标的物之日起2 年内 未通知出卖人的,视为标的物的数量或者质量符合约定;但是,对标的物有质量保证期的,适 用 质 量 保 证 期 , 不 适 用 该 2 年 的 规 定 。 (2 0 2 0 年 案 例 分 析 题 )
	
	
	\paragraph{买卖合同的特别解除规则} 
	(1)因标的物的主物不符合约定而解除合同的,解除合同的效力及于从物。因标的物的从 物不符合约定被解除的,解除的效力不及于主物。
	(2)标的物为数物,其中 一物不符合约定的,买受人可以就该物解除。但是,该物与他物 分离使标的物的价值显受损害的,买受人可以就数物解除合同。(2020 年案例分析题)
	(3 )出卖人分批交付标的物的,出卖人对其中一批标的物不交付或者交付不符合约定,致 使该批标的物不能实现合同目的的,买受人可以就该批标的物解除。出卖人不交付其中 一批标 的物或者交付不符合约定,致使之后其他各批标的物的交付不能实现合同目的的,买受人可以 就该批以及之后其他各批标的物解除。买受人如果就其中一批标的物解除,该批标的物与其他 各批标的物相互依存的,可以就 已经交付和未交付的各批标的物解除。 (4)分期付款的买受人未支付到期价款的数额达到全部价款的20\%,经催告后在合理期 限内仍未支付到期价款的,出卖人可以请求买受人支付全部价款或者解除合同。出卖人解除合 同的,可以向买受人请求支付该标的物的使用费。(2014 年案例分析题)
	【解释】分期付款要求买受人将总价款在 一定期限内至少分3 次向出卖人支付。
	
	\paragraph{试用买卖合同} (1)对试用期限没有约定或者约定不明确,依照法律有关规定仍不能确定的,由出卖人确定。 (2 )试用期限届满,买受人对是否购买标的物未作表示的,视为购买。 (3)试用买卖的买受人在试用期内已经支付部分价款或者对标的物实施出卖、出租、设立 担保物权等行为的,视为同意购买。
	
	\paragraph{凭样品买卖合同}
	凭样品买卖的当事人应当封存样品,并可以对样品质量予以说明。
	
	\subsubsection{商品房买卖合同}
	
	\paragraph{商品房销售广告的性质}(2022年案例分析题、2023年案例分析题) 商品房的销售广告和宣传资料为要约邀请,但是出卖人就商品房开发规划范围内的房屋及相关设施所作的说明和允诺具体确定,并对商品房买卖合同的订立以及房屋价格的确定有重大 影响的,构成要约。该说明和允诺即使未载人商品房买卖合同,亦应当为合同内容,当事人违 反的,应当承担违约责任。
	
	\paragraph{商品房预售合同}(2022年案例分析题、2023年案例分析题) 出卖人未取得商品房预售许可证明,与买受人订立的商品房预售合同,应当认定无效,但 是在起诉前取得商品房预售许可证明的,可以认定有效。
	
	\paragraph{解除权的行使} (1)因房屋主体结构质量不合格不能交付使用,或者房屋交付使用后,房屋主体结构质量 经核验确属不合格,买受人请求解除合同和赔偿损失的,人民法院应予支持。
	(2 )因房屋质量问题严重影响正常居住使用,买受人请求解除合同和赔偿损失的,人民法 院 应 予 支持 。
	(3)出卖人迟延交付房屋或者买受人迟延支付购房款,经催告后在3 个月的合理期限内仍 未履行,解除权人请求解除合同的,人民法院应予支持,但当事人另有约定的除外。
	
	\paragraph{贷款合同的效力}
	(1)商品房买卖合同约定,买受人以担保贷款方式付款,因当事人 一方原因未能订立商品 房担保贷款合同并导致商品房买卖合同不能继续履行的,对方当事人可以请求解除合同和赔偿 损失。因不可归责于当事人双方的事由未能订立商品房担保贷款合同并导致商品房买卖合同不 能继续履行的,当事人可以请求解除合同,出卖人应当将收受的购房款本金及其利息或者定金 返还买受人。 (2)商品房买卖合同被确认无效或者被撤销、解除后,商品房担保贷款合同也被解除的, 出卖人应当将收受的购房贷款和购房款的本金及利息分别返还担保权人和买受人。
	
	\subsubsection{租赁合同}
	
	\paragraph{租赁期限}
	(1)租货期限不得超过20年。超过20年的,超过部分无效。 (2)租赁期限届满,当事人可以续订租赁合同;但是,约定的租赁期限自续订之日起不得 超过 20 年 。
	
	\paragraph{登记备案} 当事人未依照法律、行政法规规定办理租赁合同登记备案手续的,不影响合同的效力。
	
	\paragraph{不定期租赁}(2013年案例分析题、2019年案例分析题、2023年案例分析题) (1)租赁期限6 个月以上的,应当采用书面形式。当事人未采用书面形式,无法确定租赁 期限的,视为不定期租赁。 (2)当事人对租赁期限没有约定或者约定不明确的,可以协议补充;不能达成补充协议的, 按照合同相关条款或者交易习惯确定;仍不能确定的,视为不定期租赁。 (3)租赁期限届满,承租人继续使用租赁物,出租人没有提出异议的,原租赁合同继续有 效,但是租赁期限为不定期。
	
	【解释】对于不定期租赁,当事人可以随时解除合同,但是应当在合理期限之前通知对方。
	
	租赁物危及承租人的安全或者健康的,即使承租人订立合同时明知该租赁物质量不合 格,承租人仍然可以随时解除合同。
	
	\paragraph{维修义务} (2 0 1 4 年 案 例 分 析 题 、 2 0 1 6 年 案 例 分 析 题 、 2 0 1 8 年 案 例 分 析 题 、 2 0 1 9 年 案 例分析题、2020 年案例分析题)
	(1 )出租人应当履行租赁物的维修义务,但是当事人另有约定或者因承租人的过错致使租 赁物需要维修的除外。
	(2 )承租人在租赁物需要维修时可以请求出租人在合理期限内维修。出租人未履行维修义 务的,承租人可以自行维修,维修费用由出租人负担。因维修租赁物影响承租人使用的,应当 相应减少租金或者延长租期。
	
	\paragraph{租金的支付期限}
	(1 ) 承 租 人 应 当 按 照 约 定 的 期 限 支 付 租 金 。 对 支 付 租 金 的 期 限 没 有 约 定 或 者 约 定 不 明 确 的,可以协议补充;不能达成补充协议的,按照合同相关条款或者交易习惯确定;仍不能确定,
	租赁期限 不满 1 年的,应当在租赁期限届满时 支付;租赁期限 1 年以 上的,应当在每届满 1 年 时支付,剩余期限不满1 年的,应当在租赁期限届满时支付。
	(2 )承租人无正当理由未支付或者迟延支付租金的,出租人可以请求承租人在合理期限内 支付;承租人逾期不支付的,出租人可以解除合同。(2022 年案例分析题)
	
	\paragraph{承租人的权利和义务}
	(1 )承租人未按照约定的方法或者未根据租赁物的性质使用租赁物,致使租赁物受到损失 的,出租人可以解除合同并请求赔偿损失。
	(2 )承租人经出租人同意,可以对租赁物进行改善或者增设他物。承租人未经出租人同意, 对租赁物进行改善或者增设他物的,出租人可以请求承租人恢复原状或者赔偿损失。(2016 年 案例分析题 )
	(3 )因不可归责于承租人的事由,致使租赁物部分或者全部毁损、灭失的,承租人可以请 求减少租金或者不支付租金;因租赁物部分或者全部毁损、灭失,致使不能实现合同目的的,
	承租人可以解除合同。
	【解释】租赁期间,租赁物毁损、灭失的风险由出租人承担。例如,甲将其机器设备 出租给乙使用,在租赁期间,该机器设备因遭遇泥石流全部毁损(并非乙故意将其砸烂),
	致使承租人乙的合同目的不能实现,乙可以解除合同。
	
	\paragraph{转租}
	(1 )承租人经出租人同意,可以将租赁物转租给第三人。承租人转租的,承租人与出租人 之间的租赁合同继续有效;第三人造成租赁物损失的,承租人应当赔偿损失。(2016 年案例分 析题、2020年案例分析题、2023年案例分析题)
	(2 )承租人未经出租人同意转租的,出租人可以解除合同。 (3 )次承租人的代 履行 (2024 年新增 )
	承租人拖欠租金的,次承租人可以代承租人支付其欠付的租金和违约金,但是转租合同对 出租人不具有法律约束力的除外。次承租人代为支付的租金和违约金,可以充抵次承租人应当 向承租人支付的租金;超出其应付的租金数额的,可以向承租人追偿。
	
	【解释】次承租人属于对该债务履行具有合法利益的第三人,出租人无正当理由不得 拒绝受领,否则陷入债权人迟延,承租人也无权加以反对。债权人(出租人)接受第 三人(次 承租人)履行后,其对债务人 (承租人)的债权转让给第三人。因此,次承租人代为支付 的租金和违约金,可以充抵次承租人应当向承租人支付的租金;超出其应付的租金数额的, 可以向承租人追偿。
	
	\paragraph{买卖不破租赁} 租赁物在承租人按照租赁合同占有期限内发生所有权变动的,不影响租赁合同的效力。 (2013 年案例分析题、2016 年案例分析题、2018 年案例分析题、2022年案例分析题)
	【相关链接1】抵押权设立前,抵押财产已经出租并转移占有的,原租赁关系不受该抵 押权的影响。
	【相关链接2 】动产抵押合同订立后未办理抵押登记,抵押人将抵押财产出租给他人并 移转占有,抵押权人行使抵押权的,租赁关系不受影响,但是抵押权人能够举证证明承租 人知道或者应当知道已经订立抵押合同的除外。
	
	\paragraph{承租人的优先购买权} 
	(1)出租人出卖租赁房屋的,应当在出卖之前的合理期限内通知承租人,承租人享有以同 等条件优先购买的权利;但是,房屋按份共有人行使优先购买权或者出租人将房屋出卖给近亲 属 的 除 外 。(2 0 1 6 年 案 例 分 析 题 )
	【解释1 】只有房屋租赁的承租人才享有优先购买权。其他标的物的租赁,承租人并不 享 有 优 先 购 买 权 。( 2 0 1 3 年 案 例 分 析 题 )
	【解释2 】近亲属包括配偶、父母、子女、兄弟姐妹、祖父母、外祖父母、孙子女、外 孙子女,不包括舅舅、姑姑、叔叔、表哥等。
	(2 )出租人履行通知义务后,承租人在15 日内未明确表示购买的,视为承租人放弃优先 购买权。
	(3)出租人委托拍卖人拍卖租赁房屋的,应当在拍卖5 日前通知承租人。承租人未参加拍 卖的,视为放弃优先购买权。
	(4 )出租人未通知承租人或者有其他妨害承租人行使优先购买权情形的,承租人可以请求 出租人承担赔偿责任。但是,出租人与第三人订立的房屋买卖合同的效力不受影响 。(2018年 案例分析题、2022 年案例分析题)
	
	
	11. 承租人在房屋租赁期限内死亡的,与其生前共同居住的人或者共同经营人可以按照原 租赁合同租赁该房屋。
	
	\paragraph{房屋租赁合同的无效}
	(1 )出租人就未取得建设 工程规划许可证或者未按照建设工程规划许可证的规定建设的房 屋,与承租人订立的租赁合同无效。但在一审法庭辩论终结前取得建设工程规划许可证或者经 主管部门批准建设的,人民法院应当认定有效。(2023年案例分析题)
	(2 )出租人就未经批准或者未按照批准内容建设的临时建筑,与承租人订立的租赁合同无 效。但在一审法庭辩论终结前经主管部门批准建设的,人民法院应当认定有效。
	(3)租赁期限超过临时建筑的使用期限,超过部分无效 。但在 一审法庭辩论终结前经主管 部门批准延长使用 限的,人民法院应当认定延长使用期限内的租赁期间有效。
	(4 )当事人以房屋租赁合同未按照法律、行政法规规定办理登记备案手续为由,请求确认 合同无效的,人民法院不予支持。
	(5 )房屋租赁合同无效,当事人请求参照合同约定的租金标准支付房屋占有使用费的,人 民法院一般应 予支持。
	
	
	\subsubsection{融资租赁合同}
	
	融资租赁合同是出租人根据承租人对出卖人、租赁物的选择,向出卖人购买租赁物, 提供给承租人使用,承租人支付租金的合同。融资租赁合同应当采用书面形式。
	【解释】融资租赁关系:(1)三方当事人:出租人、承租人、出卖人;(2 )两个合同: 融资租赁合同、买卖合同。
	
	 出租人根据承租人对出卖人、租赁物的选择订立的买卖合同,未经承租人同意,出租 人不得变更与承租人有关的合同内容。
	
	索赔权
	(1 )出租人、出卖人、承租人可以约定,出卖人不履行买卖合同义务的,由承租人行使索 赔的权利。 (2)承租人对出卖人行使索赔权利,不影响其履行支付租金的义务。但是,承租人依赖出 租人的技能确定租赁物或者出租人干预选择租赁物的,承租人可以请求减免相应租金。
	
	
	租赁物不符合约定或者不符合使用目的的,出租人不承担责任。但是,承租人依赖出 租人的技能确定租赁物或者出租人干预选择租赁物的除外。
	
	承租人占有租赁物期间,租赁物造成第 三人人身损害或者财产损失的,出租人不承担责任。 6. 承租人应当履行占有租赁物期间的维修义务 。(2018 年案例分析题、2020年案例分析题)
	【相关链接】在租赁合同中,出租人应当履行租赁物的维修义务,但是当事人另有约 定或者因承租人的过错致使租赁物需要维修的除外。
	
	承租人占有租赁物期间,租赁物毁损、灭失的,出租人有权请求承租人继续支付租金, 但是法律另有规定或者当事人另有约定的除外。(2022 年案例分析题)
	【解释】在租赁合同中,租赁物毁损、灭失的风险由出租人承担;在融资租赁合同中, 租赁物毁损、灭失的风险由承租人承担。
	
	承租人应当按照约定支付租金。承租人经催告后在合理期限内仍不支付租金的,出租 人可以请求支付全部租金;也可以解除合同,收回租赁物 。(2018年案例分析题)
	【相关链接】在买卖合同中,分期付款的买受人未支付到期价款的数额达到全部价款的 20\%,经催告后在合理期限内仍未支付到期价款的,出卖人可以请求买受人支付全部价款或 者解除合同。
	
	
	承租人未经出租人同意,将租赁物转让、抵押、质押、投资人股或者以其他方式处分的, 出租人可以解除融资租赁合同。
	
	有下列情形之 一,出租人请求解除融资租赁合同的,人民法院应予支持: (1)承租人未按照合同约定的期限和数额支付租金,符合合同约定的解除条件,经出租人 催告后在合理期限内仍不支付的;
	(2 )合同对于欠付租金解除合同的情形没有明确约定,但承租人欠付租金达到2期以上, 或者数额达到全部租金15\%以上,经出租人催告后在合理期限内仍不支付的;
	(3 )承租人违反合同约定,致使合同目的不能实现的其他情形。
	
	
	租赁期限届满租赁物的归属
	(1)出租人和承租人可以约定租赁期限届满租赁物的归属;对租赁物的归属没有约定或者 约定不明确的,可以协议补充;不能达成补充协议的,按照合同相关条款或者交易习惯确定; 仍不能确定的,租赁物的所有权归出租人。(2014 年案例分析题、2018 年案例分析题、2020 年案例分析题 )
	(2)当事人约定租赁期限届满租赁物归承租人所有,承租人已经支付大部分租金,但是无 力支付剩余租金,出租人因此解除合同收回租赁物,收回的租赁物的价值超过承租人欠付的租 金以及其他费用的,承租人可以请求相应返还。 (3)当事人约定租赁期限届满租赁物归出租人所有,因租赁物毁损、灭失或者附合、混合 于他物致使承租人不能返还的,出租人有权请求承租人给予合理补偿。
	(4 )当事人约定租赁期限届满,承租人仅需向出租人支付象征性价款的,视 约定的租金 义务履行完毕后租赁物的所有权归承租人。
	
	售后回租 承租人将其自有物出卖给出租人,再通过融资租赁合同将租赁物从出租人处租回的,人民 法院不应仅以承租人和出卖人系同一人为由认定不构成融资租赁法律关系。
	
	行政许可 依照法律、行政法规的规定,对于租赁物的经营使用应当取得行政许可的,出租人未取得 行政许可不影响融资租赁合同的效力。
	
	\subsubsection{借款合同}
	
	借款合同应当采用书面形式,但是自然人之间借款另有约定的除外。
	
	自然人之间的借款合同,自贷款人提供借款时成立。(2021年案例分析题、2023年案 例分析题 )
	
	借款合同对支付利息没有约定的,视为没有利息。
	
	借款合同对支付利息约定不明确,当事人不能达成补充协议的,按照当地或者当事人 的交易方式、交易习惯、市场利率等因素确定利息;自然人之间借款的 ,视为没有利息。
	
	借款的利息不得预先在本金中扣除。利息预先在本金中扣除的,应当按照实际借款数 额返还借款并计算利息。(2021年案例分析题、2023 年案例分析题)
	
	借款人未按照约定的借款用途使用借款的,贷款人可以停止发放借款、提前收回借款 或 者 解 除 合 同 。 (2 0 2 2 年 案 例 分 析 题 )
	
	借款人应当按照约定的期限支付利息。对支付利息的期限没有约定或者约定不明确的, 可以协议补充;不能达成补充协议的,按照合同相关条款或者交易习惯确定;仍不能确定,借 款期间不满 1 年的,应当在返还借款时一并支付;借款期间 1 年以 上的,应当在每届满 1 年时 支付,剩余期间不满1年的,应当在返还借款时 一并支付。
	
	借款人应当按照约定的期限返还借款。对借款期限没有约定或者约定不明确的,可以 协议补充;不能达成补充协议的,按照合同相关条款或者交易习惯确定;仍不能确定的,借款 人可以随时返还;贷款人可以催告借款人在合理期限内返还。
	
	借款人提前返还借款的,除当事人另有约定外,应当按照实际借款的期间计算利息。
	
	
	\subsubsection{民间借贷合同}
	【解释】民间借贷是指自然人、法人和非法人组织之间进行资金融通的行为。
	\paragraph{借期内利息}
	(1 )未约定
	借贷双方没有约定利息,出借人主张支付利息的,人民法院不予支持。
	(2 )约定不明 自然人之间借贷对利息约定不明,出借人主张支付利息的,人民法院不予支持。除自然 人之间借贷的外,借贷双方对借贷利息约定不明,出借人主张利息的,人民法院应当结合民 间借贷合同的内容,并根据当地或者当事人的交易方式、交易习惯、市场报价利率等因素确 定利息。
	(3 )有 约 定 出借人请求借款人按照合同约定利率支付利息的,人民法院应予支持,但是双方约定的利 率超过合同成立时 一年期贷款市场报价利率四倍的除外。
	
	\paragraph{逾期利率}
	借贷双方对逾期利率有约定的,从其约定,但是以不超过合同成立时 一年期贷款市场报价 利率四倍为限。未约定逾期利率或者约定不明的,人民法院可以区分不同情况处理: (1)既未约定借期内利率,也未约定逾期利率,出借人主张借款人自逾期还款之日起 参照当时 一年期贷款市场报价利率标准计算的利息承担逾期还款违约责任的,人民法院应 予
	支持。 (2)约定了借期内利率但是未约定逾期利率,出借人主张借款人自逾期还款之日起按照借 期内利率支付资金占用期间利息的,人民法院应予支持。
	
	出借人与借款人既约定了逾期利率,又约定了违约金或者其他费用,出借人可以选择 主张逾期利息、违约金或者其他费用,也可以一并主张,但是总计超过合同成立时 一年期贷款 市场报价利率四倍的部分,人民法院不予支持。
	【相关链接】在同一合同中,当事人既约定违约金,又约定定金的,一方违约时,对 方可以选择适用违约金或者定金条款,二者不能并用。
	
	\paragraph{诉讼当事人}
	(1 )保证人为借款人提供连带责任保证,出借人仅起诉借款人的,人民法院可以不追加保 证人为共同被告;出借人仅起诉保证人的,人民法院可以追加借款人为共同被告。
	(2 )保证人为借款人提供一般保证,出借人仅起诉保证人的,人民法院应当追加借款人为 共同被告;出借人仅起诉借款人的,人民法院可以不追加保证人为共同被告。
	
	\paragraph{自然人之间的借款合同} 自然人之间的借款合同,自贷款人提供借款时成立。自然人之间的借款合同具有下列情形 之 一的,可以视为合同成立:
	(1 )以现金支付的,自借款人收到借款时;
	(2 )以银行转账、网上电子汇款等形式支付的,自资金到达借款人账户时;
	(3 )以票据交付的,自借款人依法取得票据权利时;
	(4 )出借人将特定资金账户支配权授权给借款人的,自借款人取得对该账户实际支配权时; (5 )出借人以与借款人约定的其他方式提供借款并实际履行完成时。
	
	\paragraph{民间借贷合同的无效} 法人之间、非法人组织之间以及它们相互之间为生产、经营需要订立的民间借贷合同 ,原 则 上有效。具有 下列情形之一的,人民法院应当认定民间借贷合同无效: (1)套取金融机构贷款转贷的;
	(2 )以向其他营利法人借贷、向本单位职工集资或者以向公众非法吸收存款等方式取得的 资金转货的;
	(3 )未依法取得放贷资格的出借人,以营利为目的向社会不特定对象提供借款的;
	(4 )出借人事先知道或者应当知道借款人借款用于违法犯罪活动仍然提供借款的;
	(5 )违反法律、行政法规强制性规定的;
	(6 )违背公序良俗的。
	【 解 释 1 】 借 款 人 或 者 出 借 人 的 借 贷 行 为 涉 嫌 犯 罪 ,或 者 已 经 生 效 的 裁 判 认 定 构 成 犯 罪 , 当事人提起民事诉讼的,民间借贷合同并不当然无效。人民法院应当依据《民法典》规定 的民事法律行为无效情形以及 《民间借贷规定》规定的民间借贷合同无效情形,认定民间 借贷合同的效力。(2023 年案例分析题)
	【解释2 】担保人以借款人或者出借人的借贷行为涉嫌犯罪或者已经生效的裁判认定构 成犯罪为由,主张不承担民事责任的,人民法院应当依据民间借贷合同与担保合同的效力、 当事人的过错程度,依法确定担保人的民事责任。
	
	
	\paragraph{网络贷款平台} (1)借贷双方通过网络贷款平台形成借贷关系,网络贷款平台的提供者仅提供媒介服务, 当事人请求其承担担保责任的,人民法院不予支持。
	(2 )网络贷款平台的提供者通过网页、广告或者其他媒介明示或者有其他证据证明其为借 贷提供担保,出借人请求网络贷款平台的提供者承担担保责任的,人民法院应予支持。
	8. 法定代表人 (1)法人的法定代表人或者非法人组织的负责人以单位名义与出借人签订民间借贷合同, 有证据证明所借款项系法定代表人或者负责人个人使用,出借人请求将法定代表人或者负责人 列为共同被告或者第三人的,人民法院应予准许。
	(2 )法人的法定代表人或者非法人组织的负责人以个人名义与出借人订立民间借贷合同, 所借款项用于单位生产经营,出借人请求单位与个人共同承担责任的,人民法院应予支持。
	
	\paragraph{以订立买卖合同作为民间借贷合同的担保}
	(1)当事人以订立买卖合同作为民间借贷合同的担保,借款到期后借款人不能还款,出借 人请求履行买卖合同的,人民法院应当按照民间借贷法律关系审理。当事人根据法庭审理情况 变更诉讼请求的,人民法院应当准许。
	(2)按照民间借贷法律关系审理作出的判决生效后,借款人不履行生效判决确定的金钱债 务,出借人可以申请拍卖买卖合同标的物,以偿还债务。就拍 卖所得的价款与应偿还借款本息 之间的差额,借款人或者出借人有权主张返还或者补偿。
	【相关链接】根据《民法典合同编通则解释》第28条的规定,债务人或者第三人与债 权人在债务履行期限届满前达成以物抵债协议的,人民法院应当在审理债权债务关系的基 础上认定该协议的效力。当事人约定债务人到期没有清偿债务,债权人可以对抵债财产拍 卖、变卖、折价以实现债权的,人民法院应当认定该约定有效。当事人约定债务人到期没 有清偿债务,抵债财产归债权人所有的,人民法院应当认定该约定无效,但是不影响其他 部分的效力;债权人请求对抵债财产拍卖、变卖、折价以实现债权的,人民法院应予支持。 当事人订立上述以物抵债协议后,债务人或者第三人未将财产权利转移至债权人名下,债 权人主张优先受偿的,人民法院不予支持;债务人或者第三人已将财产权利转移至债权人
	名下的,依据《民法典担保制度解释》第68 条的规定处理。
	
	\subsubsection{赠与合同}
	赠与合同是赠与人将自己的财产无偿给予受赠人,受赠人表示接受赠与的合同 。赠与 合同是诺成合同、单务合同、无偿合同。
	【解释1 】赠与合同属于诺成合同,而非实践合同。
	【解释2 】贈与合同属于单务合同,订立赠与合同属于双方民事法律行为。
	
	赠与人的义务如下
	\begin{enumerate}
		\item 赠与的财产有瑕疵的,赠与人不承担责任。附义务的赠与,赠与的财产有瑕疵的,赠 与人在附义务的限度内承担与出卖人相同的责任。
		
		\item 赠与人故意不告知瑕疵或者保证无瑕疵,造成受赠人损失的,应当承担赔偿责任。
		
		\item 赠与人的经济状况显著恶化,严重影响其生产经营或者家庭生活的,可以不再履行赠 与义务。
	\end{enumerate}
	
	\paragraph{任意撤销} (2 0 2 2 年 案 例 分 析 题 )
	(1) 一般情况下,赠与人在赠与财产的权利转移之前可以撤销赠与。
	(2 )经过公证的赠与合同或者依法不得撤销的具有救灾、扶贫、助残等公益、道德义务性质的赠与合同,不得任意撤销。赠与人不交付赠与财产的,受赠人可以请求交付。依据前述规 定应当交付的赠与财产因赠与人故意或者重大过失致使毁损、灭失的,赠与人应当承担赔偿责任。
	
	\paragraph{赠与的法定撤销} 
	(1 )赠 与 人 的 撤 销 权
	受赠人有下列情形之一的,赠与人可以撤销赠与: 1严重侵害赠与人或者赠与人近亲属的合法权益;
	2对赠与人有扶养义务而不履行;
	3不履行赠与合同约定的义务。
	(2 )赠与人的继承人或者法定代理人的撤销权 因受赠人的违法行为致使赠与人死亡或者丧失民事行为能力的,赠与人的继承人或者法定 代理人可以撤销赠与。
	【解释1】受赠人有忘思行为时,无论赠与财产的权利是否转移,贈与是否经过公证或 者具有救灾、扶贫、助残等公益、道德义务性质,赠与人或其继承人、法定代理人均可撤 销该赠与。
	【解释2 】(1 )赠与人的撤销权,自知道或者应当知道撤销事由之日起1 年内行使; (2)贈与人的继承人或者法定代理人的撤销权,自知道或者应当知道撤销事由之日起6 个 月内行使。
	
	
	\subsubsection{承揽合同}
	
	
	承揽合同是双务、有偿、诺成合同。
	
	承揽人应当按照定作人的要求保守秘密,未经定作人许可,不得留存复制品或者技术 资料。
	
	承揽人应当以自己的设备、技术和劳力,完成主要工作,但是当事人另有约定的除外。 承揽人将其承揽的主要工作交由第三人完成的,应当就该第三人完成的工作成果向定作人负责; 未经定作人同意的,定作人也可以解除合同。
	
	承揽人可以将其承揽的辅助工作交由第三人完成 。承揽人将其承揽的辅助工作交由第 三人完成的,应当就该第三人完成的工作成果向定作人负责。
	【解释】(1)主要工作:应当经定作人同意;(2 )辅助工作:可以不经同意。
	
	承揽人交付的工作成果不符合质量要求的,定作人可以合理选择请求承揽人承担修理、 重作、减少报酬、赔偿损失等违约责任。
	
	定作人中途变更承揽工作的要求,造成承揽人损失的,应当赔偿损失。(2022年案例分 析题)
	
	定作人在承揽人完成工作前可以随时解除合同,造成承揽人损失的,应当赔偿损失。 (2 0 2 2 年 案 例 分 析 题 )
	
	
	\subsubsection{建设工程合同}
	\paragraph{建设工程合同}
	建设工程合同应当采用书面形式。建设工程合同 、融资租赁合同属于要式合同。
	
	
	建设工程施工合同具有下列情形之 一的,应当依据《民法典》的规定,\textbf{认定无效}: 
	\begin{enumerate}
		\item 承包人未取得建筑施工企业资质或者超越资质等级的;
		
		\item 没有资质的实际施工人借用有资质的建筑施 工企业名义的;
		
		\item 建设工程必须进行招标而未招标或者中标无效的。
	\end{enumerate}
	【解 释 1 】 承 包 人 超 越 资 质 等 级 许 可 的 业 务 范 围 签 订 建 设 工 程 施 工 合 同 , 在 建 设 工 程 竣 工前取得相应资质等级,当事人请求按照无效合同处理的,人民法院不予支持。(2018 年 案例分析题)
	【解释2 】缺乏资质的单位或者个人借用有资质的建筑施工企业名义签订建设工程施工 合同,发包人请求出借方与借用方对建设工程质量不合格等因出借资质造成的损失承担连 带赔偿责任的,人民法院应予支持。
	
	\paragraph{分包} (1)总承包人或者勘察、设计、施工承包人经发包人同意,可以将自己承包的部分工作交 由第 三人完成。第 三人就其完成的工作成果与总承包人或者勘察、设计、施工承包人向发包人 承 担 连 带 责 任 。( 2 0 2 3 年 案 例 分 析 题 )
	(2 )承包人不得将其承包的全部建设工程转包给第 三人或者将其承包的全部建设工程支解 以后以分包的名义分别转包给第 三人。
	(3 )禁止承包人将工程分包给不具备相应资质条件的单位。禁止分包单位将其承包的工程 再分包。
	(4 )建 设 工程 主体 结 构 的 施 工必 须 由 承 包 人 自 行 完 成 。
	(5 )承包人将建设工程转包、违法分包的,发包人可以解除合同。
	【解释1】因建设工程质量发生争议的,发包人可以以总承包人、分包人和实际施工人 为共同被告提起诉讼。
	【解释2 】因承包人的原因致使建设工程在合理使用期限内造成人身损害和财产损失 的,承包人应当承担賠偿责任。
	
	\paragraph{委托监理合同}
	(1 )建 设 工程 实 行 监 理 的 , 发 包 人 应 当 与 监 理 人 采 用 书面 形 式 订 立 委 托 监 理 合 同 。
	(2 )发包人与监理人的权利和义务以及法律责任,应当依照委托合同以及其他有关法律、 行政法规的规定。
	
	\paragraph{实际竣工日期} 当事人对建设工程实际竣工日期有争议的,人民法院应当分别按照以下情形予以认定:
	(1 )建设工程经竣工验收合格的,以竣工验收合格之日为竣工日期; (2)承包人已经提交竣工验收报告,发包人拖延验收的 ,以承包人提交验收报告之日为竣 工日期; (3)建设工程未经竣工验收,发包人擅自使用的,以转移占有建设工程之日为竣工日期。
	
	建设工程竣工经验收合格后,方可交付使用;未经验收或者验收不合格的,不得交付 使用。建设 工程未经竣 工验收,发包人擅自使用后,又以使用部分质量不符合约定为由 主张权 利的,人民法院不 予支持;但是承包人应当在建设工程的合理使用寿命内对地基基础 工程和主 体 结 构 质 量 承 担 民 事 责 任 。 (2 0 2 2 年 案 例 分 析 题 )
	
	\paragraph{垫资}
	(1 )当事人对垫资和垫资利息有约定,承包人请求按照约定返还垫资及其利息的,人民法 院应 予支持,但是约定的利息计算标准高 于垫资时的同类贷款利率或者同期贷款市场报价利率 的部分除外 。
	(2 )当事人对垫资没有约定的,按照工程欠款处理。
	(3 )当事人对垫资利息没有约定,承包人请求支付利息的,人民法院不予支持。
	
	\paragraph{工程欠款}
	(1 )当事人对欠付 工程价款利息计付标准有约定的,按照约定处理。 (2)没有约定的,按照同期同类贷款利率或者同期贷款市场报价利率计息。 (3)利息从应付工程价款之日开始计付。当事人对付款时间没有约定或者约定不明的,下 列时间视为应付款时间:
	1建设工程已实际交付的,为交付之日; 2建设工程没有交付的,为提交竣工结算文件之日; 3建设工程未交付,工程价款也未结算的,为当事人起诉之日。
	
	% \usepackage{tabularray}
	\begin{table}
		\centering
		\begin{tblr}{
				width = \linewidth,
				colspec = {Q[83]Q[554]Q[302]},
				hlines,
				vlines,
			}
			& 垫资                                             & 工程欠款                      \\
			未约定利息 & 不支付利息                                          & 按照同期同类贷款利率或者同期 贷款市场报价利率计息 \\
			约定利息  & 按照约定,但是约定的利息计算标准高于垫资时的同类贷款利率或者同期贷款市场 报价利率的部分除外 & 按照约定                      
		\end{tblr}
	\end{table}
	
	9. 承包人建设工程价款的优先受偿权
	(1 )发包人未按照约定支付价款的,承包人可以催告发包人在合理期限内支付价款。发包 人逾期不支付的,除根据建设工程的性质不宜折价、拍卖外,承包人可以与发包人协议将该工 程折价,也可以请求人民法院将该工程依法拍卖。建设工程的价款就该工程折价或者拍卖的价 款优先受偿。
	(2 )承包人享有的建设工程价款优先受偿权优于抵押权和其他债权。(2018 年案例分析题) (3 )承包人就逾期支付建设工程价款的利息、违约金、损害赔偿金等主张优先受偿的,人 民法院不予支持。
	(4 ) 承 包 人 应 当 在 合 理 期 限 内 行 使 建 设 工 程 价 款 优 先 受 偿 权 , 但 最 长 不 得 超 过 1 8 个 月 , 自发包人应当给付建设工程价款之日起算。
	(5 )发包人与承包人约定放弃或者限制建设工程价款优先受偿权,损害建筑工人利益,发 包 人 根 据 该 约 定 主张 承 包 人 不 享 有 建 设 工程 价 款 优 先 受 偿 权 的 , 人 民 法 院 不 予 支 持 。
	
	【相关链接】根据最高人民法院2023 年发布的《关于商品房消费者权利保护问题的批复》, 商品房消费者以居住为目的购买房屋并已支付全部价款,主张其房屋交付请求权优先于建设 工程价款优先受偿权、抵押权以及其他债权的,人民法院应当予以支持。只支付了部分价款 的商品房消费者,在一审法庭辩论终结前已实际支付剩余价款的,可以适用上述规定。在房 屋不能交付且无实际交付可能的情况下,商品房消费者主张价款返还请求权优先于建设工程 价款优先受偿权、抵押权以及其他债权的,人民法院应当予以支持。(2024 年新增)
	
	\subsubsection{客运合同}
	客运合同自承运人向旅客出具客票时成立,但是当事人另有约定或者另有交易习惯的 除外。
	
	旅客可以自行决定解除客运合同。旅客因自己的原因不能按照客票记载的时间乘坐的, 应当在约定的期限内办理退票或者变更手续;逾期办理的,承运人可以不退票款,并不再承担 运输义务。
	
	承运人应当按照有效客票记载的时间、班次和座位号运输旅客。承运人迟延运输或者 有其他不能正常运输情形的,应当及时告知和提醒旅客,采取必要的安置措施,并根据旅客的要求安排改乘其他班次或者退票;由此造成旅客损失的,承运人应当承担赔偿责任,但是不可 归责于承运人的除外。
	
	承运人擅自降低服务标准的,应当根据旅客的请求退票或者减收票款;提高服务标准的, 不得加收票款。
	
	在运输过程中旅客随身携带物品毁损、灭失,承运人有过错的,应当承担赌偿责任。
	
	
	\subsubsection{货运合同}
	在承运人将货物交付收货人之前,托运人可以要求承运人中止运输、返还货物、变更 到达地或者将货物交给其他收货人,但是应当赔偿承运人因此受到的损失。
	
	承运人对运输过程中货物的毁损、灭失承担赔偿责任。但是,承运人证明货物的毁损、 灭失是因不可抗力、货物本身的自然性质或者合理损耗以及托运人、收货人的过错造成的,不 承 担 赔 偿 责 任 。(2 0 2 0 年 案 例 分 析 题 )
	
	货物在运输过程中因不可抗力灭失,未收取运费的,承运人不得请求支付运费;已经 收取运费的,托运人可以请求返还。法律另有规定的,依照其规定。
	
	货物的毁损、灭失的赔偿额,当事人有约定的,按照其约定;没有约定或者约定不明 确的,可以协议补充;不能达成补充协议的,按照合同相关条款或者交易习惯确定;仍不能确 定的,按照交付或者应当交付时货物到达地的市场价格计算。法律、行政法规对赔偿额的计算 方法和赔偿限额另有规定的,依照其规定。
	
	托运人或者收货人不支付运费、保管费或者其他费用的,承运人对相应的运输货物享 有留置权,但是当事人另有约定的除外。
	
	收货人不明或者收货人无正当理由拒绝受领货物的,承运人依法可以提存货物。
	
	
	\subsubsection{委托合同}
	委托人可以特别委托受托人处理 一项或者数项事务,也可以概括委托受托人处理一切 事务。两个以上的受托人共同处理委托事务的,对委托人承担连带责任。
	\paragraph{受托人的报酬}
	(1 )受托人完成委托事务的,委托人应当按照约定向其支付报酬。
	(2 )因不可归责于受托人的事由,委托合同解除或者委托事务不能完成的,委托人应当向 受托人支付相应的报酬。当事人另有约定的,按照其约定。
	
	\paragraph{委托人的费用}
	(1 )委托人应 当预付处理 委托事务的费用。
	(2 )受 托 人 为 处 理 委 托 事 务 垫 付 的 必 要 费 用 , 委 托 人应 当 偿 还 该 费 用 并 支付 利 息 。
	
	\paragraph{损失赔偿} (1)有偿的委托合同,因受托人的过错造成委托人损失的,委托人可以请求赌偿损失。 (2)无偿的委托合同,因受托人的故意或者重大过失造成委托人损失的,委托人可以请求 赔偿损失。
	6. 受托人处理委托事务时,因不可归责于自己的事由受到损失的,可以向委托人请求赔 偿损失。
	
	\paragraph{随时解除}
	(1 )委托人或者受托人可以随时解除委托合同。 (2)因解除合同造成对方损失的,除不可归责于该当事人的事由外,无偿委托合同的解除 方应当赔偿因解除时间不当造成的直接损失,有偿委托合同的解除方应当赔偿对方的直接损失 和合同履行后可以获得的利益。
	
	\paragraph{转委托}
	 (1)经委托人同意,受托人可以转委托。转委托经同意或者追认的,委托人可以就委托事 务直接指示转委托的第三人,受托人仅就第三人的选任及其对第 三人的指示承担责任。 (2)转委托未经同意或者追认的,受托人应当对转委托的第三人的行 承担责任;但是, 在紧急情况下受托人为了维护委托人的利益需要转委托第三人的除外。
	
	\paragraph{隐名代理}
	(1)受托人以自己的名义,在委托人的授权范围内与第三人订立的合同,第 三人在订立合 同时知道受托人与委托人之间的代理关系的,该合同直接约束委托人和第三人;但是,有确切 证据证明该合同只约束受托人和第三人的除外。
	(2 )受托人以自己的名义与第三人订立合同时,第三人不知道受托人与委托人之间的代理 关系的,受托人因第三人的原因对委托人不履行义务,受托人应当向委托人披露第 三人,委托 人因此可以行使受托人对第三人的权利。但是,第三人与受托人订立合同时如果知道该委托人 就不会订立合同的除外。
	(3 )受托人因委托人的原因对第三人不履行义务,受托人应当向第三人披露委托人,第三 人因此可以选择受托人或者委托人作为相对人主张其权利,但是第三人不得变更选定的相对人。 (4)委托人行使受托人对第 三人的权利的,第 三人可以向委托人主张其对受托人的抗辩。 第三人选定委托人作为其相对人的,委托人可以向第三人主张其对受托人的抗辩以及受托人对 第 三人的抗辩。
	
	\subsubsection{行纪合同}
	行纪合同是行纪人以自己的名义为委托人从事贸易活动,委托人支付报酬的合同。行纪人处理委托事务支出的费用,由行纪人负担,但是当事人另有约定的除外。
	
	行纪人低于委托人指定的价格卖出或者高于委托人指定的价格买入的,应当经委托人 同意;未经委托人同意,行纪人补偿其差额的,该买卖对委托人发生效力。
	
	行纪人高于委托人指定的价格卖出或者低于委托人指定的价格买人的,可以按照约定 增加报酬;没有约定或者约定不明确的,可以协议补充;不能达成补充协议的,按照合同相关 条款或者交易习惯确定;仍不能确定的,该利益属于委托人。
	
	行纪人卖出或者买入具有市场定价的商品,\textbf{除委托人有相反的意思表示外},行纪人自已可以作为买受人或者出卖人。在这种情形下,\textbf{行纪人仍然可以请求委托人支付报酬}。
	 
	 【相关链接】代理人不得以被代理人的名义与自己实施民事法律行为,但是被代理人同意或者追认的除外。
	
	\textbf{行纪人与第三人订立合同的},行纪人对该合同直接享有权利、承担义务。第三人不履行 义务致使委托人受到损害的,行纪人应当承担赔偿责任,但是行纪人与委托人另有约定的除外。 7. 行纪人完成或者部分完成委托事务的,委托人应当向其支付相应的报酬。委托人逾期 不支付报酬的,行纪人对委托物享有留置权,但是当事人另有约定的除外。
	
	% \usepackage{tabularray}
	\begin{table}
		\centering
		\begin{tblr}{
				width = \linewidth,
				colspec = {Q[400]Q[312]Q[221]},
				hlines,
				vlines,
			}
			& 委托合同       & 行纪合同    \\
			委托事项的范围       & 非常宽泛       & 仅限于贸易活动 \\
			以谁的名义与第三人订立合同 & 可以委托人也可以自己 & 必须自己    \\
			是否支付报酬        & 有偿或无偿      & 均为有偿    \\
			费用承担          & 委托人        & 行纪人     
		\end{tblr}
	\end{table}
	
	\subsubsection{技术合同}
	\paragraph{技术合同的种类}
	技术合同包括
	\begin{enumerate}
		\item 技术开发合同。当事人之间就新技术、新产品、新 工艺、新品种或者新材料及其系统 的研究开发所订立的合同,包括委托开发合同和合作开发合同。
		
		\item 技术转让合同。是合法拥有技术的权利人 ,将现有特定的专利、专利申请、技术秘密的相关权利让与他人所订立的合同。包括专利权转让、专利申请权转让、技术秘密 转让等合同
		
		\item 技术许可合同。是合法拥有技术的权利人,将现有特定的专利、技术秘密的相关权利许 可他人实施、使用所订立的合同 。包括专利实施许可、技术秘密使用许可等合同。
		
		\item 技术咨询合同。是当事人 一方以技术知识为对方就特定技术项目提供可行性论证、技术 预测、专题技术调查、分析评价报告等所订立的合同。
		
		\item 技术服务合同。是当事人 一方以技术知识为对方解决特定技术问题所订立的合同,不包 括承揽合同和建设工程合同。
	\end{enumerate}
	
	【解释】技术转让合同和技术许可合同的规定不仅适用于专利权、技术秘密,还可以 适用于计算机软件署作权、集成电路布图设计专有权、植物新品种权等其他知识产权。
	
	\paragraph{职务技术成果} (1)职务技术成果是执行法人或者非法人组织的工作任务,或者主要是利用法人或者非法 人组织的物质技术条件所完成的技术成果。
	(2 )职务技术成果的使用权、转让权属 于法人或者非法人组织的,法人或者非法人组织可 以就该项职务技术成果订立技术合同。法人或者非法人组织订立技术合同转让职务技术成果时, 职务技术成果的完成人享有以同等条件优先受让的权利。(2022 年案例分析题) (3)非职务技术成果的使用权、转让权属于完成技术成果的个人,完成技术成果的个人可 以就该项非职务技术成果订立技术合同。
	【相关链接】以专利权出质的,质权自办理出质登 记时设立。专利权出质后,出质人 不得转让或者许可他人使用,但是出质人与质权人协商同意的除外。
	
	\paragraph{后续改进的技术成果} 当事人可以按照互利的原则,在合同中约定实施专利、使用技术秘密后续改进的技术成果 的分享办法;没有约定或者约定不明确,依据《民法典》的规定仍不能确定的, 一方后续改进 的技术成果,其他各方无权分享。

	
	\newpage
	\section{合伙企业法律制度}
	
	\subsection{普通合伙企业}
	\subsubsection{合伙企业的特征} 
	1. 合伙企业是合伙人共同出资、共同经营、共享收益、共担风险的自愿联合 。其中,最 关键的是共担风险,这是合伙关系不同 于其他合同关系的最关键之处。
	2. 合伙企业的信用基础最终取决于普通合伙人的偿债能力。
	3. 合伙企业不具有法人资格,但可以以自己的名义从事法律行为,可以以自己的名义起 诉和应诉。合伙企业拥有自己的、与合伙人财产相区别的财产,合伙企业的债务应当先以合伙 企业的财产清偿;合伙企业的财产不足以清偿时,普通合伙人才承担清偿责任。
	4. 合伙企业的内部治理和利益分配高度灵活。合伙企业的内部事务管理和利益分配主要 由合伙协议规范,而合伙协议由合伙人在自愿协商的基础上订立,法律上的强制性规范很少。 (2 0 2 4 年 调 整 )
	S. 合伙企业并非企业所得税的纳税人。合伙企业的生产经营所得和其他所得,按照国家 有关税收规定,由合伙人分别缴纳所得税。
	
	\subsubsection{合伙企业的类型}
	1. 合伙企业分为普通合伙企业(其中包括特 的普通合伙企业)和有限合伙企业。
	2. 普通合伙企业的特征是“所有”:所有的合伙人(不论其出资比例、不论其是否执行合 伙事务)对所有的企业债务均应承担无限连带责任。
	3. 特殊的普通合伙企业的特征是“先看债务再找人” :(1)一般的企业债务,所有的合伙 人均应承担无限连带责任;(2 )特定的企业债务,由特定的合伙人承担无限 (连带)责任, 一 个合伙人或者数个合伙人在执业活动中因故意或者重大过失造成合伙企业债务的,应当承担无 限责任或者无限连带责任,其他合伙人以其在合伙企业中的财产份额为限承担有限责任。
	
	4. 有限合伙企业的特征是“先找人再确定责任” :(1)有限合伙企业至少应当有1 个普通 合 伙 人 和 1 个 有 限 合 伙 人 ;(2 ) 普 通 合 伙 人 应 当 对 所 有 的 企 业 债 务 承 担 无 限 连 带 责 任 ;(3 )有 限合伙人以其认缴的出资额为限对合伙企业债务承担责任。 【例外情形】第三人有理由相信有限合伙人为普通合伙人并与其交易的,该有限合伙人对该 笔交易承担与普通合伙人同样的责任(无限连带责任)。有限合伙人未经授权以有限合伙企业名义 与他人进行交易,给有限合伙企业或者其他合伙人造成损失的,该有限合伙人应当承担赔偿责任。
	
	\subsubsection{普通合伙企业的设立条件}
	1. 有两个以上合伙人
	(1)合伙人至少为2 个以上,对于普通合伙企业合伙人数的最高限额,《合伙企业法》未 作规定。
	(2 )合伙人可以是自然人,也可以是法人或者其他组织。
	【解释】个人独资企业、合伙企业可以成为合伙人。
	(3 )合伙人为自然人的,应当具有完全民事行为能力,无民事行为能力人和限制民事行为 能力人不得成为普通合伙人。
	(4 )国有独资公司、国有企业、上市公司以及公益性的事业单位、社会团体不得成为普通 合伙人,但可以成为有限合伙人。
	
	2. 有书面合伙协议
	(1)合伙协议应当依法由全体合伙人协商 一致,以书面形式订立。合伙协议经全体合伙人 签名、盖章后生效。 (2)修改或者补充合伙协议,应当经全体合伙人一致同意;但是,合伙协议另有约定的 除外。
	【解释】修改或者补充合伙协议时,先约定后法定:先看合伙协议是否另有约定(爱 怎么约定就怎么约定),合伙协议没有约定的,才看法定(经全体合伙人一致同意)。
	3. 有合伙人认缴或者实际缴纳的出资 (1)合伙人可以用货币、实物、知识产权、土地使用权或者其他财产权利出资,普通合伙 人也可以用劳务出资。
	(2 )合伙人应当按照合伙协议约定的出资方式、数额和缴付期限履行出资义务。
	(3 )合伙人以实物、知识产权、土地使用权或者其他财产权利出资,需要评估作价的,可 以由全体合伙人协商确定,也可以由全体合伙人(不能是个别合伙人)委托法定评估机构评估。 (4 )合伙人以劳务出资的,其评估办法由全体合伙人协商确定,并在合伙协议中载明。
	【解释1 】只有普通合伙人可以以劳务出资,有限合伙人不得以劳务出资。
	【解释2 】(1)实物、知识产权、土地使用权或者其他财产权利:协商确定或者委托法 定评估机构评估;(2 )劳务:只能协商确定。
	
	4. 有合伙企业的名称和生产经营场所
	(1)普通合伙企业应当在名称中标明“普通合伙” 字样。 (2)特殊的普通合伙企业应当在名称中标明“特殊普通合伙” 字样。
	(3 )有限合伙企业名称中应当标明“ 有限合伙” 字样。
	(4 )经企业登记机关登记的合伙企业主要经营场所只能有一个,并且应当在其企业登记机 关登记管辖区域内。
	
	\subsubsection{合伙企业的财产}
	1. 合伙企业财产的构成(2024年调整)
	(1 )合伙人的出资,形成合伙企业的原始财产。 (2)以合伙企业名义取得的收益(如营业收入、投资收益等)。
	( 3 ) 依 法 取 得 的 其 他 财 产 ( 如 合 法 接 受 的 赠 与 财 产 、 因 遭 受 侵 权 而 获 得 的 赔 偿 金 等 )。
	
	2. 合伙企业财产的性质
	(1 )独 立性 合伙企业的财产独立于合伙人,合伙人缴纳出资之后,便丧失了对作为其出资财产的所有 权,合伙企业财产的所有权属于合伙企业,而不是某 一个合伙人。某一个合伙人对合伙企业财 产的权益,表现形式为依照合伙协议所确定的财产收益份额或者比例。 (2)合伙人在合伙企业清算前私自转移或者处分合伙企业财产的,合伙企业不得以此对抗 善 意第三 人。
	
	3. 财产份额的转让
	(1 ) 对 内 转 让 普通合伙人之间转让在合伙企业中的全部或者部分财产份额时,应当通知其他合伙人。 【解释1】对内转让时,其他合伙人并无优先购买权。
	【解释2 】合伙协议如果明确约定合伙人之间转让合伙财产份额需经全体合伙人一致同 意,且该约定不违反法律、行政法规的强制性规定,亦不违背公序良俗,人民法院通常认 定其合法有效。基于此种特别约定,在其他合伙人未同意合伙财产份额转让之前,当事人 就合伙财产份额转让签订的转让协议,应当认定为成立但未生效 。如其他合伙人明确不同 意该合伙财产份额转让,则转让协议确定不生效,不能在当事人之间产生履行力。当事人 请求履行转让协议的,人民法院不予支持。
	(2 ) 对 外 转 让 1除合伙协议另有约定外,普通合伙人向合伙人以外的人转让其在合伙企业中的全部或者 部分财产份额时,须经其他合伙人一致同意。 2普通合伙人向合伙人以外的人转让其在合伙企业中的财产份额的,在同等条件下,其他 合伙人有优先购买权;但是,合伙协议另有约定的除外。 3合伙人以外的人依法受让合伙人在合伙企业中的财产份额的,经修改合伙协议即成为合 伙企业的合伙人。
	【解释1】普通合伙人对外转让财产份额时,先约定后法定:先看合伙协议是否另有约 定(爱怎么约定就怎么约定);合伙协议未约定的,才看法定(必须经其他合伙人一致同意)。
	【解释2 】其他合伙人是否享有优先购买权?先约定后法定:先看合伙协议是否另有约 定(爱怎么约定就怎么约定);合伙协议未约定的,才看法定(在同等条件下其他合伙人 才有优先购买权)。
	
	4. 财产份额的出质 普通合伙人以其在合伙企业中的财产份额出质的,须经其他合伙人一致同意;未经其他合 伙人 一致同意,其行为无效,由此给善意第三人造成损失的,由行为人依法承担赔偿责任。
	【解释】普通合伙人以其财产份额出质的,必须经其他合伙人一致同意,这是法律的 强制性规定,合伙协议不得任意约定。
	
	\subsubsection{合伙企业的事务执行和利润分配}
	1. 合伙事务执行的形式
	(1 )普通合伙 人无论其出资多少,都需 要对企业债务承担无限连带责任,因此,各合伙人 无论其出资多少,都有权平等享有执行合伙企业事务的权利。
	( 2 ) 作 为 合 伙 人 的 法 人 、 其 他 组 织 执 行 合 伙 企 业 事 务 的 ,由 其 委 托 的 代 表 执 行 。
	( 3 ) 一般 情 况 下 , 由 全 体 合 伙 人 共 同 执 行 合 伙 事 务 。 按 照 合 伙 协 议 的 约 定 或 者 经 全 体 合 伙 人决定,也可以委托 一个或者数个合伙人执行合伙事务,其他合伙人不再执行合伙事务。由一 个或者数个合伙人执行合伙 事务的,执行 事务的合伙人应当定期向其他合伙人报告事务执行情 况以及合伙企业的经营和财务状况,其执行合伙事务所产生的收益归合伙企业,所产生的费用 和亏损由合伙企业承担。
	2. 由 一个或者数个合伙人执行合伙事务的
	(1 )对外代表权 委托一个或者数个合伙人执行合伙事务的,执行合伙事务的合伙人对外代表合伙企业,其 他合伙人不得对外代表合伙企业。
	(2 )监 督 权
	委托 一个或者数个合伙人执行合伙事务的,其他合伙人不再执行合伙事务,但不执行合伙 事务的合伙 人有权监督执行事务合伙 人执行合伙事务的情况。
	(3 )查阅账簿权
	合伙人有权查阅合伙企业会计账簿等财务资料。
	
	(4 )撤销委托权
	受委托执行合伙事务的合伙人不按照合伙协议或者全体合伙人的决定执行事务的,其他合 伙人可以决定撤销该委托。
	(5 )异议权 合伙人分别执行合伙事务的,执行事务合伙人可以对其他合伙人执行的事务提出异议。提 出异议时,应当暂停该项事务的执行。如果发生争议,按照合伙协议约定的表决办法办理,合 伙协议未约定或者约定不明确的,实行合伙人一人一票并经全体合伙人过半数通过的表决办法。
	【解释】不执行合伙事务的合伙人有监督权、撤销委托权、查阅账簿权,但无异议权。 异议权是指当两个以上的合伙人分别执行合伙事务的(一个负责卖西瓜,一个负责卖南 瓜),执行事务合伙人可以对其他合伙人执行的事务提出异议。异议权的行使主体只能是 执行合伙事务的合伙人。
	
	3. 重大事项的决议办法 除合伙协议另有约定外,合伙企业的下列事项应当经全体合伙人一致同意: (1 )改 变 合 伙 企 业 的 名 称 ;
	(2 )改变合伙企业的经营范围、主要经营场所的地点;
	(3 )处 分 合 伙 企 业 的 不 动 产 ;
	(4 )转让或者处分合伙企业的知识产权和其他财产权利; (5 )以合伙企业名义为他人提供担保; (6)聘任合伙人以外的人担任合伙企业的经营管理人员。
	【解释】重大事项的决议办法,先约定后法定:先看合伙协议是否另有约定(爱怎么 约定就怎么约定),合伙协议没有约定的,才看法定(经全体合伙人一致同意)。
	
	4. 非合伙人参与经营管理 (1)除合伙协议另有约定外,经全体合伙人一致同意,可以聘任合伙人以外的人担任合伙 企业的经营管理人员。 (2)被聘任的合伙企业的经营管理人员,超越合伙企业授权范围履行职务,或者在履行职 务过程中因故意或者重大过失给合伙企业造成损失的,依法承担赔偿责任。 【解释】合伙人以外的经营管理人员属于非合伙人,无须对企业债务承担无限连带责任。
	
	5. 合伙企业的损益分配 (1)合伙企业的利润分配、亏损分担,按照合伙协议的约定办理。
	(2 )合伙协议未约定或者约定不明确的,由合伙人协商决定;协商不成的,由合伙人按照 实缴出资比例分配、分担;无法确定出资比例的,由合伙人平均分配、分担。 (3)合伙协议不得约定将全部利润分配给部分合伙人或者由部分合伙人承担全部亏损。该 规定体现 了普通合伙人应当共同承担企业经营风险的原则。
	【解 释 1 】 亏 损 是 指 以 合 伙 企 业 的 名 义 从 事 经 营 活 动 所 形 成 的 亏 损 ( 利 润 为 负 数 的
	状 态 )。
	【解释2 】合伙企业损益分配原则:约定一协商一实缴出资比例一平均。
	【解释3 】合伙企业是契约式企业,损益分配首先看合伙协议的约定,合伙协议爱怎么 约定就怎么约定,但有 一个例外,普通合伙企业的合伙协议绝对不得约定将全部利润分配 给部分合伙人 或者由部分合伙人承 担全 部亏损。
	
	\subsubsection{合伙企业与第三人的关系}
	1. 对外代表权的效力 合伙企业对合伙人执行合伙事务以及对外代表合伙企业权利的限制,不得对抗善意第三人。
	【相关链接1】合伙人在合伙企业清算前私自转移或者处分合伙企业财产的,合伙企业 不得以此对抗善意第三人。
	【相关链接2 】普通合伙人以其在合伙企业中的财产份额出质的,须经其他合伙人 一致 同意;未经其他合伙人一致同意,其行为无效,由此给善意第三人造成损失的 ,由行为人 依法承担赔偿责任。
	2. 企业的债务清偿
	(1 )先企业后个人 合伙企业对其债务,应先以其全部财产进行清偿。
	
	【解释】只有合伙企业以其全部财产不能清偿到期债务时,普通合伙人才承担无限连 带 责任。
	(2 )无限连带责任 合伙人之间的分担比例对债权人没有约束力。债权人可以根据自己的清偿利益,请求全体合 伙人中的 一人或者数人承担全部清偿责任,也可以按照自己确定的清偿比例向各合伙人分别追索。 【解释】债权人爱找谁找谁,爱要多少要多少。
	(3 )内部追偿 如果某一合伙人实际支付的清偿数额超过其依照既定比例所应承担的数额,该合伙人有权 就超过部分向其他未支付或者未足额支付应承担数额的合伙人追偿。
	【解释】普通合伙人对外承担连带责任,对内承担按份责任。合伙人之间进行内部追 偿时,不能爱找谁找谁,爱要多少要多少。
	
	3. 合伙人的债务清偿 (1)合伙人发生与合伙企业无关的债务,相关债权人不得以其债权抵销其对合伙企业的债 务;也不得代位行使合伙人在合伙企业中的权利。
	(2 )合伙人的自有财产不足清偿其与合伙企业无关的债务的,该合伙人可以以其从合伙企 业中分取的收益用于清偿;债权人也可以依法请求人民法院强制执行该合伙人在合伙企业中的 财产份额用于清偿。
	(3 )债权人不能自行接管或者直接变卖该合伙人在合伙企业中的财产份额。
	(4 )人民法院强制执行普通合伙人的财产份额时,应当通知全体合伙人,其他合伙人有优 先购买权;其他合伙人未购买,又不同意将该财产份额转让给他人的,依法为该合伙人办理退 伙结算,或者办理削减该合伙人相应财产份额的结算。
	
	\subsubsection{退伙的类型}
	 点 0 7 退 伙 的 类 型 重要程度 | * * *
	【 解 释 】 退 伙 包 括 自 愿 退 伙 ( 协 议 退 伙 、 通 知 退 伙 ) 和 强 制 退 伙 ( 当 然 退 伙 、 除 名 )。
	1. 协议退伙 合伙协议约定合伙期限的,在合伙企业存续期间,有下列情形之一的,合伙人可以退伙: (1 )合伙协议约定的退伙事由出现;
	(2 )经全体合伙人 一致同意;
	(3 )发生合伙人难以继续参加合伙的事由;
	(4 )其他合伙人严重违反合伙协议约定的义务。
	2. 通知退伙 合伙协议未约定合伙期限的,合伙人在不给合伙企业事务执行造成不利影响的情况下,可 以退伙,但应当提前30 日通知其他合伙人。 【解释】协议退伙与通知退伙的主要区别是:合伙协议是否约定了合伙期限。
	3. 当然退伙
	(1 )作为合伙人的自然人死亡或者被依法宜告死亡;
	(2 )个人丧失偿债能力;
	(3 )作为合伙人的法人或者其他组织依法被吊销营业执照、责令关闭、撤销,或者被宣告 破产;
	(4 )法律规定或者合伙协议约定合伙人必须具有相关资格而丧失该资格;
	(5 )合伙人在合伙企业中的全部财产份额被人民法院强制执行。
	【解释1】合伙人被依法宣告失踪,不属于当然退伙的法定事由。为主张权利而申请宣 告义务人失踪或者死亡,引起诉讼时效的中断。
	【解释2 】合伙人被宣告破产,当然退伙;如果只是被债权人申请破产,是否被宣告破 产并不确定,并不导致当然退伙。
	【解释3 】退伙事由实际发生之日为退伙生效日。例如,合伙人甲公司于2024 年4 月 1 日被人民法院宣告破产,其退伙生效日为2024 年4 月1 日。
	【解释4 】(1 )普通合伙人丧失偿债能力的,当然退伙;(2 )有限合伙人丧失偿债能力 的,无须退伙。
	【解释5 】(1)普通合伙人被依法认定为无民事行为能力人或者限制民事行为能力人 的,经其他合伙人一致同意,可以依法转汐有限合伙人,普通合伙企业依法转为有限合伙 企业;其他合伙人未能一致同意的,该无民事行为能力或者限制民事行为能力的合伙人当 然退伙;(2 )作为有限合伙人的自然人在有限合伙企业存续期间丧失民事行为能力的,其 他合伙人不得因此要求其退伙。
	
	4. 除名 合伙人有下列情形之一的,经其他合伙人一致同意,可以决议将其除名: (1 )未履行出资义务;
	(2 )因故意或者重大过失给合伙企业造成损失; (3 )执行合伙事务时有不正当行为; (4)发生合伙协议约定的事由。
	【解释】对合伙人的除名决议应当书面通知被除名人。被除名人接到除名通知之日, 除名生效,被除名人退伙。被除名人对除名决议有异议的,可以自接到除名通知之日起 30 日内,向人民法院起诉。
	
	
	
	\subsubsection{财产继承}
	1. 合伙人死亡或者被依法宣告死亡的,对该合伙人在合伙企业中的财产份额享有合法继 承权的继承人,依照合伙协议的约定或者经全体合伙人同意,从继承开始之日起,即取得该合 伙企业的合伙人资格。若有数个继承人,数人只能作为一个整体继承被继承人的合伙份额,否 则就会破坏合伙企业原有的结构。合法继承人不愿意成为该合伙企业的合伙人的,合伙企业应 退还其依法继承的财产份额。
	2. 能否成为合伙人?
	(1 )有 限 合 伙 人 作为有限合伙人的自然人死亡、被依法宣告死亡或者作为有限合伙人的法人及其他组织终 止时,其继承人或者权利承受人可以依法取得该有限合伙人在有限合伙企业中的资格。
	【解释】有限合伙人不执行合伙事务,无须对企业债务承担无限连带责任,因此 ,有 限合伙人死亡后,无论其继承人是否具备完全民事行为能力,都可以依法取得该有限合伙 人在有限合伙企业中的资格。
	(2 )普通合伙 人 1继承人具备完全民事行为能力的,按照合伙协议的约定或者经全体合伙人一致同意,从 继承开始之日起,可以取得普通合伙人资格。 2继承人为无民事行为能力人或者限制民事行为能力人的,经全体合伙人一致同意,可以 依法成为有限合伙人,普通合伙企业依法转为有限合伙企业。全体合伙人未能一致同意的,合 伙企业应当将被继承合伙人的财产份额退还该继承人。
	【解释】普通合伙人应具备承担无限连带责任的能力:(1)如果其继承人为无民事行 能力人或者限制民事行为能力人的,不可能成为普通合伙人,但有机会成为有限合伙人。
	经全体合伙人一致同意,可以依法成为有限合伙人(而非自动取得有限合伙人的资格), 此时普通合伙企业依法转为有限合伙企业;(2 )如果其继承人具备完全民事行为能力,有 机会成为普通合伙人,而非自动取得普通合伙人的资格。先看合伙协议对普通合伙人的继 承资格问题有没有约定,有约定按照约定处理。如果合伙协议对此没有约定,经全体合伙 人一致同意,才能取得普通合伙人的资格;如果不能取得 一致同意,合伙企业应当向其退 还被继承合伙人的财产份额。此时,该继承人能否成为有限合伙人?法律并未明确规定, 希望考生就此打住,千万别作茧自缚。
	
	
	\subsubsection{特殊的普通合伙企业}
	【解释】特殊的普通合伙企业(如会计师事务所、律师事务所)适用普通合伙企业的 一般规定,其特珠性主要体现在“债务责任的承担” 上。注意该考点在案例分析题中可以 跟上市公司虚假陈述相结合。
	1. 普通债务 合伙人在执业活动中非因故意或者重大过失造成的合伙企业债务以及合伙企业的其他债 务,由全体合伙人承担无限连带责任。
	2. 特定债务
	(1 )对 外 一个合伙人或者数个合伙人在执业活动中因故意或者重大过失造成合伙企业债务的,应当 承担无限责任或者无限连带责任,其他合伙人以其在合伙企业中的财产份额为限承担责任。 (2 )对内
	合伙人在执业活动中因故意或者重大过失造成的合伙企业债务,以合伙企业财产对外承担 责任后,该合伙人应当按照合伙协议的约定对给合伙企业造成的损失承担赔偿责任。
	3. 执业风险防范
	(1)特殊的普通合伙企业应当建立执业风险基金,办理职业责任保险。
	(2 )执业风险基金,主要是指为了化解经营风险,特 的普通合伙企业从其经营收益中提 取相应比例的资金,用于偿付合伙人执业活动造成的债务。执业风险基金应当单独立户管理。
	 
	
	\subsection{有限合伙企业}
	
	\subsubsection{有限合伙企业的设立}
	
	\subsubsection{有限合伙企业的合伙协议}
	
	\subsubsection{有限合伙企业的事务执行}
	
	\subsubsection{交易}
	
	\subsubsection{竞争}
	
	\subsubsection{出质}
	
	\subsubsection{财产份额的对外转让}
	
	\subsubsection{有限合伙人的债务清偿}
	
	\subsubsection{入伙}
	
	\subsubsection{退伙}
	
	\subsubsection{合伙人性质的转变}
	
	\subsubsection{法定事项与约定事项}
	
	\subsubsection{合伙企业的设立登记}
	
	\subsubsection{合伙企业的解散与清算}
	

	 
	
	    
	
	\newpage
	\section{公司法律制度}
	
	
	\newpage
	\section{证券法律制度}
	
	\newpage
	\section{企业破产法律制度}
	
	\newpage
	\section{票据与支付结算法律制度}
	
	\newpage
	\section{企业国有资产法律制度}
	
	\newpage
	\section{反垄断法律制度}
	
	\newpage
	\section{涉外法律制度}
	
	\newpage
	\section{附录}
	\subsection{所有和时间有关的知识点}
	
\end{document}