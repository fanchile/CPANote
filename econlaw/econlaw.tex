\documentclass[UTF8,12pt]{ctexart}
\usepackage{amsmath,amssymb,geometry,bm,graphicx,fontspec,amssymb,amsthm}
\usepackage[mathscr]{euscript}

\usepackage[colorlinks,
linkcolor=black,
anchorcolor=blue,
citecolor=green
]{hyperref} % 目录中的超链接

\newtheorem{Def}{定义}[section]
\newtheorem{Theo}{定理}[section]
\newtheorem{Lemm}{引理}[section]
\newtheorem{Prop}{命题}[section]
\newtheorem{Assu}{假设}[section]
\newtheorem{Axiom}{Axiom}

\numberwithin{equation}{section} % 按章节进行排序与编号
\numberwithin{figure}{section}
\numberwithin{table}{section}

\usepackage{draftwatermark} % 所有页加水印
\SetWatermarkText{EconNerd} % 设置水印内容
\SetWatermarkLightness{0.99} % 设置水印透明度 0-1
\SetWatermarkScale{1} % 设置水印大小    

\title{经济法} % 文档相关信息
\author{EconNerd}
\date{\today}
\geometry{scale=0.8}

\begin{document}
	\maketitle
	\tableofcontents
	\newpage
	
	\section{法律基本原理}
	
	\subsection{法律基本概念}
	\subsubsection{我国的法律渊源}
	我国的法律渊源主要表现为制定法,不包括判例法。在效力等级上:(1)宪法>法律>行政法规>地方性法规>本级和下级地方政府规章;(2)宪法>法律>行政法规>部门规章
	
	\paragraph{宪法}由全国人民代表大会制定
	
	\paragraph{法律}
	\begin{enumerate}
		\item 基本法律:由全国人民代表大会制定
		
		\item 一般法律:由全国人民代表大会常务委员会制定
	\end{enumerate}
	全国人民代表大会制定和修改的,调整国家和社会生活中带有普遍性的社会关系的规范性法律文件,属于基本法律。如《刑法》、《民法总则》。
	
	全国人民代表大会常务委员会制定和修改的,调整国家和社会生活中某一方面社会关系的规范性法律文件,属于一般法律。如《公司法》、《证券法》。
	
	在全国人民代表大会闭会期间,全国人民代表大会常务委员会可对基本法律进行部分补充和修改,但是不得同该法律的基本原则相抵触。
	
	全国人民代表大会常务委员会负责解释法律,其作出的法律解释与法律具有同等效力。例如,2016年11月7日,全国人民代表大会常务委员会对《香港特别行政区基本法》第104条进行了解释,明确了相关公职人员“就职时必须依法宣誓”的具体含义。
	
	\paragraph{法规}
	\begin{enumerate}
		\item 行政法规:由国务院制定
		
		\item 地方性法规:由有地方立法权的地方人民代表大会及其常务委员会制定
	\end{enumerate}
	省、自治区、直辖市的人民代表大会及其常务委员会,有权制定地方性法规;自治州和设区的市的人民代表大会及其常务委员会有权对\textbf{城乡建设与管理、环境保护、历史文化保护}等方面的事项制定地方性法规。
	
	\paragraph{规章}
	\begin{enumerate}
		\item 部门规章:由国务院各部、委员会、中国人民银行、审计署和具有行政管理职能的直属机构制定
		
		\item 地方政府规章:由有地方立法权的地方人民政府制定
	\end{enumerate}
	自治州和设区的市的人民政府有权就城乡建设与管理、环境保护、历史文化保护等方面的事项制定地方政府规章。
	
	没有法律或者国务院的行政法规、决定、命令的依据,部门规章不得设定减损公民、法人和其他组织权利或者增加其义务的规范,不得增加本部门的权力或者减少本部门的法定职责。
	
	没有法律、行政法规、地方性法规的依据,地方政府规章不得设定减损公民、法人和其他组织权利或者增加其义务的规范。
	
	\paragraph{司法解释}由最高人民法院、最高人民检察院制定
	最高人民法院和最高人民检察院的解释如果有原则性的分歧,报请全国人民代表大会常务委员会解释或者决定。法律解释权归全国人民代表大会常务委员会,司法解释权归最高人民法院和最高人民检察院。

	
	\subsubsection{法律规范}
	\paragraph{法律规范的概念}法律规范是由国家制定或者认可的,具体规定主体权利、义务及法律后果的行为准则。
	
	\paragraph{法律规范与法律条文}
	\begin{enumerate}
		\item 法律规范不同于法律条文,法律规范是法律条文的内容,法律条文是法律规范的表现形式。
		
		\item 法律规范是法律条文的主要内容,但法律条文的内容还可能包含其他法要素(如法律原则)。
		
		\item 法律规范与法律条文不是一一对应的,一项法律规范的内容可以表现在不同法律条文甚至不同的规范性法律文件中。同样,一个法律条文中也可以反映若干法律规范的内容。
	\end{enumerate}
	
	\paragraph{法律规范的种类}
	\begin{enumerate}
		\item 授权性规范和义务性规范。这是根据法律规范为主体提供行为模式的不同方式进行的区分。其中,义务性规范可分为命令性规范和禁止性规范。
		\begin{enumerate}
			\item 授权性规范是规定人们可以作出一定行为或者可以要求别人作出一定行为的法律规范。授权性规范的立法语言表达式为“可以……”、“有权……”、“享有……权利”等。
			
			\item 命令性规范是指规定人们的积极义务,即规定主体应当或者必须作出一定积极行为的规范。命令性规范的立法语言表达式为“应当……”、“必须……”、“有……义务”等。
			
			\item 禁止性规范是指规定人们的消极义务(不作为义务),即禁止人们作出一定行为的规范。禁止性规范的立法语言表达式为“不得……”、“禁止……”等。
		\end{enumerate}
		
		\item 强行性规范和任意性规范。
		\begin{enumerate}
			\item 强行性规范是指所规定的义务具有确定的性质,不允许任意变动和伸缩。
			
			\item 任意性规范是指在法定范围内允许行为人自行确定其权利义务的具体内容。
		\end{enumerate}
		
		
		\item 确定性规范和非确定性规范。
		\begin{enumerate}
			\item 确定性规范是指内容已经完备明确,无须再援引或者参照其他规范来确定其内容的法律规范。
			
			\item 非确定性规范是指没有明确具体的行为模式或者法律后果,需要引用其他法律规范来说明或者补充的规范,具体包括委任性规范和准用性规范。
		\end{enumerate}
		
	\end{enumerate}
	
	
	\subsection{法律关系}
	\subsubsection{法律关系的主体}
	\paragraph{法律关系的概念}
	法律关系是根据法律规范产生的,以主体之间的权利与义务为内容的特殊的社会关系(如合同关系)。法律关系包括三个要素:主体、内容和客体。
	
	并非所有的社会关系均属于法律关系。法律关系(如夫妻关系)是以相应法律规范的存在为前提的社会关系,没有相应的法律规范就不可能产生相应的法律关系。例如,同学关系、恋人关系,因不存在相应的法律规范,也就不存在相应的法律关系。
	
	\paragraph{法律关系主体的种类}
	\begin{enumerate}
		\item 自然人
		
		\item 法人和非法人组织
		
		\item 国家
	\end{enumerate}

	自然人既包括本国公民,也包括居住在一国境内或者在境内活动的外国公民和无国籍人。国家可以直接以自己的名义参与国内法律关系(如发行国债)。
	
	\paragraph{法人的分类}
	\begin{enumerate}
		\item 法人分为营利法人、非营利法人和特别法人。
		
		\item 营利法人包括有限责任公司、股份有限公司和其他企业法人等。
		
		\item 非营利法人包括事业单位、社会团体、基金会、社会服务机构等。
		
		\item 特别法人包括特定的机关法人、农村集体经济组织法人、城镇农村的合作经济组织法人、基层群众性自治组织法人。
	\end{enumerate}
	
	\paragraph{非法人组织}
	\begin{enumerate}
		\item 非法人组织是不具有法人资格,但是能够依法以自己的名义从事民事活动的组织。
	
		\item 非法人组织包括个人独资企业、合伙企业、不具有法人资格的专业服务机构等。
	\end{enumerate}
	
	\subsubsection{权利能力和行为能力}
	\paragraph{权利能力和行为能力}
	\begin{enumerate}
		\item 权利能力是指权利主体享有权利和承担义务的能力,它反映了权利主体取得权利和承担义务的资格。行为能力是指权利主体能够通过自己的行为取得权利和承担义务的能力。
		
		\item 法律关系主体要自己参与法律活动,必须具备相应的行为能力。
		
		\item 行为能力必须以权利能力为前提,无权利能力就谈不上行为能力。
		
		\item 作为民事法律关系主体的法人,其权利能力从法人成立时产生,其行为能力伴随着权利能力的产生而同时产生;法人终止时,其权利能力和行为能力同时消灭。
		
		\item 自然人从出生时起到死亡时止,具有民事权利能力,依法享有民事权利,承担民事义务。自然人的民事权利能力一律平等。
	\end{enumerate}
	
	
	\paragraph{自然人的民事行为能力}
	\begin{enumerate}
		\item 完全民事行为能力人。
		\begin{enumerate}
			\item 18周岁以上(≥18周岁)的自然人为成年人,成年人为完全民事行为能力人,可以独立实施民事法律行为。
			
			\item 16周岁以上(≥16周岁)的未成年人,以自己的劳动收入为主要生活来源的,视为完全民事行为能力人,可以独立实施民事法律行为。
		\end{enumerate}
		
		\item 限制民事行为能力人。8周岁以上(≥8周岁)的未成年人和不能完全辨认自己行为的成年人为限制民事行为能力人。
		
		\item 无民事行为能力人。不满8周岁(<8周岁)的未成年人,不能辨认自己行为的成年人,以及8周岁以上的未成年人不能辨认自己行为的,为无民事行为能力人,由其法定代理人代理实施民事法律行为。
	\end{enumerate}
	无民事行为能力人、限制民事行为能力人的监护人是其法定代理人。
	
	\subsubsection{法律关系的客体}
	法律关系的客体,是指法律关系主体间权利义务所指向的对象。法律关系的客体通常包括:
	\begin{enumerate}
		\item 物。物(如土地、机器设备、货币、有价证券)是物权法律关系的客体。
		
		\item 行为。行为包括作为和不作为(如竞业禁止合同的客体是不从事相同或者相似的经营活动)。给付行为是债权法律关系(如合同之债)的客体。
		
		\item 人格利益。人格利益(如公民的肖像、名誉、人身)是人身权法律关系的客体,也是诸多行政、刑事法律关系的客体。
		
		\item 智力成果。智力成果(如文学艺术作品、科学著作、专利、商标)是知识产权法律关系的客体。
	\end{enumerate}
	
	
	\subsubsection{法律事实}
	法律事实是指法律规范所规定的,能够引起法律关系产生、变更或者消灭的客观现象。根据是否以权利主体的意志为转移,法律事实分为行为和事件两类。
	\begin{enumerate}
		\item 事件(与当事人的意志无关)
		(1)人的出生与死亡
		(2)自然灾害与意外事件
		(3)时间的经过
		
		\item 行为
		(1)法律行为(以行为人的\textbf{意思表示}为要素,如订立合同)
		(2)事实行为(与意思表示无关,如创作行为、侵权行为)
	\end{enumerate}
	
	
	
	\subsection{全面依法治国基本方略}
	
	\section{基本民事法律制度}
	
	\subsection{民事法律行为制度}
	
	
	\subsection{代理制度}
	
	
	\subsection{诉讼时效制度}
	
\end{document}