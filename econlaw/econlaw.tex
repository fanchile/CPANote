\documentclass[UTF8,12pt]{ctexart}
\usepackage{amsmath,amssymb,geometry,bm,graphicx,fontspec,amssymb,amsthm}
\usepackage[mathscr]{euscript}

\usepackage{tabularray}

\usepackage[colorlinks,
linkcolor=black,
anchorcolor=blue,
citecolor=green
]{hyperref} % 目录中的超链接

\newtheorem{Def}{定义}[section]
\newtheorem{Theo}{定理}[section]
\newtheorem{Lemm}{引理}[section]
\newtheorem{Prop}{命题}[section]
\newtheorem{Assu}{假设}[section]
\newtheorem{Axiom}{Axiom}

\numberwithin{equation}{section} % 按章节进行排序与编号
\numberwithin{figure}{section}
\numberwithin{table}{section}

\usepackage{draftwatermark} % 所有页加水印
\SetWatermarkText{EconNerd} % 设置水印内容
\SetWatermarkLightness{0.99} % 设置水印透明度 0-1
\SetWatermarkScale{1} % 设置水印大小    

\title{经济法} % 文档相关信息
\author{EconNerd}
\date{\today}
\geometry{scale=0.8}

\begin{document}
	\maketitle
	\tableofcontents
	\newpage
	
	\section{法律基本原理}
	
	\subsection{法律基本概念}
	\subsubsection{我国的法律渊源}
	我国的法律渊源主要表现为制定法,不包括判例法。在效力等级上:(1)宪法>法律>行政法规>地方性法规>本级和下级地方政府规章;(2)宪法>法律>行政法规>部门规章
	
	\paragraph{宪法}由全国人民代表大会制定
	
	\paragraph{法律}
	\begin{enumerate}
		\item 基本法律:由全国人民代表大会制定
		
		\item 一般法律:由全国人民代表大会常务委员会制定
	\end{enumerate}
	全国人民代表大会制定和修改的,调整国家和社会生活中带有普遍性的社会关系的规范性法律文件,属于基本法律。如《刑法》、《民法总则》。
	
	全国人民代表大会常务委员会制定和修改的,调整国家和社会生活中某一方面社会关系的规范性法律文件,属于一般法律。如《公司法》、《证券法》。
	
	在全国人民代表大会闭会期间,全国人民代表大会常务委员会可对基本法律进行部分补充和修改,但是不得同该法律的基本原则相抵触。
	
	全国人民代表大会常务委员会负责解释法律,其作出的法律解释与法律具有同等效力。例如,2016年11月7日,全国人民代表大会常务委员会对《香港特别行政区基本法》第104条进行了解释,明确了相关公职人员“就职时必须依法宣誓”的具体含义。
	
	\paragraph{法规}
	\begin{enumerate}
		\item 行政法规:由国务院制定
		
		\item 地方性法规:由有地方立法权的地方人民代表大会及其常务委员会制定
	\end{enumerate}
	省、自治区、直辖市的人民代表大会及其常务委员会,有权制定地方性法规;自治州和设区的市的人民代表大会及其常务委员会有权对\textbf{城乡建设与管理、环境保护、历史文化保护}等方面的事项制定地方性法规。
	
	\paragraph{规章}
	\begin{enumerate}
		\item 部门规章:由国务院各部、委员会、中国人民银行、审计署和具有行政管理职能的直属机构制定
		
		\item 地方政府规章:由有地方立法权的地方人民政府制定
	\end{enumerate}
	自治州和设区的市的人民政府有权就城乡建设与管理、环境保护、历史文化保护等方面的事项制定地方政府规章。
	
	没有法律或者国务院的行政法规、决定、命令的依据,部门规章不得设定减损公民、法人和其他组织权利或者增加其义务的规范,不得增加本部门的权力或者减少本部门的法定职责。
	
	没有法律、行政法规、地方性法规的依据,地方政府规章不得设定减损公民、法人和其他组织权利或者增加其义务的规范。
	
	\paragraph{司法解释}由最高人民法院、最高人民检察院制定
	最高人民法院和最高人民检察院的解释如果有原则性的分歧,报请全国人民代表大会常务委员会解释或者决定。法律解释权归全国人民代表大会常务委员会,司法解释权归最高人民法院和最高人民检察院。

	
	\subsubsection{法律规范}
	\paragraph{法律规范的概念}法律规范是由国家制定或者认可的,具体规定主体权利、义务及法律后果的行为准则。
	
	\paragraph{法律规范与法律条文}
	\begin{enumerate}
		\item 法律规范不同于法律条文,法律规范是法律条文的内容,法律条文是法律规范的表现形式。
		
		\item 法律规范是法律条文的主要内容,但法律条文的内容还可能包含其他法要素(如法律原则)。
		
		\item 法律规范与法律条文不是一一对应的,一项法律规范的内容可以表现在不同法律条文甚至不同的规范性法律文件中。同样,一个法律条文中也可以反映若干法律规范的内容。
	\end{enumerate}
	
	\paragraph{法律规范的种类}
	\begin{enumerate}
		\item 授权性规范和义务性规范。这是根据法律规范为主体提供行为模式的不同方式进行的区分。其中,义务性规范可分为命令性规范和禁止性规范。
		\begin{enumerate}
			\item 授权性规范是规定人们可以作出一定行为或者可以要求别人作出一定行为的法律规范。授权性规范的立法语言表达式为“可以……”、“有权……”、“享有……权利”等。
			
			\item 命令性规范是指规定人们的积极义务,即规定主体应当或者必须作出一定积极行为的规范。命令性规范的立法语言表达式为“应当……”、“必须……”、“有……义务”等。
			
			\item 禁止性规范是指规定人们的消极义务(不作为义务),即禁止人们作出一定行为的规范。禁止性规范的立法语言表达式为“不得……”、“禁止……”等。
		\end{enumerate}
		
		\item 强行性规范和任意性规范。
		\begin{enumerate}
			\item 强行性规范是指所规定的义务具有确定的性质,不允许任意变动和伸缩。
			
			\item 任意性规范是指在法定范围内允许行为人自行确定其权利义务的具体内容。
		\end{enumerate}
		
		
		\item 确定性规范和非确定性规范。
		\begin{enumerate}
			\item 确定性规范是指内容已经完备明确,无须再援引或者参照其他规范来确定其内容的法律规范。
			
			\item 非确定性规范是指没有明确具体的行为模式或者法律后果,需要引用其他法律规范来说明或者补充的规范,具体包括委任性规范和准用性规范。
		\end{enumerate}
		
	\end{enumerate}
	
	
	\subsection{法律关系}
	\subsubsection{法律关系的主体}
	\paragraph{法律关系的概念}
	法律关系是根据法律规范产生的,以主体之间的权利与义务为内容的特殊的社会关系(如合同关系)。法律关系包括三个要素:主体、内容和客体。
	
	并非所有的社会关系均属于法律关系。法律关系(如夫妻关系)是以相应法律规范的存在为前提的社会关系,没有相应的法律规范就不可能产生相应的法律关系。例如,同学关系、恋人关系,因不存在相应的法律规范,也就不存在相应的法律关系。
	
	\paragraph{法律关系主体的种类}
	\begin{enumerate}
		\item 自然人
		
		\item 法人和非法人组织
		
		\item 国家
	\end{enumerate}

	自然人既包括本国公民,也包括居住在一国境内或者在境内活动的外国公民和无国籍人。国家可以直接以自己的名义参与国内法律关系(如发行国债)。
	
	\paragraph{法人的分类}
	\begin{enumerate}
		\item 法人分为营利法人、非营利法人和特别法人。
		
		\item 营利法人包括有限责任公司、股份有限公司和其他企业法人等。
		
		\item 非营利法人包括事业单位、社会团体、基金会、社会服务机构等。
		
		\item 特别法人包括特定的机关法人、农村集体经济组织法人、城镇农村的合作经济组织法人、基层群众性自治组织法人。
	\end{enumerate}
	
	\paragraph{非法人组织}
	\begin{enumerate}
		\item 非法人组织是不具有法人资格,但是能够依法以自己的名义从事民事活动的组织。
	
		\item 非法人组织包括个人独资企业、合伙企业、不具有法人资格的专业服务机构等。
	\end{enumerate}
	
	\subsubsection{权利能力和行为能力}
	\paragraph{权利能力和行为能力}
	\begin{enumerate}
		\item 权利能力是指权利主体享有权利和承担义务的能力,它反映了权利主体取得权利和承担义务的资格。行为能力是指权利主体能够通过自己的行为取得权利和承担义务的能力。
		
		\item 法律关系主体要自己参与法律活动,必须具备相应的行为能力。
		
		\item 行为能力必须以权利能力为前提,无权利能力就谈不上行为能力。
		
		\item 作为民事法律关系主体的法人,其权利能力从法人成立时产生,其行为能力伴随着权利能力的产生而同时产生;法人终止时,其权利能力和行为能力同时消灭。
		
		\item 自然人从出生时起到死亡时止,具有民事权利能力,依法享有民事权利,承担民事义务。自然人的民事权利能力一律平等。
	\end{enumerate}
	
	
	\paragraph{自然人的民事行为能力}
	\begin{enumerate}
		\item 完全民事行为能力人。
		\begin{enumerate}
			\item 18周岁以上(≥18周岁)的自然人为成年人,成年人为完全民事行为能力人,可以独立实施民事法律行为。
			
			\item 16周岁以上(≥16周岁)的未成年人,以自己的劳动收入为主要生活来源的,视为完全民事行为能力人,可以独立实施民事法律行为。
		\end{enumerate}
		
		\item 限制民事行为能力人。8周岁以上(≥8周岁)的未成年人和不能完全辨认自己行为的成年人为限制民事行为能力人。
		
		\item 无民事行为能力人。不满8周岁(<8周岁)的未成年人,不能辨认自己行为的成年人,以及8周岁以上的未成年人不能辨认自己行为的,为无民事行为能力人,由其法定代理人代理实施民事法律行为。
	\end{enumerate}
	无民事行为能力人、限制民事行为能力人的监护人是其法定代理人。
	
	\subsubsection{法律关系的客体}
	法律关系的客体,是指法律关系主体间权利义务所指向的对象。法律关系的客体通常包括:
	\begin{enumerate}
		\item 物。物(如土地、机器设备、货币、有价证券)是物权法律关系的客体。
		
		\item 行为。行为包括作为和不作为(如竞业禁止合同的客体是不从事相同或者相似的经营活动)。给付行为是债权法律关系(如合同之债)的客体。
		
		\item 人格利益。人格利益(如公民的肖像、名誉、人身)是人身权法律关系的客体,也是诸多行政、刑事法律关系的客体。
		
		\item 智力成果。智力成果(如文学艺术作品、科学著作、专利、商标)是知识产权法律关系的客体。
	\end{enumerate}
	
	
	\subsubsection{法律事实}
	法律事实是指法律规范所规定的,能够引起法律关系产生、变更或者消灭的客观现象。根据是否以权利主体的意志为转移,法律事实分为行为和事件两类。
	\begin{enumerate}
		\item 事件(与当事人的意志无关)
		(1)人的出生与死亡
		(2)自然灾害与意外事件
		(3)时间的经过
		
		\item 行为
		(1)法律行为(以行为人的\textbf{意思表示}为要素,如订立合同)
		(2)事实行为(与意思表示无关,如创作行为、侵权行为)
	\end{enumerate}
	
	
	
	\subsection{全面依法治国基本方略}
	
	\newpage
	
	\section{基本民事法律制度}
	
	\subsection{民事法律行为制度}
	\paragraph{民事法律行为的分类}
	民事法律行为是民事主体通过\textbf{意思表示}设立、变更、终止民事法律关系的行为。民事法律行为是法律关系变动的原因之一,是民法最重要的\textbf{法律事实}
	
	民事法律行为是\textbf{有目的}的行为(包括后果、不包括动机),此处的“目的”仅指当事人实施民事法律行为所追求的法律后果,不包括行为人实施行为的动机。这一特征使得民事法律行为区别于其他法律事实,如侵权行为。侵权行为虽然也产生一定的法律后果,但该法律后果并非由当事人自己主张,而是由法律规定的
	
	民事法律行为有很多中分类方式,最基础的可以分为以下几类
	\begin{enumerate}
		\item 单方民事法律行为,是指根据一方当事人的意思表示而成立的民事法律行为(如订立遗嘱、撤销权的行使、解除权的行使、效力待定行为的追认等)。
		
		\item 双方民事法律行为,是指两个当事人之间意思表示一致而成立的民事法律行为。
		
		\item 多方民事法律行为,是指三个以上的当事人意思表示一致而成立的民事法律行为。
	\end{enumerate}
	
	合同是常见的双方民事法律行为,决议则是典型的多方民事法律行为。决议是指多个主体依据表决规则作出的决定,决议当事人的意思表示可以多数决的方式作出,而且对没有表示同意的成员也具有拘束力;决议中的意思表示不仅对发出表示的成员有拘束力,而且主要对表示者共同代表的法人有拘束力。例如公司股东会依法作出的决议,对全体股东(包括投反对票的股东、弃权的股东、未出席会议的股东)均有约束力
	
	负担行为与处分行为
	\begin{enumerate}
		\item \textbf{负担行为}是使一方(义务人)相对于他方(权利人)承担一定给付义务的法律行为,这种给付义务既可以是作为的(如交付货物),也可以是不作为的(如保密义务)。因此负担行为产生的是债法上的法律效果,其中负有给付义务的主体是债务人。负担行为中的权利人可以享有要求履行的请求权,义务人的履行行为是请求权实现的重要前提。
		
		\item \textbf{处分行为}是直接导致权利发生变动的法律行为(如甲抛弃自己的动产),并不需要义务人积极履行给付义务。物权行为是典型的处分行为。
	\end{enumerate}
	
	要式民事法律行为与不要式民事法律行为
	\begin{enumerate}
		\item 要式民事法律行为,是指法律规定必须采取一定的形式或者履行一定的程序才能成立,否则民事法律行为不能成立。例如,票据行为属于法定要式民事法律行为。根据《民法典》的规定,融资租赁合同、建设工程合同应当采用书面形式。
		
		\item 不要式民事法律行为,是指法律不要求采取特定形式,当事人自由选择形式即可成立。例如,买卖合同为非要式合同。
	\end{enumerate}
	
	主民事法律行为与从民事法律行为
	(1)主民事法律行为(如借款合同)不成立,从民事法律行为(如担保合同)则不能成立;主民事法律行为无效,从民事法律行为当然不能生效。
	(2)主民事法律行为履行完毕,并不必然导致从民事法律行为效力的丧失。
	
	\paragraph{意思表示}
	民事法律行为以意思表示为核心,对于单方面民事法律行为可以分为有相对人(撤销权的行使、法定代理人的追认、授予代理权)和无相对人(遗嘱、抛弃动产)的意思表示
	
	有无相对人也影响了民事法律行为的生效时间。无相对人的意思表示,表示完成时生效。法律另有规定的,依照其规定。
	
	有相对人的意思表示分为对话的意思表示和非对话的意思表示。
	\begin{enumerate}
		\item 以对话方式作出的意思表示,相对人知道其内容时生效。
		
		\item 以非对话方式作出的意思表示,到达相对人时生效。订立合同过程中的要约和承诺、授予代理权、合同解除等意思表示,均采取到达主义。
		
		\item 以非对话方式作出的采用数据电文形式的意思表示,相对人指定特定系统接收数据电文的,该数据电文进入该特定系统时生效;未指定特定系统的,相对人知道或者应当知道该数据电文进入其系统时生效。当事人对采用数据电文形式的意思表示的生效时间另有约定的,按照其约定。
		
		\item 以公告方式作出的意思表示,公告发布时生效。
	\end{enumerate}
	
	行为人可以明示或者默示作出意思表示。沉默只有在有法律规定、当事人约定或者符合当事人之间的交易习惯时,才可以视为意思表示。(遗产中的沉默视为接受继承)
	
	行为人可以撤回意思表示。撤回意思表示的通知应当在意思表示到达相对人前或者与意思表示同时到达相对人。
	
	对意思表示的解释:有相对人应当按照所用词句,无相对人不能完全拘泥于所用词句
	
	\paragraph{民事法律行为的生效}
	民事法律行为主要可以分为四种,以下会分别进行介绍
	\begin{enumerate}
		\item 有效的
		
		\item 无效的
		
		\item 可撤销的
		
		\item 效力待定的
	\end{enumerate}
	
	民事法律行为有效需要分别满足实质要件和形式要件,实质要件包括
	\begin{enumerate}
		\item 行为人具有相应的民事行为能力;
		
		\item 行为人的意思表示真实;
		
		\item 不违反法律、行政法规的强制性规定,不违背公序良俗。
	\end{enumerate}
	
	形式要件包括
	\begin{enumerate}
		\item 民事法律行为可以采用\textbf{书面形式、口头形式或者其他形式}(如推定形式、沉默形式);法律、行政法规规定或者当事人约定采用特定形式的,应当采用特定形式。
		
		\item \textbf{推定形式},是指当事人并不直接用书面形式或者口头形式进行意思表示,而是通过实施某种\textbf{积极的行为},使得他人可以推定其意思表示。例如,王某在超市购物,王某向售货员交付货币的行为可以推定为王某具有购买商品的意思。
		
		\item \textbf{沉默形式},是指行为人没有以积极的作为进行意思表示,而是以\textbf{消极的不作为代替意思表示}。根据《民法典》的规定,沉默只有在有法律规定、当事人约定或者符合当事人之间的交易习惯时,才可以视为意思表示。
	\end{enumerate}
	
	\paragraph{无效的民事法律行为}
	无效的民事法律行为有着如下的三个特征
	\begin{enumerate}
		\item 自始无效:无效的民事法律行为从行为开始时就没有法律约束力。
		
		\item 当然无效:不论当事人是否主张,是否知道,也不论是否经过人民法院或者仲裁机构的确认,该民事法律行为当然无效。
		
		\item 绝对无效:无效的民事法律行为绝对不发生法律效力,不能通过当事人的行为进行补正。
	\end{enumerate}
	
	无效民事法律行为当事人通过一定行为消除无效原因,使之有效,这不是对无效民事法律行为的补正,而是消灭旧的民事法律行为,成立新的民事法律行为。
	
	无效的民事法律行为有以下几种
	\begin{enumerate}
		\item 无民事行为能力人独立实施的民事法律行为无效。
		
		\item 违背公序良俗的民事法律行为无效。
		
		\item 行为人与相对人恶意串通,损害他人合法权益的民事法律行为无效。
		
		\item 行为人与相对人以虚假的意思表示实施的民事法律行为无效。
		
		\item 违反法律、行政法规的强制性规定的民事法律行为无效,但是该强制性规定不导致该民事法律行为无效的除外。
	\end{enumerate}
	
	合同违反法律、行政法规的强制性规定,有下列情形之一,由行为人承担行政责任或者刑事责任能够实现强制性规定的立法目的的,人民法院可以认定该合同不因违反强制性规定无效:
	\begin{enumerate}
		\item 强制性规定虽然旨在维护社会公共秩序,但是合同的实际履行对社会公共秩序造成的影响显著轻微,认定合同无效将导致案件处理结果有失公平公正;
		
		\item 强制性规定旨在维护政府的税收、土地出让金等国家利益或者其他民事主体的合法利益而非合同当事人的民事权益,认定合同有效不会影响该规范目的的实现;
		
		\item 强制性规定旨在要求当事人一方加强风险控制、内部管理等,对方无能力或者无义务审查合同是否违反强制性规定,认定合同无效将使其承担不利后果;
		
		\item 当事人一方虽然在订立合同时违反强制性规定,但是在合同订立后其已经具备补正违反强制性规定的条件却违背诚信原则不予补正;
		
		\item 法律、司法解释规定的其他情形。
	\end{enumerate}
	
	法律、行政法规的强制性规定旨在规制合同订立后的履行行为,当事人以合同违反强制性规定为由请求认定合同无效的,人民法院不予支持。但是,合同履行必然导致违反强制性规定或者法律、司法解释另有规定的除外。(2024年新增)
	
	\paragraph{可撤销的民事法律行为}
	首先应当区别一下可撤销和无效的民事法律行为之间的区别
	\begin{enumerate}
		\item \textbf{法律效力不同}。可撤销的民事法律行为在撤销前已经生效,在被撤销之前,其法律效果可以对抗除撤销权人以外的任何人。而无效的民事法律行为在法律上当然无效,从一开始即不发生法律效力。
		
		\item \textbf{主张权利的主体不同}。可撤销的民事法律行为的撤销,应由撤销权人申请,人民法院不主动干预。而无效的民事法律行为的确认,不以当事人的意志为转移,人民法院或者仲裁机构可以在诉讼或者仲裁程序中主动宣告其无效。
		
		\item \textbf{行为效果不同}。可撤销的民事法律行为的撤销权人对权利的行使拥有选择权,如果撤销权人未在法定的期限内行使撤销权的,可撤销的民事法律行为将终局有效,不得再被撤销。可撤销的民事法律行为一经撤销,则视同无效的民事法律行为,其效力溯及至行为开始,即自行为开始时无效。而无效的民事法律行为则自始无效、绝对无效。
		
		\item \textbf{行使时间不同}。可撤销的民事法律行为,其撤销权的行使有时间限制。而无效的民事法律行为不存在此种限制。
	\end{enumerate}
	
	在以下几种情况下,民事法律行为属于可撤销的民事法律行为
	\begin{enumerate}
		\item 重大误解。基于重大误解实施的民事法律行为,行为人(误解方)有权请求人民法院或者仲裁机构予以撤销。交易习惯除外(知道90日内,发生5年内)
		
		\item 显失公平。一方利用对方处于危困状态、缺乏判断能力等情形,致使民事法律行为\textbf{成立时}显失公平的,受损害方有权请求人民法院或者仲裁机构予以撤销。(知道1年内,发生5年内)
		
		在民事法律行为\textbf{成立之后}发生的情势变化,导致双方利益显失公平的,不属于可撤销的民事法律行为,而应当按照诚实信用原则处理。
		
		\item 一方或者第三人以胁迫手段,使对方在违背真实意思的情况下实施的民事法律行为,受胁迫方有权请求人民法院或者仲裁机构予以撤销。(胁迫终止1年内,发生5年内)
		
		\item 一方以欺诈手段,使对方在违背真实意思的情况下实施的民事法律行为,受欺诈方有权请求人民法院或者仲裁机构予以撤销。(欺诈终止1年内,发生5年内)
		
		第三人实施欺诈行为,使一方在违背真实意思的情况下实施的民事法律行为,对方知道或者应当知道该欺诈行为的,受欺诈方有权请求人民法院或者仲裁机构予以撤销。(善意第三人不可撤销)
	\end{enumerate}
	
	撤销权在性质上属于\textbf{形成权},依撤销权人单方的意思表示即可产生相应的法律效力,无须相对人同意。形成权是指依照权利人单方意思表示就可以使已经成立的民事法律关系发生变化的权利。如追认权、解除权、撤销权等
	
	撤销权的存续期间为除斥期间
	
	无效的或者被撤销的民事法律行为\textbf{自始没有}法律约束力。民事法律行为部分无效,不影响其他部分效力的,其他部分仍然有效。
	
	民事法律行为无效、被撤销或者确定不发生效力后,行为人因该行为取得的财产,应当予以返还;不能返还或者没有必要返还的,应当折价补偿。有过错的一方应当赔偿对方由此所受到的损失;各方都有过错的,应当各自承担相应的责任。法律另有规定的,依照其规定。(占有资金按LPR或存款利率来计算利息)
	
	\paragraph{效力待定的民事法律行为}
	效力待定的民事法律行为,是指民事法律行为成立时尚未生效,须经权利人追认才能生效。追认的意思表示自到达相对人时生效。一旦追认,则民事法律行为自成立时起生效;如果权利人拒绝追认,则民事法律行为自成立时起无效。
	
	可能导致效力待定的情况只有两个:限制民事行为能力人独立实施的民事法律行为和无权代理
	
	限制民事行为能力人实施的\textbf{纯获利益}(如接受奖励、赠与)的民事法律行为或者与其年龄、智力、精神健康状况相适应的民事法律行为\textbf{直接有效}。除此以外效力待定。
	
	相对人可以催告法定代理人自收到通知之日起30日内予以追认;法定代理人未作表示的,视为拒绝追认。民事法律行为被追认前,善意相对人有撤销的权利。撤销应当以通知的方式作出。
	
	无权代理还要细分两种情况。狭义的无权代理效力待定、表见代理直接有效。
	
	狭义的无权代理:行为人没有代理权、超越代理权或者代理权终止后,仍然实施代理行为,未经被代理人追认的,\textbf{对被代理人不发生效力}。撤销权和催告权与限制性民事行为能力人类似。
	
	无权代理人以被代理人的名义订立合同,被代理人已经开始履行合同义务或者接受相对人履行的,视为对合同的追认。
	
	行为人实施的行为未被追认的:(1)善意相对人有权请求行为人履行债务或者就其受到的损害请求行为人赔偿,但是赔偿的范围不得超过被代理人追认时相对人所能获得的利益;(2)相对人知道或者应当知道行为人无权代理的,相对人和行为人按照各自的过错承担责任。
	
	表见代理:行为人没有代理权、超越代理权或者代理权终止后,仍然实施代理行为,相对人有理由相信行为人有代理权的,\textbf{代理行为有效}。要成立表见代理,应当具备如下构成要件:
	\begin{enumerate}
		\item 代理人无代理权
		
		\item 相对人主观上为善意且无过失
		
		\item 客观上有使相对人相信无权代理人具有代理权的情形,即存在代理权的外观
		
		\item 相对人基于这种客观情形而与无权代理人成立民事法律行为
	\end{enumerate}
	
	相对人有理由相信无权代理人具有代理权的情形包括但不限于:
	\begin{enumerate}
		\item 被代理人对相对人表示已将代理权授予无权代理人,而实际并未授权
		
		\item 无权代理人持有被代理人的介绍信或者盖有印章的空白合同书,使得相对人相信其有代理权
		
		\item 代理关系终止后,被代理人未采取必要的措施而使相对人仍然相信行为人有代理权,并与之进行民事法律行为
	\end{enumerate}
	
	\paragraph{附条件和附期限的民事法律行为}
	对于附条件的民事法律行为
	\begin{enumerate}
		\item 附\textbf{生效条件}(延缓条件)的民事法律行为,自条件成就时生效。
		
		\item 附\textbf{解除条件}的民事法律行为,自条件成就时失效。
		
		\item 附条件的民事法律行为,当事人为自己的利益不正当地阻止条件成就的,视为条件已成就;不正当地促成条件成就的,视为条件不成就。
	\end{enumerate}
	
	延缓条件亦称“停止条件”,在延缓条件成就之前,民事法律行为已经成立,但是效力却处于停止状态。条件成就之后,民事法律行为发生法律效力。
	
	解除条件亦称“消灭条件”,附解除条件的民事法律行为,在所附条件成就之前,已经发生法律效力,行为人已经开始行使权利和承担义务。当条件成就时,权利和义务则失去法律效力。
	
	\textbf{所附条件应当是双方当事人约定的},如果是法律规定的特定民事法律行为的成立条件,不属于此处所谓的“条件”。
	
	所附条件,可以是自然现象、事件,也可以是人的行为。但应当具备下列特征:(1)必须是将来发生的事实;(2)必须是将来不确定的事实;(3)条件应当是双方当事人约定的;(4)条件必须合法;(5)条件是可能发生的事实。
	
	下列民事法律行为不得附条件:(1)条件与行为性质相违背的,如根据《民法典》的规定,法定抵销不得附条件或者附期限;(2)条件违背社会公共利益或者社会公德的,如结婚、离婚等身份性民事法律行为,原则上不得附条件。
	
	如果条件不可能发生,对于生效条件,视为法律行为不发生效力。对于解除条件,视为未附条件。
	
	附期限的民事法律行为
	\begin{enumerate}
		\item 附生效期限(延缓期限,也称初期)的民事法律行为,自期限届至时生效。
		
		\item 附终止期限(解除期限,也称终期)的民事法律行为,自期限届满时失效。
	\end{enumerate}
	
	所附的期限可以是未来一个确定的日期(如2028年11月11日),也可以是一个不确定的日期(如雷某死亡之日),但无论是不是一个确定的日期,期限的到来是一个必然发生的事件。因此,附期限的民事法律行为的效力的产生或者消灭是确定的、可预知的。
	
	
	\subsection{代理制度}
	
	\paragraph{代理的概念}
	代理是指代理人在代理权限内,以被代理人的名义与第三人实施民事法律行为,由此产生的法律后果直接由被代理人承担的一种法律制度。应当由本人实施的法律行为不得代理。
	
	行纪是指行纪人接受他人委托以自己的名义从事商业活动的行为。拍卖公司(行纪人)与委托人之间的合同是一种典型的行纪合同。与代理的主要区别在于
	\begin{enumerate}
		\item 行纪人以自己的名义实施民事法律行为,而代理人以被代理人的名义实施民事法律行为。
		
		\item 行纪的法律后果由行纪人自行承担,然后会通过其他法律关系(如委托合同)转给委托人;而代理的法律效果直接由被代理人承受。
	
		\item 行纪必须为有偿民事法律行为,而代理既可以有偿,也可以无偿
	\end{enumerate}
	
	代理与传达的区别
	\begin{enumerate}
		\item 传达的任务是忠实传递委托人的意思表示,传达人自己不进行意思表示,传达人不以具有民事行为能力为条件。
		
		\item 代理人在代理权限内可以独立向第三人进行意思表示,因此代理人必须具有相应的民事行为能力。
		
		\item 身份行为(如结婚行为、收养行为)\textbf{不能代理,但可以借助传达人传递意思表示}。
		
	\end{enumerate}

	
	\paragraph{委托代理}
	委托代理是指基于被代理人授权的意思表示而发生的代理。
	2、委托授权为不要式行为,既可以采用书面形式,也可以采用口头或者其他方式授权。
	3、委托代理中的授权行为是一种单方民事法律行为,仅凭被代理人一方的意思表示,即可发生授权的效果。被代理人的授权行为,既可以向代理人进行,也可以向相对人进行,二者效力相同。
	
	执行法人或者非法人组织工作任务的人员,就其职权范围内的事项,以法人或者非法人组织的名义实施民事法律行为,对法人或者非法人组织发生效力。法人或者非法人组织对执行其工作任务的人员职权范围的限制,\textbf{不得对抗善意相对人}。
	
	代理权会存在如下的滥用情况
	\begin{enumerate}
		\item 自己代理:代理人不得以被代理人的名义与自己实施民事法律行为,但是被代理人同意或者追认的除外。
		
		\item 双方代理:代理人不得以被代理人的名义与自己同时代理的其他人实施民事法律行为,但是被代理的双方同意或者追认的除外。
		
		\item 恶意串通:代理人和相对人恶意串通,损害被代理人合法权益的,代理人和相对人应当承担连带责任
	\end{enumerate}
	
	\subsection{诉讼时效制度}
	
	\paragraph{诉讼时效的基本理论}
	诉讼时效的概念
	\begin{enumerate}
		\item 诉讼时效期间届满的,义务人可以提出不履行义务的抗辩。诉讼时效期间届满后,义务人同意履行的,不得以诉讼时效期间届满为由抗辩;义务人已经自愿履行的,不得请求返还。
		
		\item 诉讼时效期间届满时债务人获得抗辩权,但债权人的实体权利并不消灭。
		
		\item 权利人超过诉讼时效期间后起诉的,人民法院应当受理(起诉权并不丧失)。义务人提出诉讼时效抗辩的,人民法院查明无中止、中断、延长事由的,判决驳回权利人的诉讼请求(权利人丧失胜诉权),但权利人的实体权利并不消灭。
		
		\item 义务人未提出诉讼时效抗辩的,人民法院不应对诉讼时效问题进行释明及主动适用诉讼时效的规定进行裁判。
		
		\item 当事人在一审期间未提出诉讼时效抗辩,在二审期间提出的,人民法院不予支持;但其基于新的证据能够证明对方当事人的请求权已过诉讼时效期间的情形除外。
		
	\end{enumerate}
	
	
	\textbf{诉讼时效具有强制性}
	(1)当事人对诉讼时效利益的预先放弃无效。
	(2)诉讼时效的期间、计算方法以及中止、中断的事由由法律规定,当事人约定无效。
	
	下列\textbf{请求权不适用诉讼时效}的规定
	(1)请求停止侵害、排除妨碍、消除危险;
	(2)不动产物权和登记的动产物权的权利人请求返还财产;
	(3)请求支付抚养费、赡养费或者扶养费;
	(4)依法不适用诉讼时效的其他请求权。
	
	下列\textbf{债权请求权不适用诉讼时效}的规定
	(1)支付存款本金及利息请求权;
	(2)兑付国债、金融债券以及向不特定对象发行的企业债券本息请求权;
	(3)基于投资关系产生的缴付出资请求权;
	(4)其他依法不适用诉讼时效规定的债权请求权。
	
	诉讼时效和除斥区间一般有着如下的不同
	\begin{enumerate}
		\item 适用对象不同
		①诉讼时效一般适用于债权请求权;
		②除斥期间一般适用于形成权(如追认权、解除权、撤销权等),也可能适用于请求权(如受遗赠权)。
		
		\item 可以援用的主体不同
		①人民法院不能主动援用诉讼时效,诉讼时效须由当事人主张后,人民法院才能审查;
		②除斥期间无论当事人是否主张,人民法院均可主动审查。
		
		\item 法律效力不同
		①诉讼时效届满只是让债务人取得抗辩权,债权人的实体权利并不消灭;
		②除斥期间届满,实体权利消灭。
	\end{enumerate}
	
	\paragraph{诉讼时效的种类和起算}
	诉讼时效有以下两种
	\begin{enumerate}
		\item 普通诉讼时效。向人民法院请求保护民事权利的诉讼时效期间为3年。法律另有规定的,依照其规定(可以中止、中断,不可以延长)。因国际货物买卖合同和技术进出口合同争议提起诉讼或者申请仲裁的时效期间为4年。
		
		\item 最长诉讼时效。自权利受到损害之日起超过20年的,人民法院不予保护;有特殊情况的,人民法院可以根据权利人的申请决定延长。(不可以中止、中断,可以延长)
	\end{enumerate}
	
	诉讼时效期间的起算
	(1)附条件或者附期限的债的请求权,从条件成就或者期限届满之日起算。
	(2)约定有履行期限的债的请求权,从清偿期限届满之日起算;当事人约定同一债务分期履行的,诉讼时效期间从最后一期履行期限届满之日起计算。
	(3)未约定履行期限或者履行期限不明确的债的请求权,依照《民法典》的规定可以确定履行期限的,诉讼时效期间从履行期限届满之日起计算;不能确定履行期限的,诉讼时效期间从债权人要求债务人履行义务的宽限期届满之日起计算,但债务人在债权人第一次向其主张权利之时明确表示不履行义务的,诉讼时效期间从债务人明确表示不履行义务之日起计算。
	(4)请求他人不作为的债权请求权,应当自权利人知道义务人违反不作为义务时起算。
	(5)国家赔偿的诉讼时效的起算,自赔偿请求人知道或者应当知道国家机关及其工作人员行使职权时的行为侵犯其人身权、财产权之日起计算,但被羁押等限制人身自由期间不计算在内。
	(6)未成年人遭受性侵害的损害赔偿请求权的诉讼时效期间,自受害人年满18周岁之日起算。
	(7)无民事行为能力人或者限制民事行为能力人对其法定代理人的请求权,诉讼时效期间自该法定代理终止之日起算。
	(8)无民事行为能力人或者限制民事行为能力人的权利受到损害的,诉讼时效期间自其法定代理人知道或者应当知道权利受到损害以及义务人之日起计算。法律另有规定的,依照其规定。
	
	\paragraph{诉讼时效的中止}
	诉讼时效中止的事由
	在诉讼时效期间的最后6个月内,因下列障碍,不能行使请求权的,\textbf{诉讼时效中止}:
	\begin{enumerate}
		\item 不可抗力;
		
		\item 无民事行为能力人或者限制民事行为能力人没有法定代理人,或者法定代理人死亡、丧失民事行为能力、丧失代理权;
		
		\item 继承开始后未确定继承人或者遗产管理人;
		
		\item 权利人被义务人或者其他人控制;
		
		\item 其他导致权利人不能行使请求权的障碍。
	\end{enumerate}
	
	自中止时效的原因消除之日起满6个月,诉讼时效期间届满。
	
	\paragraph{诉讼时效的中断事由}
	有下列情形之一的,诉讼时效中断,从中断、有关程序终结时起,\textbf{诉讼时效期间重新计算}:
	\begin{enumerate}
		\item 权利人向义务人、义务人的代理人、财产代管人或者遗产管理人等提出履行请求;以下情形认定为权利人向义务人提出履行请求
		\begin{enumerate}
			\item 当事人一方直接向对方当事人送交主张权利文书,对方当事人在文书上签名、盖章、按指印或者虽未签名、盖章、按指印但能够以其他方式证明该文书到达对方当事人的。
			
			\item 当事人一方以发送信件或者数据电文方式主张权利,信件或者数据电文到达或者应当到达对方当事人的。
			
			\item 当事人一方为金融机构,依照法律规定或者当事人约定从对方当事人账户中扣收欠款本息的。
			
			\item 当事人一方下落不明,对方当事人在国家级或者下落不明的当事人一方住所地的省级有影响的媒体上刊登具有主张权利内容的公告的,但法律和司法解释另有特别规定的,适用其规定。
			
			\item 权利人对同一债权中的部分债权主张权利,诉讼时效中断的效力及于剩余债权,但权利人明确表示放弃剩余债权的情形除外。
		\end{enumerate}
		
		
		\item 义务人同意履行义务;义务人作出分期履行、部分履行、提供担保、请求延期履行、制定清偿债务计划等承诺或者行为的,应当认定为“义务人同意履行义务”
		
		\item 权利人提起诉讼或者申请仲裁;
		\begin{enumerate}
			\item 当事人一方向人民法院提交起诉状或者口头起诉的,诉讼时效从提交起诉状或者口头起诉之日起中断。
			
			\item 权利人向人民调解委员会以及其他依法有权解决相关民事纠纷的国家机关、事业单位、社会团体等社会组织提出保护相应民事权利的请求,诉讼时效从提出请求之日起中断。
			
			\item 权利人向公安机关、人民检察院、人民法院报案或者控告,请求保护其民事权利的,诉讼时效从其报案或者控告之日起中断。
			
			\item 上述机关决定不立案、撤销案件、不起诉的,诉讼时效期间从权利人知道或者应当知道不立案、撤销案件、不起诉之日起重新计算。
		\end{enumerate}
	
		
		\item 与提起诉讼或者申请仲裁具有同等效力的其他情形。
		\begin{enumerate}
			\item 申请支付令;
			
			\item 申请破产、申报破产债权;
			
			\item 为主张权利而申请宣告义务人失踪或者死亡;
			
			\item 申请诉前财产保全、诉前临时禁令等诉前措施;
			
			\item 申请强制执行;
			
			\item 申请追加当事人或者被通知参加诉讼;
		\end{enumerate}
	\end{enumerate}
	
	诉讼时效中断的其他情形
	\begin{enumerate}
		\item 对于连带债权人、连带债务人中的一人发生诉讼时效中断效力的事由,应当认定对其他连带债权人、连带债务人也发生诉讼时效中断的效力。
		
		\item 债权人提起代位权诉讼的,应当认定对债权人的债权和债务人的债权均发生诉讼时效中断的效力。
		
		\item 债权转让的,应当认定诉讼时效从债权转让通知到达债务人之日起中断。
		
		\item 债务承担情形下,构成原债务人对债务承认的,应当认定诉讼时效从债务承担意思表示到达债权人之日起中断。
	\end{enumerate}
	
	\newpage
	
	\section{物权法律制度}
	
	首先基本介绍一下什么是物权,其次就从动产和不动产的角度分别考虑两者的物权是如何变动的。
	
	\subsection{物权法律制度概述}
	\paragraph{物的种类} 民法典规定物分为动产和不动产,如果有其他的法律规定,则权利也可以作为物权客体。物权有着如下的特征
	\begin{enumerate}
		\item 有体性
		
		\item 可支配性
		
		\item 在人的身体之外
	\end{enumerate}
	
	物有着如下的分类
	\begin{enumerate}
		\item \textbf{动产与不动产}。不动产包括土地、海域以及房屋、林木等地上定着物。
		 
		\item \textbf{可分物与不可分物}。可分物是指不因分割而变更其性质或者减损其价值的物。牛肉属于可分物,一头牛则属于 不可分物。
		 
		\item (仅限于动产)\textbf{可替代物与不可替代物}。交易客体为可替代物(如冰棍)时,可以同类物替代履行;不可替代 物 ( 如 齐 白 石 的 某 一幅 字 画 ) 一旦 发 生 毁 损 、 灭 失 , 就 只 能 转 化 为 金 钱 赔 偿 。
		
		\item (仅限于动产) \textbf{消耗(费)物与非消耗(费)物}。消耗物只能 一次性使用或者让与,非消耗物则相反。以让与为目的的 消耗物(如金钱)转移占有即转移所有权。
		
		\item \textbf{流通物、限制流通物与禁止流通物。}流通物可以自由进入市场流通(如冰棍),限制流通物是指被法律限制市场流通之物(如 文物、黄金、药品),禁止流通物是指法律规定禁止流通之物。根据《民法典》的规定,法律 规定专属于国家所有的不动产和动产,任何组织或者个人不能取得所有权。
		
		\item \textbf{主物与从物}。认定主物、从物关系,必须同时具备两个条件:(1 )二者在物理上互相独立;(2 )二者在 经济用途上存在主从关系。A物脱离B物,不损害A物的独立用途,则A物为主物;B物脱 离A物,丧失其本来的用途,则B物为从物。
	\end{enumerate}
	
	物的划分上还有一种划分方式,即原物与孳息
	\begin{enumerate}
		\item 孳息是独立于原物的物,原物 、孳息属 于两个物。因此,尚在母牛身体里的小牛属于母牛的组成部分,不属于孳息;尚未与苹果树相分离的苹果,也不属于孳息。

		\item 孳息分为天然孳息和法定孳息(如储蓄存款的利息、出租房屋获得的租金)。
		
		\item 天然孳息,由所有权人取得;既有所有权人又有用益物权人的,由用益物权人取得。 当事人另有约定的,按照其约定。
		
		\item 法定孳息,当事人有约定的,按照约定取得;没有约定或者约定不明确的,按照交易习惯取得。
	\end{enumerate}
	
	\paragraph{物权的概念}
	物权是权利人依法对特定的物享有直接支配和排他的权利。与债权相比,物权具有以下特征:
	\begin{enumerate}
		\item 支配性:物权人有权仅以自己的意志实现权利,无须第三人的积极行为协助,属于支配权。而债权属于请求权,其实现有赖于债务人的履行行为。
		
		\item 排他性:物权具有排他性, 一物之上只能成立一项所有权。而债权具有兼容性,同一标的物之上可 以成立数个买卖合同,几个买卖合同均可有效,并不相互排斥。
		
		\item 绝对性:物权是可以对抗所有人的财产权,排除任何他人的干涉,他人有义务予以尊重,属于绝对权、对世权。而债权仅对特定的债务人存在,属于相对权、对人权。
	\end{enumerate}
	
	物权可以分为以下几类
	\begin{enumerate}
		\item 所有权
		
		\item 用益物权。《民法典》规定的用益物权包括土地承包经营权、建设用地使用权、宅基地使用权、居住权和地役权
		
		\item 担保物权。担保物权包括抵押权、质权和留置权
	\end{enumerate}
	
	物权的分类上可以分类为自物权和他物权以及独立物权和从物权。独立物权是指能够独立存在的物权,包括所有权、建设用地使用权、土地承包经营权、 宅基地使用权和居住权。从物权是指从属于其他权利、不能独立存在的物权,包括担保物权和地役权。具体的分类如下表所示
	
	
	\begin{table}
		\centering
		\begin{tblr}{
				width = \linewidth,
				colspec = {Q[154]Q[256]Q[119]Q[119]Q[154]Q[119]},
				cell{1}{1} = {c=2}{0.41\linewidth},
				cell{2}{1} = {c=2}{0.41\linewidth},
				cell{3}{1} = {r=5}{},
				cell{8}{1} = {c=2}{0.41\linewidth},
				vlines,
				hline{1-3,8-9} = {-}{},
				hline{4-7} = {2-6}{},
			}
			物权的类型 &         & 自物权 & 他物权 & 独立物权 & 从物权 \\
			所有权   &         & 是   &     & 是    &     \\
			用益物权  & 建设用地使用权 &     & 是   & 是    &     \\
			& 土地承包经营权 &     & 是   & 是    &     \\
			& 宅基地使用权  &     & 是   & 是    &     \\
			& 居住权     &     & 是   & 是    &     \\
			& 地役权     &     & 是   &      & 是   \\
			担保物权  &         &     & 是   &      & 是   
		\end{tblr}
	\end{table}
	
	\paragraph{物权法律制度的基本原则}
	主要有以下三个原则
	\begin{enumerate}
		\item 物权法定原则。根据《民法典》第116条的规定,物权的种类和内容,由法律规定。
		
		\item 物权客体特定原则(一物一权原则)
		\begin{enumerate}
			\item 物权只存在于确定的一物之上,物尚未确定谈不上物权;而债权的客体是当事人的给 付行为,即使物尚未确定、尚不存在,也不影响债权合同(如贾某将正在研发的汽车作价10 0 万卖给孙某)的有效性。
			
			\item 一物之上只能有 一个所有权,但所有权人可以为多人(如甲、乙按份共有或者共同共 有 一辆 汽 车 )。
			
			\item 一物之上只能有 一个所有权,但一物之上可以成立数个互不冲突的物权 。如所有权和他物权的共容、用益物权与担保物权的共容。
		\end{enumerate}
		
		\item 物权公示原则
		\begin{enumerate}
			\item 不动产物权的设立、变更、转让和消灭,应当依照法律规定登记。
			
			\item 动产物权的设立和转让,应当依照法律规定交付。
		\end{enumerate}
	\end{enumerate}
	
	\paragraph{物权行为和债权行为}
	我们对比一下物权行为和债权行为,首先先区分一下两者的概念
	
	债权行为的效力在当事人之间确立债权债务关系,债务人为此负有法律上的义务。例如, 甲、乙双方就某套商品房订立买卖合同,买卖合同生效后,出卖人甲负有向买受人乙转移房屋所有权的义务,乙负有向甲支付相应价款的义务。
	
	买卖合同只是债权行为,并不会直接导致房 屋所有权的转移。\textbf{房屋所有权的转移依赖于出卖人向买受人为了履行买卖合同而转移所有权的行为},该行为在消灭合同之债的意义上称为合同的履行行为,在转移物权的意义上称为物权 行为。
	
	接下来从法律效果、处分权和兼容性三方面进行分析
	
	\textbf{法律效果}上,\textbf{债权行为不会直接引起积极财产 (物权)的减少},\textbf{却会导致消极财产(义务)的增加};\textbf{物权行为则直接导致积极财产的减少}。例如,房屋买卖合同生效后,出卖人负有向买受人转移所 有权的义务,但在履行义务之前,标的物的所有权仍属于出卖人。在出卖人实际履行义务(向 买受人转移所有权)之后,出卖人才失去标的物的所有权。
	
	处分权上
	\begin{enumerate}
		\item 物权行为直接导致物权发生变动,因此出让人应当对标的物享有处分权,否则将构成 无权处分。无权处分行为处于效力待定状态,在得到真权利人的追认或者出让人取得处分权之 后,该行为有效;否则,该行为归于无效。
		
		\item 债权行为只是负担行为而不直接转移物权,因此对出卖人无处分权的要求。出卖他人 之物的买卖合同亦可有效,当出卖人无法履行合同时,买受人可以基于有效的买卖合同主张违 约救济。根据《民法典》的规定,因出卖人未取得处分权致使标的物所有权不能转移的,买受 人可以解除合同并请求出卖人承担违约责任。
	\end{enumerate}
	
	
	兼容性上
	\begin{enumerate}
		\item 物权只能被转让一次,出让人在实施转让物权的物权行为后,即失去所转让标的物的 物权,因此对 于同一标的物不能实施两次有效的处分行为。
		
		\item 债权行为因其仅负担义务,而不涉及物权变动,因此可以反复作出,在同一标的物上 成立的数个买卖合同均可有效,但出卖人只能履行其中一项买卖合同,其他未能获得标的物所 有权的买受人有权基于有效的买卖合同请求出卖人承担违约责任(具体规定见第四单元)。
	\end{enumerate}
	
	\subsection{不动产的物权变动}
	物权变动可能基于法律行为,也可能不基于。以下分别进行讨论
	
	\subsubsection{基于法律行为的物权变动}
	\paragraph{登记生效} 
	不动产物权(包括抵押权)的设立、变更、转让和消灭,\textbf{经依法登记,发生效力}(包括建设用地使用权、居住权);未经登记,不发生效力,但是法律另有规定的除外。

	当事人之间订立有关设立、变更、转让和消灭不动产物权的合同,除法律另有规定或 者当事人另有约定外,自合同成立时生效;未办理物权登记的,不影响合同效力。(2016年案 例分析题 )
	
	\paragraph{登记对抗} 
	土地承包经营权自土地承包经营权合同生效时设立。土地承包经营权互换、转让的, 当事人可以向登记机构申请登记
	
	地役权自地役权合同生效时设立。当事人要求登记的,可以向登记机构申请地役权登记
	
	以上两种情况下,\textbf{未经登记,不得对抗善意第三人}。
	
	\subsubsection{非基于法律行为的物权变动}
	物权变动也可能基于事实行为、法律规定以及公法行为。基于非法律行为的物权变动则不以登记为前提。
	\begin{enumerate}
		\item 基于事实行为。因合法建造、拆除房屋等事实行为设立或者消灭物权的,自事实行为成就时发生效力。
		
		非基于法律行为的不动产物权变动\textbf{不以登记为前提},但获得不动产物权之人\textbf{再处分该不动产时},依照法律规定需要办理登记的,未经登记,不发生物权效力
		
		\item 基于法律规定。因继承取得物权的,自继承开始时发生效力。
		
		\item 基于公法行为。因人民法院、仲裁机构的法律文书或者人民政府的征收决定等,导致物权设立、变更、转 让或者消灭的,自法律文书或者征收决定等生效时发生效力。
		
		【解释】人民法院、仲裁机构在分割共有不动产或者动产等案件中作出并依法生效的 改变原有物权关系的判决书、裁决书、调解书,以及人民法院在执行程序中作出的拍卖成 交裁定书、变卖成交裁定书、以物抵债裁定书,应当认定为“ 导致物权设立、变更、转让 或者消灭的人民法院、仲裁机构的法律文书”。这些 “法律文书” 具有直接改变原有物权 关系、不必由当事人履行的形成效力 。如果判决内容是一方当事人向另一方履行,那么, 让物权发生变动的,是当事人的履行行为而非判决本身。
		
	\end{enumerate}

	\subsubsection{不动产登记制度}
	登记这块主要解释各种登记是什么意思,以及在什么情况下可以进行登记。指的注意的是异议登记有15天的限制。预告登记有90天的限制
	
	\paragraph{首次登记} 是指不动产权利第一次登记。未办理不动产首次登记的,不得办理不动 产其他类型登记,但法律、行政法规另有规定的除外。
	
	除了首次登记外,不动产还会进行\textbf{变更登记、转移登记、注销登记、更正登记、异议登记和预告登记}
	
	\paragraph{变更登记}
	变更登记,是指不动产登记事项发生\textbf{不涉及权利转移的变更}所需进行的登记 。有下列情形之一的,不动产权利人可以向不动产登记机构申请变更登记:
	\begin{enumerate}
		\item 权利人的姓名、名称、身份证明类型或者身份证明号码发生变更的;
		
		\item 不动产的坐落、界址、用途、面积等状况变更的;
		
		\item 不动产权利期限、来源等状况发生变化的;
		
		\item 同一权利人分割或者合并不动产的;
		
		\item 抵押担保的范围、主债权数额、债务履行期限、抵押权顺位发生变化的;
		
		\item 最高额抵押担保的债权范围、最高债权额、债权确定期间等发生变化的;
		
		\item 地役权的利用目的、方法等发生变化的;
		
		\item 共有性质发生变更的;
		
		\item 法律、行政法规规定的其他不涉及不动产权利转移的变更情形。
	\end{enumerate}
	
	
	\paragraph{转移登记} 
	转移登记,是指\textbf{不动产权利在不同主体之间发生转移}所需进行的登记。因下列情形导致不动产权利转移的,当事人可以向不动产登记机构申请转移登记: 
	\begin{enumerate}
		\item 买卖、互换、赠与不动产的;
		
		\item 以不动产作价出资 (入股 )的;
		
		\item 法人或者其他组织因合并、分立等原因致使不动产权利发生转移的;
		
		\item 不动产分割、合并导致权利发生转移的;
		
		\item 继承、受遗赠导致权利发生转移的;
		
		\item 共有人增加或者减少以及共有不动产份额变化的;
		
		\item 因人民法院、仲裁委员会的生效法律文书导致不动产权利发生转移的;
		
		\item 因主债权转移引起不动产抵押权转移的;
		
		\item 因需役地不动产权利转移引起地役权转移的; (10)法律、行政法规规定的其他不动产权利转移情形。
	\end{enumerate}
	
	
	\paragraph{注销登记} 
	\textbf{不动产权利消灭时,需要办理注销登记}。有下列情形之一的,当事人可以申请办理注销登记: 
	\begin{enumerate}
		\item 不动产灭失的;
		
		\item 权利人放弃不动产权利的;
		
		\item 不动产被依法没收、征收或者收回的;
		
		\item 人民法院、仲裁委员会的生效法律文书导致不动产权利消灭的;
		
		\item 法律、行政法规规定的其他情形。
	\end{enumerate}
	
	
	\paragraph{更正登记} 权利人、利害关系人认为不动产登记簿记载的事项错误的,可以申请更正登记。不动产登 记簿记载的权利人书面同意更正或者有证据证明登记确有错误的,登记机构应当予以更正。
	
	\paragraph{异议登记}
	不动产登记簿记载的权利人不同意更正的,利害关系人可以申请异议登记。登记机构予以异议登记,申请人自\textbf{异议登记之日起15日内不提起诉讼的},异议登记失效。异议登记不当, 造成权利人损害的,权利人可以向申请人请求损害赔偿。
	
	\paragraph{预告登记}
	当事人签订买卖房屋的协议或者签订其他不动产物权的协议,\textbf{为保障将来实现物权, 按照约定可以向登记机构申请预告登记}。如以下情形
	\begin{enumerate}
		\item 商品房 等不动产预售的;
		
		\item 不动产买卖、抵押的;
		
		\item 以预购商品房设定抵押权的;
		
		\item 法律、 行政法规规定的其他情形。
	\end{enumerate}
	
	预告登记后,未经预告登记的权利人同意,处分该不动产的(包括转让不动产所有权等物权,或者设立建设用地 使用权、居住权、地役权、抵押权等其他物权),不发生物权效力。
	
	预告登记后,\textbf{债权消灭或者自能够进行不动产登记之日起90日内未申请登记的},预告登记失效。
	
	其中债权消灭是指买卖不动产物权的协议被认定无效、被撤销,或者预告登记的权利人放弃债 权的
	
	\subsection{动产动物权变动}
	不动产的物权变动通常以登记作为节点,动产的物权变动通常以交付为节点
	
	\subsubsection{动产的所有权}
	对于一般动产交付生效。动产物权的设立和转让,自交付时发生效力,但是法律另有规定的除外。
	
	对于特殊动产(船舶、航空器和机动车等)则是交付生效+登记对抗。
	
	\subsubsection{特殊的交付方式}
	除了普通交付,还有简易交付、指示交付、占有交付等特殊的交付方式
	\begin{enumerate}
		\item 简易交付。动产物权设立和转让前,权利人已经占有该动产的,物权自民事法律行为生效时发生效力。
		
		\item 指示交付。动产物权设立和转让前,第三人占有该动产的,负有交付义务的人可以通过转让请求第 三 人返还原物的权利代替交付。
		
		\item 占有交付。动产物权转让时,当事人又约定由出让人继续占有该动产的,物权自该约定生效时发生效力。
	\end{enumerate}

	\subsubsection{动产所有权的特殊取得方式}
	特殊取得方式有先占和添附两种
	\begin{enumerate}
		\item 先占是指以所有权人的意思\textbf{占有无主动产}。先占人基于先占行为取得无主动产的所有权。
		
		\item 添附是附合、混合与加工的总称。
		\begin{enumerate}
			\item 附合是指不同所有权人的物密切结合,构成不可分割的一物。 
			
			动产附合于不动产,由不动产所有权人取得该附合物的所有权。例如,甲错拿乙的钢筋 建造自己的房屋,由甲取得该房屋的所有权。
			
			动产附合于动产, 一般情况下,各动产所有权人按其动产附合时的价值,共有附合物。 但附合的动产,有可视为主物者,则该主物的所有权人取得附合物的所有权。例如,甲错拿乙 的油漆粉刷自己的办公桌,办公桌是主物,因此甲单独取得新办公桌的所有权。 
			
			\item 混合,是指不同所有权人的动产相互混杂合并,不能识别或者识别所需费用过大。例 如,甲错拿乙的牛奶 ,将其倒人自己的咖啡中,难以识别分离。
			
			\item 加工,是指在他人的动产之上进行改造或者劳作,并生成新物的法律事实。例如,甲 将 乙 的 木 板 加 工成 办 公 桌 。
		\end{enumerate}
	\end{enumerate}
	
	\subsection{有权处分和无权处分}
	处分行为是民事法律行为的一种,而处分权也是物权中重要的关注点。因此本节主要考虑分别在什么情况下有权处分以及无权处分
	
	\subsubsection{有权处分}
	这里考虑一下针对普通动产以及特殊动产的一物二卖,什么样的买受人有权处分。总结来讲判断标准分别为: 普通动产:交付 > 付款 > 合同成立时间 ;特殊动产 : 交付 > 登记 > 合同成立时间
	
	\paragraph{普通动产的一物二卖} ( 第 四 章 ) 
	出卖人就同一普通动产订立多重买卖合同,在买卖合同均有效的情况下,买受人均要求实 际履行合同的,应当按照以下情形分别处理: 
	\begin{enumerate}
		\item 先行受领交付的买受人请求确认所有权已经转移的,人民法院应予支持;
		
		\item 均未受领交付,先行支付价款的买受人请求出卖人履行交付标的物等合同义务的,人 民法院应予支持;
		
		\item 均未受领交付,也未支付价款,依法成立在先合同的买受人请求出卖人履行交付标的 物等合同义务的,人民法院应予支持。
	\end{enumerate}
	
	\paragraph{特殊动产的 一物二卖}(第四章)
	出卖人就同 一船舶、航空器、机动车等特殊动产订立多重买卖合同,在买卖合同均有效的 情况下,买受人均要求实际履行合同的,应当按照以下情形分别处理: 
	\begin{enumerate}
		\item 先行受领交付的买受人请求出卖人履行办理所有权转移登记手续等合同义务的,人民 法院应予支持;
		
		\item 均未受领交付,先行办理所有权转移登记手续的买受人请求出卖人履行交付标的物等 合同义务的,人民法院应予支持;
		
		\item 均未受领交付,也未办理所有权转移登记手续,依法成立在先合同的买受人请求出卖 人履行交付标的物和办理所有权转移登记手续等合同义务的,人民法院应予支持; (4)出卖人将标的物交付给买受人之一,又 其他买受人办理所有权转移登记,已受领交 付的买受人请求将标的物所有权登记在自己名下的,人民法院应予支持。
	\end{enumerate}
	
	\subsubsection{无权处分与善意取得制度}
	出卖人因未取得处分权致使标的物所有权不能转移,此时合同有效,买受人可以解除合同并请求出卖人承担违约责任(区分无权处分和无权代理)
	
	无处分权人将不动产或者动产转让给受让人的,所有权人有权追回;除法律另有规定外, 符合下列情形的,受让人取得该不动产或者动产的所有权(善意取得制度):
	\begin{enumerate}
		\item 受让人受让该不动产或者动产时是善意;
		
		\item 以合理的价格转让;
	
		\item 转让的不动产或者动产依照法律规定应当登记的已经登记,不需要登记的已经交付给 受让人。
	\end{enumerate} 
	受让人依据 上述规定取得不动产或者动产的所有权的,原所有权人有权向无处分权人请求损害赔偿。
	
	\subsubsection{拾得遗失物}
	拾得漂流物、发现埋藏物或者隐藏物的,参照适用拾得遗失物的有关规定。 法律另有规定的,依照其规定。这属于准用性规范。
	\begin{enumerate}
		\item 拾得遗失物,应当返还权利人。拾得人应当及时通知权利人领取,或者送交公安等有 关部门。有关部门收到遗失物,知道权利人的,应当及时通知其领取;不知道的,应当及时发 布招领公告。遗失物自发布招领公告之日起1 年内无人认领的,归国家所有。
		
		\item 权利人领取遗失物时,应当向抬得人或者有关部门支付保管遗失物等支出的必要费用。
		
		\item 权利人悬赏寻找遗失物的,领取遗失物时应当按照承诺履行义务。
		
		\item 所有权人或者其他权利人有权追回遗失物。该遗失物通过转让被他人占有的,权利人有权向无处分权人请求损害赔偿,或者自知道或者应当知道受让人之日起\textbf{2年内向受让人请求返还原物};
		
		但是,受让人通过\textbf{拍卖或者向具有经营资格的经营者购得该遗失物的},权利人请求返还原物时应当支付受让人所付的费用。权利人向受让人支付所付费用后,有权向无处分权人 追偿。
		
	\end{enumerate}
	
	\subsection{共有}
	
	\subsubsection{基本规定}
	\paragraph{共有的确定}共有可以分为按份共有和共同共有,无约定一般为按份共有(除共有人具有家庭关系等外)。按份共有的份额无约定一般按照出资额确定,不能确定出资额的视为等额享有
	
	\paragraph{共有物的管理}在管理上,共有人按照约定管理共有的不动产或者动产;没有约定或者约定不明确的,\textbf{各共有人都有管理的权利和义务}。共有人对共有物的管理费用以及其他负担,有约定的,按照其约定;没有约定或者约 定不明确的,\textbf{按份共有人按照其份额负担,共同共有人共同负担}。(2019 年案例分析题)
	
	\paragraph{共有物的分割}共有人约定不得分割共有的不动产或者动产,以维持共有关系的,应当按照约定,但是共 有人有重大理由需要分割的,可以请求分割;没有约定或者约定不明确的,按份共有人可以随 时请求分割,共同共有人在共有的基础丧失或者有重大理由需要分割时可以请求分割。因分割 造成其他共有人损害的,应当给予赔偿。
	
	\paragraph{共有物的对内以及对外责任}因共有的不动产或者动产产生的债权债务,在对外关系上,共有人享有连带债权、承 担连带债务,但是法律另有规定或者第三人知道共有人不具有连带债权债务关系的除外。(2019 年案例分析题 )
	
	在共有人内部关系上,除共有人另有约定外,按份共有人按照份额享有债权、承担债务,共同共有人共同享有债权、承担债务 。偿还债务超过自己应当承担份额的按份共有人,有权向其他共有人追偿。
	
	总结:\textbf{按份共有人}对外承担连带责任、内部承担按份责任。
	
	\subsubsection{按份共有人转让自己的个人份额}
	 按份共有人可以转让其享有的共有的不动产或者动产份额,其他共有人在同等条件下 享有优先购买的权利。民法典所称的“同等条件”,应当综合共有份额的转让价格、价款履行 方式及期限等因素确定。
	 
	按份共有人转让其享有的共有的不动产或者动产份额的,应当将转让条件及时通知其 他共有人。其他共有人应当在合理期限内行使优先购买权。(2019 年案例分析题)
	
	优先购买权的行使期间,按份共有人之间有约定的,按照约定处理;没有约定或者约定不明的,按照下列情形确定:
	\begin{enumerate}
		\item 转让人向其他按份共有人发出的包含同等条件内容的通知中载明行使期间的,以该期 间为准;
		
		\item 通知中未载明行使期间,或者载明的期间短于通知送达之日起15 日的,为15 日;
		
		\item 转让人未通知的,为其他按份共有人知道或者应当知道最终确定的同等条件之日起 15 日;
		
		\item 转让人未通知,且无法确定其他按份共有人知道或者应当知道最终确定的同等条件的, 为共有份额权属转移之日起6 个月。
	\end{enumerate}
	
	两个以上其他共有人主张行使优先购买权的,协商确定各自的购买比例;协商不成的, 按照转让时各自的共有份额比例行使优先购买权。
	
	按份共有人向共有人之外的人转让其份额,其他按份共有人根据法律、司法解释规定, 请求按照同等条件优先购买该共有份额的,人民法院应予支持。其他按份共有人的请求具有下 列情形之 一的,人民法院不予支持: 
	\begin{enumerate}
		\item 未在司法解释规定的期间内主张优先购买,或者虽主张优先购买,但提出减少转让价 款 、增加转让人负担等实质性变更要求;
		
		\item 以其优先购买权受到侵害为由,仅请求撤销共有份额转让合同或者认定该合同无效。 
	\end{enumerate}
	
	按\textbf{份共有人之间转让共有份额}以及\textbf{共有份额的权利主体因继承、遗赠等原因发生变化时},其他按份共有人主张依据《民法典》的规定优先购买的, \textbf{人民法院不予支持},但按份共有人之间另有约定的除外。
	
	
	\subsubsection{共有物的处分}
	处分共有的不动产或者动产以及对共有的不动产或者动产作重大修缮、变更性质或者用途的。不同类型的共有有着不同的条件
	\begin{enumerate}
		\item 按份共有,\textbf{应当经占份额2/3以上的按份共有人同意},但是共有人之间另有约定的除外 。(2019年案 例分析题 )
		
		\item 共同共有 ,\textbf{应当经全体共同共有人同意},但是共有人之间另有约定的除外。(2019 年案例分析题)
	\end{enumerate}
	
	\subsection{建设用地使用权}
	之前主要从物的种类角度考虑了物权,现在我们着重考虑几个特殊的物权(建设用地使用权、担保物权(包括抵押权、质权、留置权))
	
	\subsubsection{建设用地使用权的设立}
	设立建设用地使用权的,应当向登记机构申请建设用地使用权登记。\textbf{建设用地使用权自登记时设立}。设立建设用地使用权,\textbf{可以采取出让或者划拨}等方式。
	
	严格限制以划拨方式设立建设用地使用权(用于商业开发的建设用地,不得以划拨方式取得建设用地使用权)。下列建设用地的土地使用权,确属必需的, 可以由县级以 上人民政府依法批准划拨:
	\begin{enumerate}
		\item 国家机关用地和军事用地;
		
		\item 城市基础设施用地和公益事业用地; 
		
		\item 国家重点扶持的能源、交通、水利等项目用地;
		
		\item 法律、行政法规规定的其他用地。
	\end{enumerate}
	
	建设用地使用权出让,可以采取\textbf{拍卖、招标或者双方协议}的方式。其中,工业 、商业、旅游、娱乐和商品住宅等经营性用地以及同 一土地有两个以上意向用地者的,应当采取招标、 拍卖等公开竞价的方式出让;没有条件,不能采取拍卖、招标方式的,可以采取双方协议的 方式。
	
	\subsubsection{建设用地使用权的期限}
	以无偿划拨方式取得的建设用地使用权,除法律、行政法规另有规定外,没有使用期 限的限制。
	
	以有偿出让方式取得的建设用地使用权,出让最高年限按下列用途确定:
	\begin{enumerate}
		\item 居住用地70年 ;
		
		\item 工业用地50年 ;
		
		\item 教育、科技、文化、卫生、体育用地5 0年;
		
		\item 商业 、 旅 游 、 娱 乐 用 地 4 0 年 ;
		
		\item 综合或者其他用地 5 0 年 。
	\end{enumerate}

	 
	住宅建设用地使用权期限届满的,自动续期。续期费用的缴纳或者减免,依照法律、 行政法规的规定办理。
	
	非住宅建设用地使用权期限届满后的续期,依照法律规定办理。该土地上的房屋以及 其他不动产的归属,有约定的,按照约定;没有约定或者约定不明确的 ,依照法律、行政法规 的规定办理。
	
	\subsubsection{建设用地使用权的转让}
	建设用地使用权转让、互换、出资、赠与或者抵押的,当事人应当采用\textbf{书面形式订立相应的合同}。使用期限由当事人约定 ,但是\textbf{不得超过建设用地使用权的剩余期限}。
	
	建设用地使用权转让、互换、出资或者赠与的,应当向\textbf{登记机构申请变更登记}。
	
	以划拨方式取得土地使用权的,转让房地产时,\textbf{应当按照国务院规定,报有批准权的人民政府审批}。有批准权的人民政府准予转让的,应当由受让方办理土地使用权出让手续,并 依照国家有关规定\textbf{缴纳土地使用权出让金}。
	
	以出让方式取得土地使用权的,转让房地产时,应当符合下列条件:
	\begin{enumerate}
		\item 按照出让合同约定已经支付全部土地使用权出让金,并取得 土地使用权证 书;
		
		\item 按照出让合同约定进行投资开发,属于房屋建设工程的,完成开发投资总额的25\% 以 上,属于成片开发 土地的,形成 工业用地或者其他建设用地条件; 
		
		\item 转让房地产时房屋已经建成的,还应当持有房屋所有权证书。
	\end{enumerate}
	
	
	\subsubsection{集体土地的建设使用}
	对于集体土地有着不同的法律
	\paragraph{农田}
	建设占用土地,涉及农用地转为建设用地的,应当办理农用地转用审批手续。其中:
	\begin{enumerate}
		\item \textbf{永久基本农田转为建设用地的,由国务院批准}。
		
		\item 在土地利用总体规划确定的城市和村庄、集镇建设用地\textbf{规模范围内},为实施该规划而将永久基本农田以外的农用地转为建设用地的,按 土地利用年度计划分批次按照国务院规定由\textbf{原批准土地利用总体规划的机关或者其授权的机关批准}。在已批准的农用地转用范围内,\textbf{具体建设项目用地可以由市、县人民政府批准}。
		
		\item 在土地利用总体规划确定的城市和村庄、集镇建设用地\textbf{规模范围外},将永久基本农田 以外的农用地转为建设用地的,由\textbf{国务院或者国务院授权的省、自治区、直辖市人民政府批准}。
	\end{enumerate}
	
	\paragraph{集体经营性建设用地}城市规划区内的集体所有的 土地,经依法征收转为国有土地后,该幅国有土地的使用权方 可有偿出让,但法律另有规定的除外。
	
	\subsection{担保的一般规定}
	
	\subsubsection{担保的类型}
	保证、抵押、质押和定金,都是依据当事人的合同而设立,称为\textbf{约定担保}。留置则是 直接依据法律的规定而设立,无须当事人之间特别约定,称为\textbf{法定担保}。担保物权分为
	\begin{enumerate}
		\item 意定担保物权(抵押权、质权)
		
		\item 法定担保物权(留置权)。
	\end{enumerate}
	
	担保也可以分为人保、物保、金钱担保
	\begin{enumerate}
		\item 保证是以保证人的财产和信用为担保的基础,属于人的担保 。
		
		\item 抵押、质押和留置,是 以 一定的财产为担保的基础,属于物的担保。
		
		\item 定金是以一定的金钱为担保的基础,称为金钱担保。
	\end{enumerate}
	此外,所有权保留、融资租赁也可具有担保的功能。
	
	
	在担保上还可以有反担保,当第三人为债务人向债权人提供担保的,可以要求债务人提供反担保。反担保方式可以是债务人提供的抵押或者质押,也可以是其他人提供的 保证、抵押、 质押。留置和定金不能作为反担保方式。
	
	\subsubsection{担保合同的效力}
	登记为\textbf{营利法人}的学校、幼儿园、医疗机构、养老机构等提供担保,当事人以其不具有担保资格为由\textbf{主张担保合同无效的,人民法院不予支持}。
	 
	以\textbf{公益为目的的非营利性}学校、幼儿园、医疗机构、养老机构等提供担保的,\textbf{人民法院应当认定担保合同无效},但是有下列情形之 一的除外: 
	\begin{enumerate}
		\item 在购人或者以融资租赁方式承租教育设施、医疗卫生设施、养老服务设施和其他公益 设施时,出卖人、出租人 担保价款或者租金实现而在该公益设施上保留所有权;
		
		\item 以教育设施、医疗卫生设施、养老服务设施和其他公益设施以外的不动产、动产或者 财产权利设立担保物权。
	\end{enumerate}
	
	\subsubsection{担保合同无效的法律责任}
	
	\begin{enumerate}
		\item 主合同有效而第三人提供的担保合同无效
		\begin{enumerate}
			\item 债权人与担保人均有过错的,担保人承担的赔偿责任不应超过债务人不能清偿部分的1/2; 
			
			\item 担保人有过错而债权人无过错的,担保人对债务人不能清偿的部分承担赔偿责任; 
			
			\item 债权人有过错而担保人无过错的,担保人不承担赔偿责任。
		\end{enumerate}
		
		\item 主合同无效导致第三人提供的担保合同无效
		\begin{enumerate}
			\item 担保人无过错的,不承担赔偿责任;
			
			\item 担 保 人 有 过 错 的 , 其 承 担 的 赔 偿 责 任 不 应 超 过 债 务 人 不 能 清 偿 部 分 的 1 / 3 。
		\end{enumerate}
	\end{enumerate}
	
	【相关链接】主合同解除后,担保人对债务人应当承担的民事责任仍应当承担担保责任, 但是担保合同另有约定的除外。
	
	【解释】主合同解除后,担保合同继续有效,担保人仍应按照担保合同承担担保责任, 除非担保合同另有约定。担保合同被确认无效后,债务人、担保人、债权人有过错的,应 当根据其过错各自承担相应的民事责任,即承担《民法典》规定的缔约过失责任。
	
	\subsubsection{借新还旧的担保责任}
	
	\begin{enumerate}
		\item 主合同当事人协议以新贷偿还旧贷,债权人请求旧贷的担保人承担担保责任的,人民 法 院 不 予 支持 。(2 0 2 3 年 案 例 分 析 题 )
		
		\item 主合同当事人协议以新贷偿还旧贷,债权人请求新货的担保人承担担保责任的,按照 下列情形处理:
		\begin{enumerate}
			\item 新货与旧贷的担保人相同的,人民法院应 予支持;
			
			\item 新贷与旧贷的担保人不同,或者旧贷无担保新贷有担保的,人民法院不予支持,但 是债权人有证据证明新贷的担保人提供担保时对以新贷偿还旧贷的事实知道或者应当知道的 除外。
		\end{enumerate}
		
		\item 主合同当事人协议以新贷偿还旧贷,旧贷的物的担保人在登记尚未注销的情形下同意继续为新贷提供担保,在订立新的贷款合同前又以该担保财产为其他债权人设立担保物权,其 他债权人主张其担保物权顺位优先于新贷债权人的,人民法院不予支持。
	\end{enumerate}

	
	
	\subsubsection{不可分性}
	\begin{enumerate}
		\item 主债权未受全部清偿,担保物权人主张就担保财产的全部行使担保物权的,人民法院 应子支持,但是留置权人行使留置权,如果留置财产 可分物的,留置财产的价值应当相当于 债务的金额。
		
		\item 担保财产被分割或者部分转让,担保物权人主张就分割或者转让后的担保财产行使担 保物权的,人民法院应予支持,但是法律或者司法解释另有规定的除外。
		
		\item 主债权被分割或者部分转让,各债权人主张就其享有的债权份额行使担保物权的,人 民法院应予支持,但是法律另有规定或者当事人另有约定的除外。
		
		\item 主债务被分割或者部分转移,债务人自己提供物的担保,债权人请求以该担保财产担 保全部债务履行的,人民法院应予支持;第三人提供物的担保,主张对未经其书面同意转移的 债务不再承担担保责任的,人民法院应予支持。
	\end{enumerate}
	
	
	\subsection{抵押权}
	
	\subsubsection{抵押财产}
	所谓抵押,是指为担保债务的履行,债务人或者第三人\textbf{不转移财产的占有}, 将该财产抵押给债权人,债务人不履行到期债务或者发生当事人约定的实现抵押权的情形, 债权人有权\textbf{就该财产优先受偿}。其中,债务人或者第三人为抵押人,债权人为抵押权人, 提供担保的财产为抵押财产。
	
	债务人或者第三人有权处分的下列财产可以抵押: 
	\begin{enumerate}
		\item 建筑物和其他土地附着物;
		
		\item 建设用地使用权;
		
		\item 海 域 使 用 权 ;
		
		\item 生产设备、原材料、半成品、产品;
		
		\item 正在建造的建筑物、船舶、航空器;
		
		\item 交通运输 工具;
		
		\item 法律、行政法规未禁止抵押的其他财产。
	\end{enumerate}
	
	下列财产不得抵押:
	\begin{enumerate}
		\item 土地所有权;
		
		\item 宅基地、自留地、自留山等集体所有土地的使用权,但是法律规定可以抵押的除外; 
		
		\item 学校、幼儿园、医疗机构等为公益目的成立的非营利法人的教育设施、医疗卫生设施 和其他公益设施;
		
		\item 所有权、使用权不明或者有争议的财产;
		 
		\item 依 法 被 查 封 、 扣 押 、 监 管 的 财 产 ;
		
		\item 法律、行政法规规定不得抵押的其他财产
	\end{enumerate}
	
	此外抵押中还有房地一体原则
	\begin{enumerate}
		\item 城市房地产(房随地走、地随房走):以建筑物抵押的,该建筑物占用范围内的建设用地使用权一并抵押。以建设用地使用权抵 押的,该土地上的建筑物一并抵押。
		
		\item 农村集体土地(地随房走):乡镇、村企业的建设用地使用权不得单独抵押。以乡镇、村企业的厂房等建筑物抵押的, 其占用范围内的建设用地使用权一并抵押。
	\end{enumerate}
	
	
	\subsubsection{抵押合同}
	设立抵押权,当事人应当采用书面形式订立抵押合同
	
	流押条款:抵押权人在债务履行期限届满前,与抵押人约定债务人不履行到期债务时抵押财产归债权 人所有的,只能依法就抵押财产优先受偿(所有权不直接归债权人所有,质押和抵押相同)
	
	担保物权的担保范围包括主债权及其利息、违约金、损害赔偿金、保管担保财产和实 现担保物权的费用。当事人另有约定的,按照其约定。保证的范围不包括保管担保财产的费用
	
	\subsubsection{不动产的抵押}
	以建筑物和其他土地附着物、建设用地使用权、海域使用权和正在建造的建筑物抵押的, 应当办理抵押登记 ,\textbf{抵押权自登记时设立} 
	
	\paragraph{不能办理抵押登记时的责任承担}
	\begin{enumerate}
		\item  不动产抵押合同生效后未办理抵押登记手续,债权人请求抵押人办理抵押登记手续的 , 人民法院应予支持。
		
		\item 抵押财产因不可归责于抵押人自身的原因灭失或者被征收等导致不能办理抵押登记, 债权人请求抵押人在约定的担保范围内承担责任的,人民法院不予支持;但是抵押人已经获得 保险金、赔偿金或者补偿金等,债权人请求抵押人在其所获金额范围内承担赔偿责任的,人民法院依法予以支持 。
		
		\item 因抵押人转让抵押财产或者其他可归责于抵押人自身的原因导致不能办理抵押登记, 债权人请求抵押人在约定的担保范围内承担责任的,人民法院依法予以支持,但是不得超过抵 押权能够设立时抵押人应当承担的责任范围。
		
		\item 当事人申请办理抵押登记手续时,因登记机构的过错致使其不能办理抵押登记,当事 人请求登记机构承担赔偿责任的,人民法院依法予以支持。
	\end{enumerate}
	
	
	\paragraph{以划拨方式取得的建设用地使用权}
	当事人以划拨方式取得的建设用地使用权抵押,抵押人以未办理批准手续为由主张抵 押合同无效或者不生效的,人民法院不予支持。已经依法办理抵押登记,抵押权人主张行使抵 押权的,人民法院应予支持。抵押权依法实现时所得的价款,应当优先用于补缴建设用地使用 权 出 让 金 。 (2 0 1 3 年 案 例 分 析 题 )
	
	抵押人以划拨建设用地上的建筑物抵押,当事人以该建设用地使用权不能抵押或者未 办理批准手续为由主张抵押合同无效或者不生效的,人民法院不子支持。抵押权依法实现时, 拍卖、变卖建筑物所得的价款,应当优先用于补缴建设用地使用权出让金。
	
	拍卖、变卖建筑物所得的价款,应当优先用于补缴建设用地使用权出让金。
	
	\paragraph{土地上新增的建筑物} 
	\begin{enumerate}
		\item 建设用地使用权抵押后,该土地上新增的建筑物不属于抵押财产。该建设用地使用权 实现抵押权时,应当将该土地上新增的建筑物与建设用地使用权一并处分。但是,新增建筑物 所得的价款,抵押权人无权优先受偿。(2013 年案例分析题)
		
		\item 当事人仅以建设用地使用权抵押,债权人主张抵押权的效力及于土地上已有的建筑物 以及正在建造的建筑物已完成部分的,人民法院应予支持。债权人主张抵押权的效力及于正在 建造的建筑物的续建部分以及新增建筑物的,人民法院不予支持。 
		
		\item 当事人以正在建造的建筑物抵押,抵押权的效力范围限于已办理抵押登记的部分。当 事人按照担保合同的约定,主张抵押权的效力及于续建部分、新增建筑物以及规划中尚未建造 的建筑物的,人民法院不予支持。
		
		\item 抵押人将建设用地使用权、土地上的建筑物或者正在建造的建筑物分别抵押给不同债 权人的,人民法院应当根据抵押登记的时间先后确定清偿顺序。		
	\end{enumerate}

	\paragraph{预告登记}如果存在尚未办理建筑物所有权首次登记、预告登记的财产与办理建筑物所有权首次登记时的财产不一致、 抵押预告登记已经失效等情形,则预告登记失效。
	
	如果不存在预告登记失效等情形,认定抵押权自预告登记之日起设立。
	
	
	\subsubsection{动产的抵押}
	以动产抵押的,抵押权自\textbf{抵押合同生效时设立};未经登记,不得对抗善意第三人
	
	动产抵押合同订立后\textbf{未办理抵押登记},动产抵押权的效力按照下列情形分别处理:
	\begin{enumerate}
		\item 抵押人转让抵押财产,受让人占有抵押财产后,抵押权人向受让人请求行使抵押权的, 人民法院不予支持,但是抵押权人能够举证证明受让人知道或者应当知道已经订立抵押合同的 除外;
		
		\item 抵押人将抵押财产出租给他人并移转占有,抵押权人行使抵押权的,租赁关系不受影 响,但是抵押权人能够举证证明承租人知道或者应当知道已经订立抵押合同的除外; 
		
		\item 抵押人的其他债权人向人民法院申请保全或者执行抵押财产,人民法院已经作出财产 保全裁定或者采取执行措施,抵押权人主张对抵押财产优先受偿的,人民法院不予支持;
		
		\item 抵押人破产,抵押权人主张对抵押财产优先受偿的,人民法院不予支持。
	\end{enumerate}
	
	
	以动产抵押的,不得对抗\textbf{正常经营活动中已经支付合理价款并取得抵押财产的买受人}。(和善意取得要件进行对比)
	
	 买受人在出卖人正常经营活动中通过支付合理对价取得已被设立担保物权的动产,担保物权人请求就该动产优先受偿的,人民法院不予支持,但是有下列情形之一的 除外:
	 \begin{enumerate}
	 	\item  购买商品的数量明显超过一般买受人;
	 	
	 	\item 购买出卖人的生产设备;
	 	
	 	\item 订立买卖合同的目的在于担保出卖人或者第三人履行债务;
	 	
	 	\item 买受人与出卖人存在直接或者 间接的控制关系;
	 	
	 	\item 买受人应当查询抵押登记而未查询的其他情形。
	 \end{enumerate}
	
	\subsubsection{动产的浮动抵押}
	企业、个体工商户、农业生产经营者可以将\textbf{现有的以及将有的生产设备、原材料、半成品、 产品抵押},债务人不履行到期债务或者发生当事人约定的实现抵押权的情形,债权人有权就抵 押财产确定时的动产优先受偿。抵押权自抵押合同生效时设立;未经登记,不得对抗善意第 三人。
	
	动产浮动抵押\textbf{无论是否办理抵押登记,均不得对抗正常经营活动中已支付合理价款并取得抵押财产的买受人}。
	
	抵押财产自下列情形之一发生时确定:
	\begin{enumerate}
		\item 债务履行期限届满,债权未实现;
		
		\item 抵押人被宣告破产或者解散;
		
		\item 当事人约定的实现抵押权的情形;
		
		\item 严重影响债权实现的其他情形 。
	\end{enumerate}
	
	
	\subsubsection{抵押物的转让}

	抵押期间,抵押人可以转让抵押财产。当事人另有约定的,按照其约定。抵押财产转 让的,抵押权不受影响。(2021年案例分析题)
	
	抵押人转让抵押财产的,\textbf{应当及时通知抵押权人}。抵押权人能够证明抵押财产转让\textbf{可能损害抵押权的},\textbf{可以请求抵押人将转让所得的价款向抵押权人提前清偿债务或者提存}。转让 的价款超过债权数额的部分归抵押人所有,不足部分由债务人清偿。
	
	关于“当事人约定禁止或者限制转让抵押财产” 的司法解释
	\begin{enumerate}
		\item 未将约定登记
		\begin{enumerate}
			\item 抵押权人请求确认转让合同无效的,人民法院不予支持; 
			
			\item 抵押财产已经交付或者登记,抵押权人请求确认转让不发生物权效力的,人民法院不予 支持,但是抵押权人有证据证明受让人知道的除外; 
			
			\item 抵押权人请求抵押人承担违约责任的,人民法院依法予以支持。
		\end{enumerate}
		
		\item 已经将约定登记
		\begin{enumerate}
			\item 抵押权人请求确认转让合同无效的,人民法院不予支持; 
			
			\item 抵押财产已经交付或者登记,抵押权人主张转让不发生物权效力的,人民法院应予支持, 但是因受让人代替债务人清偿债务导致抵押权消灭的除外
		\end{enumerate}
	\end{enumerate} 
	
	动产抵押合同订立后\textbf{未办理抵押登记,抵押人转让抵押财产},受让人占有抵押财产后, 抵押权人向受让人请求行使抵押权的,人民法院不予支持,但是抵押权人能够举证证明受让人 知道或者应当知道已经订立抵押合同的除外。(动产抵押中未办理抵押登记就转让抵押财产,不对对抗善意第三人)
	
	
	\subsubsection{抵押物的出租}
	
	\paragraph{先出租后抵押} 抵押权设立前,抵押财产已经出租并转移占有的,原租赁关系不受该抵押权的影响。
	
	\paragraph{先抵押后出租} 动产抵押合同订立后未办理抵押登记,抵押人将抵押财产出租给他人并移转占有,抵押权 人行使抵押权的,租赁关系不受影响,但是抵押权人能够举证证明承租人知道或者应当知道已 经订立抵押合同的除外。
	
	\subsubsection{抵押权的效力}
	
	\paragraph{物上代位性}
	\begin{enumerate}
		\item 担保期间,担保财产毁损、灭失或者被征收等,担保物权人可以就获得的保险金、赔偿金或者补偿金等优先受偿。
		
		\item 被担保债权的履行限未届满的,也可以提存该保险金、赔偿金或者补偿金等。
	\end{enumerate} 
	
	\paragraph{孳息} 债务人不履行到期债务或者发生当事人约定的实现抵押权的情形,致使抵押财产被人民法院依法扣押的,自扣押之日起,\textbf{抵押权人有权收取该抵押财产的天然孳息或者法定孳息},但是 抵押权人未通知应当清偿法定孳息义务人的除外。收取的孳息应当先充抵收取孳息的费用。关于孳息的处理总结如下
	\begin{enumerate}
		\item 抵押物被扣押之前的孳息归抵押人,与抵押权人没有关系;
		
		\item 抵押物被 扣押之后,抵押权人有权收取该抵押物的天然孳息(不用通知),但并非取得该孽息的所 有 权 , 而 是 将 孳 息 一 并 计 入 抵 押 财 产 ; 
		
		\item 如果收取法定孳息(如房租), 应通知义务人(如承租人)。
	\end{enumerate}

	
	\paragraph{从物}(2022年案例分析题)
	
	这里主要考虑从物的抵押权效力如何判断。如果产生于抵押权设立前,效力及于从物。反之同理。具体来说
	\begin{enumerate}
		\item 从物产生于抵押权依法设立前,抵押权人主张抵押权的效力及于从物的,人民法院应予支持,但是当事人另有约定的除外。
		
		\item 从物产生于抵押权依法设立后,抵押权人主张抵押权的效力及于从物的,人民法院不予支持,但是在抵押权实现时可以 一并处分。
	\end{enumerate}
	
	\paragraph{添附}
	这里主要分析抵押财产被添附后抵押权的效力问题
	\begin{enumerate}
		\item 抵押权依法设立后,抵押财产被添附,添附物归第 三人所有,抵押权人主张抵押权效 力及于补偿金的,人民法院应予支持。 
		
		\item 抵押权依法设立后,抵押财产被添附,抵押人对添附物享有所有权,抵押权人主张抵 押权的效力及于添附物的,人民法院应予支持,但是添附导致抵押财产价值增加的,抵押权的 效力不及于增加的价值部分。 
		
		\item 抵押权依法设立后,抵押人与第三人因添附成为添附物的共有人,抵押权人主张抵押权的效力及于抵押人对共有物享有的份额的,人民法院应予支持 。		
	\end{enumerate}

	
	\paragraph{同一财产向两个以上债权人设定抵押时的清偿顺序}(2018 年案例分析题) 
	
	同一财产向两个以上债权人抵押的,拍卖、变卖抵押财产所得的价款依照下列规定清偿: 
	\begin{enumerate}
		\item 抵押权已经登记的,按照登记的时间先后确定清偿顺序;
		
		\item 抵押权已经登记的先于未登记的受偿;
		
		\item 抵押权未登记的,按照债权比例清偿。 【解释】抵押权均未登记的,按照债权比例(而非抵押权设立的时间先后)清偿。
	\end{enumerate}
	
	\paragraph{抵押权顺位的变更} 抵押权人与抵押人可以协议变更抵押权顺位以及被担保的债权数额等内容。但是,抵押权 的变更未经其他抵押权人书面同意的,不得对其他抵押权人产生不利影响。
	
	债务人以自己的财产设定抵押,抵押权人放弃该抵押权、抵押权顺位或者变更抵押权的, 其他担保人在抵押权人丧失优先受偿权益的范围内免除担保责任,但是其他担保人承诺仍然提 供担保的除外。
	
	\paragraph{抵押权的消灭}
	\begin{enumerate}
		\item 债权消灭
		
		\item 抵押权实现
		
		\item 抵押物灭失。抵押物灭失,将导致抵押权消灭。但是,如果抵押物灭失之后存在保险金、赔偿金等价值 转换形态,则抵押权并不消灭。
		
		\item 混同。如果抵押权人获得抵押物的所有权,集抵押权人与抵押人于一身,将导致抵押权消灭。
	\end{enumerate}
	
	
	\subsection{质权}
	
	\subsubsection{动产质押}
	动产、不动产均可抵押。质押包括动产质押和权利质押。所谓动产质押,是 指为担保债务的履行 ,\textbf{债务人或者第三人将其动产出质给债权人占有的},债务人不履行到 期债务或者发生当事人约定的实现质权的情形,债权人有权就该动产优先受偿。其中,债 务人或者第三人为出质人,债权人为质权人,交付的动产为质押财产。
	
	 动产质权自出质人交付质押财产时设立。(和动产抵押对比)
	 
	质权人对于质物处分也存在限制 
	\begin{enumerate}
		\item 质权人在质权存续期间,未经出质人同意,擅自使用、处分质押财产,造成出质人损 害的,应当承担赔偿责任。
		
		\item 质权人在质权存续期间,未经出质人同意转质,造成质押财产毁损、灭失的,应当承 担赔偿责任。
	\end{enumerate}
	
	质权人有着妥善保管质物的义务
	\begin{enumerate}
		\item 因保管不善致使质押财产毁损、灭失的,应当承担赔偿责任。
		
		\item 质权人的行为可能使质押财产毁损、灭失的,出质人可以请求质权人将质押财产提存, 或者请求提前清偿债务并返还质押财产。
	\end{enumerate} 
	
	出质人有及时行使质权请求权 
	\begin{enumerate}
		\item 出质人可以请求质权人在债务履行期限届满后及时行使质权;质权人不行使的,出质 人可以请求人民法院拍卖、变卖质押财产。
		
		\item 出质人请求质权人及时行使质权,因质权人怠于行使权利造成出质人损害的,由质权 人承担赔偿责任。
	\end{enumerate}
	
	\subsubsection{权利质押}
	债务人或者第三人有权处分的下列权利可以出质:
	\begin{enumerate}
		\item 汇票、本票、支票;
		
		\item 债券、存款单;
		
		\item 仓单、提单;
		
		\item 可以转让的基金份额、股权;
		
		\item 可以转让的注册商标专用权、专利权、著作权等知识产权中的财产权;
		
		\item 现有的以及将有的应收账款;
		
		\item 法律、行政法规规定可以出质的其他财产权利。 
	\end{enumerate}【解释】根据物权法定原则,不动产、建设用地使用权、海域使用权可以设定抵押, 但不能设定质押。可以转让的股权、应收账款等权利可以设定质押,但不能设定抵押。
	
	质权设立的时点上以汇票、本票、支票、债券、存款单、仓单、提单出质的,\textbf{质权自权利凭证交付质权人时设立};\textbf{没有权利凭证的,质权自办理出质登记时设立}。
	\begin{enumerate}
		\item 以汇票出质,当事人以背书记载“质押” 字样并在汇票上签章,汇票已经交付质权人 的,人民法院应当认定质权自汇票交付质权人时设立。 
		
		\item 存货人或者仓单持有人在仓单上以背书记载“质押” 字样,并经保管人签章,仓单已 经交付质权人的,人民法院应当认定质权自仓单交付质权人时设立。没有权利凭证的仓单,依 法可以办理出质登记的,仓单质权自办理出质登记时设立。
		
		\item 出质人既以仓单出质,又以仓储物设立担保,按照公示的先后确定清偿顺序;难以确 定先后的,按照债权比例清偿。 
		
		\item 保管人为同一货物签发多份仓单,出质人在多份仓单上设立多个质权,按照公示的先 后确定清偿顺序;难以确定先后的,按照债权比例受偿。
	\end{enumerate} 
	
	以\textbf{可以转让的基金份额、股权出质}的,质权自办理出质登记时设立。基金份额、股权出质后,不得转让,但是出质人与质权人协商同意的除外。(2022年案例分析题)
	
	以\textbf{可以转让的注册商标专用权、专利权、著作权等知识产权中的财产权出质}的,质权 自办理出质登记时设立。知识产权中的财产权出质后,出质人不得转让或者许可他人使用,但 是出质人与质权人协商同意的除外。
	
	以现有的以及将有的应收账款出质的,质权自办理出质登记时设立。应收账款出质后, 不得转让,但是出质人与质权人协商同意的除外。
	\begin{enumerate}
		\item  以现有的应收账款出质,应收账款债务人向质权人确认应收账款的真实性后,又以应 收账款不存在或者已经消灭为由主张不承担责任的,人民法院不予支持。
		
		\item 以现有的应收账款出质,应收账款债务人未确认应收账款的真实性,质权人以应收账 款债务人为被告,请求就应收账款优先受偿,能够举证证明办理出质登记时应收账款真实存在 的,人民法院应予支持;质权人不能举证证明办理出质登记时应收账款真实存在,仅以已经办 理出质登记为由,请求就应收账款优先受偿的,人民法院不予支持。 
		
		\item 以现有的应收账款出质,应收账款债务人已经向应收账款债权人履行了债务,质权人 请求应收账款债务人履行债务的,人民法院不予支持,但是应收账款债务人接到质权人要求向 其履行的通知后,仍然向应收账款债权人履行的除外。
		
		\item 以基础设施和公用事业项目收益权、提供服务或者劳务产生的债权以及其他将有的应 收账款出质,当事人为应收账款设立特定账户,发生法定或者约定的质权实现事由时,质权人 请求就该特定账户内的款项优先受偿的,人民法院应予支持;特定账户内的款项不足以清偿债 务或者未设立特定账户,质权人请求折价或者拍卖、变卖项目收益权等将有的应收账款,并以 所得的价款优先受偿的,人民法院依法予以支持。
	\end{enumerate}


	\begin{table}[h!]
		\centering
		\begin{tblr}{
				width = \linewidth,
				colspec = {Q[129]Q[535]Q[273]},
				cell{1}{1} = {r=4}{},
				cell{5}{1} = {r=7}{},
				cell{5}{3} = {r=4}{},
				cell{9}{3} = {r=3}{},
				cell{12}{1} = {r=4}{},
				cell{12}{3} = {r=3}{},
				vlines,
				hline{1,5,12,16} = {-}{},
				hline{2-4,9,15} = {2-3}{},
				hline{6-8,10-11,13-14} = {2}{},
			}
			动产   & 一般动产的所有权              & 交付生效      \\
			& 船舶、航空器、机动车的所有权        & 交付生效、登记对抗 \\
			& 动产的抵押权                & 登记生效      \\
			& 动产的质权                 & 交付生效      \\
			不动产  & 房屋的转让、抵押              & 登记生效      \\
			& 建设用地使用权的设立、转让、抵押      &           \\
			& 海域使用权的抵押              &           \\
			& 居住权的设立                &           \\
			& 土地承包经营权的设立            & 登记对抗      \\
			& 地役权的设立                &           \\
			& 以家庭承包方式取得的土地使用权的抵押    &           \\
			权利质押 & 可以转让的基金份额、股权          & 登记生效      \\
			& 可以转让的知识产权中的财产权        &           \\
			& 现 有的 以 及将 有 的 应 收 账 款 &           \\
			& 票据、债券、存款单、仓单、提单       & 交付(登记)生效  
		\end{tblr}
	\end{table}
	
	\subsection{留置权}
	
	\subsubsection{留置权}
	债务人不履行到期馈务,债权人可以留置已经合法占有的债务人或者第 三人的动产, 并有权就该动产优先受偿。其中,债权人为留置权人,占有的动产为留置财产 。(2022年案例 分析题 )
	
	【解释1】留置权的行使对象\textbf{仅限于动产}。留置权属于法定担保物权,只要具备法定要件,债权人就可以依法行使留置权 。但是,\textbf{当事人可以通过合同约定事先排除留置权的适用}。例如,甲、乙公司订立保管合同,双方事先约定,即使债务人甲公司不能按期支付保 管费,债权人乙公司也不得行使留置权,该约定有效。
	
	【解释2 】留置财产为可分物的,留置财产的价值应当相当于债务的金额。
	
	
	债权人留置的动产,\textbf{应当与债权属于同一法律关系},但是企业之间留置的除外。 【解释】所谓“同一法律关系 ”,是指占有人交付或者返还占有物的义务与留置所担保 的债权属于同一法律关系。因此在同一法律关系以及非同一法律关系的情况下,债权人与第三人有着不同的权利
	\begin{enumerate}
		\item 债务人不履行到期债务,债权人\textbf{因同一法律关系}留置合法占有的第 三人的动产,并主 张就该留置财产优先受偿的,人民法院应予支持。第三人以该留置财产并非债务人的财产为由 请求返还的,人民法院不予支持。
		
		\item 企业之间留置的动产与债权\textbf{并非同一法律关系},债权人留置第 三人的财产,第三人请 求债权人返还留置财产的,人民法院应予支持。
	\end{enumerate}
	
	企业之间留置的动产与债权并非同 一法律关系,债务人以该债权不属于企业持续经营 中发生的债权为由请求债权人返还留置财产的,人民法院应予支持。
	
	留置权人与债务人应当约定留置财产后的债务履行期限;没有约定或者约定不明确的, 留置权人应当给债务人60 日以上履行债务的期限,但是鲜活易腐等不易保管的动产除外。债 务人逾期未履行的,留置权人可以与债务人协议以留置财产折价,也可以就拍卖、变卖留置财 产所得的价款优先受偿。留置财产折价或者变卖的,应当参照市场价格。
	
	债务人可以请求留置权人在债务履行期限届满后行使留置权;留置权人不行使的,债 务人可以请求人民法院拍卖、变卖留置财产。
	【相关链接】动产质押的出质人可以请求质权人在债务履行期限届满后及时行使质权; 质权人不行使的,出质人可以请求人民法院拍卖、变卖质押财产。
	
	留置权的消灭原因
	\begin{enumerate}
		\item 债权消灭;
		
		\item 留置权人对留置财产丧失占有; 
		
		\item 债务人另行提供担保并被留置权人接受。
	\end{enumerate}
	
	
	\subsubsection{担保物权与诉讼时效}
	
	\paragraph{抵押权} 抵押权人应当在主债权诉讼时效期间行使抵押权;未行使的,人民法院不予保护。抵押人 以主债权诉讼时效期间届满为由,主张不承担担保责任的,人民法院应予支持。
	
	\paragraph{留置权 }主债权诉讼时效期间届满后,财产被留置的债务人或者对留置财产享有所有权的第三人请 求债权人返还留置财产的,人民法院不予支持;债务人或者第 三人请求拍卖、变卖留置财产并 以所得价款清偿债务的,人民法院应予支持。
	
	\paragraph{质权}
	主债权诉讼时效期间届满的法律后果,以 登记作为公示方式的权利质权,参照适用抵押权 的规定;动产质权、以交付权利凭证作为公示方式的权利质权,参照适用留置权的规定。
	
	【相关链接1 】以可以转 让的基金份额、股权出质的,质权自办理出质登记时设立。 
	
	【相关链接2 】以汇票、本票、支票、债券、存款单、仓单、提单出质的,质权自权利 凭证交付质权人时设立;没有权利凭证的,质权自办理出质登记时设立。
	
	\subsection{担保的并存}
	
	\subsubsection{人保+物保}
	被担保的债权既有物的担保又有人的担保的,债务人不履行到期债务或者发生当事人约定 的实现担保物权的情形,债权人应当按照约定实现债权;没有约定或者约定不明确:
	
	\textbf{债务人自己提供物的担保的},债权人应当先就该物的担保实现债权。同一债权既有债 务人自己提供的物的担保,又有第 三人提供的担保,承担了担保责任或者赔偿责任的第 三人, 主张行使债权人对债务人享有的担保物权的,人民法院应予支持。
	
	【解释】之所以先就债务人提供的物保实现债权,是因为这样既可以避免法律关系的 复杂化,又有助于节省司法成本。如果先由保证人承担责任,那保证人必然再向债务人追偿, 其仍然可能要就债务人的物保变价求偿,会造成较多资源浪费。在债务人的物保与第三人 的物保并存时,也应同样处理,除非债务人提供的物保不足以清偿全部债务。(2024 年新增)
	
	\textbf{第三人提供物的担保的},债权人可以就物的担保实现债权,也可以请求保证人承担保 证责任。提供担保的第三人承担担保责任后,有权向债务人追偿。
	
	【相关链接】主债务被分割或者部分转移,债务人自己提供物的担保,债权人请求以 该担保财产担保全部债务履行的,人民法院应予支持;第三人提供物的担保,主张对未经 其书面同意转移的债务不再承担担保责任的,人民法院应予支持。

	
	\subsubsection{动产抵押权、质权与留置权的竞存}
	同一动产向两个以上债权人设定抵押时的清偿顺序
	\begin{enumerate}
		\item 抵押权已经登记的,按照登记的时间先后确定清偿顺序;
		
		\item 抵押权已经登记的先于未登记的受偿;
		
		\item 抵押权未登记的,按照债权比例清偿。
	\end{enumerate}
	
	同一财产既设立抵押权又设立质权的,拍卖、变卖该财产所得的价款按照登记、交付的时 间先后确定清偿顺序。
	
	同一动产上已经设立抵押权或者质权,该动产又被留置的,留置权人优先受偿。
	
	\subsubsection{动产抵押权人的超级优先权}
	《民法典》的规定,动产抵押担保的主债权是抵押物的价款,标的物交付后10 日内办理抵押登记的,该抵押 权人优先于抵押物买受人的其他担保物权人受偿,但是留置权人除外。
	\begin{enumerate}
		\item 担保人在设立动产浮动抵押并办理抵押登记后又购人或者以融资租赁方式承租新的 动产,下列权利人为担保价款债权或者租金的实现而订立担保合同,并在该动产交付后10 日内 办理登记,主张其权利优先于在先设立的浮动抵押权的,人民法院应予支持:
		\begin{enumerate}
			\item 在该动产上设立抵押权或者保留所有权的出卖人; 
			
			\item 为价款支付提供融资而在该动产上设立抵押权的债权人; 
			
			\item 以融资租赁方式出租该动产的出租人。
		\end{enumerate}
		
		\item 买受人取得动产但未付清价款或者承租人以融资租赁方式占有租赁物但是未付清全部 租金, 又以标的物 为他 人设 立担保物权, 上述权利 人为担保价款债权或者租 金的实现而订立担保合同,并在该动产交付后10 日内办理登记,主张其权利优先于买受人为他人设立的担保物 权的,人民法院应予支持。(2022 年案例分析题)
		
		\item 同一动产上存在多个价款优先权的,人民法院应当按照登记的时间先后确定清偿顺序。
	\end{enumerate}
	
	
	抵押人购买动产时,一般需要向出卖人或者金融机构融资,并就该债权提供 担保,通常是在该动产上设定抵押。与此同时,抵押人可能之前已经设定过浮动抵押,该 动产可能会被纳入抵押财产中,进而成为其他债权担保的一部分。在这种情况下,对于该 动产的拍卖价款,谁优先受偿?
	
	\subsubsection{让与担保}
	\paragraph{让与担保中所有权的效力}
	根据《民法典担保制度解释》第68 条第 一款的规定,债务人或者第三人与债权人约定将 财产形式上转移至债权人名下,债务人不履行到期债务,债权人有权对财产折价或者以拍卖、 变卖该财产所得价款偿还债务的,人民法院应当认定该约定有效。当事人已经完成财产权利变 动的公示,债务人不履行到期债务,债权人请求参照 《民法典》关于担保物权的有关规定就该 财产优先受偿的,人民法院应予支持。
	
	【解释】所谓“将财产形式上转移”,意味着当事人所转移的所有权并非真正意义上的 所有权,而是仅具有担保功能的所有权。形式上的受让人并不享有对财产的全面支配权, 而只享有就该财产进行变价、优先受偿的权利。
	
	\paragraph{流质的禁止}
	(1 )根据 《民法典担保制度解释》第 68 条第二款的规定,债务人或者第三人与债权人约 定将财产形式上转移至债权人名下,债务人不履行到期债务,财产归债权人所有的,人民法院 应当认定该约定无效,但是不影响当事人有关提供担保的意思表示的效力。当事人已经完成财 产权利变动的公示,债务人不履行到期债务,债权人请求对该财产享有所有权的,人民法院不 予支持;债权人请求参照《民法典》关于担保物权的规定对财产折价或者以拍卖 、变卖该财产 所得的价款优先受偿的,人民法院应予支持;债务人履行债务后请求返还财产,或者请求对财 产折价或者以拍卖、变卖所得的价款清偿债务的,人民法院应予支持。
	
	【解释】在让与担保等非典型担保中,仍应适用流质(流押)禁止规定 。換言之,在 债务人不履行到期债务时,债权人不能直接获得真正的所有权,而只能对财产进行合理折 价,或者以拍卖、变卖该财产所得的价款优先受偿。
	
	(2)根据《民法典合同编通则解释》第28条的规定,债务人或者第 三人与债权人在债务 履行期限届满前达成以物抵债协议的,人民法院应当在审理债权债务关系的基础上认定该协议 的效力。当事人约定债务人到期没有清偿债务,债权人可以对抵债财产拍卖、变卖、折价以实 现债权的,人民法院应当认定该约定有效。当事人约定债务人到期没有清偿债务,抵债财产归 债权人所有的,人民法院应当认定该约定无效,但是不影响其他部分的效力;债权人请求对抵 债财产拍卖、变卖、折价以实现债权的,人民法院应子支持。当事人订立上述以物抵债协议后, 债务人或者第 三人未将财产权利转移至债权人名下,债权人主张优先受偿的,人民法院不予支 持;债务人或者第三人已将财产权利转移至债权人名下的,依据《民法典担保制度解释》第 68 条的规定处理。
	
	\newpage
	\section{合同法律制度}
	
	
	\newpage
	\section{合伙企业法律制度}
	
	
	
	\newpage
	\section{公司法律制度}
	
	
	\newpage
	\section{证券法律制度}
	
	\section{企业破产法律制度}
	
	\section{票据与支付结算法律制度}
	
	\section{企业国有资产法律制度}
	
	\section{反垄断法律制度}
	
	\section{涉外法律制度}
	
\end{document}